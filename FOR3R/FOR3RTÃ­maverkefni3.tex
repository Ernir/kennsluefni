\documentclass{article}

\usepackage[top=0.9in, bottom=1in, left=1.5in, right=1.5in]{geometry}
\usepackage[utf8]{inputenc}
\usepackage[icelandic]{babel}
\usepackage[T1]{fontenc}
\usepackage[sc]{mathpazo} % Palatino font

\usepackage[parfill]{parskip}
\usepackage{booktabs,tabularx}
\usepackage{multirow}
\usepackage{graphicx}
\usepackage{gensymb}
\usepackage{amsmath}
\usepackage{minted} %Minted and configuration

\usepackage[pdftex,bookmarks=true,colorlinks=true,pdfauthor={Eirikur Ernir Thorsteinsson},linkcolor=blue,urlcolor=blue]{hyperref}

\usemintedstyle{default}
\renewcommand{\theFancyVerbLine}{\sffamily \arabic{FancyVerbLine}}

\date{Vorönn 2016}
\author{}

\hyphenpenalty=5000

\setcounter{secnumdepth}{-1} 
\pagenumbering{gobble}

\title{FOR3R3U Tímaverkefni - Strengjareiknirit}

\begin{document}

\maketitle

\section{Verkefni}
Þessi verkefni eru ætluð til æfingar. Þeim þarf ekki að skila.
\subsection{1. Einföld strengjaleit}
Skrifið einfalt hlutstrengjaleitarfall í Python, \texttt{string\_find}. Það skal taka inn tvo strengi, \texttt{pattern} og \texttt{text} (gerum ráð fyrir að lengd \texttt{pattern} sé minni en lengd \texttt{text}). Það skal skila þeim vísum (e. \emph{indices}) þar sem \texttt{pattern} hefst sem hlutstrengur í \texttt{text}. 

Þannig myndi t.d. kall á \texttt{string\_find} með \texttt{pattern = "ana"} og \texttt{text = "bananaananas"} skila vísunum $1, 3, 6$ og $8$.

Veltið fyrir ykkur hver tímaflækja reikniritsins er m.t.t. lengdar strengjanna \texttt{pattern} og \texttt{text}.

\subsection{2. Hamming fjarlægð}
Svokölluð Hamming-fjarlægð (e. \emph{Hamming distance}) milli tveggja strengja af sömu lengd er fjöldi þeirra sæta í strengjunum þar sem tákn strengjanna eru ekki eins. Skrifið Python-fallið \texttt{hamming\_distance} sem tekur inn tvo strengi, \texttt{text\_1} og \texttt{text\_2} og skilar Hamming-fjarlægð þeirra á milli.

Þannig myndi t.d. kall á \texttt{hamming\_distance} með \texttt{text\_1 = "abraca"} og \texttt{text\_2 = "ababra"} skila tölunni 3, þar sem 3 sæti í strengjunum eru ekki eins.
\subsection{3. Ónákvæm leit}
Tiltölulega ``einfalt'' er að finna hlutstrengi þar sem hlutstrengurinn kemur nákvæmlega fyrir í strengnum sem leita skal í (sjá dæmi 1).

Nokkuð algengt er hins vegar að gögnin passi ekki nákvæmlega saman, t.d. vegna villna í strengjunum.

Skrifið ónákvæmt hlutstrengjaleitarfall í Python, \texttt{approximate\_string\_find}. 

Það skal taka inn tvo strengi og eina heiltölu, \texttt{pattern}, \texttt{text} og \texttt{distance}. Það skal skila þeim vísum þar sem Hamming-fjarlægð á milli \texttt{pattern} og hlutstrengsins í \texttt{text} er minni en eða jöfn \texttt{distance}.

\end{document}