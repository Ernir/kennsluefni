\documentclass{article}

\usepackage[top=0.9in, bottom=1in, left=1.5in, right=1.5in]{geometry}
\usepackage[utf8]{inputenc}
\usepackage[icelandic]{babel}
\usepackage[T1]{fontenc}
\usepackage[sc]{mathpazo} % Palatino font

\usepackage[parfill]{parskip}
\usepackage{booktabs,tabularx}
\usepackage{multirow}
\usepackage{graphicx}
\usepackage{gensymb}
\usepackage{amsmath}
\usepackage{minted} %Minted and configuration

\usepackage[pdftex,bookmarks=true,colorlinks=true,pdfauthor={Eirikur Ernir Thorsteinsson},linkcolor=blue,urlcolor=blue]{hyperref}

\usemintedstyle{default}
\renewcommand{\theFancyVerbLine}{\sffamily \arabic{FancyVerbLine}}

\date{Vorönn 2016}
\author{}

\hyphenpenalty=5000

\setcounter{secnumdepth}{-1} 
\pagenumbering{gobble}

\title{FOR3R3U - Verklýsing fyrir nemendakynningu}

\begin{document}

\maketitle
\section{Um kynningar}
Í viku 17 (25. til 28. apríl) skuluð þið halda kynningu á einu ákveðnu efni sem tengist reikniritum og/eða tölvunarfræði.

Unnið er í tveggja manna hópum.

Við einkunnagjöf verður tekið tillit til:
\begin{itemize}
 \item Þekkingar á efninu (\emph{hafið þið kynnt ykkur efnið ítarlega?})
 \item Framsögu (\emph{komið þið efninu til skila?})
 \item Faglegra vinnubragða (\emph{komið þið fram eins og vel undirbúið fagfólk?})
\end{itemize}

\section{Dæmi um möguleg verkefni}

\begin{itemize}
 \item $P = NP$ spurningin
 \item $NP$-complete vandamál
 \item Dulkóðun (e. \emph{encryption})
 \item Þáttun heiltalna (e. \emph{integer factorization}) 
 \item Fallaforritun (e. \emph{functional programming})
 \item ``Logic programming''
 \item Ýmis verkefni sem tengjast gervigreind:
 \begin{itemize}
  \item Ákvarðanatré (e. \emph{decision trees})
  \item Skák í tölvum
  \item ``Genetic algorithms''
  \item ``Artificial neural networks''
 \end{itemize}
 \item Heapsort reikniritið
 \item Hakkaföll og hakkatöflur (e. \emph{hashing functions} og \emph{hash tables})
 \item MapReduce
 \item ``Quantum computing''
 \item Ýmis verkefni sem tengjast netum:
 \begin{itemize}
  \item Reiknirit Dijkstra
  \item ``Travelling Salesman Problem''
  \item Ford-Fulkerson reikniritið
  \item Reiknirit Kruskals
  \item Litun á netum (e. \emph{graph coloring})
 \end{itemize}
 \item Rauðsvört tré (e. \emph{Red-Black trees})
 \item ``Treap''
 \item ``Dining Philosophers problem''
\end{itemize}
Listinn er ekki tæmandi.
\end{document}
