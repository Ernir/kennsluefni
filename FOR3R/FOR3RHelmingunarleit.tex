\documentclass{beamer}

\usepackage{ts-glærur}

\title{FOR3R - Helmingunarleit}

\begin{document}
\begin{frame}
\titlepage
\end{frame}

\section{Að leita}

\begin{frame}{Að leita}
\begin{itemize}
 \item ``Leit í lista'' er frekar hversdagslegt verkefni.
 \item Dæmi:
 \begin{itemize}
  \item Hvar er talan $5$ í listanum $l = [4, 6, 7, 4, 5, 9]$?
  \item Svar: í sæti 4.
 \end{itemize}
 \item Við þekkjum þegar reikniritið ``línuleg leit'', sem athugar einfaldlega alla möguleika og finnur fyrir okkur svarið.
\end{itemize}
\end{frame}

\subsection{Helmingunarleit}

\begin{frame}{Tengt verkefni: ``Giskileikur''}
\begin{itemize}
 \item ``Giskileikur'' er svipað verkefni og ``leit í lista''. 
 \item Dæmi:
 \begin{itemize}
  \item ``Ég er að hugsa um tölu á milli 1 og 100. Þú mátt spyrja já og nei spurninga.''
 \end{itemize}
 \item Nennum við að spyrja um allar tölurnar?
 \pause
 \item Af hverju getum við hér komist upp með að spyrja ekki um allar, en ekki þegar við erum að framkvæma ``venjulega leit'' í lista?
 \pause
 \item Getum við skilið þetta vandamál nógu vel til að láta tölvu leysa það?
\end{itemize}
\end{frame}

\begin{frame}{Divide and Conquer}
Giskileikurinn hentar vel til lausnar með Divide and Conquer.
\begin{itemize}
 \item  \textbf{Divide} the problem into a number of subproblems that are smaller instances of the
same problem.
 \item \textbf{Conquer} the subproblems by solving them recursively. If the subproblem sizes are
small enough, however, just solve the subproblems in a straightforward manner.
 \item \textbf{Combine} the solutions to the subproblems into the solution for the original problem.
\end{itemize}
- Introduction to Algorithms, kafli 4
\end{frame}

\begin{frame}{Niðurstaða: Helmingunarleit}
\begin{itemize}
 \item Helmingunarleit er leitarreiknirit sem byggir á sömu hugmynd og góða lausnin á giskileiknum
 \item Við finnum stakið í miðjum listanum, og spyrjum hvort að það sé stærra eða minna en það stak se leitað er að.
 \begin{itemize}
  \item Sé miðjustakið stærra en leitarstakið finnum við miðjuna í undirlistanum til vinstri, annars í undirlistanum til hægri
  \item Haldið er áfram þar til stakið er fundið.
 \end{itemize}
 \item \textbf{Ath. vandlega} að þetta virkar ekki nema að stökin í listanum séu \emph{röðuð}
\end{itemize}
Sauðakóði: \href{http://en.wikipedia.org/wiki/Binary\_search\_algorithm\#Algorithm}{T.d. Wikipedia}
\end{frame}

\section{Skilgreiningar}

\begin{frame}{Skilgreiningaratriði}
\begin{itemize}
 \item Til að við getum leyst ákveðið vandamál formlega er oft gagnlegt að skilgreina það fyrst formlega.
 \item Getum við skilgreint ``leitarvandamál''?
\end{itemize}
\end{frame}

\begin{frame}{Hugmynd að formlegri skilgreiningu}
\begin{framed}
\texttt{LEIT Í FYLKI}

\textbf{Inntak:} Fylki af stökum $[x_0, x_1, \ldots, x_n]$ og leitargildi $y$

\textbf{Spurning:} Getum við fundið $i$ svo að $x_i = y$?
\end{framed}
Við getum notað línulega leit til að leysa þetta verkefni á $\Theta(n)$ tíma í versta tilfellinu.
\end{frame}

\begin{frame}{Önnur skilgreining}
\begin{framed}
\texttt{LEIT Í RÖÐUÐU FYLKI}

\textbf{Inntak:} Fylki af stökum $[x_0, x_1, \ldots, x_n]$ þar sem $x_i \leq x_{i+1}$, og leitargildi $y$

\textbf{Spurning:} Getum við fundið $i$ svo að $x_i = y$?
\end{framed}
Við getum notað helmingunarleit til að leysa þetta verkefni á $\Theta(\log(n))$ tíma í versta tilfellinu.
\end{frame}

\begin{frame}{Helmingunarleit í öðrum samhengjum}
\begin{itemize}
 \item Helmingunarleit er gagnlegur hugsunarháttur fyrir margar gerðir af vandamálum
 \begin{itemize}
  \item Giskileikurinn
  \item Vísar (e. indexes) í gagnagrunnum!
 \end{itemize}
 \item Seinna í áfanganum: Tvíleitartré.
\end{itemize}
\end{frame}


\end{document}
