\documentclass{article}

\usepackage{ts-skil}

\title{FOR3R3U Tímaverkefni - Endurkvæmni}

\begin{document}

\maketitle

\section{Verkefni}
Þessi verkefni eru ætluð til æfingar í endurkvæmum hugsunarhætti. Þeim þarf ekki að skila.
\subsection{1. Endurkvæm línuleg leit}
Skrifið línulega leit í Python, með því að nota endurkvæmni í stað ítrunar.
\subsection{2. Viðsnúningur á lista}
Skrifið endurkvæmt Python-fall sem tekur inn streng og skilar sama streng í öfugri röð.
\subsection{3. Betri Fibonacci}
Við höfum skoðað eftirfarandi forrit, sem reiknar út svokallaðar \href{http://en.wikipedia.org/wiki/Fibonacci_number}{Fibonacci tölur}:

\begin{minted}{python}
def fib(n):
    if n == 1 or n == 2:
        return 1
    else:
        return fib(n-1) + fib(n-2)
\end{minted}

En við sáum í hendi okkar að keyrslutíminn á þessu forriti væri hrikalegur. Gerum nú aðeins betur.

\subsubsection{a) Ítrun}
Skrifaðu útgáfu af þessu falli í Python sem notar lykkju í stað endurkvæmni.

\subsubsection{b) Halaendurkvæmni}
Skrifaðu halaendurkvæma (e. \emph{tail recursive}) útgáfu af þessu falli í Python.

\subsection{4. Ackermann}
Ackermann-föll eru sögufræg í rannsóknum á endurkvæmni. Ein framsetning á slíku falli er eftirfarandi: 

\[
 A(m,n) = \left\{
 \begin{array}{ll}
 n+1&\text{ þegar }m = 0\\
 A(m-1,1)&\text{ þegar }m > 0 \text{ og } n = 0\\
 A(m-1,A(m,n-1))&\text{ þegar }m>0 \text{ og } n > 0\\
 \end{array}
 \right.
\]

Skrifaðu þetta fall í Python.

\end{document}