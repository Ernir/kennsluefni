\documentclass{article}

\usepackage[top=0.9in, bottom=1in, left=1.5in, right=1.5in]{geometry}
\usepackage[utf8]{inputenc}
\usepackage[icelandic]{babel}
\usepackage[T1]{fontenc}
\usepackage[sc]{mathpazo} % Palatino font

\usepackage[parfill]{parskip}
\usepackage{booktabs,tabularx}
\usepackage{multirow}
\usepackage{graphicx}
\usepackage{gensymb}
\usepackage{amsmath}
\usepackage{minted} %Minted and configuration

\usepackage[pdftex,bookmarks=true,colorlinks=true,pdfauthor={Eirikur Ernir Thorsteinsson},linkcolor=blue,urlcolor=blue]{hyperref}

\usemintedstyle{default}
\renewcommand{\theFancyVerbLine}{\sffamily \arabic{FancyVerbLine}}

\date{Vorönn 2016}
\author{}

\hyphenpenalty=5000

\setcounter{secnumdepth}{-1} 
\pagenumbering{gobble}

\title{FOR3R3U Skilaverkefni 4 - gagnagrindur}

\begin{document}

\maketitle
\section{Verkefni}
Skilið inn eftirfarandi dæmum.

\subsection{1. Listi}
Skrifið Python klasa (einn fyrir eintengdan og annan fyrir tvítengdan lista) sem skilgreina lista af númeruðum stökum (númerin eru líkt og í fylki). Klasarnir skulu útfæra eftirfarandi föll:

\begin{itemize}
 \item \verb|__init__|: Býr til nýjan, tóman lista
 \item \texttt{add(value)}: Bætir hnúti (e. \emph{node}) framan á listann sem hefur ákveðið gildi
 \item \texttt{get(index)}: Skilar gildi úr listanum sem hefur ákveðinn vísi (e. \emph{index})
 \item \texttt{head()}: Skilar fyrsta gildinu í listanum
 \item \texttt{foot()}: Skilar síðasta gildinu í listanum
 \item \texttt{search(value)}: Skilar vísi á fyrsta hnútinum sem hefur ákveðið gildi
 \item \texttt{delete(value)}: Eyðir öllum hnútum úr listanum sem hafa ákveðið gildi
 \item \texttt{delete\_index(index)}: Eyðir hnúti úr listanum sem hefur ákveðinn vísi
 \item \texttt{clear()}: Tæmir listann
\end{itemize}
Þið þurfið einnig að skrifa tvo viðeigandi klasa sem tákna hnúta.

Ekki má nota innbyggðu \texttt{list} gagnagrindina!

\subsection{2. Biðröð, hlaði og forgangsbiðröð}
Skrifið eftirfarandi Python-klasa, sem tákna
\begin{itemize}
 \item biðröð (e. \emph{queue})
 \item hlaða (e. \emph{stack})
 \item og forgangsbiðröð (e. \emph{priority queue}).
\end{itemize}

Klasarnir skulu útfæra eftirfarandi föll:
\begin{itemize}
 \item \verb|__init__|: Býr til tóma gagnagrind
 \item \texttt{push(value)}: Bætir við gildi (þarf að vera push(value, priority) fyrir forgangsbiðröðina)
 \item \texttt{pop()}: Skilar (og fjarlægir) viðeigandi ``fremsta'' gildinu
 \item \texttt{clear()}: Tæmir gagnagrindina
\end{itemize}

\subsection{3. Tvíleitartré}
Skrifið Python-klasa sem skilgreinir tvíleitartré (e. \emph{Binary Search Tree}) fyrir heiltölur.

Skrifið einnig eftirfarandi föll:
\begin{itemize}
 \item \verb|tree_insert|: Fallið setur nýjan hnút á réttan stað í tréð.
 \item \verb|inorder_tree_walk|: Fallið prentar \emph{endurkvæmt} út öll gildi í trénu, í röð.
 \item \verb|tree_search|: Fallið leitar \emph{endurkvæmt} að gefnu gildi í trénu og skilar hnút sem inniheldur það gildi, annars skilar það \verb|None|.
 \item \verb|tree_minimum| og \verb|tree_maximum|: Föllin skila annars vegar minnsta gildi í trénu og hins vegar stærsta gildi í trénu.
 \item \verb|tree_delete|: Fallið eyðir hnút úr trénu. 
\end{itemize}
Þið þurfið einnig að skrifa viðeigandi klasa sem táknar hnút.

\end{document}