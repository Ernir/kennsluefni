\documentclass[handout]{beamer}

\usepackage[utf8]{inputenc}
\usepackage[icelandic]{babel}
\usepackage[T1]{fontenc}

\usepackage{booktabs}
\usepackage{minted} %Minted and configuration
\usefonttheme[onlymath]{serif}
\usepackage{framed}
\usepackage{tikz}
\usemintedstyle{default}
\renewcommand{\theFancyVerbLine}{\sffamily \arabic{FancyVerbLine}}

\setbeamertemplate{navigation symbols}{}
\hypersetup{colorlinks=true,pdfauthor={Eirikur Ernir Thorsteinsson},linkcolor=blue,urlcolor=blue}

\author{}
\institute{Tækniskólinn}
\date{Vorönn 2016}

\title{FOR3R - kynning á áfanganum}

\begin{document}

\begin{frame}
\titlepage
\end{frame}

\begin{frame}{FOR3R - um áfangann}
Hvað er reiknirit (e. \emph{algorithm})?
\begin{itemize}[<+->]
 \item Margar skilgreiningar til
 \begin{itemize}
  \item Öll forrit?
  \item ``Stærðfræðileg'' forrit?
 \end{itemize}
 \item Hér munum við helst nota þá skilgreiningu að reiknirit séu forrit sem:
 \begin{itemize}[<+->]
  \item Eru vel skilgreind (hlutverk og notkun)
  \item Taka inn breytur
  \item Beita skilgreindum reikniaðgerðum
  \item Skila niðurstöðu
 \end{itemize}
 \item Skoðum betur í næsta tíma
\end{itemize}
\end{frame}

\begin{frame}{Forritunarmál áfangans: Python}
\begin{columns}[c]
\column{.6\textwidth}
\begin{itemize}[<+->]
 \item Þessa önnina munum við notast við forritunarmálið \emph{Python}. Ástæður:
 \begin{itemize}
  \item Lítil Python-forrit eru læsileg. Líklegt er að stærðfræðileg lýsing á reikniriti líkist útfærslu þess í Python.
  \item Það er hipp og kúl.
 \end{itemize}
 \item Sýnidæmi gera ráð fyrir Python 3.
 \item Ath. að FOR3R er \emph{ekki} áfangi í Python-forritun. (Sjá: fyrsta heimaverkefnið)
\end{itemize}
\column{0.4\textwidth}
\includegraphics[width=\linewidth]{Pics/PythonLogo}
\end{columns}
\end{frame}

\begin{frame}{Engu að síður\ldots}
Nokkur atriði sem einkenna Python
\begin{itemize}
 \item Inndráttur (e. \emph{indentation}) skilgreinir blokkir
 \begin{itemize}
  \item Engir slaufusvigar, engar semíkommur
  \item Ólíkt öðrum málum kenndum í TS
 \end{itemize}
 \item Venjulega túlkað (e. \emph{interpreted}) frekar en þýtt (e. \emph{compiled})
 \begin{itemize}
  \item Líkt og PHP og JavaScript
 \end{itemize}
 \item Notar hvorki hrein gildisviðföng (e. \emph{call by value}) né hrein tilvísunarviðföng (e. \emph{call by reference)}
 \begin{itemize}
  \item Líkast Java
 \end{itemize}
 \item ``\href{https://wiki.python.org/moin/Why\%20is\%20Python\%20a\%20dynamic\%20language\%20and\%20also\%20a\%20strongly\%20typed\%20language}{Strong dynamic typing}''
 \begin{itemize}
  \item Ólíkt öðrum málum kenndum í TS
 \end{itemize}
 \item \mint{python}|import this|
\end{itemize}
\end{frame}

\begin{frame}{Dæmi}
Lítum á nokkur einföld Python-dæmi.
\end{frame}

\begin{frame}{Þróunarumhverfi}
\begin{itemize}
 \item IDLE: standardinn, fylgir með Windows bundle
 \item PyCharm: byggt á IntelliJ
 \item Eclipse + PyDev: fyrir Eclipse aðdáendur
 \item Visual studio?
\end{itemize}
Plaintext editor + skipanalínan ætti einnig að duga.
\end{frame}


\begin{frame}{Gagnlegir tenglar}
\begin{itemize}
 \item \href{http://www.codecademy.com/en/tracks/python}{Code Academy - Python}
 \item \href{http://www.learnpython.org/}{LearnPython.org}
 \item Opinber síða: \url{http://python.org/}
 \item Byrjendabók í Python: \href{http://www.swaroopch.com/notes/python/}{A Byte of Python}
 \item StackOverflow: \href{http://stackoverflow.com/questions/101268/hidden-features-of-python}{Hidden features of Python}
\end{itemize}
\end{frame}


\end{document}
