\documentclass{beamer}

\usepackage{ts-glærur}

\title{FOR3R - reiknirit}

\begin{document}

\begin{frame}
\titlepage
\end{frame}

\section{Hugtakið ``reiknirit''}

\begin{frame}{Hvað er reiknirit?}
Hvað er reiknirit (e. \emph{algorithm})?
\begin{itemize}[<+->]
 \item Margar skilgreiningar til
 \begin{itemize}
  \item Öll forrit?
  \item ``Stærðfræðileg'' forrit?
 \end{itemize}
 \item Hér munum við helst nota þá skilgreiningu að reiknirit séu forrit sem:
 \begin{itemize}[<+->]
  \item Eru vel skilgreind (hlutverk og notkun)
  \item Taka inn breytur
  \item Beita skilgreindum reikniaðgerðum
  \item Skila niðurstöðu
 \end{itemize}
\end{itemize}
\end{frame}

\begin{frame}{Skilgreiningar: Reiknirit}
\begin{quote}
``Informally, an algorithm is any well-defined computational procedure that takes
some value, or set of values, as input and produces some value, or set of values, as
output. An algorithm is thus a sequence of computational steps that transform the
input into the output.''
\end{quote} - Introduction to Algorithms
\end{frame}

\begin{frame}{Skilgreiningar: Reiknirit}
\begin{quote}
``An algorithm is an effective method that can be expressed within a finite amount of space and time$^{[1]}$ and in a well-defined formal language$^{[2]}$ for calculating a function.$^{[3]}$''
\end{quote} - \href{http://en.wikipedia.org/wiki/Algorithm}{Wikipedia}
\end{frame}

\begin{frame}{Reiknirit - bara fyrir tölur?}
\begin{quote}
``Again, it might act upon other things besides number, were objects found whose mutual fundamental relations could be expressed by those of the abstract science of operations''
\end{quote} - Ada Lovelace
\end{frame}

\begin{frame}{Örfá fræg reiknirit}

\begin{itemize}
 \item \href{http://en.wikipedia.org/wiki/Euclidean_algorithm}{Reiknirit Evklíðs}: Stærsti samnefnari tveggja talna
 \item Reiknirit \href{http://en.wikipedia.org/wiki/Dijkstra\%27s\_algorithm}{Dijkstra}/\href{http://en.wikipedia.org/wiki/A*\_search_algorithm}{A*}: Strætóleiðir, tölvuleikir, o.fl.
 \item \href{http://en.wikipedia.org/wiki/RSA\_(algorithm)}{RSA}: Dulkóðun
 \item \href{http://en.wikipedia.org/wiki/PageRank}{PageRank}: Google-niðurstöður (gamalt)
 \item \href{http://en.wikipedia.org/wiki/MapReduce}{MapReduce}: Dreifing verkefna yfir tölvukerfi
 \item \href{http://en.wikipedia.org/wiki/AKS\_primality\_test}{AKS primality test}: Prímtölur
\end{itemize}
Einnig: Reiknirit fyrir leit og röðun.
\end{frame}


\begin{frame}{Hvernig lýsum við reikniritum?}
Reikniritum má lýsa á marga vegu. T.d. má lýsa þeim með því að nota
\begin{itemize}
 \item mannamál
 \item sauðakóða
 \item forritunarmál
 \item rafrásir?
\end{itemize}
Flestar leiðir eru færar svo lengi sem útkoman er skýr.
\pause

\vspace{0.5cm}
\begin{quote}
\emph{Science is what we understand well enough to explain to a computer}
\end{quote} - Donald Knuth
\end{frame}

\begin{frame}{Skilgreiningar: Rétt reiknirit}
\begin{itemize}
 \item Reiknirit er sagt \emph{rétt} (e. \emph{correct}) ef það skilar skilgreindri niðurstöðu fyrir öll skilgreind inntök
 \begin{itemize} \pause
  \item Sum reiknirit eru gagnleg þó þau séu ekki rétt
  \item Dæmi: Mörg heuristic reiknirit (í. \emph{brjóstvitsaðferðir}) skila niðurstöðu sem er einungis ``nógu góð''
 \end{itemize}
 \item Rétt reiknirit \emph{leysir} (e. \emph{solves}) gefið, tölulegt vandamál
 \begin{itemize} \pause
  \item Ekki er nóg að reiknirit leysi vandamál til að það sé gagnlegt
  \item Dæmi: Bogosort
 \end{itemize}
\end{itemize}
\end{frame}

\section{Tímaflækjur}

\begin{frame}{Kynning: Tímaflækja}
\begin{itemize}
 \item Reiknirit geta þurft mjög mismunandi tíma til að leysa sama verkefni
 \item Mjög hefðbundin aðferð við að komast að keyrslutíma: Mæla hann
 \begin{itemize}
  \item ``Með inntak af stærð $n = 1000$ var forritið 567ms í keyrslu á 2.4 GHz örgjörva...''
 \end{itemize}
 \item Aðferð sem veitir almennari innsýn: Að telja grundvallaraðgerðir
\end{itemize}
\end{frame}

\subsection{Grunnaðgerðir}

\begin{frame}{Að telja grunnaðgerðir}
Til að einfalda talningu gerum við ráð fyrir ýmsu um tölvuna sem reikniritin keyra á.
\begin{itemize}
 \item Einn örgjörvi, engin samhliða vinnsla
 \item Enginn skortur á minni, einungis ein gerð af minni
 \begin{itemize}
  \item Wikipedia: \href{http://en.wikipedia.org/wiki/Memory_hierarchy}{Memory hierarchy}
 \end{itemize}
 \item Einfaldar aðgerðir taka fastan (e. \emph{constant}) tíma. Þar á meðal eru
 \begin{itemize}
  \item Aflestur og skrif gagna í minni
  \item Stýring: if, lykkjur, fallsköll
  \item Reikniaðgerðir: samlagning, frádráttur, margföldun, deiling
  \item Jafnvel algeng stærðfræðiföll (lograr, veldishafning, hornaföll?)
 \end{itemize}
\end{itemize}
Ath. vandlega: Einfaldanir!
\end{frame}

\begin{frame}[fragile]{Að telja grunnaðgerðir - dæmi}
Athugum Python-fall sem leggur saman þrjár tölur.

\begin{minted}{python}
def add_three(a, b, c):
    total = a  # Constant time c1
    total += b  # Constant time c2
    total += c  # Constant time c3

    return total
\end{minted}

Heildarkeyrslutíminn er summa $c_1$, $c_2$ og $c_3$. Þar sem þetta eru allt fastar segjum við að keyrslutíminn sé \emph{fastur}. Köllum hann $T = c_1 + c_2 + c_3 = \sum_i^3 c_i$.

Ath: Í raunverulegri tölvu væri raunverulegur keyrslutími háður t.d. stærð og tagi talnanna. Við lítum framhjá slíku.
\end{frame}

\begin{frame}[fragile]{Að telja grunnaðgerðir - dæmi}
Athugum Python-fall sem finnur meðaltal talna í lista.

\begin{minted}{python}
def list_average(a):
    total = 0  # Constant time c1
    n = len(a)  # Constant time c2 (documented)
    for i in a:  # Constant time c3
        total += i  # n times constant time c4
    average = total/n  # Constant time c5
    return average
\end{minted}

Heildarkeyrslutíminn er summan $c_1 + c_2 + c_3 + n\cdot c_4 + c_5$. Sameinum fastana, köllum keyrslutímann $T = c_4 \cdot n + c_{1235}$. Þar sem keyrslutíminn líkist þá fyrsta stigs margliðu (e. \emph{polynomial}) af $n$ köllum við hann \emph{línulegan}. Þar sem $n$ er lengd inntaksins er talað um \emph{línulegt fall af inntaksstærð.}
\end{frame}

\begin{frame}[fragile]{Að telja grunnaðgerðir - dæmi}
Athugum fall sem leggur saman tvo jafn langa lista.
\begin{minted}{python}
def list_summation(a, b):
    n = len(a)  # c1, assume len(a) == len(b)
    c = []  # c2
    for i in range(0, n):  # c3
        a_temp = a[i]  # n*c4
        b_temp = b[i]  # n*c5
        c.append(a_temp + b_temp)  # n*(c6 + c7)
    return c
\end{minted}

Heildarkeyrslutíminn er summan $c_1 + c_2 + c_3 + n\cdot (c_4 + c_5 + c_6 + c_7)$. Sameinum fastana, köllum keyrslutímann $T = c_{4567}\cdot n + c_{124}$ (styttra: $T = k_1\cdot n + k_2$). Aftur línulegur keyrslutími.
\end{frame}

\begin{frame}[fragile]{Innskot - ``list comprehensions'' og ``generators''}
Kóðinn á undan er auðvitað ekki skilvirkur.

Skárra:
\begin{minted}{python}
def list_summation(a, b):
    return [x+y for x, y in zip(a, b)]
\end{minted}

Enn betra?
\begin{minted}{python}
def list_sum_generator(a, b):
    for x, y in zip(a, b):
        yield x + y
\end{minted}

Lesefni: \href{https://docs.python.org/3/tutorial/datastructures.html#list-comprehensions}{list comprehensions}, \href{http://freepythontips.wordpress.com/2013/09/29/the-python-yield-keyword-explained/}{generators}.
\end{frame}

\begin{frame}{Að þekkja áhrif tímaflækju}
\begin{itemize}
 \item Oft höfum við áhuga á að skoða hvernig reiknirit hegða sér fái þau mjög stór inntök
 \begin{itemize}
  \item Stærð $n$ í dæmunum hér á undan
 \end{itemize}
 \item $T = k_1\cdot n + k_2, k_1 = 1, k_2 = 100$
 \item $T = k_1\cdot n + k_2, k_1 = 10, k_2 = 1000$
 \item $T = k_1\cdot n^2 + k_2, k_1 = 100, k_2 = 10000$
 \item Hvað gerist þegar $n \to \infty$?
\end{itemize}
\end{frame}

\begin{frame}{$\Theta$-notation: Óformleg skilgreining}
Við innleiðum svokallað $\Theta$-notation til að lýsa vexti falla. Óformleg skilgreining á því gæti verið eftirfarandi:

\begin{center}
$\Theta(g(x))$ er mengi falla sem vaxa eins og $g(x)$.
\end{center}

\begin{center}
Fallið $g(x)$ fæst út frá margliðu $f(x)$ með því að fjarlægja alla fastaliði úr $f(x)$.
\end{center}
Dæmi: 
\begin{itemize}
 \item $T(n) = k_1\cdot n + k_2$ er í menginu $\Theta(n)$
 \item $T(n) = k_1\cdot n^2 + k_2$ er í menginu $\Theta(n^2)$
\end{itemize}
\end{frame}

\begin{frame}{Frekara efni}
\begin{itemize}
 \item Kafli 3 í kennslubók fjallar um tímaflækjur
 \item \href{http://en.wikipedia.org/wiki/Time_complexity}{Wikipedia}
 \item \href{https://wiki.python.org/moin/TimeComplexity}{Tímaflækjur innbyggðra falla í Python}
\end{itemize}
\end{frame}


\end{document}

