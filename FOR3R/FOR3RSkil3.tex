\documentclass{article}
\usepackage{xfrac}

\usepackage[top=0.9in, bottom=1in, left=1.5in, right=1.5in]{geometry}
\usepackage[utf8]{inputenc}
\usepackage[icelandic]{babel}
\usepackage[T1]{fontenc}
\usepackage[sc]{mathpazo} % Palatino font

\usepackage[parfill]{parskip}
\usepackage{booktabs,tabularx}
\usepackage{multirow}
\usepackage{graphicx}
\usepackage{gensymb}
\usepackage{amsmath}
\usepackage{minted} %Minted and configuration

\usepackage[pdftex,bookmarks=true,colorlinks=true,pdfauthor={Eirikur Ernir Thorsteinsson},linkcolor=blue,urlcolor=blue]{hyperref}

\usemintedstyle{default}
\renewcommand{\theFancyVerbLine}{\sffamily \arabic{FancyVerbLine}}

\date{Vorönn 2016}
\author{}

\hyphenpenalty=5000

\setcounter{secnumdepth}{-1} 
\pagenumbering{gobble}

\title{FOR3R3U Skilaverkefni 3 - Röðunarreiknirit}

\begin{document}

\maketitle
\section{Verkefni}
Leysið eftirfarandi verkefni tvö og tvö saman í hóp.

Framkvæmið keyrslutímasamanburð á a.m.k. þremur röðunarreikniritanna. Athugið sérstaklega eftifarandi þætti:
\begin{itemize}
 \item Hvaða áhrif hefur stærð inntaksins á keyrslutímann?
 \item Hvaða áhrif hefur röðun inntaksins (vaxandi röð, lækkandi röð, slembin röð)?
 \item Hefur það áhrif að stökin séu öll ólík eða öll eins?
 \item Eru sérstakar aðstæður sem hygla öðru reikniritinu umfram annað?
\end{itemize}
Skilið þeim reikniritum sem borin eru saman inn til einkunnagjafar.

\subsection{1. Insertion Sort}
Útfærið Insertion Sort reikniritið í Python.

\paragraph{Í bók:} Kafli 2.1
\subsection{2. Merge Sort}
Útfærið Merge Sort reikniritið í Python. Búið til sérstakt hjálparfall fyrir ``merge'' hlutann.

\paragraph{Í bók:} Kafli 2.3.1

\subsection{3. Quicksort}
Útfærið Quicksort reikniritið í Python \emph{á endurkvæman máta}.

\paragraph{Í bók:} Kafli 7

\newpage
\subsection{4. Counting Sort}
Útfærið Counting Sort reikniritið í Python.

\paragraph{Í bók:} Kafli 8.2
\subsection{5. Radix Sort}
Útfærið Radix Sort reikniritið í Python. Takið fram hvort um LSD eða MSD útgáfu er að ræða.

\paragraph{Í bók:} Kafli 8.3

\subsection{6. Heapsort}
Útfærið Heapsort reikniritið í Python.

\paragraph{Í bók:} Kafli 6

\end{document}