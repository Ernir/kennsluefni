\documentclass{article}
\usepackage{xfrac}

\usepackage{ts-skil}

\hyphenpenalty=5000

\title{FOR3R3U - Atriði fyrir miðannarpróf}

\begin{document}

\maketitle
\section{Efnisatriði til prófs}

\begin{itemize}
 \item Grundvallarforritun í Python
 \begin{itemize}
  \item Breytur, tölur, skilyrði (\texttt{if}), lykkjur (\texttt{for} og \texttt{while}), fallsskilgreiningar (\texttt{def}), listar og strengir
 \end{itemize}
 \item Tímaflækjur
 \begin{itemize}
  \item Að geta áætlað tímaflækjur einfaldra reiknirita
  \item Kunna skilgreiningar á stóra-$\Theta$ og stóra-$O$
  \item Kunna skil á hvað versti, besti og meðalkeyrslutími er m.t.t. tímaflækju
 \end{itemize}
 \item Endurkvæmni
 \begin{itemize}
  \item Geta skilið einföld endurkvæm reiknirit
  \item Geta útfært einföld endurkvæm reiknirit í Python
 \end{itemize}
 \item Leitarreiknirit (helmingunarleit, línuleg leit)
 \begin{itemize}
  \item Þekkja helstu eiginleika
  \begin{itemize}
   \item Tímaflækjur
   \item Í hvaða aðstæðum er mögulegt að nota hvort reiknirit um sig
  \end{itemize}
  \item Kunna útfærslu á línulegri leit, meginatriði í útfærslu á helmingunarleit
 \end{itemize}
 \item Röðunarreiknirit (Insertion sort, Quicksort, Merge Sort)
 \begin{itemize}
  \item Þekkja helstu eiginleika
  \begin{itemize}
   \item Tímaflækjur
   \item Í hvaða aðstæðum hvert reiknirit er viðeigandi
  \end{itemize}
  \item Kunna meginatriði í útfærslum
 \end{itemize} 
\end{itemize}


\end{document}