\documentclass{article}
\usepackage{xfrac}

\usepackage[top=0.9in, bottom=1in, left=1.5in, right=1.5in]{geometry}
\usepackage[utf8]{inputenc}
\usepackage[icelandic]{babel}
\usepackage[T1]{fontenc}
\usepackage[sc]{mathpazo} % Palatino font

\usepackage[parfill]{parskip}
\usepackage{booktabs,tabularx}
\usepackage{multirow}
\usepackage{graphicx}
\usepackage{gensymb}
\usepackage{amsmath}
\usepackage{minted} %Minted and configuration

\usepackage[pdftex,bookmarks=true,colorlinks=true,pdfauthor={Eirikur Ernir Thorsteinsson},linkcolor=blue,urlcolor=blue]{hyperref}

\usemintedstyle{default}
\renewcommand{\theFancyVerbLine}{\sffamily \arabic{FancyVerbLine}}

\date{Vorönn 2016}
\author{}

\hyphenpenalty=5000

\setcounter{secnumdepth}{-1} 
\pagenumbering{gobble}

\hyphenpenalty=5000

\title{FOR3R3U - Atriði fyrir miðannarpróf}

\begin{document}

\maketitle
\section{Efnisatriði til prófs}

\begin{itemize}
 \item Grundvallarforritun í Python
 \begin{itemize}
  \item Breytur, tölur, skilyrði (\texttt{if}), lykkjur (\texttt{for} og \texttt{while}), fallsskilgreiningar (\texttt{def}), listar og strengir
 \end{itemize}
 \item Tímaflækjur
 \begin{itemize}
  \item Að geta áætlað tímaflækjur einfaldra reiknirita
  \item Kunna skilgreiningar á stóra-$\Theta$ og stóra-$O$
  \item Kunna skil á hvað versti, besti og meðalkeyrslutími er m.t.t. tímaflækju
 \end{itemize}
 \item Endurkvæmni
 \begin{itemize}
  \item Geta skilið einföld endurkvæm reiknirit
  \item Geta útfært einföld endurkvæm reiknirit í Python
 \end{itemize}
 \item Leitarreiknirit (helmingunarleit, línuleg leit)
 \begin{itemize}
  \item Þekkja helstu eiginleika
  \begin{itemize}
   \item Tímaflækjur
   \item Í hvaða aðstæðum er mögulegt að nota hvort reiknirit um sig
  \end{itemize}
  \item Kunna útfærslu á línulegri leit, meginatriði í útfærslu á helmingunarleit
 \end{itemize}
 \item Röðunarreiknirit (Insertion sort, Quicksort, Merge Sort)
 \begin{itemize}
  \item Þekkja helstu eiginleika
  \begin{itemize}
   \item Tímaflækjur
   \item Í hvaða aðstæðum hvert reiknirit er viðeigandi
  \end{itemize}
  \item Kunna meginatriði í útfærslum
 \end{itemize} 
\end{itemize}


\end{document}