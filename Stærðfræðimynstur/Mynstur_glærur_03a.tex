\documentclass{beamer}

\usepackage{Haust2016glærur}

\title{Stærðfræðimynstur í tölvunarfræði}
\subtitle{Vika 2, seinni fyrirlestur}

\begin{document}

\begin{frame}
\titlepage
\end{frame}

\section{Inngangur}

\begin{frame}{Í síðasta tíma}
\begin{itemize}
 \item Mengi!
 \begin{itemize}
  \item Stök í mengjum
  \item Hlutmengi
  \item Veldismengi og mengjamargfeldi
  \item Sniðmengi og sammengi
 \end{itemize}
 \item Föll 
\end{itemize}
\end{frame}

\section{Fylki}

\begin{frame}{Fylki}
\begin{tcolorbox}[title=Fylki]
Fylki (e. \emph{matrix}) er ferhyrnt safn af tölum sem raðast í línur og dálka. Fylki með $m$ línum og $n$ dálkum er kallað $m \times n$ fylki.
\end{tcolorbox}
Fylki og aðgerðir á þau koma mikið við sögu í forritun sem tengist vísindalegum útreikningum og í tölvugrafík.

Dæmi um fylki: Fylkið
\[
\begin{bmatrix}
1&1\\0&2\\1&4
\end{bmatrix}
\]
er $3 \times 2$ fylki.
\end{frame}

\begin{frame}{Fylkjamargföldun}
Hægt er að skilgreina margar reikniaðgerðir fyrir fylki. Aðgerð sem oft kemur við sögu er fylkjamargföldun.

\begin{tcolorbox}[title=Fylkjamargföldun]
Látum $A$ vera $m \times k$ fylki og $B$ vera $k \times n$ fylki. Margfeldi $A$ og $B$, táknað með $AB$, er $m \times n$ fylki þar sem stak $i, j$ er summa margfelda staka úr $i$-tu línu $A$ og $j$-ta dálks $B$. Þ.e.a.s. fyrir stak $c_{ij}$ í $AB$ gildir:

\[
 c_{ij} = a_{i1}b_{1j} + a_{i2}b_{2j} + \ldots + a_{ij}b_{kj}
\]
\end{tcolorbox}
\end{frame}

\begin{frame}{Fylkjamargföldun - dæmi}
\[
\begin{bmatrix}
1&0&4\\2&1&1\\3&1&0\\0&2&2
\end{bmatrix}
\begin{bmatrix}
2&4\\
1&1\\
3&0\\
\end{bmatrix}
=
\begin{bmatrix}
14&4\\
8&9\\
7&13\\
8&2\\
\end{bmatrix}
\]

\end{frame}


\section{Runur og rakningarvensl}

\begin{frame}{Runur}
\begin{tcolorbox}[title=Runur]
Runa (e. \emph{sequence}) er fall frá hlutmengi heiltalna (oftast mengi jákvæðra heiltalna) til annars mengis $S$. Mynd heiltölunnar $n$ er táknuð með $a_n$. Sagt er að $a_n$ sé liður (e. \emph{term}) í rununni.
\end{tcolorbox}
Bókin táknar runur með slaufusvigum, t.d. $\{a_n\}$. Runur eru samt ekki mengi.

Dæmi um runu:
\[
 \{a_n\} = a_1, a_2, a_3 \ldots = 1, \frac{1}{2}, \frac{1}{3}
\]
\[
 a_n = \frac{1}{n}
\]
\end{frame}

\begin{frame}[fragile]{Strengir}
\begin{itemize}
 \item Sérstök, mikið notuð gerð af runum kallast strengur (e. \emph{string})
 \item Strengur er endanleg runa af stöfum, úr endanlegu mengi (stafrófinu)
 \begin{itemize}
  \item Stafrófið getur verið hvaða mengi sem er, en algeng stafróf eru \texttt{\{'a', 'b', 'c', \ldots\}} og \texttt{\{0, 1\}}
 \end{itemize}
\end{itemize}

\begin{minted}[frame=lines]{java}
// Gagnagerðin String í Java í notkun
String s = "Ég er strengur!";
String b = "01010110";
\end{minted}
\end{frame}

\begin{frame}{Rakningarvensl}
\begin{tcolorbox}[title=Rakningarvensl]
Rakningarvensl (e. \emph{recurrence relation}) fyrir runu $\{a_n\}$ er jafna sem skilgreinir $a_n$ sem fall af einum eða fleiri fyrri liðum rununnar, þ.e.a.s. $a_0, a_1, \ldots, a_{n-1}$ fyrir öll $n > n_0$, þar sem $n_0$ er ekki-neikvæð heiltala.
\end{tcolorbox}
Runa er lausn (e. \emph{solution}) á rakningarvenslum ef liðir hennar uppfylla venslin. Liðirnir sem eru fyrir framan fyrsta liðinn sem rankningarvenslin tilgreina eru upphafsskilyrði (e. \emph{initial conditions}) þeirra.
\end{frame}

\begin{frame}{Dæmi um rakningarvensl}
Látum $\{a_n\}$ vera runu sem uppfyllir $a_n = a_{n-1} + 3$ fyrir $n=1, 2, 3, \ldots$ og setjum $a_0 = 2$. Hvað eru þá $a_1, a_2$ og $a_3$? \pause

\begin{align*}
a_1 &= a_0 + 3 = 2 + 3 = 5\\
a_2 &= 5 + 3 = 8\\
a_3 &= 8 + 3 = 11\\
\end{align*}

\end{frame}

\begin{frame}{Dæmi um rakningarvensl}
\begin{itemize}
 \item Fibonacci-rununa $f_0, f_1, f_2, \ldots$ má skilgreina með rakningavenslum
 \begin{itemize}
  \item Upphafsskilyrði: $f_0 = 0, f_1 = 1$
  \item Rakningavensl: $f_n = f_{n-1} + f_{n-2}$
 \end{itemize}
\end{itemize}
\end{frame}

\begin{frame}{Að leysa rakningarvensl}
\begin{itemize}
 \item Oft er nauðsynlegt að finna formúlu fyrir liði runu sem skilgreind er með rakningarvenslum
 \item Formúlan er sögð vera á lokuðu sniði (e. \emph{closed form}) ef hún inniheldur einungis ``einfaldar'' grunnaðgerðir og föll
 \item Algengt: Giska á rétta formúlu, sem síðan er sannreynd með þrepun (e. \emph{induction})
 \begin{itemize}
  \item Gerum líklega meira af því síðar
 \end{itemize}
\end{itemize}
\end{frame}


\section{Fjöldatölur og reiknanleiki}

\begin{frame}{Fjöldatölur}
Við höfum áður kynnst fjöldatölum. Útvíkkum nú hugtakið:

\begin{tcolorbox}[title=Eins fjöldatölur]
Mengin $A$ og $B$ hafa sömu fjöldatölu ef til er gagntækt fall frá $A$ til $B$.
\end{tcolorbox}

Þá má einnig skilgreina:

\begin{tcolorbox}[title=Teljanleiki]
Mengi sem er endanlegt eða hefur sömu fjöldatölu og mengi jákvæðra heiltalna er teljanlegt (e. \emph{countable}). Þegar óendanlegt mengi $S$ er teljanlegt skilgreinum við fjöldatölu þess sem $|S| = \aleph_0$.
\end{tcolorbox}
\end{frame}

\begin{frame}{Reiknanleiki}
\begin{itemize}
 \item Til að fall sé reiknanlegt (e. \emph{computable}) þarf að vera til forrit í einhverju forritunarmáli sem finnur gildi þess
 \item Til eru óreiknanleg föll!
 \item Þekkt atriði:
 \begin{itemize}
  \item Fjöldi forrita er teljanlega óendanlegur
  \item Fjöldi falla frá teljanlega óendanlegu mengi yfir í sjálft sig er óteljanlegur
 \end{itemize}
\end{itemize}
\end{frame}

\begin{frame}{Næst}
Reiknirit (kafli 3.1) og vöxtur falla (kafli 3.2)
\end{frame}


\end{document}
