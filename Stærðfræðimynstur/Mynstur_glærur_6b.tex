\documentclass{beamer}

\usepackage[utf8]{inputenc}
\usepackage[icelandic]{babel}
\usepackage[T1]{fontenc}

\usepackage{booktabs}
\usepackage[outputdir=.]{minted} %Minted and configuration
\usepackage{framed}
\usepackage{tikz}
\usemintedstyle{default}
\renewcommand{\theFancyVerbLine}{\sffamily \arabic{FancyVerbLine}}
\newcommand{\Mod}[1]{\ \text{mod}\ #1}

\usebackgroundtemplate%
{%
\vbox to \paperheight{
\includegraphics[width=\paperwidth]{Pics/hi-slide-head}

\vfill
\hspace{0.5cm}\includegraphics[width=0.3\paperwidth]{Pics/hi-von-logo}
\vspace{0.5cm}
    }%
}

\setbeamertemplate{navigation symbols}{}
\usecolortheme{dove}
\setbeamercolor{frametitle}{fg=white}
\hypersetup{colorlinks=true,pdfauthor={Eirikur Ernir Thorsteinsson},linkcolor=blue,urlcolor=blue}

\AtBeginSection[]
{
  \begin{frame}<beamer>
    \frametitle{Yfirlit}
    \tableofcontents[currentsection]
  \end{frame}
}

\author{Eiríkur Ernir Þorsteinsson}
\institute{Háskóli Íslands}
\date{Haust 2016}

\title{Stærðfræðimynstur í tölvunarfræði}
\subtitle{Vika 6, seinni fyrirlestur}

\begin{document}

\begin{frame}
\titlepage
\end{frame}


\section{Inngangur}

\begin{frame}{Í síðasta tíma}
\begin{itemize}
 \item Upprifjun á rakningarvenslum
 \item Endurkvæm reiknirit
\end{itemize}
\end{frame}

\begin{frame}{Af hverju virkar þrepun?}
\begin{itemize}
 \item Þrepun er ekki göldrótt!
 \item Í hefðbundnu grunnskrefi þrepunarsönnunar sýnum við að staðhæfing $P(1)$ sé sönn
 \item Við gerum svo ráð fyrir að $P(k)$ sé satt fyrir ótilgreinda jákvæða heiltölu $k$ og sýnum að $P(k) \to P(k+1)$ fyrir allar jákvæðar heiltölur $k$
 \item Þá getum við séð fyrir okkur að $P(n)$ gildi fyrir allar jákvæðar heiltölur $n$
 \begin{itemize}
  \item Af hverju?
  \item $P(1)$ er satt og við vitum að ef $P(1)$ er satt þá er $P(2)$ líka satt
  \item Þá er $P(2)$ satt og við vitum að ef $P(2)$ er satt þá er $P(3)$ líka satt\ldots
 \end{itemize}
 \item Sönnun í bók
\end{itemize}
\end{frame}

\begin{frame}{Sterk þrepun}
\begin{itemize}
 \item Sterk þrepun er sönnunaraðferð sem er jafngild ``venjulegri'' þrepun
 \begin{itemize}
  \item Þ.e.a.s. ef hægt er að nota aðra sönnunaraðferðina má nota hina líka
 \end{itemize}
 \item Þá er ekki sýnt fram á að $P(k) \to P(k+1)$, heldur að \[[P(1) \land P(2) \land \ldots \land P(k)] \to P(k+1)\]
 \item Sterka þrepun er betra að nota þegar ljósara er að $[P(1) \land \ldots \land P(k)] \to P(k+1)$ heldur en að einungis $P(k) \to P(k+1)$
\end{itemize}
\end{frame}


\section{Endurkvæm reiknirit}

\begin{frame}{Endurkvæm reiknirit}
\begin{itemize}
 \item Munum að reiknirit er endanleg runa vel skilgreindra aðgerða sem reikna út lausn á skilgreindu vandamáli
 \item Reiknirit hefur inntak, sem lýsir tilviki af vandamálinu sem það leysir
 \item Reiknirit er kallað endurkvæmt ef það leysir vandamálið með því að smætta það niður í smærra tilvik af sama vandamáli
\end{itemize}
\end{frame}

\begin{frame}{Endurkvæmt hrópmerkt}
Hrópmerkt hentar vel til endurkvæms útreiknings.
\begin{center}
\includegraphics[width=\textwidth]{factorial-algorithm}
\end{center}
\[
3! = 3\cdot 2! = 3 \cdot 2 \cdot 1! = 3 \cdot 2 \cdot 1 \cdot 0! = 3 \cdot 2 \cdot 1 \cdot 1
\]
\end{frame}

\begin{frame}{Endurkvæmt GCD}
\begin{columns}
\column{0.5\textwidth}
Við getum reiknað stærsta samdeili (gcd) með endurkvæmu reikniriti sem byggir á því að 
\[
 gcd(a,b) = gcd(b \Mod a, a)
\]
og að 
\[
 gcd(0,b) = b
\]
þegar $b > 0$.
\column{0.5\textwidth}
Þannig getum við reiknað stærsta samdeili 5 og 8:
\begin{align*}
gcd(5,8) &= gcd(8 \Mod 5, 5)\\
&= gcd(3,5)\\
&= gcd(5 \Mod 3, 3)\\
&= gcd(2,3)\\
&= gcd(3 \Mod 2, 2)\\
&= gcd(1,2)\\
&= gcd(2 \Mod 1, 1)\\
&= gcd(0,1)\\
&= 1
\end{align*}

\end{columns}
\end{frame}

\begin{frame}{Endurkvæmt GCD}
Við getum skrifað reiknirit út frá þessari innsýn:
\begin{center}
\includegraphics[width=\textwidth]{gcd-recursive}
\end{center}
\end{frame}

\begin{frame}{Línuleg leit}
Rifjum upp línulega leit:
\begin{center}
\includegraphics[width=\textwidth]{linear-search}
\end{center}
\end{frame}

\begin{frame}{Endurkvæm línuleg leit?}
Getum við gert endurkvæma útgáfu af línulegri leit?\pause

Já, við getum það:
\begin{center}
\includegraphics[width=\textwidth]{linear-search-recursive}
\end{center}
\end{frame}

\begin{frame}{Helmingunarleit}
Rifjum upp helmingunarleit:
\begin{center}
\includegraphics[width=\textwidth]{binary-search}
\end{center}
\end{frame}

\begin{frame}{Endurkvæm helmingunarleit?}
Getum við gert endurkvæma útgáfu af helmingunarleit?\pause

Já, við getum það:
\begin{center}
\includegraphics[width=0.8\textwidth]{binary-search-recursive}
\end{center}
\end{frame}

\section{Endurkvæmni og ítrun}

\begin{frame}{Endurkvæmni og ítrun}
\begin{itemize}
 \item Höfum séð endurkvæm reiknirit sem byrja á að skoða heildarmyndina og brjóta hana svo niður í smærri tilvik
 \item Gætum þess í stað séð fyrir okkur að við byrjum á að skoða smáu tilvikin og ``byggjum svo upp'' í þau stóru
 \begin{itemize}
  \item Slíka aðferðafræði köllum við ítrun
  \item Ítrun er venjulega framkvæmd með lykkjum
 \end{itemize}
 \item Ítrun og endurkvæmni framkvæma hliðstæða útreikninga!
\end{itemize}
\end{frame}

\begin{frame}{Endurkvæmni og ítrun}
\begin{itemize}
 \item Hvenær á að nota endurkvæmni og hvenær á að nota ítrun við forritun? \pause
 \begin{itemize}
  \item Ráðlegging undirritaðs: Notið það sem ykkur finnst lýsa vandamálinu best
  \begin{itemize}
   \item Tími forritarans er verðmætur
   \item Getið síðan breytt úr einu í annað ef \emph{mælingar} sýna að upprunalega nálgunin er ekki skilvirk
  \end{itemize}
  \item Þekkið forritunarmálin sem þið eruð að nota - styðja þau endurkvæmni vel?
 \end{itemize}
\end{itemize}
\end{frame}

\section{Merge Sort}

\begin{frame}{Röðun}
\begin{itemize}
 \item Höfum áður séð innsetningarröðun
 \begin{itemize}
  \item Keyrslutími innsetningarröðunar á runu af lengd $n$ er $O(n^2)$ í versta tilfellinu
 \end{itemize}
 \item Skoðum nú endurkvæma lýsingu á reikniriti sem raðar á $O(n\log(n))$ tíma í versta tilfellinu
\end{itemize}
\end{frame}

\begin{frame}{Merge sort}
\begin{itemize}
 \item Hugmyndin að baki merge sort er eftirfarandi:
 \begin{itemize}
  \item Heildarrununni er skipt upp í tvær hlutrunur af jafnri (eða næstum jafnri) lengd
  \begin{itemize}
   \item Hverri hlutrunu er svo skipt upp í tvær hluthlutrunur, o.s.frv.
   \item Uppskiptingunni er haldið áfram þar til runurnar eru af lengdinni 1
  \end{itemize}
  \item Þegar uppskiptingunni er lokið eru runurnar sameinaðar tvær og tvær svo að úr verði röðuð runa
  \begin{itemize}
   \item Þegar síðustu sameiningunni er lokið er runan röðuð!
  \end{itemize}
 \end{itemize}
\end{itemize}
\end{frame}

\begin{frame}{Keyrsla merge sort}
\vspace{0.5cm}
\begin{center}
\includegraphics[width=0.5\textwidth]{merge-sort-visual}
\end{center}
\end{frame}

\begin{frame}{Merge reikniritið}
Mikið af vinnunni sem fer fram í merge sort felst í því að sameina listana:
\begin{center}
\includegraphics[width=\textwidth]{merge}
\end{center}
\end{frame}

\begin{frame}{Merge sort}
Merge sort er síðan lýst endurkvæmt:
\begin{center}
\includegraphics[width=0.9\textwidth]{merge-sort}
\end{center}
\end{frame}

\section{Sannanleg forritun}


\begin{frame}{Næst}
Talning (6.1) og skúffureglan (6.2)
\end{frame}


\end{document}
