\documentclass{exam}

\usepackage{Haust2016verkefnablöð}

\title{Stærðfræðimynstur í tölvunarfræði \\ Skilaverkefni 3}
\author{}

\printanswers

\begin{document}
\maketitle
\thispagestyle{empty} 

Skila skal þessu verkefni á vefnum \href{https://gradescope.com/}{Gradescope}. Aðgangskóði fyrir námskeiðið er \textbf{926WD9}.

Samvinna á milli nemenda er eðlileg og æskileg. Hins vegar er hvorki æskilegt né heimilt að fá lausnir hjá öðrum (þ.m.t. Google), afrita lausnir eða láta aðra fá lausnina sína. Í slíkum tilvikum er einkunnin 0 gefin sem fyrsta viðvörun. Hikið ekki við að leita til umsjónarkennara ef þið eruð í vafa um hvað telst eðlileg samvinna og hvað ekki.

Telji nemandi að mistök hafi verið gerð við yfirferð skal tilkynna slíkt á vefnum Gradescope.

\section{Spurningar}
\begin{questions}
\question Notið sauðakóða til að lýsa reikniriti sem hefur stígandi runu heiltalna sem inntak og runu þeirra heiltalna sem koma oftar en einu sinni fyrir í inntakinu sem úttak.

Stígandi runa er runa $a_1, a_2, \ldots, a_n$ þar sem $a_1 \leq a_2 \leq \ldots \leq a_n$. Gera má ráð fyrir að $n \geq 2$.

\question Útskýrðu öll skrefin sem gráðuga skiptimyntareikniritið fer í gegnum til að skipta eftirfarandi upphæðum með $c_1 = 1, c_2 = 5, c_3 = 10, c_4 = 25$.
\begin{enumerate}[a)]
 \item Skipta 69
 \item Skipta 17
\end{enumerate}

\question Sýnið að ekki sé til neitt forrit sem tekur inn forrit $P$ og inntak $I$ sem inntök og ákvarðar hvort $P(I)$ skili einhvern tímann stafnum 1.

\emph{Ráðlegging:} Sýnið að hægt væri að nota slíkt forrit til að leysa stöðvunarvandamálið.

\question Notaðu skilgreiningu á ``$f(x)$ er $O(g(x))$'' til að sýna að $x^4 + 9x^3 + 4x + 7$ er $O(x^4)$.

\question Látum $m$ vera jákvæða heiltölu. Sýndu að $1^m + 2^m + \ldots + n^m$ er $O(n^{m+1})$.

\question Raðaðu föllunum $\sqrt{n}, 3^n, n \log n, 1000 \log n,  2n!, 2^n$ og $\frac{n^2}{1000}$ í runu svo að hvert fall sé í stóra O af þeim föllum sem á eftir koma.

\end{questions}

\vfill
\includegraphics[width=0.5\linewidth]{hi-von-logo}

\end{document}