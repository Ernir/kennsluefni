\documentclass{exam}

\usepackage[top=0.9in, bottom=1in, left=1.5in, right=1.5in]{geometry}
\usepackage[utf8]{inputenc}
\usepackage[icelandic]{babel}
\usepackage[T1]{fontenc}
\usepackage[sc]{mathpazo}

\usepackage[parfill]{parskip}
\usepackage{booktabs,tabularx}
\usepackage{multirow}
\usepackage{enumerate}
\usepackage{graphicx}
\usepackage{amsmath, amsfonts, amssymb, amsthm}
\usepackage{tikz}
\usepackage{minted} %Minted and configuration

\usepackage[pdftex,bookmarks=true,colorlinks=true,pdfauthor={Eirikur Ernir Thorsteinsson},linkcolor=blue,urlcolor=blue]{hyperref}

\setcounter{secnumdepth}{-1} 
\hyphenpenalty=5000


% Picture locations
\graphicspath{{./Pics/}}

\usemintedstyle{default}
\renewcommand{\theFancyVerbLine}{\sffamily \arabic{FancyVerbLine}}

\newcommand{\Mod}[1]{\ \text{mod}\ #1}

\runningfooter{\hspace{-2cm}\includegraphics[width=0.5\textwidth]{Pics/hi-von-logo}}{}{}

\renewcommand{\solutiontitle}{\noindent\textbf{Mögulegt svar:}}

\author{}
\date{}

\footer{}{}{}

\title{Stærðfræðimynstur í tölvunarfræði \\ Skilaverkefni 1}
\author{}

\printanswers

\begin{document}
\maketitle
\thispagestyle{empty} 

Skila skal þessu verkefni á vefnum \href{https://gradescope.com/}{Gradescope}. Aðgangskóði fyrir námskeiðið er \textbf{926WD9}.

Samvinna á milli nemenda er eðlileg og æskileg. Hins vegar er hvorki æskilegt né heimilt að fá lausnir hjá öðrum (þ.m.t. Google), afrita lausnir eða láta aðra fá lausnina sína. Í slíkum tilvikum er einkunnin 0 gefin sem fyrsta viðvörun. Hikið ekki við að leita til umsjónarkennara ef þið eruð í vafa um hvað telst eðlileg samvinna og hvað ekki.

Telji nemandi að mistök hafi verið gerð við yfirferð skal tilkynna slíkt á vefnum Gradescope.

\section{Spurningar}
\begin{questions}
\question Látum $p$, $q$ og $r$ vera eftirfarandi yrðingar:

\begin{align*}
p &: \text{Nemandi fær háa einkunn fyrir skilaverkefnin}\\
q &: \text{Nemandi gerir öll æfingaverkefnin}\\
r &: \text{Nemandi fær háa vetrareinkunn}
\end{align*}

Skrifið eftirfarandi yrðingar með því að nota $p$, $q$ og $r$ ásamt rökvirkjum:

\begin{enumerate}[a)]
 \item Til að nemandi fái háa vetrareiknunn er nauðsynlegt að nemandi fái háa einkunn fyrir skilaverkefnin.
 \item Nemandi sem fær háa einkunn fyrir skilaverkefnin og gerir öll æfingaverkefnin fær háa vetrareinkunn.
\end{enumerate}

\question Eftirfarandi setningar nota orðið ``eða''. Í hverju tilfelli um sig, útskýrið hvort um er að ræða ``eða'', táknað með $\lor$ eða ``aðgreint eða'', táknað með $\oplus$.

\begin{enumerate}[a)]
 \item Starfið krefst reynslu af Java eða C\#.
 \item Með hádegismatnum fylgir súpa eða salat.
 \item Til að ferðast til landsins þarftu að vera með gilt vegabréf eða ökuskírteni.
 \item Nú skal duga eða drepast.
\end{enumerate}

\question Við erum stödd á eyju riddara og ribbalda. Sem fyrr segja riddarar alltaf satt og ribbaldar alltaf ósatt. Að þessu sinni segir $A$ ``við erum bæði riddarar'' og $B$ segir ``$A$ er ribbaldi''.

Útskýrið hvað má álykta um hvort $A$ og $B$ séu riddarar eða ribbaldar.

\newpage

\question Skrifið upp sanntöflur fyrir eftirfarandi yrðingar.
\begin{enumerate}[a)]
\item $(q \rightarrow \lnot p) \lor (\lnot p \rightarrow \lnot q)$
\item $(p \leftrightarrow q) \oplus (\lnot p \rightarrow r)$
\end{enumerate}

\question Notið þekktar umritunarreglur til að sýna eftirfarandi:
\[ (( p \lor \lnot ~q) \land r) \rightarrow p \equiv (p \lor q) \lor \lnot ~r \]
Sýnið útreikninga. Notið eina reglu í einu.

\question Setjið fram eftirfarandi staðhæfingar með því að nota rökvirkja, umsagnir og magnara. Tilgreinið viðkomandi óðal.
\begin{enumerate}[a)]
 \item Einhver hlutur er ekki á réttum stað.
 \item Öll verkfæri eru á réttum stað og í góðu ástandi.
 \item Ekkert er á réttum stað og í góðu ástandi.
 \item Eitt verkfæranna er ekki á réttum stað, en það er í góðu ástandi.
\end{enumerate}

\end{questions}

\vfill
\includegraphics[width=0.5\linewidth]{hi-von-logo}

\end{document}