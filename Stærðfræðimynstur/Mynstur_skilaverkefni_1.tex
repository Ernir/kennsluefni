\documentclass{exam}

\usepackage[top=0.9in, bottom=1in, left=1.5in, right=1.5in]{geometry}
\usepackage[utf8]{inputenc}
\usepackage[icelandic]{babel}
\usepackage[T1]{fontenc}
\usepackage[sc]{mathpazo}

\usepackage[parfill]{parskip}
\usepackage{booktabs,tabularx}
\usepackage{multirow}
\usepackage{enumerate}
\usepackage{graphicx}
\usepackage{amsmath, amsfonts, amssymb, amsthm}
\usepackage{tikz}
\usepackage{minted} %Minted and configuration

\usepackage[pdftex,bookmarks=true,colorlinks=true,pdfauthor={Eirikur Ernir Thorsteinsson},linkcolor=blue,urlcolor=blue]{hyperref}

\setcounter{secnumdepth}{-1} 
\hyphenpenalty=5000


% Picture locations
\graphicspath{{./Pics/}}

\usemintedstyle{default}
\renewcommand{\theFancyVerbLine}{\sffamily \arabic{FancyVerbLine}}

\newcommand{\Mod}[1]{\ \text{mod}\ #1}

\runningfooter{\hspace{-2cm}\includegraphics[width=0.5\textwidth]{Pics/hi-von-logo}}{}{}

\renewcommand{\solutiontitle}{\noindent\textbf{Mögulegt svar:}}

\author{}
\date{}

\footer{}{}{}

\title{Stærðfræðimynstur í tölvunarfræði \\ Skilaverkefni 1}
\author{}

% \printanswers

\begin{document}
\maketitle
\thispagestyle{empty} 

Skila skal þessu verkefni á vefnum \href{https://gradescope.com/}{Gradescope}. Aðgangskóði fyrir námskeiðið er \textbf{926WD9}.

Samvinna á milli nemenda er eðlileg og æskileg. Hins vegar er hvorki æskilegt né heimilt að fá lausnir hjá öðrum (þ.m.t. Google), afrita lausnir eða láta aðra fá lausnina sína. Í slíkum tilvikum er einkunnin 0 gefin sem fyrsta viðvörun. Hikið ekki við að leita til umsjónarkennara ef þið eruð í vafa um hvað telst eðlileg samvinna og hvað ekki.

Telji nemandi að mistök hafi verið gerð við yfirferð skal tilkynna slíkt á vefnum Gradescope.

\section{Spurningar}
\begin{questions}
\question Látum $p$, $q$ og $r$ vera eftirfarandi yrðingar:

\begin{align*}
p &: \text{Nemandi fær háa einkunn fyrir skilaverkefnin}\\
q &: \text{Nemandi gerir öll æfingaverkefnin}\\
r &: \text{Nemandi fær háa vetrareinkunn}
\end{align*}

Skrifið eftirfarandi yrðingar með því að nota $p$, $q$ og $r$ ásamt rökvirkjum:

\begin{enumerate}[a)]
 \item Til að nemandi fái háa vetrareiknunn er nauðsynlegt að nemandi fái háa einkunn fyrir skilaverkefnin.
 \item Nemandi sem fær háa einkunn fyrir skilaverkefnin og gerir öll æfingaverkefnin fær háa vetrareinkunn.
\end{enumerate}

\begin{solution}
\begin{enumerate}[a)]
 \item $r \rightarrow p$.\\
 \item $(p \land q) \rightarrow r$.
\end{enumerate}
\end{solution}


\question Eftirfarandi setningar nota orðið ``eða''. Í hverju tilfelli um sig, útskýrið hvort um er að ræða ``eða'', táknað með $\lor$ eða ``aðgreint eða'', táknað með $\oplus$.

\begin{enumerate}[a)]
 \item Starfið krefst reynslu af Java eða C\#.
 \item Með hádegismatnum fylgir súpa eða salat.
 \item Til að ferðast til landsins þarftu að vera með gilt vegabréf eða ökuskírteni.
 \item Nú skal duga eða drepast.
\end{enumerate}

\begin{solution}

\begin{enumerate}[a)]
 \item Eða
 \item Aðgreint eða
 \item Eða
 \item Aðgreint eða
\end{enumerate}

\end{solution}

\question Við erum stödd á eyju riddara og ribbalda. Sem fyrr segja riddarar alltaf satt og ribbaldar alltaf ósatt. Að þessu sinni segir $A$ ``við erum bæði riddarar'' og $B$ segir ``$A$ er ribbaldi''.

Útskýrið hvað má álykta um hvort $A$ og $B$ séu riddarar eða ribbaldar.

\begin{solution}

If A is a knight, then his statement that both of them are knights is true, and both will be telling the truth. But that is impossible, because B is asserting otherwise (that A is a knave). If A is a knave, then B’s assertion is true, so he must be a knight, and A’s assertion is false, as it should be. Thus we conclude that A is a knave and B is a knight.
\end{solution}

\newpage

\question Skrifið upp sanntöflur fyrir eftirfarandi yrðingar.
\begin{enumerate}[a)]
\item $(q \rightarrow \lnot p) \lor (\lnot p \rightarrow \lnot q)$
\item $(p \leftrightarrow q) \oplus (\lnot p \rightarrow r)$
\end{enumerate}

\begin{solution}
\begin{verbatim}
a)
p    q    ~p   q->~p  ~q   ~p->~q   (q->~p)v(~p->~q)
---------------------------------------------------
0    0    1    1      1	   1        1
0    1    1    1      0    0        1
1    0    0    1      1    1        1
1    1    0    0      0    1        1
[~ táknar neitunarvirkjan]
Samsetta yrðingin er sísanna (alltaf sönn).

b)
p    q    r    p<->q  ~p   ~p->r    (p<->q)o(~p->r)
---------------------------------------------------
0    0    0    1      1    0        1
0    0    1    1      1    1        0
0    1    0    0      1    0        0
0    1    1    0      1    1        1
1    0    0    0      0    1        1
1    0    1    0      0    1        1
1    1    0    1      0    1        0
1    1    1    1      0    1        0
[o táknar aðgreint-eða virkjann (xor)]
\end{verbatim}
\end{solution}

\question Notið þekktar umritunarreglur til að sýna eftirfarandi:
\[ (( p \lor \lnot ~q) \land r) \rightarrow p \equiv (p \lor q) \lor \lnot ~r \]
Sýnið útreikninga. Notið eina reglu í einu.

\begin{solution}
\begin{eqnarray*}
\begin{array}{llll}
	(( p \lor \lnot ~q) \land r) \rightarrow p
& \equiv & \lnot((p \lor \lnot q)\land r) \lor p &\textrm{Umritun leiðingar} \\
& \equiv & ( \lnot (p \lor \lnot q) \lor \lnot r) \lor p &\textrm{Regla De Morgan}\\
& \equiv & (( \lnot p \land q) \lor \lnot r) \lor p &	\textrm{Regla De Morgan}\\
& \equiv & (( \lnot p \land q) \lor p) \lor \lnot r &	\textrm{Tengiregla}\\
& \equiv & ((\lnot p \lor p) \land (q \lor p) \lor \lnot r)	&\textrm{Dreifiregla}\\
& \equiv & T \lor (q \lor p) \lor \lnot r	&\textrm{Regla um eiginleika neitunar}\\
& \equiv & (q \lor p) \lor \lnot r	&\textrm{Drottnunarregla}\\
\end{array}
\end{eqnarray*}
\end{solution}


\question Setjið fram eftirfarandi staðhæfingar með því að nota rökvirkja, umsagnir og magnara. Tilgreinið viðkomandi óðal.
\begin{enumerate}[a)]
 \item Einhver hlutur er ekki á réttum stað.
 \item Öll verkfæri eru á réttum stað og í góðu ástandi.
 \item Ekkert er á réttum stað og í góðu ástandi.
 \item Eitt verkfæranna er ekki á réttum stað, en það er í góðu ástandi.
\end{enumerate}

\begin{solution}

Let $R(x)$ be ``x is in the correct place''; let $E(x)$ be ``x is in excellent condition''; let $T(x)$ be ``$x$ is a tool''; and let the domain of discourse be all things.

\begin{enumerate}[a)]
\item There exists something not in the correct place: $\exists x \lnot R(x)$.
\item If something is a tool, then it is in the correct place place and in excellent condition: $\forall x (T(x) \to (R(x) \land E(x))).$
\item This is saying that everything fails to satisfy the condition: $\forall x \lnot (R(x) \land E(x)).$
\item There exists a tool with this property: $\exists x (T(x) \land  \lnot R(x) \land E(x)).$
\end{enumerate}
\end{solution}



\end{questions}

\vfill
\includegraphics[width=0.5\linewidth]{hi-von-logo}

\end{document}