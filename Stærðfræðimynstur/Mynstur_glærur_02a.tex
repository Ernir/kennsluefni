\documentclass{beamer}

\usepackage{Haust2017glærur}

\title{Stærðfræðimynstur í tölvunarfræði}
\subtitle{Vika 2, fyrri fyrirlestur}

\begin{document}

\begin{frame}
    \titlepage
\end{frame}

\section{Inngangur}

\begin{frame}{Í síðasta tíma}
    \begin{itemize}
        \item Skoðuðum hagnýtingar á yrðingum
        \item Sáum staðalsnið fyrir yrðingar
        \begin{itemize}
            \item Eð-að staðalsnið
            \item Og-að staðalsnið
        \end{itemize}
        \item Kynntumst umsögnum
        \item Kynntumst mögnurum
    \end{itemize}
\end{frame}

\section{Kynning á sönnunum (1.7)}

\begin{frame}{Sannanir}
    \begin{itemize}
        \item Setning (e. \emph{theorem}) er staðhæfing sem hægt er að sýna fram á að sé sönn
        \begin{itemize}
            \item Sönnun (e. \emph{proof}) er sú sýning
        \end{itemize}
        \item Setningar og sannanir á þeim eru ær og kýr stærðfræðinga
        \item Í tölvunarfræði rekumst við á sannanir og eðlislíkar aðferðir m.a. í formlegum skilgreiningum og lýsingum
    \end{itemize}
\end{frame}

\begin{frame}{Orðaforði}
    \begin{itemize}
        \item Frumsenda (e. \emph{axiom}) er staðhæfing sem gert er ráð fyrir að sé sönn og notuð til frekari rökleiðslu
        \item Hjálparsetning (e. \emph{lemma}) er setning sem fyrst og fremst er notuð við sönnun á annarri setningu sem álitin er merkilegri
        \item Fylgisetning (e. \emph{corollary}) er setning sem er bein afleiðing annarrar setningar
        \item Tilgáta (e. \emph{conjecture}) er staðhæfing sem sett er fram án sönnunar, venjulega byggt á einhvers konar hyggjuviti
    \end{itemize}
\end{frame}

\section{Aðferðir til sannana}

\begin{frame}{Sannanir}
    \begin{itemize}
        \item Margar leiðir eru færar þegar sanna á nýjar setningar
        \item Sannanir eiga það sameiginlegt að nota
        \begin{itemize}
            \item Skilgreiningar
            \item Frumsendur
            \item Aðrar setningar sem hafa verið sannaðar
        \end{itemize}
        \item Algeng villa: Gert ráð fyrir of miklu
    \end{itemize}
\end{frame}

\subsection{Bein sönnun}

\begin{frame}{Bein sönnun}
    \begin{itemize}
        \item Algengt vandamál: Sýna að eitt leiði af öðru
        \item Bein sönnun (e. \emph{direct proof}) á afleiðingu sýnir að $p \to q$
        \begin{itemize}
            \item Sýnum þá að ef $p$ er satt er $q$ það líka
            \item Oft þarf að sýna fram á yrðingar á forminu \[\forall x (P(x) \to Q(x))\]
        \end{itemize}
        \item Margar beinar sannanir eru einfaldar aflesturs, en þarfnast ákveðins innsæis til að finna til að byrja með
    \end{itemize}
\end{frame}

\begin{frame}{Dæmi um beina sönnun}
    Byrjum á skilgreiningu:
    \begin{tcolorbox}[title=Oddatölur og sléttar tölur]
        Heiltala $n$ er slétt tala sé til heiltala $k$ svo að $n = 2k$ og oddatala sé til heiltala $k$ svo að $n = 2k+1$. Sérhver heiltala er slétt tala eða oddatala en aldrei hvort tveggja.
    \end{tcolorbox}
    Við getum með beinum hætti sannað eftirfarandi:
    \begin{tcolorbox}
        Sé $n$ oddatala, þá er $n^2$ oddatala.
    \end{tcolorbox}
\end{frame}

\begin{frame}{Dæmi um beina sönnun}
    \begin{columns}
        \column{0.5\linewidth}
        \begin{itemize}[<+->]
            \item Látum $p$ vera forsenduna ``$n$ er oddatala'', viljum sýna afleiðinguna $q$ sem er ``$n^2$ er oddatala''
            \item Gerum ráð fyrir því að $p$ sé sönn
            \item Út frá skilgreiningu á oddatölum er $n = 2k+1$, þar sem $k$ er heiltala
        \end{itemize}
        \column{0.5\linewidth}
        \begin{itemize}[<+->]
            \item Skv. þekktum reiknireglum:
            \begin{align*}
                n^2 &= (2k+1)^2\\
                &=4k^2+4k+1\\
                &=2(2k^2+2k)+1
            \end{align*}
            \item Þar sem $2k^2+2k$ er heiltala er $n^2$ þá oddatala skv. skilgreiningu
            \item Höfum þá sannað að ef $p$, þá $q$
        \end{itemize}
    \end{columns}
\end{frame}

\subsection{Sönnun með mótskilyrðingu}

\begin{frame}{Sönnun með mótskilyrðingu}
    Önnur leið til að sanna staðhæfingu á forminu $\forall x (P(x) \to Q(x))$ er að nota mótskilyrðingu (e. \emph{contraposition}). Athugum að $p \to q$ er jafngilt $\lnot q \to \lnot p$:
    \begin{center}
        \begin{tabular}{cccccc}
            \toprule
            $p$&$q$&$p \to q$&$\lnot p$&$\lnot q$&$\lnot q \to \lnot p$\\
            \midrule
            0&0&1&1&1&1\\
            1&0&0&0&1&0\\
            0&1&1&1&0&1\\
            1&1&1&0&0&1\\
            \bottomrule
        \end{tabular}
    \end{center}
    Ef við getum sýnt fram á $\lnot q \to \lnot p$ er $p \to q$ þar með sannað.
\end{frame}

\begin{frame}{Dæmi um sönnun með mótskilyrðingu}
    Notum mótskilyrðingu til að sanna eftirfarandi:
    \begin{tcolorbox}
        Sé $n$ heiltala og $3n+2$ oddatala, þá er $n$ oddatala.
    \end{tcolorbox}
    Getum séð að $3n + 2 = 2k+1$, en hvað svo?
\end{frame}

\begin{frame}{Dæmi um beina sönnun}
    \begin{columns}
        \column{0.5\linewidth}
        \begin{itemize}[<+->]
            \item Látum $p$ vera forsenduna ``$3n+2$ er oddatala'', viljum sýna afleiðinguna $q$ sem er ``$n$ er oddatala''
            \item Gerum ráð fyrir því að $q$ sé ósönn, að $n$ sé slétt tala
            \item Út frá skilgreiningu á oddatölum er $n = 2k$, þar sem $k$ er heiltala
        \end{itemize}
        \column{0.5\linewidth}
        \begin{itemize}[<+->]
            \item Skv. þekktum reiknireglum:
            \begin{align*}
                3n+2 &= 3(2k)+2\\
                &=6k+2\\
                &=2(3k+1)
            \end{align*}
            \item Þar sem $(3k+1)$ er heiltala er $3n+2$ þá slétt tala skv. skilgreiningu, sem er staðhæfingin $\lnot p$
            \item Höfum þá sannað $\lnot q \to \lnot p$ og þar með $p \to q$
        \end{itemize}
    \end{columns}
\end{frame}

\subsection{Sönnun með mótsögn}

\begin{frame}{Sönnun með mótsögn}
    \begin{itemize}
        \item Mótsagnir geta verið gagnlegar til sannana. Látum $p$ vera staðhæfingu sem sanna skal
        \item Til þess dugar að finna mótsögn $q$ sem er þeim eiginleikum gædd að $\lnot p \to q$ er sönn
        \begin{itemize}
            \item $q$ er ósönn, svo ef $\lnot p \to q$ er sönn þá hlýtur $p$ að vera sönn
        \end{itemize}
        \item Algeng aðferðafræði:
        \begin{itemize}
            \item Finna staðhæfingu $r$ svo að $\lnot p \to (r \land \lnot r)$ sé sönn
        \end{itemize}
    \end{itemize}
\end{frame}

\begin{frame}{Dæmi um sönnun með mótsögn}
    Byrjum á skilgreiningu:
    \begin{tcolorbox}[title=Ræðar og óræðar tölur]
        Rauntalan $r$ er ræð (e. \emph{rational}) séu til heiltölur $p$ og $q$, $q \neq 0$ svo að $r = \frac{p}{q}$. Rauntala sem er ekki ræð er óræð (e. \emph{irrational}).
    \end{tcolorbox}
    Notum mótsögn til að sanna að $\sqrt{2}$ sé óræð tala.
\end{frame}

\begin{frame}{Dæmi um sönnun með mótsögn}
    \begin{columns}
        \column{0.5\linewidth}
        \begin{itemize}[<+->]
            \item Látum $p$ vera yrðinguna ``$\sqrt{2}$ er óræð tala''
            \item Gerum ráð fyrir $\lnot p$, ``$\sqrt{2}$ er óræð tala''
            \item Veljum gildi á heiltölum $a$ og $b$, $b \neq 0$ svo að $\sqrt{2} = \frac{a}{b}$ sé fullstytt brot
        \end{itemize}
        \column{0.5\linewidth}
        \begin{itemize}[<+->]
            \item Skv. þekktum reiknireglum:
            \begin{align*}
                \sqrt{2} &= \frac{a}{b}\\
                \text{svo } 2 &= \frac{a^2}{b^2}\\
                \text{svo } 2b^2 &= a^2
            \end{align*}
            \item Þá er $a^2$ slétt tala. Skrifum $a = 2c$, þar sem $c$ er heiltala, fáum 
            \begin{align*}
                2b^2 &= (2c)^2\\
                \text{svo } b^2 &= 2c^2
            \end{align*}
            \item Þá er $b^2$ líka slétt tala
        \end{itemize}
    \end{columns}
\end{frame}

\begin{frame}{Dæmi um sönnun með mótsögn}
    \begin{itemize}
        \item Þekkt er að ef heiltala $n^2$ er slétt, þá er $n$ slétt \pause
        \begin{itemize}
            \item Contrapositive ``Sé $n$ oddatala, þá er $n^2$ oddatala''
        \end{itemize}
        \item En nú er $\frac{a}{b}$ fullstytt brot \emph{og} $a$ og $b$ eru bæði slétt
        \item Sléttar tölur eru deilanlegar með 2, svo fullstytta brotið er styttanlegt. Mótsögn.
        \item $\lnot p$ er þá ósatt, svo $p$, ``$\sqrt{2}$ er óræð tala'' er satt
    \end{itemize}
\end{frame}

\section{Fleiri gerðir sannana (1.8)}

\begin{frame}{Fleiri gerðir sannana}
    \begin{itemize}
        \item Erum ekki alltaf að sýna fram á afleiðingu
        \item Jafngildissannanir (e. \emph{proofs of equivalence})
        \begin{itemize}
            \item Sýnt fram á $p \to q \land q \to p$
            \item Mikilvægt að sýna fram á ``báðar áttir''!
        \end{itemize}
        \item Þaulleitarsannanir (e. \emph{exhaustive proof})
        \begin{itemize}
            \item Sísanna: $(p_1 \lor p_2 \lor \ldots \lor p_n) \to q \leftrightarrow (p_1 \to q) \land \ldots \land (p_n \to q)$
            \item Hvert tilvik rannsakað fyrir sig
        \end{itemize}
        \item Tilvistarsannanir (e. \emph{existence proof})
        \begin{itemize}
            \item $\exists x P(x)$
        \end{itemize}
        \item Ótvíræðnisannanir (e. \emph{uniqueness proof})
        \begin{itemize}
            \item Tilvistarsönnun á staki $x$, svo sýnt að ef $x \neq y$ þá uppfylli $y$ ekki skilyrðin
        \end{itemize}
    \end{itemize}
\end{frame}

\begin{frame}{Næst}
Í næsta tíma: Mengi og mengjaaðgerðir (kaflar 2.1 og 2.2)
\end{frame}


\end{document}
