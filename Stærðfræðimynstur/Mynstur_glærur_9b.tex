\documentclass[handout]{beamer}

\usepackage[utf8]{inputenc}
\usepackage[icelandic]{babel}
\usepackage[T1]{fontenc}

\usepackage{booktabs}
\usepackage[outputdir=.]{minted} %Minted and configuration
\usepackage{framed}
\usepackage{tikz}
\usemintedstyle{default}
\renewcommand{\theFancyVerbLine}{\sffamily \arabic{FancyVerbLine}}
\newcommand{\Mod}[1]{\ \text{mod}\ #1}

\usebackgroundtemplate%
{%
\vbox to \paperheight{
\includegraphics[width=\paperwidth]{Pics/hi-slide-head}

\vfill
\hspace{0.5cm}\includegraphics[width=0.3\paperwidth]{Pics/hi-von-logo}
\vspace{0.5cm}
    }%
}

\setbeamertemplate{navigation symbols}{}
\usecolortheme{dove}
\setbeamercolor{frametitle}{fg=white}
\hypersetup{colorlinks=true,pdfauthor={Eirikur Ernir Thorsteinsson},linkcolor=blue,urlcolor=blue}

\AtBeginSection[]
{
  \begin{frame}<beamer>
    \frametitle{Yfirlit}
    \tableofcontents[currentsection]
  \end{frame}
}

\author{Eiríkur Ernir Þorsteinsson}
\institute{Háskóli Íslands}
\date{Haust 2016}

\title{Stærðfræðimynstur í tölvunarfræði}
\subtitle{Vika 9, seinni fyrirlestur}

\begin{document}

\begin{frame}
\titlepage
\end{frame}


\section{Inngangur}

\begin{frame}{Í síðasta tíma}
\begin{itemize}
 \item Inngangur að venslum
\end{itemize}
\end{frame}

\begin{frame}[fragile]{Leiðrétting}
\begin{tcolorbox}[title=Gegntæk vensl]
Vensl $R$ á mengi $A$ eru gegntæk (e. \emph{transitive}) ef, hvenær sem $(a, b) \in R$ og $(b, c) \in R$, þá sé $(a, c) \in R$, fyrir öll $a, b, c \in A$. Þ.e.a.s. vensl $R$ á mengi $A$ eru gegntæk þegar $\forall a\forall b\forall c(((a, b) \in R \land (b, c) \in R) \to (a, c) \in R)$
\end{tcolorbox}
\begin{columns}
\column{0.5\textwidth}
Gegnvirk vensl:
\begin{align*}
R_1 &= \{(a, b)|a \leq b\}\\
R_2 &= \{(a, b)|a > b\}\\
R_3 &= \{(a, b)|a = b \text{ eða } a = -b\}\\
R_4 &= \{(a, b)|a = b\}\\
\end{align*}
\column{0.5\textwidth}
Ekki gegnvirk vensl:
\begin{align*}
R_5 &= \{(a, b)|a = b+1\}\\
R_6 &= \{(a, b)|a+b \leq 3\}\\
\end{align*}
\end{columns}
\end{frame}


\section{N-vensl}

\begin{frame}{$n$-undarvensl}
\begin{itemize}
 \item Í síðasta tíma skoðuðum við tvíundarvensl (vensl á tvö mengi) og vensl á mengi (eitt mengi)
 \item Oft koma upp vensl á milli tveggja eða fleiri mengja
 \item Köllum vensl á milli $n$ mengja $n$-undarvensl (e. \emph{n-ary relations})
\end{itemize}
\end{frame}

\begin{frame}{Skilgreining}
\begin{tcolorbox}[title=n-undarvensl]
Látum $A_1, A_2, \ldots, A_n$ vera mengi. $n$-undarvensl á mengin er hlutmengi $A_1 \times A_2 \times \ldots \times A_n$. Mengin $A_1, A_2, \ldots, A_n$ eru þá óðöl (e. \emph{domains}) venslanna og talan $n$ er stig (e. \emph{degree}) þeirra.
\end{tcolorbox}

\end{frame}



\begin{frame}{Næst}

Kafli 9.2 (Hagnýting vensla), Kafli 10.1 (Net), vonandi Kafli 9.3 (Framsetning vensla).

\end{frame}


\end{document}
