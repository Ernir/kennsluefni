\documentclass{article}
    
\usepackage{Haust2017skil}

\title{Stærðfræðimynstur í tölvunarfræði \\ Skilaverkefni 2}
\author{}

\begin{document}
\maketitle

Skila skal þessu verkefni á vefnum \href{https://gradescope.com/courses/9487}{Gradescope}. Aðgangskóði fyrir námskeiðið er \textbf{9N834D}.

Gradescope tekur við .pdf skjölum. Frágangur á þeim skiptir máli. Þau skulu að vera hreinskrifuð í tölvu. Kerfi eins og \LaTeX, Google Docs og Microsoft Word geta búið til .pdf skjöl.

Samvinna á milli nemenda er eðlileg og æskileg. Hins vegar er afritun það aldrei. Miklu gagnlegra er að reyna við dæmin upp á eigin spýtur en að reyna að hala inn hærri einkunn með því að skila inn lausnum annarra.

Telji nemandi að mistök hafi verið gerð við yfirferð skal tilkynna slíkt á Gradescope.

\section{Kafli 1.7}

\question

\textbf{(Ísl)} Notið beina sönnun til að sýna að margfeldi tveggja ræðra talna sé ræð tala.

\textbf{(En)} Use a direct proof to show that the product of two rational numbers is rational.

\paragraph{Í bók} Dæmi 1.7.10 í Icelandic edition, ekki til staðar í Global edition

\question 

\textbf{(Ísl)} Sannið að að sé $n$ heiltala og $3n+2$ slétt tala, þá sé $n$ slétt tala með því að nota

\begin{enumerate}[a)]
    \item sönnun með mótskilyrðingu
    \item sönnun með mótsögn
\end{enumerate}

\textbf{(En)} Prove that if $n$ is an integer and $3n+2$ is even, then $n$ is even using

\begin{enumerate}[a)]
    \item a proof by contraposition
    \item a proof by contradiction
\end{enumerate}

\paragraph{Í bók} Dæmi 1.7.18 í Icelandic edition, ekki til staðar í Global edition

\section{Kafli 1.8}

\question

Sannið eða afsannið eftirfarandi: Séu $a$ og $b$ ræðar tölur, þá er $a^b$ líka ræð tala.

\paragraph{Í bók} Dæmi 1.8.14 í Icelandic edition, dæmi 1.9.10 í Global edition.

\section{Kafli 2.1}

\question 

\textbf{(Ísl)} Finnið sannmengi þessara umsagna. Óðalið er mengi heiltalna.

\textbf{(En)} Find the truth set of each of thes predicates where the domain is the set of all integers.

\begin{enumerate}[a)]
    \item $P(x): x \geq 1$
    \item $Q(x): x^2 = 2$
    \item $R(x): x < x^2$
\end{enumerate}

\paragraph{Í bók} Dæmi 2.1.44 í Icelandic edition, ekki til staðar í Global edition.

\section{Kafli 2.2}

\question Látum $A$, $B$ og $C$ vera mengi. Sýnið að

\begin{itemize}
    \item[c)] $(A - B) - C \subseteq A - C$
    \item[d)] $(A - C) \cap (C - B) = \emptyset$
\end{itemize}

\paragraph{Í bók} Hluti af dæmi 2.2.18 í Icelandic edition, dæmi 2.2.10 í Global edition.

\section{Kafli 2.3}

\question
Látum $f$ vera fall frá mengi $A$ til mengis $B$. Látum $S$ og $T$ vera hlutmengi $A$. Sýnið fram á eftirfarandi: \[f(S \cup T) \subseteq f(S) \cap f(T)\]

\paragraph{Í bók} Hluti af dæmi 2.3.40 í Icelandic edition, dæmi 2.3.26 í Global edition.

\end{document}
