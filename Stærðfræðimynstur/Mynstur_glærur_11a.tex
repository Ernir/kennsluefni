\documentclass{beamer}

\usepackage[utf8]{inputenc}
\usepackage[icelandic]{babel}
\usepackage[T1]{fontenc}

\usepackage{booktabs}
\usepackage[outputdir=.]{minted} %Minted and configuration
\usepackage{framed}
\usepackage{tikz}
\usemintedstyle{default}
\renewcommand{\theFancyVerbLine}{\sffamily \arabic{FancyVerbLine}}
\newcommand{\Mod}[1]{\ \text{mod}\ #1}

\usebackgroundtemplate%
{%
\vbox to \paperheight{
\includegraphics[width=\paperwidth]{Pics/hi-slide-head}

\vfill
\hspace{0.5cm}\includegraphics[width=0.3\paperwidth]{Pics/hi-von-logo}
\vspace{0.5cm}
    }%
}

\setbeamertemplate{navigation symbols}{}
\usecolortheme{dove}
\setbeamercolor{frametitle}{fg=white}
\hypersetup{colorlinks=true,pdfauthor={Eirikur Ernir Thorsteinsson},linkcolor=blue,urlcolor=blue}

\AtBeginSection[]
{
  \begin{frame}<beamer>
    \frametitle{Yfirlit}
    \tableofcontents[currentsection]
  \end{frame}
}

\author{Eiríkur Ernir Þorsteinsson}
\institute{Háskóli Íslands}
\date{Haust 2016}

\title{Stærðfræðimynstur í tölvunarfræði}
\subtitle{Vika 11, fyrri fyrirlestur}

\begin{document}

\begin{frame}
\titlepage
\end{frame}


\section{Inngangur}

\begin{frame}{Í síðasta tíma}
\begin{itemize}
 \item Eiginleikar neta
 \item Nokkrar gerðir sérstakra neta
 \item Spyrðingar
 \item Framsetning á netum
\end{itemize}
\end{frame}

\subsection{Um misserislok}

\begin{frame}{Námsáætlun}
\begin{itemize}
 \item Tímar sem við eigum eftir:
 \begin{itemize}
  \item Þri 1. nóvember (net/stoðtími)
  \item Fös 4. nóvember (net)
  \item Þri 8. nóvember (net/tré?/stoðtími)
  \item Fös 11. nóvember (tré)
  \item Þri 15. nóvember (tré/stoðtími)
  \item Fös 18. nóvember (mál og stöðuvélar)
  \item Þri 22. nóvember (mál og stöðuvélar)
  \item Fös 25. nóvember (dæmayfirferð og stoðtími)
 \end{itemize}
 \item 12. kafla er sleppt
\end{itemize}

\end{frame}


\begin{frame}{Síðustu skiladæmi}
\begin{itemize}
 \item Útlit er fyrir að við náum n skilaverkefnum
 \begin{itemize}
  \item Skilaverkefni 9: Skil 3. nóvember
  \item Skilaverkefni 10: Skil 10. nóvember 
  \item Skilaverkefni 11: Skil 17. nóvember
  \item Skilaverkefni 12: Skil 24. nóvember
 \end{itemize}
 \item Af þessum 12 verkefnum mun einkunn 10 bestu skilaverkefna gilda 20\% til lokaeinkunnar
\end{itemize}
\end{frame}

\begin{frame}{Connect-einkunnin}
\begin{itemize}
 \item \emph{Skil} á Connect-verkefnum gildir 10\% til lokaeinkunnar
 \item $\text{Einkunn} = (\text{hlutfall skilaðra fyrirlestraræfinga} + \text{hlutfall skilaðra dæmatímaverkefna}) \cdot 5$
 \item Tekið er við Connect-verkefnum út nóvember
\end{itemize}
\end{frame}

\section{Vegur í neti}

\begin{frame}{Vegur í neti}
Skilgreining á vegi (e.  path) kemur víða við í netafræði.

\begin{tcolorbox}[title=Vegur]
Látum $n$ vera ekki-neikvæða heiltölu. Þá er vegur af lengd $n$ frá hnúti $u$ í $G$ til hnúts $v$ í $G$ runa $e_1, \ldots, e_n$ af leggjum í $G$ sem hefur þann eiginleika að til sé runa $x_0 = u, x_1, \ldots, x_{n-1}, x_n = v$ af hnútum í $G$ svo að fyrir öll $i = 1, \ldots, n$, hafi leggurinn $e_i$ endahnútana $x_{i-1}$ og $x_i$.
\end{tcolorbox}
\end{frame}

\begin{frame}{Um vegi}
\begin{itemize}
 \item Veg þar sem allir leggirnir eru aðskildir má kalla einfaldan (e. \emph{simple})
 \item Veg af jákvæðri lengd þar sem upphafs- og endahnúturinn er sami hnútur má kalla rás (e. \emph{circuit})
 \item Þegar um er að ræða einfalt net getum við táknað veg með því að telja upp hnútana sem hann snertir
 \item Í stefndu neti getum við á mjög svipaðan hátt skilgreint stefnda vegi/örvavegi (e. \emph{directed paths})
 \item Athugum vandlega: Mismunandi skilgreiningar eru til á vegum og einföldum vegum, þetta er skilgreiningin sem bókin notar
\end{itemize}
\end{frame}



\end{document}
