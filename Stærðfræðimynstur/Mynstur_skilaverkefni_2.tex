\documentclass{exam}

\usepackage[top=0.9in, bottom=1in, left=1.5in, right=1.5in]{geometry}
\usepackage[utf8]{inputenc}
\usepackage[icelandic]{babel}
\usepackage[T1]{fontenc}
\usepackage[sc]{mathpazo}

\usepackage[parfill]{parskip}
\usepackage{booktabs,tabularx}
\usepackage{multirow}
\usepackage{enumerate}
\usepackage{graphicx}
\usepackage{amsmath, amsfonts, amssymb, amsthm}
\usepackage{tikz}
\usepackage{minted} %Minted and configuration

\usepackage[pdftex,bookmarks=true,colorlinks=true,pdfauthor={Eirikur Ernir Thorsteinsson},linkcolor=blue,urlcolor=blue]{hyperref}

\setcounter{secnumdepth}{-1} 
\hyphenpenalty=5000


% Picture locations
\graphicspath{{./Pics/}}

\usemintedstyle{default}
\renewcommand{\theFancyVerbLine}{\sffamily \arabic{FancyVerbLine}}

\newcommand{\Mod}[1]{\ \text{mod}\ #1}

\runningfooter{\hspace{-2cm}\includegraphics[width=0.5\textwidth]{Pics/hi-von-logo}}{}{}

\renewcommand{\solutiontitle}{\noindent\textbf{Mögulegt svar:}}

\author{}
\date{}

\footer{}{}{}

\title{Stærðfræðimynstur í tölvunarfræði \\ Skilaverkefni 2}
\author{}

\printanswers

\begin{document}
\maketitle
\thispagestyle{empty} 

Skila skal þessu verkefni á vefnum \href{https://gradescope.com/}{Gradescope}. Aðgangskóði fyrir námskeiðið er \textbf{926WD9}.

Samvinna á milli nemenda er eðlileg og æskileg. Hins vegar er hvorki æskilegt né heimilt að fá lausnir hjá öðrum (þ.m.t. Google), afrita lausnir eða láta aðra fá lausnina sína. Í slíkum tilvikum er einkunnin 0 gefin sem fyrsta viðvörun. Hikið ekki við að leita til umsjónarkennara ef þið eruð í vafa um hvað telst eðlileg samvinna og hvað ekki.

Telji nemandi að mistök hafi verið gerð við yfirferð skal tilkynna slíkt á vefnum Gradescope.

\section{Spurningar}
\begin{questions}
\question 

Látum $S$ vera mengi sléttra talna og $O$ vera mengi oddatalna. $Z$ er, venju samkvæmt, mengi allra heiltalna.

Í hverju tilviki um sig, hvaða mengi er verið að lýsa?

\begin{enumerate}[a)]
 \item $S \cup O$
 \item $S \cap O$
 \item $Z - S$
 \item $Z - O$
\end{enumerate}

\question

Látum $A$ vera hlutmengi í almenginu $U$. Notið skilgreiningar á mengjahugtökum og umritunarreglur fyrir yrðingar til að sýna fram á eftirfarandi jafngildi:

\begin{enumerate}[a)]
 \item $A - \emptyset = A$
 \item $A \cap U = A$
\end{enumerate}

\question Samhverfur mismunur mengjanna $A$ og $B$ er mengi þeirra staka sem er í $A$ eða $B$, en ekki í báðum. Samhverfan mismun $A$ og $B$ má tákna með $A \oplus B$.

Gildir eftirfarandi? Rökstyðjið.

\[
 A \oplus (B \oplus C) = (A \oplus B) \oplus C
\]

\question Eru eftirfarandi föll gagntæk föll frá $\mathbf{R}$ til $\mathbf{R}$?

\begin{itemize}
 \item $f(x) = 2x + 1$
 \item $f(x) = x^2 + 1$
\end{itemize}

\newpage

\question Nemandi nokkur tekur 4 milljóna króna lán á 8\% árlegum heildarvöxtum. Á hverju ári greiðir nemandinn 300 þúsund krónur inn á lánið.

Setjið upp rakningarvensl sem lýsa stöðu höfuðsstóls lánsins eftir heiltölufjölda ára, tiltakið upphafsskilyrði. Hver er staða höfuðstólsins eftir 3 ár?

\question Sýnið að þessi mengi séu teljanleg með því að sýna gagntæk vensl við mengi jákvæðu heiltalnanna.

\begin{enumerate}[a)]
 \item Mengi heiltalna stærri en 10
 \item Mengi neikvæðra oddatalna
 \item Mengi talna sem eru margfeldi af 10
\end{enumerate}

\end{questions}

\vfill
\includegraphics[width=0.5\linewidth]{hi-von-logo}

\end{document}