\documentclass{article}

\usepackage{Haust2016skil}

\title{Stærðfræðimynstur í tölvunarfræði \\ Námsáætlun \semester}
\author{}

\begin{document}
\maketitle
\hypersetup{pdftitle={Námsáætlun Stærðfræðimynstur 2016}}

Kaflavísanir eru í kennslubók námskeiðsins, \emph{Discrete Mathematics and its Applications} eftir Kenneth H. Rosen, 7. útgáfu.
\vspace{0.5cm}
\begin{center}
\renewcommand{\arraystretch}{1.2}
\begin{tabularx}{\linewidth}{lccXp{1cm}}
\toprule
&\multicolumn{2}{c}{Dagsetningar}&&\\
\cmidrule{2-3}
Vika&Þri&Fös&Námsefni&Kafli\\
\midrule
1	&23/8	&26/8	& Kynning, rök og sannanir&1\\
2	&30/8	&2/9	& Mengi, föll, runur, fylki&2\\
3	&6/9	&9/9	& Reiknirit, vöxtur falla, tímaflækjur&3\\
4	&13/9	&16/9	& Talnafræði, leifar&4\\
5	&20/9	&-	& Þrepun, endurkvæmni&5\\
6	&27/9	&30/9	& Talningar, skúffureglan &6\\
7	&4/10	&7/10	& Líkindareikningur, væntigildi, fervik&7\\
8	&11/10	&18/10	& Rakningarvensl, deila-og-drottna reiknirit&8\\
9	&18/10	&21/10	& Vensl, venslaaðgerðir, vensl í gagnagrunnum&9\\
10	&25/10	&28/10	& Net, mismunandi gerðir neta, Euler-vegir, Hamilton-vegir, stysta-vegs vandamál, lagnet, hnútalitanir&10\\
11	&1/11	&4/11	& Tré, hagnýtingar á trjám, spanntré&11\\
12	&8/11	&11/11	& Bool-algebra, rökhlið&12\\
13	&15/11	&18/11	& Framsetning reiknanleika, endanlegar stöðuvélar, Turing-vélar&13\\
14	&22/11	&25/11	& Eftirlegukindur, valið efni, upprifjun&-\\
\bottomrule
\end{tabularx}
\end{center}
\vspace{0.5cm}
Kennsla fellur niður föstudaginn 23. september. Stefnt er að því að hafa miðmisserispróf laugardaginn 15. október.
\vfill
\includegraphics[width=0.5\linewidth]{hi-von-logo}
\end{document}