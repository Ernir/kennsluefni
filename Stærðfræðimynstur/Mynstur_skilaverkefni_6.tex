\documentclass{exam}

\usepackage[top=0.9in, bottom=1in, left=1.5in, right=1.5in]{geometry}
\usepackage[utf8]{inputenc}
\usepackage[icelandic]{babel}
\usepackage[T1]{fontenc}
\usepackage[sc]{mathpazo}

\usepackage[parfill]{parskip}
\usepackage{booktabs,tabularx}
\usepackage{multirow}
\usepackage{enumerate}
\usepackage{graphicx}
\usepackage{amsmath, amsfonts, amssymb, amsthm}
\usepackage{tikz}
\usepackage{minted} %Minted and configuration

\usepackage[pdftex,bookmarks=true,colorlinks=true,pdfauthor={Eirikur Ernir Thorsteinsson},linkcolor=blue,urlcolor=blue]{hyperref}

\setcounter{secnumdepth}{-1} 
\hyphenpenalty=5000


% Picture locations
\graphicspath{{./Pics/}}

\usemintedstyle{default}
\renewcommand{\theFancyVerbLine}{\sffamily \arabic{FancyVerbLine}}

\newcommand{\Mod}[1]{\ \text{mod}\ #1}

\runningfooter{\hspace{-2cm}\includegraphics[width=0.5\textwidth]{Pics/hi-von-logo}}{}{}

\renewcommand{\solutiontitle}{\noindent\textbf{Mögulegt svar:}}

\author{}
\date{}

\footer{}{}{}

\title{Stærðfræðimynstur í tölvunarfræði \\ Skilaverkefni 6}
\author{}

\printanswers

\begin{document}
\maketitle
\thispagestyle{empty} 

Skila skal þessu verkefni á vefnum \href{https://gradescope.com/}{Gradescope}. Aðgangskóði fyrir námskeiðið er \textbf{926WD9}.

Samvinna á milli nemenda er eðlileg og æskileg. Hins vegar er hvorki æskilegt né heimilt að fá lausnir hjá öðrum (þ.m.t. Google), afrita lausnir eða láta aðra fá lausnina sína. Í slíkum tilvikum er einkunnin 0 gefin sem fyrsta viðvörun. Hikið ekki við að leita til umsjónarkennara ef þið eruð í vafa um hvað telst eðlileg samvinna og hvað ekki.

Telji nemandi að mistök hafi verið gerð við yfirferð skal tilkynna slíkt á vefnum Gradescope.

\section{Spurningar}

\subsection{Kafli 5.3}

\begin{questions}

\question Finnið $f(1), f(2), f(3)$ og $f(4)$ sé $f(n)$ skilgreint endurkvæmt með $f(0) = 1$ og með eftirfarandi formúlum fyrir ekki-neikvæða heiltölu $n$:

\begin{enumerate}[a)]
 \item $f(n+1) = f(n)+2$
 \item $f(n+1) = 3f(n)$
\end{enumerate}

\paragraph{Í bók} Hluti af exercise 5.3.1

\question Skrifið endurkvæma skilgreiningu á eftirfarandi mengjum:

\begin{enumerate}[a)]
 \item Mengi jákvæðra oddatalna
 \item Mengi talnanna $3^n$, þar sem $n$ er jákvæð heiltala
\end{enumerate}

\paragraph{Í bók} Hluti af exercise 5.3.24

\question Látum $w^i$ vera strenginn sem myndaður er með því að skeyta strengnum $w$ við sjálfan sig $i$ sinnum. Sé t.d. $w = abc$ væri $w^2 = abcabc$.

Skrifið endurkvæma skilgreiningu á $w^i$.

\paragraph{Í bók} Exercise 5.3.37 

\subsection{Kafli 5.4}

\question Skrifið endurkvæmt reiknirit sem finnur minnstu heiltölu í endanlegri runu heiltalna.

Ráðlegging: Athugum að minnsta heiltala í $n$ staka runu heiltalna er annaðhvort síðasta talan í rununni eða minnsta talan af fyrstu $n - 1$ tölunum í rununni.

\paragraph{Í bók} Exercise 5.4.11

\question Skrifið endurkvæmt reiknirit sem reiknar út $n^2$ með því að nota eftirfarandi staðreynd: \[(n + 1)^2 = n^2 + 2n + 1\].

\paragraph{Í bók} Hluti af exercise 5.4.23

\subsection{Kafli 5.5}

Sýnið að forritsbúturinn 

\begin{align*}
y: &= 1\\
z: &= x + y\\
\end{align*}
sé naumréttur með tilliti til forskilyrðisins $x = 0$ og eftirskilyrðisins $z = 1$.

\paragraph{Í bók} Exercise 5.5.1

\end{questions}

\vfill
\includegraphics[width=0.5\linewidth]{hi-von-logo}

\end{document}