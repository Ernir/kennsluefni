\documentclass{exam}

\usepackage[top=0.9in, bottom=1in, left=1.5in, right=1.5in]{geometry}
\usepackage[utf8]{inputenc}
\usepackage[icelandic]{babel}
\usepackage[T1]{fontenc}
\usepackage[sc]{mathpazo}

\usepackage[parfill]{parskip}
\usepackage{booktabs,tabularx}
\usepackage{multirow}
\usepackage{enumerate}
\usepackage{graphicx}
\usepackage{amsmath, amsfonts, amssymb, amsthm}
\usepackage{tikz}
\usepackage{minted} %Minted and configuration

\usepackage[pdftex,bookmarks=true,colorlinks=true,pdfauthor={Eirikur Ernir Thorsteinsson},linkcolor=blue,urlcolor=blue]{hyperref}

\setcounter{secnumdepth}{-1} 
\hyphenpenalty=5000


% Picture locations
\graphicspath{{./Pics/}}

\usemintedstyle{default}
\renewcommand{\theFancyVerbLine}{\sffamily \arabic{FancyVerbLine}}

\newcommand{\Mod}[1]{\ \text{mod}\ #1}

\runningfooter{\hspace{-2cm}\includegraphics[width=0.5\textwidth]{Pics/hi-von-logo}}{}{}

\renewcommand{\solutiontitle}{\noindent\textbf{Mögulegt svar:}}

\author{}
\date{}

\footer{}{}{}

\title{Stærðfræðimynstur í tölvunarfræði \\ Skilaverkefni 7}
\author{}

\printanswers

\begin{document}
\maketitle
\thispagestyle{empty} 

Skila skal þessu verkefni á vefnum \href{https://gradescope.com/}{Gradescope}. Aðgangskóði fyrir námskeiðið er \textbf{926WD9}.


\section{Spurningar}

\begin{questions}

\subsection{Kafli 6.1}

\question 
\begin{enumerate}[a)]
 \item Hversu mörg föll eru til frá mengi með 10 stökum til mengis með 4 stökum?
 \item Gefið er mengi með 10 stökum. Hversu mörg af hlutmengjum þess hafa meira en eitt stak?
\end{enumerate}

\paragraph{Í bók:} Dæmi 6.1.34c, 6.1.40

\subsection{Kafli 6.2}

\question Gefnar eru 51 heiltala, hver þeirra er á lokaða bilinu frá 0 til 99. Sýnið að a.m.k. tvær talnanna séu aðliggjandi, þ.e.a.s. að til sé $n$ þannig að bæði $n$ og $n+1$ séu meðal gefnu talnanna.

\paragraph{Í bók:} Byggt á dæmi 6.2.44

\subsection{Kafli 6.3}

\question Hversu margir bitastrengir af lengd 19 innihalda nákvæmlega fimm 0 og fjórtán 1 ef á eftir hverju 0 þurfa að koma tvö 1?

\paragraph{Í bók:} Dæmi 6.3.36

\newpage
\subsection{Kafli 6.4}

\question Sýnið að

\[
 \binom{2n}{2} = 2\binom{n}{2} + n^2
\]

\begin{enumerate}[a)]
 \item með bókstafareikningi
 \item með tvöfaldri talningu
\end{enumerate}

\paragraph{Í bók:} Dæmi 6.4.28

\subsection{Kafli 8.2}

Finnið lokaða formúlu fyrir $a_n$ í rakningarvenslunum
\[
 a_n = a_{n-1} + 6a_{n-2}
\]
þegar $n \geq 2$. Upphafsskilyrði eru $a_0 = 3$ og $a_1 = 6$.

\paragraph{Í bók:} Dæmi 8.2.4a

\subsection{Kafli 8.3}

Notið ``master theorem'' til að finna stóra-O mat á $f(n) = 9f\left(\frac{n}{3}\right) + n$.

\paragraph{Í bók:} Þetta dæmi er ekki í bókinni.
\end{questions}

\vfill
\includegraphics[width=0.5\linewidth]{hi-von-logo}

\end{document}