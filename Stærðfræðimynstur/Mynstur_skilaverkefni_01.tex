\documentclass{article}
    
\usepackage{Haust2017skil}

\title{Stærðfræðimynstur í tölvunarfræði \\ Skilaverkefni 1}
\author{}

\begin{document}
\maketitle

Skila skal þessu verkefni á vefnum \href{https://gradescope.com/courses/9487}{Gradescope}. Aðgangskóði fyrir námskeiðið er \textbf{9N834D}.

Gradescope tekur við .pdf skjölum. Frágangur á þeim skiptir máli. Þau skulu að vera hreinskrifuð í tölvu. Kerfi eins og \LaTeX, Google Docs og Microsoft Word geta búið til .pdf skjöl.

Samvinna á milli nemenda er eðlileg og æskileg. Hins vegar er afritun það aldrei. Miklu gagnlegra er að reyna við dæmin upp á eigin spýtur en að reyna að hala inn hærri einkunn með því að skila inn lausnum annarra.

Telji nemandi að mistök hafi verið gerð við yfirferð skal tilkynna slíkt á Gradescope.

\section{Kafli 1.1}
\subsection{Verkefni 1}
Látum $p$, $q$ og $r$ vera eftirfarandi yrðingar:

\begin{align*}
p &: \text{Nemandi fær háa prófseinkunn}\\
q &: \text{Nemandi gerir öll verkefnin}\\
r &: \text{Nemandi fær háa lokaeinkunn}
\end{align*}

Skrifið eftirfarandi yrðingar með því að nota $p$, $q$ og $r$ ásamt rökvirkjum:

\begin{itemize}
    \item[a)] Nemandi fær háa lokaeinkunn en gerir ekki öll verkefnin.
    \item[c)] Til að fá háa lokaeinkunn er nauðsynlegt að fá háa prófseinkunn.
    \item[e)] Það fá háa prófseinkunn og að gera öll verkefnin er nemanda nægjanlegt til að fá háa lokaeinkunn.
    \item[f)] Nemandi fær háa lokaeinkunn ef og aðeins ef nemandinn annaðhvort gerir öll verkefnin eða fær háa prófseinkunn.
\end{itemize}

\paragraph{Í bók} Hluti af 1.1.14 í Icelandic edition, 1.1.8 í Global edition
\newpage
\subsection{Verkefni 2} 
Eftirfarandi setningar nota orðið ``eða''. Í hverju tilfelli um sig, útskýrið hvort um er að ræða ``eða'' (táknað með $\lor$) eða ``aðgreint eða'' (táknað með $\oplus$).

\begin{enumerate}[a)]
 \item Starfið krefst reynslu af Java eða C\#.
 \item Með hádegismatnum fylgir súpa eða salat.
 \item Til að ferðast til landsins þarftu að vera með gilt vegabréf eða ökuskírteni.
 \item Nú skal duga eða drepast.
\end{enumerate}

\paragraph{Í bók} 1.1.20 í Icelandic edition, 1.1.14 í Global edition

\subsection{Verkefni 3} 
Skrifið upp sanntöflur fyrir eftirfarandi yrðingar.
\begin{itemize}
\item[b)] $(p \lor  q) \land     r$
\item[f)] $(p \land q) \lor \neg r$
\end{itemize}

\paragraph{Í bók} Hluti af 1.1.36 í Icelandic edition, ekki til staðar í Global edition

\section{Kafli 1.2}

\subsection{Verkefni 4}

Verkefni af lokaprófi 2016.

\textbf{(Ísl)} Vitað er að riddarar ljúga aldrei, ribbaldar ljúga alltaf og að allar manneskjur séu annaðhvort riddarar eða ribbaldar. Nú hittum við Önnu, Ásmund og Birnu. Hvað eru Anna, Ásmundur og Birna ef Anna segir ``ég er ribbaldi og Ásmundur er riddari'' en Ásmundur segir ``nákvæmlega eitt af okkur þremur er riddari''? Útskýrið af hverju.

\textbf{(En)} It is known that knights always tell the truth, knaves never tell the truth, and that every person is either a knight or a knave. Suppose that we meet Anna, Ásmundur and Birna. What are Anna, Ásmundur and Birna if Anna says ``I am a knave and Ásmundur is a knight'' and Ásmundur says ``Exactly one of the three of us is a knight''? Explain your reasoning.

\section{Kafli 1.3}

\subsection{Verkefni 5}
Sýnið að eftirfarandi yrðing sé sísanna á tvo vegu. Fyrst með því að nota sanntöflu, svo án þess að nota sanntöflu.
\[ (p \land (p \to q)) \to q \]

\paragraph{Í bók} Hluti af 1.3.10 og 1.3.12 í Icelandic edition, 1.3.6 og ``1.3.7'' í Global edition.

\section{Kafli 1.4}

\subsection{Verkefni 6} 
Setjið fram eftirfarandi staðhæfingar með því að nota rökvirkja, umsagnir og magnara. Tilgreinið viðkomandi óðal.
\begin{enumerate}[a)]
 \item Einhver hlutur er ekki á réttum stað.
 \item Öll verkfæri eru á réttum stað og í góðu ástandi.
 \item Ekkert er á réttum stað og í góðu ástandi.
 \item Eitt verkfæranna er ekki á réttum stað, en það er í góðu ástandi.
\end{enumerate}

\paragraph{Í bók} 1.4.28 í Icelandic edition, 1.4.18 í Global edition.

\end{document}
