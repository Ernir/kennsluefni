\documentclass{exam}

\usepackage[top=0.9in, bottom=1in, left=1.5in, right=1.5in]{geometry}
\usepackage[utf8]{inputenc}
\usepackage[icelandic]{babel}
\usepackage[T1]{fontenc}
\usepackage[sc]{mathpazo}

\usepackage[parfill]{parskip}
\usepackage{booktabs,tabularx}
\usepackage{multirow}
\usepackage{enumerate}
\usepackage{graphicx}
\usepackage{amsmath, amsfonts, amssymb, amsthm}
\usepackage{tikz}
\usepackage{minted} %Minted and configuration

\usepackage[pdftex,bookmarks=true,colorlinks=true,pdfauthor={Eirikur Ernir Thorsteinsson},linkcolor=blue,urlcolor=blue]{hyperref}

\setcounter{secnumdepth}{-1} 
\hyphenpenalty=5000


% Picture locations
\graphicspath{{./Pics/}}

\usemintedstyle{default}
\renewcommand{\theFancyVerbLine}{\sffamily \arabic{FancyVerbLine}}

\newcommand{\Mod}[1]{\ \text{mod}\ #1}

\runningfooter{\hspace{-2cm}\includegraphics[width=0.5\textwidth]{Pics/hi-von-logo}}{}{}

\renewcommand{\solutiontitle}{\noindent\textbf{Mögulegt svar:}}

\author{}
\date{}

\footer{}{}{}

\title{Stærðfræðimynstur í tölvunarfræði \\ Skilaverkefni 12}
\author{}

\printanswers

\begin{document}
\maketitle
\thispagestyle{empty} 

Skila skal þessu verkefni á vefnum \href{https://gradescope.com/}{Gradescope}. Aðgangskóði fyrir námskeiðið er \textbf{926WD9}. Allar heiðarlegar tilraunir til að leysa þetta verkefni gefa einkunnina 10.

\section{Spurningar}

\begin{questions}

\section{Kafli 11.4}

\question Notið dýptarleit til að finna spanntré fyrir eftirfarandi net. Veljið $a$ sem rót. Þegar velja þarf á milli jafngildra hnúta, veljið í stafrófsröð. Gefið lausn með því að telja leggina upp í röð.

\begin{center}
\includegraphics[width=0.5\textwidth]{tree-spanning-exercise}
\end{center}

\paragraph{Í bók:} Exercise 11.4.14

\section{Kafli 11.5}

\question Finnið tvö léttustu spanntré í eftirfarandi neti, annars vegar með reikniriti Prims og hins vegar með reikniriti Kruskals. Gefið lausn með því að telja leggina upp í röð.

\begin{center}
\includegraphics[width=0.2\textwidth]{tree-spanning-minimum-exercise}
\end{center}

\paragraph{Í bók:} Byggt á Exercise 11.5.2

\section{Kafli 13.2}

\question Skrifið endanlega stöðuvél sem hefur bitastreng sem inntak og bitastreng sem úttak. Úttaksstrengurinn skal hefjast á $00$ en annars vera eins og inntaksstrengurinn seinkaður um tvo bita.

Gefið lausn með því að teikna stöðuvélina.

\paragraph{Í bók:} Exercise 13.2.9

\question Skrifið endanlega stöðuvél sem hefur bitastreng sem inntak og bitastreng sem úttak. Úttaksstrengurinn skal hafa $1$ ef fjöldi bita sem lesinn hefur verið inn er deilanlegur með 3, en $0$ annars.

Gefið lausn með því að teikna stöðuvélina.

\paragraph{Í bók:} Exercise 13.2.16

\section{Kafli 13.3}

\question Lýsið mengi strengjanna sem eftirfarandi stöðuvél samþykkir.

\begin{center}
\includegraphics[width=0.5\textwidth]{dfa-exercise}
\end{center}

\paragraph{Í bók:} Exercise 13.3.17

\question Skrifið endanlega stöðuvél sem samþykkir (eingöngu) mengi þeirra bitastrengja sem samanstanda af $0$ og svo oddatölufjölda $1$.

Gefið lausn með því að teikna stöðuvélina.

\paragraph{Í bók:} Exercise 13.3.35

\end{questions}

\end{document}