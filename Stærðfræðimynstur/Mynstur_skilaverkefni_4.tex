\documentclass{exam}

\usepackage{Haust2016verkefnablöð}

\title{Stærðfræðimynstur í tölvunarfræði \\ Skilaverkefni 4}
\author{}

\printanswers

\begin{document}
\maketitle
\thispagestyle{empty} 

Skila skal þessu verkefni á vefnum \href{https://gradescope.com/}{Gradescope}. Aðgangskóði fyrir námskeiðið er \textbf{926WD9}.

Samvinna á milli nemenda er eðlileg og æskileg. Hins vegar er hvorki æskilegt né heimilt að fá lausnir hjá öðrum (þ.m.t. Google), afrita lausnir eða láta aðra fá lausnina sína. Í slíkum tilvikum er einkunnin 0 gefin sem fyrsta viðvörun. Hikið ekki við að leita til umsjónarkennara ef þið eruð í vafa um hvað telst eðlileg samvinna og hvað ekki.

Telji nemandi að mistök hafi verið gerð við yfirferð skal tilkynna slíkt á vefnum Gradescope.

\section{Spurningar}

\subsection{Kafli 3.3}
\begin{questions}
\question Gefinn er eftirfarandi forritsbútur á sauðakóðaformi:

\begin{verbatim}
t := 0
fyrir i:= 1 upp í n
    fyrir j:= 1 upp í n
        t := t + i + j
\end{verbatim}

Svarið um hann eftirfarandi spurningum:
\begin{enumerate}[a)]
 \item Hversu margar samlagningar framkvæmir forritið, sem fall af $n$?
 \item Hver er keyrslutími forritsins, á stóra-$\Theta$ formi? Ekki nota óþarfa fasta í stóra-$\Theta$ framsetningunni.
\end{enumerate}

\question Gefinn er eftirfarandi forritsbútur á sauðakóðaformi:

\begin{verbatim}
i := 1
t := 0
meðan i er minna en eða jafnt n
    t := t + i
    i := 2 * i
\end{verbatim}

Svarið um hann eftirfarandi spurningum:
\begin{enumerate}[a)]
 \item Hversu margar samlagningar og margfaldanir framkvæmir forritið, sem fall af $n$?
 \item Hver er keyrslutími forritsins, á stóra-$\Theta$ formi? Ekki nota óþarfa fasta í stóra-$\Theta$ framsetningunni.
\end{enumerate}

\newpage

\question Látum $n$ tákna inntaksstærð reiknirits. Hversu stórt verkefni (hversu stórt $n$) er hægt að leysa á einni mínútu ef reikniritið þarfnast $f(n)$ bitaaðgerða, með því að nota tölvu sem framkvæmir bitaaðgerð á $10^{-12}s$, fyrir eftirfarandi föll $f(n)$?

\begin{enumerate}[a)]
 \item $10000n$
 \item $2^n$
 \item $(\log_2 n)^2$
 \item $\log_2 \log_2 n$
\end{enumerate}
Setjið mjög stórar heiltölur fram á skynsamlegan máta.

\subsection{Kafli 4.1}

\question Reiknið eftirfarandi með heiltöludeilingu. Tiltakið hver deilir (e. \emph{divisor}), deilistofn (e. \emph{dividend}), kvóti (e. \emph{quotient}) og afgangur (e. \emph{remainder}) eru í hvoru tilviki um sig.

\begin{enumerate}[a)]
 \item \texttt{67/3}
 \item \texttt{88/5}
\end{enumerate}

\subsection{Kafli 4.2}

\question Skiptið um grunntölu á eftirfarandi máta. Sýnið aðferð.

\begin{enumerate}[a)]
 \item Setjið tugakerfistöluna 230 fram sem tvíundarkerfistölu
 \item Setjið tvíundarkerfistöluna 00001101 fram sem tugakerfistölu
 \item Setjið sextándakerfistöluna 4A fram sem tvíundarkerfistölu
\end{enumerate}

\subsection{Kafli 4.3}

\question Finnið þáttanir á eftirfarandi samsettum tölum með endurtekinni deilingu með þekktum prímtölum. Sýnið útreikninga.

\begin{enumerate}[a)]
 \item $729$
 \item $2431$
\end{enumerate}

\end{questions}

\vfill
\includegraphics[width=0.5\linewidth]{hi-von-logo}

\end{document}