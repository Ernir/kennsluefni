\documentclass{article}
    
\usepackage{Haust2017skil}

\title{Stærðfræðimynstur í tölvunarfræði \\ Skilaverkefni 9}
\author{}

\begin{document}
\maketitle

Skila skal þessu verkefni á vefnum \href{https://gradescope.com/courses/9487}{Gradescope}. Aðgangskóði fyrir námskeiðið er \textbf{9N834D}. 

Gradescope tekur við .pdf skjölum. Frágangur á þeim skiptir máli. Þau skulu vera hreinskrifuð í tölvu. Kerfi eins og \LaTeX, Google Docs og Microsoft Word geta búið til .pdf skjöl. Mikilvægt er að merkja á hvaða blaðsíðu .pdf skjalsins hver lausn kemur fyrir, ekki er hægt að gera ráð fyrir að dæmatímakennarar fari yfir ómerkt dæmi.

Skila má þessum dæmum sem einstaklingar eða \emph{tvö og tvö saman}.

Telji nemandi að mistök hafi verið gerð við yfirferð skal tilkynna slíkt með tölvupósti til dæmatímakennara. Nálgast má lista yfir hvaða dæmatímakennari fór yfir hvaða dæmi á Piazza-vef námskeiðsins.

\section{Kafli 5.3}

\question

Sýnið útreikninga á $f(1), f(2), f(3), f(4)$ og $f(5)$ sé $f(n)$ skilgreint endurkvæmt með $f(0)=3$ og fyrir $n=0, 1, 2,\ldots$ með

\begin{enumerate}[a)]
    \item $f(n+1) = -2f(n)$
    \item $f(n+1) = 3f(n)+7$
\end{enumerate}

\paragraph{Í bók:} 5.3.2 í International/Icelandic, svipar til sama dæmis í Global

\question

Sýnið endurkvæma skilgreiningu á mengi jákvæðu oddatalnanna.

\paragraph{Í bók:} Hluti af 5.3.24 í International/Icelandic, 5.3.16 í Global

\section{Kafli 5.4}

\question

Sýnið öll skrefin sem endurkvæma reikniritið fyrir stærsta samdeili fer í gegnum til að reikna $gcd(12,17)$.

\paragraph{Í bók:} 5.4.4 í International/Icelandic, svipar til 5.4.2 í Global

\section{Kafli 8.2}
\question

Finnið lokaða formúlu fyrir $a_n$ í rakningarvenslunum
\[
 a_n = a_{n-1} + 6a_{n-2}
\]
þegar $n \geq 2$. Upphafsskilyrði eru $a_0 = 3$ og $a_1 = 6$. Sýnið útreikninga.

\paragraph{Í bók:} 8.2.4a í International/Icelandic, svipar til 8.2.2 í Global


\question
Sjómenn nokkrir hafa sett upp rakningarvenslalíkan til að áætla árlegan humarafla. Líkanið segir að fjöldi humra sem veiðist á hverju ári sé meðaltal af fjöldanum sem veiddist árin tvö á undan.

\begin{enumerate}[a)]
    \item Sýnið rakningarvenslin $\{L_n\}$ sem lýsa fjölda humra sem veiðist á ári $n$ samkvæmt líkaninu.
    \item Sýnið lokaða formúlu fyrir $L_n$ hafi 100000 humrar veiðst árið 1 og 300000 veiðst árið 2 
\end{enumerate}

\paragraph{Í bók:} 8.2.8 í International/Icelandic, 8.2.6 í Global

\section{Kafli 8.3}
\question 

\paragraph{(Ísl):} Finnið stóra-O mat á eftirfarandi deila-og-drottna rakningarvenslum. Útskýrið aðferð.

\[f(n) = 9f\left(\frac{n}{3}\right) + n\].

\paragraph{(En):} Give a big-O estimate of the divide-and-conquer relation given above. Explain your approach.

\paragraph{Í bók:} Þetta dæmi er ekki í bókinni.

\end{document}
