\documentclass[handout]{beamer}

\usepackage[utf8]{inputenc}
\usepackage[icelandic]{babel}
\usepackage[T1]{fontenc}

\usepackage{booktabs}
\usepackage[outputdir=.]{minted} %Minted and configuration
\usepackage{framed}
\usepackage{tikz}
\usemintedstyle{default}
\renewcommand{\theFancyVerbLine}{\sffamily \arabic{FancyVerbLine}}
\newcommand{\Mod}[1]{\ \text{mod}\ #1}

\usebackgroundtemplate%
{%
\vbox to \paperheight{
\includegraphics[width=\paperwidth]{Pics/hi-slide-head}

\vfill
\hspace{0.5cm}\includegraphics[width=0.3\paperwidth]{Pics/hi-von-logo}
\vspace{0.5cm}
    }%
}

\setbeamertemplate{navigation symbols}{}
\usecolortheme{dove}
\setbeamercolor{frametitle}{fg=white}
\hypersetup{colorlinks=true,pdfauthor={Eirikur Ernir Thorsteinsson},linkcolor=blue,urlcolor=blue}

\AtBeginSection[]
{
  \begin{frame}<beamer>
    \frametitle{Yfirlit}
    \tableofcontents[currentsection]
  \end{frame}
}

\author{Eiríkur Ernir Þorsteinsson}
\institute{Háskóli Íslands}
\date{Haust 2016}

\title{Stærðfræðimynstur í tölvunarfræði}
\subtitle{Vika 1, seinni fyrirlestur}

\begin{document}

\begin{frame}
\titlepage
\end{frame}

\section{Inngangur}

\begin{frame}{Í síðasta tíma}
\begin{itemize}
 \item Kynntumst hugtakinu um \emph{yrðingar}
 \begin{itemize}
  \item Yrðingar eru fullyrðingar sem eru áreiðanlega sannar eða áreiðanlega ósannar
  \item Yrðingar má tákna með rökbreytum
 \end{itemize}
 \item Sáum rökvirkja sem vinna á yrðingum, sanntöflur, skilgreiningu á jafngildi yrðinga
\end{itemize}
\end{frame}

\section{Hagnýting yrðinga}

\begin{frame}[fragile]{Í forritun}
\begin{columns}
\column{0.5\textwidth}
\column{0.5\textwidth}
\end{columns}
\end{frame}

\begin{frame}{Lausnir á gátum}

\end{frame}

\begin{frame}{Rökrásir}

\end{frame}

\section{Jafngildi yrðinga}

\subsection{Ýmsar reglur}

\section{Magnarar}


\begin{frame}{Næst}
Í næsta tíma:
\end{frame}


\end{document}
