\documentclass{beamer}

\usepackage{Haust2017glærur}

\title{Stærðfræðimynstur í tölvunarfræði}
\subtitle{Vika 6, fyrri fyrirlestur}

\begin{document}

\begin{frame}
\titlepage
\end{frame}


\section{Inngangur}

\begin{frame}{Í þarsíðasta tíma}
\begin{itemize}
 \item Talnafræði
 \item Deiling og afgangur
 \item Framsetning heiltalna
 \begin{itemize}
  \item Tvíundarkerfið og önnur kerfi
 \end{itemize}
 \item Prímtölur
 \item Stærsti samdeilir og minnsta samfeldi
\end{itemize}
\end{frame}

\section{Leifajöfnur}

\begin{frame}{Leifajöfnur}
    Út frá hugmyndum um deilingarafganga fáum við leifajöfnur.
    \begin{tcolorbox}
        Séu $a$ og $b$ heiltölur og $m$ jákvæð heiltala, þá eru $a$ og $b$ samleifa mátað við $m$ (e. \emph{a is congruent to b modulo m}) ef $m$ deilir $(a-b)$.

        Þá skrifum við $a \equiv b (\Mod m)$, sem er leifajafna (e. \emph{congruence}).
    \end{tcolorbox}
    Hér að ofan er $m$ leifastofn (e. \emph{modulus}).
\end{frame}

\begin{frame}{Athugasemd við rithátt}
    \[a \equiv b (\Mod m)\] og \[a \Mod m = b\] nota bæði orðið ``mod'', en merkingin er mismunandi
    \begin{itemize}
        \item Fyrri notkunin lýsir venslum (e. \emph{relation})
        \item Seinni notkunin lýsir falli (e. \emph{function})
    \end{itemize}    
\end{frame}

\begin{frame}{Mátun}
    Notkun hugtakanna er (eðlilega) nátengd:
    \begin{tcolorbox}
        Séu $a$ og $b$ heiltölur og $m$ jákvæð heiltala er $a \equiv b (\Mod m)$ þá og því aðeins að $a \Mod m = b \Mod m$.
    \end{tcolorbox}
    Svipuð hugmynd:
    \begin{tcolorbox}
        Séu $a$ og $b$ heiltölur og $m$ jákvæð heiltala er $a \equiv b (\Mod m)$ þá og því aðeins að til sé heiltala $k$ svo að $a = b + km$.
    \end{tcolorbox}
\end{frame}

\begin{frame}{Dæmi}
    \begin{itemize}
        \item Sumar leifajöfnur ganga ekki upp
        \item Eru $17$ og $5$ samleifa mátað við $6$? \pause
        \begin{itemize}
            \item Já, sjáum að 6 gengur upp í $(17-5)$
        \end{itemize}
        \item Eru $57$ og $37$ samleifa mátað við $10$? \pause
        \begin{itemize}
            \item Já, sjáum að $57 \Mod 10 = 37 \Mod 10$ = 7
        \end{itemize}
        \item Eru $24$ og $14$ samleifa mátað við $6$? \pause
        \begin{itemize}
            \item Nei, sjáum að 6 gengur ekki upp í $(24-10)$
        \end{itemize}
    \end{itemize}    
\end{frame}

\begin{frame}{Lausn á leifajöfnum}
    Lausn á jöfnum á forminu 
    \[
        ax = b(\Mod m)
    \]
    þar sem $x$ er breyta, $a$, $b$ og $m$ eru heiltölur ($m$ jákvæð) er viðfangsefni kafla 4.4, sem við sleppum.

    \vspace{0.5cm}

    Mikilvæg niðurstaða: Þessar jöfnur eru leysanlegar séu $a$ og $m$ ósamþátta.
\end{frame}

\section{Klassísk dulkóðun}

\begin{frame}{Klassísk dulkóðun}
\begin{itemize}
 \item Ein elsta þekkta leiðin til að dulkóða skilaboð er kennd við Júlíus Sesar
 \item Sesar gerði skilaboðin sín ólæsileg með því að skipta hverjum bókstaf í skilaboðunum út fyrir þann bókstaf sem er þremur sætum aftar í stafrófinu
 \begin{itemize}
  \item $B$ verður að $E$, $N$ verður að $Q$
 \end{itemize}
 \item Það að gera skilaboð ólæsileg á kerfisbundinn, afkallanlegan máta kallast dulkóðun (e. \emph{encryption})
 \item Það að ná upprunalegu skilaboðunum aftur úr dulkóðuninni kallast afkóðun (e. \emph{decryption})
\end{itemize}
\end{frame}

\begin{frame}{Reikniritið hans Sesars}
\begin{itemize}
 \item Í klassíska latneska stafrófinu eru 23 bókstafir
 \begin{itemize}
  \item $A	B	C	D	E	F	G	H	 I	K	L	M	N	O	P	Q	R	S	T	V	X	Y	Z$
 \end{itemize}
 \item Sesar hefur því séð fyrir sér að hvern staf megi tákna með heiltölu $p$, þar sem $0 \leq p \leq 22$
 \item Sesar hefur næst skilgreint fall til dulkóðunar:
\end{itemize}
\[
 f(p) = (p + 3) \Mod 23
\]
\end{frame}

\begin{frame}{Dulkóðun}
Latína er lítið notuð í dag. Notum frekar enska stafrófið (sem er af lengd 26) og dulkóðum skilaboðin \texttt{MEET YOU IN THE PARK} með reikniriti Sesars. \pause

\texttt{MEET YOU IN THE PARK} má setja fram með tölum:
\begin{center}
\texttt{12 4 4 19\quad
24 14 20\quad
8 13\quad
19 7 4\quad
15 0 17 10}
\end{center}\pause
skiptum svo hverri tölu $p$ út fyrir $f(p) = (p+3) \Mod 26$:
\begin{center}
\texttt{
15 7 7 22\quad
1 17 23\quad
11 16\quad
22 10 7\quad
18 3 20 13}
\end{center}
sem má lesa sem \texttt{PHHW BRX LQ WKH SDUN}.
\end{frame}

\begin{frame}{Afkóðun}
Til að afkóða þarf að hliðra stöfunum til baka niður stafrófið. Þetta má gera með andhverfu dulkóðunarfallsins, andhverfa dulkóðunarfallsins á glærunni á undan er:

\[
 f^{-1}(p) = (p-3)\Mod 26
\]
\end{frame}

\begin{frame}{Almennari hliðrunardulkóðun}
Almennt má fá fram dulkóðun með því að hliðra hverjum staf í stafrófinu um $k$ sæti:
\[
 f(p) = (p + k) \Mod 26
\]
og afkóðun með
\[
 f^{-1}(p) = (p - k) \Mod 26
\]
Hér er $k$ lykill (e. \emph{key}) fyrir dulkóðunina.
\end{frame}

\begin{frame}{Betri hliðrunardulkóðun}
Hægt er að gera hliðrunardulkóðun minna óörugga með því að nota fall:
\[
 f(p) = (ap + b)\Mod 26
\]
þar sem $a$ og $b$ eru heiltölur og $gcd(a,26) = 1$. 

\vspace{0.5cm}

Þegar $gcd(a,26) = 1$ er vörpunin gagntæk.
\end{frame}

\begin{frame}{Að brjóta hliðrunardulkóðun}
\begin{itemize}
 \item Hliðrunardulkóðun er viðkvæm fyrir tíðnigreiningu
 \item Tökum dulkóðaðan texta af góðri lengd
 \item Skoðum hvaða stafir koma oftast fyrir í textanum
 \item Berum saman við hvaða stafir eru algengastir í viðkomandi máli
 \item Prófa sig áfram
\end{itemize}
\end{frame}

\section{Blokkardulkóðun}

\begin{frame}{Blokkardulkóðun}
\begin{itemize}
 \item Við getum verndað okkur gegn tíðnigreiningu með því að skipta ekki út einum staf í einu, heldur mörgum stöfum
 \item Skiptum textanum upp í blokkir
 \item Skiptum blokkunum út í heilu lagi
 \item Kallast \emph{block cypher} (ísl. \emph{blokkardulkóðun}?)
\end{itemize}
\end{frame}

\begin{frame}{Blokkardulkóðun}
\begin{itemize}
 \item Skoðum umskiptingardulkóðun (e. \emph{transposition cypher})
 \item Notum umröðun $\sigma$ á hlutmengi í jákvæðu heiltölunum $\{1, 2, \ldots, m\}$
 \item Látum $\sigma$ sem sagt vera gagntæka vörpun frá $\{1, 2, \ldots, m\}$ í sjálft sig
 \item Til dulkóðunar skiptum við textanum upp í blokkir af stærð $m$
\end{itemize}
\[
 p_1p_2\ldots p_m = c_1c_2\ldots c_m = p_{\sigma(1)}p_{\sigma(2)}\ldots p_{\sigma(m)}
\]

\end{frame}

\begin{frame}{Dæmi um blokkardulkóðun}
Gefin er blokkarstærðin $m=4$ með $\sigma(1) = 3, \sigma(2) = 1, \sigma(3) = 4$ og $\sigma(4) = 2$.

Dulkóðum strenginn \texttt{PIRATE ATTACK}. Fáum grunntextann \texttt{PIRA TEAT TACK}.

Í samræmi við $\sigma$, setjum fyrsta stafinn í þriðja sætið, annan stafinn í fyrsta sætið\ldots

Fáum \texttt{IAPR ETTA AKTC}.
\end{frame}

\begin{frame}{Dæmi um blokkarafkóðun}
Gefin er blokkarstærðin $m=4$ með $\sigma(1) = 3, \sigma(2) = 1, \sigma(3) = 4$ og $\sigma(4) = 2$.

Afkóðum strenginn \texttt{SWUETRAEOEHS}. Athugum að $\sigma^{-1}$ varpar $3$ í $1$, $1$ í $2$, \ldots

Fáum strenginn \texttt{USEW ATER HOSE}. Störum og sjáum að hér stendur \texttt{USE WATER HOSE}.
\end{frame}

\section{Public Key Cryptography}

\begin{frame}{Public Key Cryptography}
\begin{itemize}
 \item Galli við klassíska dulkóðun: Sami lykill er notaður til að dulkóða og til að afkóða
 \item Komist einhver að lyklinum til að framkvæma dulkóðun má líka framkvæma afkóðun
 \begin{itemize}
  \item Slíkt kerfi er kallað ``private key'' dulkóðunarkerfi
  \item Góð private key kerfi eru til, t.d. AES
 \end{itemize}
 \item Á 8. áratugnum voru ``public key'' dulkóðunarkerfi búin til
 \begin{itemize}
  \item Mismunandi dulkóðunar- og afkóðunarlyklar, einungis afkóðunarlyklinum haldið leyndum
 \end{itemize}
\end{itemize}
\end{frame}

\begin{frame}{RSA kerfið}
\begin{itemize}
 \item Í RSA kerfinu er hver aðili með opinberan dulkóðunarlykil og leynilegan afkóðunarlykil
 \item Hluti af opinbera lyklinum er margfeldi tveggja prímtalna
 \item Hluti af leynilega lyklinum eru prímtölurnar sjálfar
 \item Þáttun er erfitt verkefni, svo leynilegi lykillinn er öruggur svo lengi sem prímtölurnar eru sæmilega stórar
\end{itemize}
\end{frame}

\begin{frame}{RSA kerfið}
\begin{itemize}
 \item Nokkrir hlutir koma við sögu í RSA dulkóðun
 \begin{itemize}
  \item Stórar prímtölur $p$ og $q$ (haldið leyndum)
  \item Margfeldi prímtalnanna, $n = pq$ (gert opinbert)
  \item Veldisvísir $e$ (gerður opinber)
  \begin{itemize}
   \item $e$ og $(p-1)(q-1)$ þurfa að vera ósamþátta, þ.e.a.s. $gcd(e, (p-1)(q-1)) = 1$
   \item $1 < e < n$
  \end{itemize}
  \item $d$, tala svo að $de \equiv 1 (\Mod (p-1)(q-1))$ (haldið leyndri)
  \begin{itemize}
   \item Þetta má reikna út (með aðferðum úr kafla 4.4) ef $p$ og $q$ eru þekktar
  \end{itemize}
 \end{itemize}
\end{itemize}
\end{frame}

\begin{frame}{Einföld dulkóðun og afkóðun}
Gerum ráð fyrir að skilaboðin $m$ séu heiltala, $0 \leq m < n$ og $gcd(m, n) = 1$. Þá má búa til dulkóðuð skilaboð $c$ með
\[
c = m^e \Mod n
\]
og afkóða með 
\[
 m = c^d \Mod n
\]
Opinberi lykillinn er þá tvenndin $(n, e)$ og leyndi lykillinn talan $d$.

\end{frame}

\begin{frame}{Af hverju virkar þetta?}
Það er talan $d$ sem er notuð til afkóðunar. $d$ er reiknað með því að leysa $de \equiv 1 (\Mod (p-1)(q-1))$.

Tölurnar $n = pq$ og $e$ eru gerðar opinberar. Væri hægt að reikna $p$ og $q$ út frá $n$ þegar $n$ er mjög stórt (meira en 300-400 stafir) væri RSA óöruggt, en það virðist vera erfitt verkefni.
\end{frame}

\begin{frame}{Dulkóðun og afkóðun - dæmi}
Byrjum á að finna tölurnar sem til þarf. Veljum $p=61$, $q = 53$ svo $n = 3233$. Fáum $(61 - 1)(53 - 1) = 3120$. Veljum okkur prímtölu sem veldisvísi $e$ svo auðveldara sé að sannfæra okkur um að $gcd(e,3120)=1$. Veljum $e = 17$. Leysum fyrir $d$, fáum $d = 2753$.

Dulkóðum töluna 65.
\[
 c = 65^{17} \Mod 3233 = 2790
\]
Afkóðum töluna 2790 með sama reikniriti.
\[
 m = 2790^{2753} \Mod 3233 = 65
\]
\end{frame}

\begin{frame}{Meira um RSA}
\begin{itemize}
 \item Til að dulkóða texta með RSA má skipta textanum upp í blokkir, tákna hverja blokk með heiltölu og dulkóða hverja tölu um sig
 \begin{itemize}
  \item Sýnidæmi 8 í kafla 4.6 í bók
 \end{itemize}
 \item Hægt er að nota RSA til að skiptast á lyklum þó að allar boðleiðir séu óöruggar
 \begin{itemize}
  \item ``Cryptographic Protocols'' í bók
  \item Eftir að public key dulkóðun hefur verið notuð til að skiptast á lyklum má skipta yfir í private key dulkóðunarkerfi
 \end{itemize}
\end{itemize}
\end{frame}

\begin{frame}{Næst}
Þrepun (5.1)
\end{frame}



\end{document}
