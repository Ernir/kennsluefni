\documentclass[handout]{beamer}

\usepackage[utf8]{inputenc}
\usepackage[icelandic]{babel}
\usepackage[T1]{fontenc}

\usepackage{booktabs}
\usepackage[outputdir=.]{minted} %Minted and configuration
\usepackage{framed}
\usepackage{tikz}
\usemintedstyle{default}
\renewcommand{\theFancyVerbLine}{\sffamily \arabic{FancyVerbLine}}
\newcommand{\Mod}[1]{\ \text{mod}\ #1}

\usebackgroundtemplate%
{%
\vbox to \paperheight{
\includegraphics[width=\paperwidth]{Pics/hi-slide-head}

\vfill
\hspace{0.5cm}\includegraphics[width=0.3\paperwidth]{Pics/hi-von-logo}
\vspace{0.5cm}
    }%
}

\setbeamertemplate{navigation symbols}{}
\usecolortheme{dove}
\setbeamercolor{frametitle}{fg=white}
\hypersetup{colorlinks=true,pdfauthor={Eirikur Ernir Thorsteinsson},linkcolor=blue,urlcolor=blue}

\AtBeginSection[]
{
  \begin{frame}<beamer>
    \frametitle{Yfirlit}
    \tableofcontents[currentsection]
  \end{frame}
}

\author{Eiríkur Ernir Þorsteinsson}
\institute{Háskóli Íslands}
\date{Haust 2016}

\title{Stærðfræðimynstur í tölvunarfræði}
\subtitle{Vika 6, fyrri fyrirlestur}

\begin{document}

\begin{frame}
\titlepage
\end{frame}


\section{Inngangur}

\begin{frame}{Í síðasta tíma}
\begin{itemize}
 \item Þrepun
 \item Sterk þrepun
\end{itemize}
\end{frame}

\section{Rakningarvensl - upprifjun}

\begin{frame}{Rakningarvensl}
\begin{tcolorbox}[title=Rakningarvensl]
Rakningarvensl (e. \emph{recurrence relation}) fyrir runu $\{a_n\}$ er jafna sem skilgreinir $a_n$ sem fall af einum eða fleiri fyrri liðum rununnar, þ.e.a.s. $a_0, a_1, \ldots, a_{n-1}$ fyrir öll $n > n_0$, þar sem $n_0$ er ekki-neikvæð heiltala.
\end{tcolorbox}
Runa er lausn (e. \emph{solution}) á rakningarvenslum ef liðir hennar uppfylla venslin. Liðirnir sem eru fyrir framan fyrsta liðinn sem rankningarvenslin tilgreina eru upphafsskilyrði (e. \emph{initial conditions}) þeirra.
\end{frame}

\begin{frame}{Dæmi um rakningarvensl}
Látum $\{a_n\}$ vera runu sem uppfyllir $a_n = a_{n-1} + 3$ fyrir $n=1, 2, 3, \ldots$ og setjum $a_0 = 2$. Hvað eru þá $a_1, a_2$ og $a_3$? \pause

\begin{align*}
a_1 &= a_0 + 3 = 2 + 3 = 5\\
a_2 &= 5 + 3 = 8\\
a_3 &= 8 + 3 = 11\\
\end{align*}

\end{frame}

\begin{frame}{Dæmi um rakningarvensl}
\begin{itemize}
 \item Fibonacci-rununa $f_0, f_1, f_2, \ldots$ má skilgreina með rakningavenslum
 \begin{itemize}
  \item Upphafsskilyrði: $f_0 = 0, f_1 = 1$
  \item Rakningarvensl: $f_n = f_{n-1} + f_{n-2}$
 \end{itemize}
\end{itemize}
\end{frame}

\section{Endurkvæmni}

\begin{frame}{Endurkvæm lýsing á föllum}
\begin{itemize}
 \item Rakningarvensl eru runur þar sem seinni liðir eru skilgreindir út frá fyrri liðum
 \item Hægt er að nota endurkvæmni (e. \emph{recursion}) til að skilgreina föll út frá sjálfum sér
 \item Endurkvæm skilgreining á falli (með mengi ekki-neikvæðu heiltalnanna sem formengi) fer fram í tvennu lagi:
 \begin{enumerate}
  \item Grunnskref (e. \emph{basis step}) skilgreinir fallsgildi fallsins í núlli eða nálægt núlli
  \item Endurkvæmt skref (e. \emph{recursive step}) gefur reglu til að finna fallsgildi fallsins í gefinni heiltölu með því að nota gildi fallsins af lægri heiltölum
 \end{enumerate}
 \item Endurkvæmni er líka kölluð rakning, endurkvæma skrefið er líka kallað þrepunarskref (e. \emph{inductive step})
\end{itemize}
\end{frame}

\begin{frame}{Endurkvæmt fall}
Við getum skilgreint fall á eftirfarandi hátt:
\begin{align*}
f(0) &= 3\\
f(n+1) &= 2f(n) + 3
\end{align*}
Hvernig finnum við gildi $f(4)$? \pause Finnum öll minni fallsgildi, og reiknum:
\begin{align*}
f(1) = 2f(0) + 3 &= 2 \cdot 3 + 3 = 9\\
f(2) = 2f(1) + 3 &= 2 \cdot 9 + 3 = 21\\
f(3) = 2f(2) + 3 &= 2 \cdot 21 + 3 = 45\\
f(4) = 2f(3) + 3 &= 2 \cdot 45 + 3 = 93
\end{align*}
\end{frame}

\begin{frame}{Endurkvæm skilgreining}
Hvernig getum við á endurkvæman máta skilgreint stærðina $a^n$, þar sem $a$ er rauntala önnur en $0$ og $n$ er jákvæð heiltala?
\pause
\begin{itemize}
 \item Grunnskref: Skilgreinum $a^0 = 1$.
 \item Endurkvæmt skref: $a^{n+1} = a \cdot a^n$
\end{itemize}
\end{frame}

\section{Endurkvæm skilgreining mengja}

\begin{frame}{Endurkvæm skilgreining mengja}
\begin{itemize}
 \item Hægt er að skilgreina mengi á endurkvæman máta 
 \begin{itemize}
  \item Grunnskref: Tilgreinum safn staka sem tilheyrir menginu
  \item Endurkvæmt skref: Tilgreinum reglu til að búa til ný stök út frá upphaflegu stökunum
 \end{itemize}
\end{itemize}
\end{frame}

\begin{frame}{Endurkvæm skilgreining}
\begin{itemize}
 \item Skilgreinum $S$, mengi allra jákvæðra margfelda af $3$:
 \begin{itemize}
  \item Grunnskref: $3 \in S$
  \item Endurkvæmt skref: Sé $x \in S$ og $y \in S$, þá er $x+y \in S$
 \end{itemize}
\end{itemize}
\end{frame}

\begin{frame}[fragile]{Upprifjun - Strengir}
\begin{itemize}
 \item Sérstök, mikið notuð gerð af runum kallast strengur (e. \emph{string})
 \item Strengur er endanleg runa af stöfum, úr endanlegu mengi (stafrófinu)
 \begin{itemize}
  \item Stafrófið getur verið hvaða mengi sem er, en algeng stafróf eru \texttt{\{'a', 'b', 'c', \ldots\}} og \texttt{\{0, 1\}}
 \end{itemize}
 \item Strengur með engum stöfum (af lengdinni $0$) er kallaður tómur og táknaður með $\lambda$
\end{itemize}
\end{frame}

\begin{frame}{Strengir - endurkvæm skilgreining}
\begin{itemize}
 \item Við getum skilgreint mengi strengja $\Sigma^*$ yfir stafrófið $\Sigma$ á endurkvæman hátt:
 \begin{itemize}
  \item Grunnskref: $\lambda \in \Sigma^*$
  \item Endurkvæmt Skref: Sé $w \in \Sigma^*$ og $x \in \Sigma$, þá er $wx \in \Sigma^*$
 \end{itemize}
 \item Þ.e.a.s. ef við skeytum staf úr stafrófinu aftan á streng fáum við annan streng
\end{itemize}
\end{frame}

\begin{frame}{Strengjasamskeyting}
\begin{itemize}
 \item Getum skilgreint samskeytingu (e. \emph{concatenation}) tveggja strengja á endurkvæman átt:
 \begin{itemize}
  \item Grunnskref: Sé $w \in \Sigma^*$ er $w\lambda = w$
  \item Endurkvæmt Skref: Sé $w_1 \in \Sigma^*$ og $w_2 \in \Sigma^*$ og $x \in \Sigma$, þá er $w_1(w_2x) = (w_1w_2)x$
 \end{itemize}
 \item Dæmi: Sé $w_1 = abra$ og $w_2 = cadabra$ er $w_1w_2 = abracadabra$.
\end{itemize}
\end{frame}

\begin{frame}{Næst}
Kaflar 5.4 (endurkvæm reiknirit) og 5.5 (sönnun forrita).
\end{frame}


\end{document}
