\documentclass{article}
    
\usepackage{Haust2017skil}

\title{Stærðfræðimynstur í tölvunarfræði \\ Skilaverkefni 7}
\author{}

\begin{document}
\maketitle

Skila skal þessu verkefni á vefnum \href{https://gradescope.com/courses/9487}{Gradescope}. Aðgangskóði fyrir námskeiðið er \textbf{9N834D}. 

Gradescope tekur við .pdf skjölum. Frágangur á þeim skiptir máli. Þau skulu vera hreinskrifuð í tölvu. Kerfi eins og \LaTeX, Google Docs og Microsoft Word geta búið til .pdf skjöl. Mikilvægt er að merkja á hvaða blaðsíðu .pdf skjalsins hver lausn kemur fyrir, ekki er hægt að gera ráð fyrir að dæmatímakennarar fari yfir ómerkt dæmi.

Samvinna á milli nemenda er eðlileg og æskileg. Hins vegar er afritun það aldrei. Miklu gagnlegra er að reyna við dæmin upp á eigin spýtur en að reyna að hala inn hærri einkunn með því að skila inn lausnum annarra.

Telji nemandi að mistök hafi verið gerð við yfirferð skal tilkynna slíkt á Gradescope.

\section{Kafli 6.1}

\question Breytuheiti í forritunarmálinu C er strengur sem getur innihaldið hástafi og lágstafi úr enska stafrófinu, tölustafi og undirstrik (\_). Að auki má fyrsta tákn breytuheitisins má ekki vera tölustafur.

Hversu mörg breytuheiti eru möguleg í C ef öll táknin eftir þau fyrstu 8 eru hunsuð? Munum að breytur geta innihaldið færri en 8 tákn. Sýnið útreikninga.

\paragraph{Í bók:} 6.1.56 í International/Icelandic, 6.1.38 í Global

\section{Kafli 6.2}

\question 10 rauðir boltar og 10 bláir boltar eru settir í skál. Kona tekur bolta úr skálinni af handahófi.

\begin{enumerate}[a)]
    \item Hversu marga bolta þarf hún að velja til að fá a.m.k. 3 bolta í sama lit? Útskýrið svarið með vísun í skúffureglu.
    \item Hversu marga bolta þarf hún að velja til að fá a.m.k. 3 bláa bolta?
\end{enumerate}

\paragraph{Í bók:} Dæmi 6.2.4 í báðum útgáfum

\question 

\textbf{(Ísl):} Gefnar eru 51 heiltala, hver þeirra er á lokaða bilinu frá 0 til 99. Sýnið með skúffureglu að a.m.k. tvær talnanna séu aðliggjandi, þ.e.a.s. að til sé $n$ þannig að bæði $n$ og $n+1$ séu meðal gefnu talnanna.

\textbf{(En):} 51 integers on the closed interval from 0 to 99 are given. Use the pigeonhole principle to show that at least two of the integers are consecutive, that is, that there is an integer $n$ so that both $n$ and $n+1$ are among the given numbers.

\paragraph{Í bók:} Byggt á dæmi 6.2.44 í International/Icelandic

\section{Kafli 6.3}

\question

Í stærðfræðideild nokkurri starfa sjö konur og níu karlar. Á hve marga vegu er hægt að velja í fimm meðlima nefnd ef\ldots

\begin{enumerate}[a)]
    \item í nefndinni þarf að vera a.m.k. ein kona?
    \item í nefndinni þarf að vera a.m.k. ein kona og a.m.k. einn karl?
\end{enumerate}

\paragraph{Í bók:} 6.3.30 í International/Icelandic, 6.3.20 í Global

\paragraph{Athugasemd:} Dæmið gerir ráð fyrir að karlar og konur myndi sundurlæg mengi og að allt starfsfólk deildarinnar falli innan þeirra.

\section{Kafli 6.4}

\question Sýnið að sé $p$ prímtala og $k$ heiltala $1 \leq k \leq p-1$, þá gangi $p$ upp í $\binom{p}{k}$.

\paragraph{Í bók:} Dæmi 6.4.24 í International/Icelandic, 6.4.16 í Global

\question Sýnið að

\[
    \binom{2n}{2} = 2\binom{n}{2} + n^2
\]

\begin{enumerate}[a)]
    \item með tvöfaldri talningu
    \item með bókstafareikningi
\end{enumerate}

\paragraph{Í bók:} Dæmi 6.4.28 í International/Icelandic, 6.4.20 í Global

\end{document}
