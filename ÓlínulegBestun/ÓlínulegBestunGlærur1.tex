\documentclass{beamer}

\usepackage{Vor2018glærur}

\title{Ólínuleg Bestun}
\subtitle{Fyrirlestur 1}

\begin{document}

\begin{frame}
\titlepage
\end{frame}

\section{Matlab}

\begin{frame}{Hvað er þetta Matlab?}
    \begin{columns}
        \column{0.5\textwidth}
        \begin{itemize}
            \item Matlab er bæði \emph{forritunarmál} og \emph{forritunarumhverfi}
            \item Nafnið stendur fyrir \textbf{Mat}rix \textbf{Lab}oratory
            \item Sérsniðið fyrir vísindalega útreikninga
            \begin{itemize}
                \item \href{http://se.mathworks.com/company/newsletters/articles/the-origins-of-matlab.html}{Upphaflega (1984) búið til sem þægilegra viðmót á Fortran-forritunarsöfn fyrir fylkjareikninga}
            \end{itemize}
            \item Talið auðvelt að læra
        \end{itemize}
        \column{0.5\textwidth}
        \begin{quote}
            Half of the students in the class were from math and computer science, and they were not impressed 
        
            ...
            
            The other half of the students were from engineering, and they liked MATLAB.
        \end{quote}
    \end{columns}
\end{frame}

\begin{frame}{Uppsetning Matlab}
    \begin{itemize}
    \item Matlab er hvorki opið né gjaldfrjálst
    \begin{itemize}
        \item Fyrirtækið \emph{Mathworks Inc.} á Matlab
        \item Háskólinn býður upp á notkunarleyfi
        \item Uppsetningarleiðbeiningar má finna á: \url{https://notendur.hi.is/\~jonasson/matlab/}
    \end{itemize}
        \item Ókeypis og opið kerfi sem líkist Matlab er til, \href{https://www.gnu.org/software/octave/}{GNU Octave}
        \begin{itemize}
            \item Erfitt að nota Octave í þessu námskeiði
        \end{itemize}
    \end{itemize}
\end{frame}

\section{Breytur og gagnatög}

\begin{frame}[fragile]{Tögun í Matlab}
Matlab er dýnamískt tagað mál með sjálfvirkum tögunarbreytingum

\begin{minted}{matlab}
>> x = 1.4 % Fleytitala búin til
x =
    1.4000
>> x + 'a' % Stafbreytan a túlkuð sem tala
ans =
    98.4000
\end{minted}

Meðal grundvallargagnataga eru \texttt{double}, \texttt{char} og \texttt{logical}. Meðal einkennandi taga fyrir Matlab eru \texttt{cell} og \texttt{struct}.

\end{frame}

\begin{frame}{Fylki og vigrar í Matlab}
    \begin{itemize}
        \item Flest í Matlab er einhvers konar fylki
        \begin{itemize}
            \item ``venjulegar'' breytur eru $1 \times 1$ fylki
            \item Vigrar eru $1 \times n$ eða $n \times 1$ fylki
        \end{itemize}
        \item Vísun í stök vigra í Matlab byrjar í \textbf{1}
        \item Höfum sérstakan virkja til að búa til og vísa í stök vigra, \texttt{:}
        \item Flest innbyggð föll í Matlab geta tekið við vigurinntökum
        \item Virkjarnir \texttt{*} og \texttt{/} framkvæma aðgerðir úr línulegri algebru, \texttt{.*} og \texttt{./} framkvæma stakvísar aðgerðir
        \item Skoðum sérstaklega: rökvigrar og rökvísanir
    \end{itemize}
\end{frame}

\section{Föll í Matlab}

\begin{frame}[fragile]{Innbyggð föll}
    \begin{itemize}
        \item Fjöldinn allur af innbyggðum föllum er til staðar í Matlab
        \item Allt í sama nafnarýminu, engar innflutningsskipanir
        \item Besti vinur ykkar: \texttt{help [fallsnafn]} og \texttt{doc [fallsnafn]}
    \end{itemize}
    \begin{minted}{matlab}
>> x = linspace(-pi,pi,5)
x =
    -3.1416   -1.5708         0    1.5708    3.1416
>> sin(x)
ans =
    -0.0000   -1.0000         0    1.0000    0.0000  
    \end{minted}
\end{frame}

\begin{frame}[fragile]{Notendaskilgreind forrit}
    \begin{itemize}
        \item Höfum nokkrar gerðir forrita, sem eru geymdar í \texttt{.m} skrám
        \begin{itemize}
            \item Skipanaskrár \eng{scripts} eru samansöfn af Matlab-skipunum keyrðar í röð
            \item Notendaskilgreind föll \eng{user-defined functions}
            \item Klasar
        \end{itemize}
        \item Fallshaus í Matlab:

        \begin{minted}{matlab}
function [skil1, skil2, ...] = nafnfalls(inn1, ...)
        \end{minted}

        \item Meginreglan (með þó nokkrum undantekningum) er að hvert fall sé skilgreint í sinni eigin skrá, sem heitir það sama og fallið
    \end{itemize}
\end{frame}

\section{Lykkjur og flæðisstýring}

\begin{frame}{Lykkjur og flæðisstýring í Matlab}
    \begin{itemize}
        \item Höfum kunnuglega flæðisstýringu
        \begin{itemize}
            \item \texttt{if}, \texttt{elseif}, \texttt{else}
            \item Höfum líka \texttt{switch}
        \end{itemize}
        \item Tvær gerðir af lykkjum
        \begin{itemize}
            \item For-lykkja, sem reyndar er ``foreach'' lykkja
            \item While-lykkja
        \end{itemize}
        \item Svigar og slaufusvigar koma ekki fyrir, lokun með \texttt{end} lykilorðinu
    \end{itemize}
\end{frame}

\begin{frame}[fragile]{Uppbygging}
\begin{minted}{matlab}
if skilyrdi
    skipanir
end

for itrunarbreyta = svid
    skipanir
end

while skilyrdi
    skipanir
end
\end{minted}
\end{frame}

\begin{frame}{Fyrirlestraræfing 1}
    \begin{itemize}
        \item (Æfing 3.1f) Þýðið gildinguna $r := \text{ ekki } (q \text{ eða } (p \text{ og } a \neq 2))$ yfir í Matlab
        \item (Byggt á æfingu 3.2) Skrifið Matlab-fall sem tekur inn gildi á $m$, $k_1$, $k_2$ og skilar $T$ reiknuðu út frá formúlunni
        \[
            T = 2\pi\sqrt{\frac{m}{k_1+k_2}}
        \]
        \item (Æfing 3.8) Skrifið forrit sem býr til tvö $4 \times 4$ slembifylki $A$ og $B$, reiknar summu þeirra $C$ og skrifar öll fylkin út með \texttt{disp}
    \end{itemize}
\end{frame}

\end{document}
