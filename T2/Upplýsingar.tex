\documentclass{article}

\input{../Vor2017skil.tex}

\title{Tölvunarfræði 2, \semester \\ Námsáætlun og upplýsingar}
\author{}

\begin{document}
\maketitle
\hypersetup{pdftitle={Tölvunarfræði 2 - \semester}}

\section{Námsáætlun}
\label{sec:schedule}

Námskeiðið er tvískipt. 

Fyrri hluti námskeiðsins er þrjár vikur. Í þeim hluta verður farið í grundvallaratriði forritunarmálsins C++.

Aðalviðfangsefni námskeiðsins, gagnagrindur og reiknirit sem á þeim vinna, koma fyrir í seinni hluta. Forritun í Java og C++ kemur við sögu. Seinni hluti spannar afgang kennslumisserisins, samtals 11 vikur.

\begin{table}
\caption{Námsáætlun eftir vikum}
\label{tab:schedule}
\begin{center}
\renewcommand{\arraystretch}{1.2}
\begin{tabularx}{\linewidth}{lcXp{1cm}}
\toprule
Vika&Dagsetning&Námsefni&Kafli\\
\midrule
1	&13/1	& Kynning á námskeiðinu, kynning á C++, minnismeðhöndlun í C++&C++ 1\\
2	&20/1	& Hlutbundin forritun í C++&C++ 2\\
3	&21/1	& Staðalsafn C++ (STL)&C++ 3\\
4	&27/9	& Uppbygging forrita, gagnagrindur/gagnatög&Alg 1\\
5	&3/2	& Skjóður, biðraðir, hlaðar, greining reiknirita&Alg 1\\
6	&10/2	& Röðunarreiknirit&Alg 2\\
7	&17/2	& Röðunarreiknirit, forgangsbiðraðir&Alg 2\\
8	&24/2	& Leit, leitartré&Alg 3\\
9	&3/3	& Leitartré, hakkatöflur&Alg 3\\
10	&10/3	& Net, stefnd net&Alg 4\\
11	&17/3	& Minnstu spanntré, stystu vegir&Alg 4\\
12	&24/3	& Strengir, trie&Alg 5\\
13	&31/3	& Strengjaleit, reglulegar segðir&Alg 5\\
14	&7/4	& Skekkjumörk í námsáætlun&-\\
15	&14/4	& Páskaleyfi&-\\
16	&21/4	& ?&-\\
\bottomrule
\end{tabularx}
\end{center}
\end{table}

\section{Kennari}
Aðalkennari er Eiríkur Ernir Þorsteinsson. Aðsetur er í Tæknigarði, 2. hæð, stofa 214. Sjá mynd \ref{fig:taeknigardur}.

Til að hafa samband við kennara er ráðlagt að setja inn þráð á \href{https://piazza.com/hi.is/spring2017/tl203g/}{Piazza}. Allar fyrirspurnir sem ekki fela í sér persónulegar upplýsingar ættu að fara þangað. 

Tölvupóstfang er \href{mailto:ernir@hi.is}{ernir@hi.is}.

\begin{figure}
\caption{Önnur hæð í Tæknigarði. Kennara má finna í stofu 214.}
\label{fig:taeknigardur}
\begin{center}
\includegraphics[width=0.5\linewidth]{taeknigardur}
\end{center}
\end{figure}

\section{Tímar og námstilhögun}
Aðalkennsla fer fram í vikulegum fyrirlestri. Fyrirlestrarnir eru á föstudögum frá 10:00 til 12:20 í Háskólabíói, sal 1.

Dæmatímar eru á mánudögum og þriðjudögum, ætlaðir til stuðnings fyrir skilaverkefni. Dæmahópum er úthlutað eftir fögum og stundatöflum, nemendur eru vinsamlegast beðnir um að mæta í úthlutaðan dæmatíma ef kostur er til að dreifingin á milli hópa sé sem jöfnust.

Ekki er skylda að mæta í fyrirlestra eða dæmatíma (en sjá \nameref{sec:lecture-exercises}).
\section{Námsbækur}
Aðalkennslubók er \emph{Algorithms}, fjórða útgáfa, eftir Sedgewick og Wayne. Kennsla eftir þeirri bók hefst í fjórðu kennsluviku, eftir kynningu á C++ (sjá \nameref{tab:schedule}). 

Til stuðnings við er bent á \emph{C++ in One Hour a Day}, 8. útgáfa, eftir Siddhartha Rao. Litið verður til hennar við yfirferð á C++, en ekki verður vísað beint í bókina.

\begin{figure}
\caption{Kennslubækur}
\begin{multicols}{2}
\begin{center}
\includegraphics[height=8cm]{algorithms4th}
\end{center}

\begin{center}
\includegraphics[height=8cm]{teachyourself}
\end{center}
\end{multicols}
\end{figure}

\newpage
\section{Námsmat, einkunnir og próf}
\subsection{Fyrirlestraræfingar}
\label{sec:lecture-exercises}
Fyrirlestraræfingar eru í hverjum föstudagsfyrirlestri. Vægi þeirra er 10\% af lokaeinkunn. Þátttaka í $n$ fyrirlestraræfingum gefur hluteinkunnina $n$, að hámarki 10.

Tenglar á fyrirlestraræfingar verða gerðir aðgengilegir í hverri viku. Fyrirlestraræfingarnar verða nafni sínu samkvæmt hannaðar til að vera leystar í fyrirlestrartímunum sjálfum, en miðað verður við að hver þeirra verði opin í um það bil sólarhring.
\subsection{Skilaverkefni}
Vikuleg verkefnaskil eru í námskeiðinu. Meðaleinkunn 10 bestu skilaverkefnanna er 20\% af lokaeinkunn. Verkefnunum skal skila á Gradescope.com (sjá \nameref{sec:tools}). Ekki er tekið við seinum skilum.

Nauðsynlegt er að skila fjórum af fyrstu sex skilaverkefnunum til að öðlast próftökurétt.
\subsection{Próf}
Vægi lokaprófs er 70\% af lokaeinkunn. Leyfilegt verður að taka með eitt blað af glósum (skrifað á báðar hliðar) í lokapróf.

Lágmarkseinkunn er 5. Nauðsynlegt er að ná lágmarkseinkunn á lokaprófi sem og í námskeiðinu sem heild til að standast námskeiðið. 

Ekki er miðmisserispróf í námskeiðinu.
\section{Kennslutól}
\label{sec:tools}
Vefkennslutól verða notað eftir föngum. Mælt er með að nemendur skrái sig inn á eftirfarandi þjónustur sem allra fyrst:
\begin{itemize}
 \item \href{https://piazza.com/hi.is/spring2017/tl203g/home}{Piazza} er fyrirspurnavefurinn sem notaður er í námskeiðinu. Skráningin í námskeiðið á að vera sjálfvirk, en hafi það misfarist á að vera hægt að skrá sig handvirkt. Allar spurningar sem snúa að námskeiðinu ættu að fara inn á Piazza frekar en í tölvupóst.
 \item \href{https://gradescope.com/courses/5640}{Gradescope.com} er vefkerfið sem notað er til að taka við skilaverkefnum. Nemendur þurfa að skrá sig sjálfir á þennan vef. Aðgangskóði námskeiðsins er \texttt{95434M}. Mikilvægt er að skrá sig á Gradescope með fullu nafni (íslenskir stafir eru leyfilegir) og með því að nota HÍ-netfang. Kerfið tekur við \texttt{.pdf} skrám.
\end{itemize}
Fyrir utan vefkennslutól er nauðsynlegt að nemendur setji upp þýðendur fyrir C++ og Java ásamt viðeigandi ritlum á eigin tölvum eða útvegi sér aðgang að slíkri vél.

\vfill
\includegraphics[width=0.5\linewidth]{hi-von-logo}
\end{document}