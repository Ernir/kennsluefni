\documentclass{beamer}

\usepackage[utf8]{inputenc} % Language and font encoding
\usepackage[icelandic]{babel}
\usepackage[T1]{fontenc}


\usepackage{tikz}
\usepackage[listings,theorems]{tcolorbox}
\usepackage{booktabs}
\usepackage{minted} %Minted and configuration
\usemintedstyle{default}

\renewcommand{\theFancyVerbLine}{\sffamily \arabic{FancyVerbLine}}

%%%%%%%%%%%%%%%%%%%%%%
% Beamer configuration
%%%%%%%%%%%%%%%%%%%%%%
\setbeamertemplate{navigation symbols}{}
\usecolortheme{dove}
\setbeamercolor{frametitle}{fg=white}

\usebackgroundtemplate%
{%
\vbox to \paperheight{
\includegraphics[width=\paperwidth]{Pics/hi-slide-head-2016}

\vfill
\hspace{0.5cm}\includegraphics[width=0.3\paperwidth]{Pics/hi-von-logo}
\vspace{0.4cm}
    }%
}

\AtBeginSection[]
{
  \begin{frame}<beamer>
    \frametitle{Yfirlit}
    \tableofcontents[currentsection]
  \end{frame}
}

\setbeamerfont{frametitle}{size=\normalsize}
\addtobeamertemplate{frametitle}{}{\vspace*{0.5cm}}

%%%%%%%%%%%%%%%%%%%%%%%%%
% tcolorbox configuration
%%%%%%%%%%%%%%%%%%%%%%%%%

% Setup from: http://tex.stackexchange.com/a/43329/21638
\tcbset{%
    noparskip,
    colback=gray!10, %background color of the box
    colframe=gray!40, %color of frame and title background
    coltext=black, %color of body text
    coltitle=black, %color of title text 
    fonttitle=\bfseries,
    alerted/.style={coltitle=red, colframe=gray!40},
    example/.style={coltitle=black, colframe=green!20, colback=green!5},
}


% Shortcuts
\newcommand{\Mod}[1]{\ \text{mod}\ #1}
\newcommand{\eng}[1]{(e.\ \emph{#1})}
\newmintedfile[cppfile]{cpp}{frame=lines, linenos=true}
\newmintedfile[javafile]{java}{frame=lines, linenos=true}


%%%%%%%%%%%%%%%%%%%%%%%
% Further configuration
%%%%%%%%%%%%%%%%%%%%%%%
\hypersetup{colorlinks=true,pdfauthor={Eirikur Ernir Thorsteinsson},linkcolor=blue,urlcolor=blue}
\graphicspath{{./Pics/}}


\author{Eiríkur Ernir Þorsteinsson}
\institute{Háskóli Íslands}
\date{Vor 2017}

\title{Tölvunarfræði 2}
\subtitle{Vika 10}

\begin{document}

\begin{frame}
\titlepage
\end{frame}

\section{Inngangur}

\begin{frame}{Í þessum tíma}
\begin{itemize}
 \item Skoðum net!
 \item Notum glærur Sedgewick \& Wayne stíft:
 \begin{itemize}
  \item \url{http://algs4.cs.princeton.edu/lectures/41UndirectedGraphs.pdf}
  \item \url{http://algs4.cs.princeton.edu/lectures/42DirectedGraphs.pdf}
 \end{itemize}
\end{itemize}
\end{frame}

\section{Óstefnd net}

\section{Stefnd net}

\section{Leit í netum}

\section{Lok}

\begin{frame}{Þessi glærupakki}
Tengill á fyrirlestraræfingu: \url{https://goo.gl/forms/XZpf2C6DcNE3H4Cb2}
\vspace{1cm}

Kóða fyrir algs4 reiknirit má finna á \url{http://algs4.cs.princeton.edu/code/}.

Gagnaskrár má finna á \href{https://github.com/Ernir/kennsluefni/tree/master/T2/Code/w10}{Github} og algs4-data.zip.
\end{frame}

\begin{frame}{Næst}
Spanntré, vegin net
\end{frame}

\end{document}
