\documentclass[handout]{beamer}

\usepackage{Vor2017glærur}

\title{Tölvunarfræði 2}
\subtitle{Vika 1}

\begin{document}

\begin{frame}
\titlepage
\end{frame}

\section{Um námskeiðið}

\begin{frame}{Kennari}
Upplýsingaskjal um námskeiðið má finna á Uglu!
\end{frame}

\section{C++ og þýðing}

\begin{frame}{C++}
\begin{itemize}
 \item C++ er almennt forritunarmál, notað á ýmsum sviðum
 \item Yfirleitt litið á það sem ``mid-level'' forritunarmál í dag
 \begin{itemize}
  \item Höfum ýmis þægindi sem við búumst við af nútímaforritunarmálum
  \item Höfum líka ýmsa möguleika á beinni minnisstjórnun
 \end{itemize}
 \item Kom út snemma á 9. áratugnum
\end{itemize}
\end{frame}

\begin{frame}{Af hverju C++?}
\begin{itemize}
 \item Mörg mikilvæg tölvunarfræðihugtök sem eru vel sýnileg í C++
 \item Málið hefur verið mjög áhrifamikið í málum sem á eftir koma
 \item Praktískt, líklegt að fólk sem vinnur við forritun muni rekast á C-mál
\end{itemize}
\end{frame}


\begin{frame}[fragile]{Hello World forrit í C++}
Hið hefðbundna forrit sem skrifar út ``halló heimur'', í C++:
\cppfile[label=hello.cpp]{Code/w1/hello.cpp}
Þessa skrá mætti búa til í hvaða textaritli (e. \emph{plaintext editor}) sem er.
\end{frame}

\begin{frame}[fragile]{Hvað erum við að horfa á?}
\begin{itemize}
 \item \texttt{\#include <iostream>}: Veitir okkur aðgang að staðalstraumum
 \item \texttt{int main()}: Skilgreining falls að nafni \texttt{main} sem tekur engin inntök og skilar heiltölu
 \begin{itemize}
  \item Afmarkað með slaufusvigum
 \end{itemize}
 \item \verb|std::cout << "Halló heimur!" << std::endl;|: Strengurinn ``halló heimur'' skrifaður á ``console out'' strauminn með straumsinnsetningarvirkjanum \verb|<<| ásamt boðum um að línunni sé lokið
 \item \texttt{return 0;}: Heiltölunni 0 skilað, sem þýðir að fallið hafi lokið keyrslu rétt
\end{itemize}
\end{frame}

\begin{frame}[fragile]{Þýðing C++ forrits}
C++ kóði er undantekningalítið þýddur. Dæmi um þýðanda er \href{https://gcc.gnu.org/}{GCC}, sem kalla má á af skipanalínu:
\begin{minted}[frame=lines]{bash}
$ g++ hello.cpp -o hello
$ ./hello
Halló heimur!
\end{minted}
Hér er \texttt{g++} skipunin sem keyrir upp GCC þýðandann fyrir forritskóðaskrána \texttt{hello.cpp} og býr til keyranlegu skrána \texttt{hello}.
\end{frame}

\begin{frame}{Undir húddinu}
\begin{columns}
\column{0.66\textwidth}
Það að mynda keyrsluskrá út frá C++ kóða fer fram í nokkrum (hér einfölduðum) skrefum.
\begin{enumerate}
 \item Forritskóðinn fer í gegnum forþýðanda (e. \emph{preprocessor}), sem sækir \texttt{\#include}-aðan kóða, framkvæmir textaútskiptingar o.fl.
 \item Útvíkkaði kóðinn er þýddur (e. \emph{compiled})
 \item Þýddi kóðinn er tengdur (e. \emph{linked}) svo úr verði keyranleg skrá
\end{enumerate}
\column{0.33\textwidth}
\begin{center}
\includegraphics[width=1.1\linewidth]{compile}

{\tiny \href{http://faculty.cs.niu.edu/~mcmahon/CS241/Notes/compile.html}{Útskýring/uppruni myndar} }
\end{center}
\end{columns}
\end{frame}

\begin{frame}[fragile]{Að skoða milliskrár}
Við getum stöðvað GCC á ýmsum stigum.

Stöðvað eftir forþýðingu, skrifar oft út mikinn textavegg:
\begin{minted}[frame=lines]{bash}
$ g++ -E p.cpp
\end{minted}
Fá smalamálsframsetningu sem þýðandin býr til í skrána \texttt{p.s}:
\begin{minted}[frame=lines]{bash}
$ g++ -S p.cpp
\end{minted}
Sleppa tengingu, fá ``object code'' í skrána \texttt{p.o}:
\begin{minted}[frame=lines]{bash}
$ g++ -c p.cpp
\end{minted}
\end{frame}



\begin{frame}[fragile]{Hvar er þessi þýðandi keyrður?}
\begin{itemize}
 \item Við munum kynnast skipanalínunni
 \begin{itemize}
  \item Á Windows: keyra cmd.exe
  \item Á Mökkum: finna terminal
  \item Á Linux: Ýmis nöfn, en á að vera auðfinnanlegt
 \end{itemize}
 \item Hægt er að athuga hvort að GCC sé uppsett með því að keyra
\end{itemize}
\begin{minted}[frame=lines]{bash}
$ g++ --version
g++ (Ubuntu 5.4.0-6ubuntu1~16.04.4) 5.4.0 20160609
\end{minted}

\end{frame}


\begin{frame}{Uppsetning GCC}
\begin{itemize}
 \item Windows (bein uppsetning):
 \begin{itemize}
  \item Setja upp MinGW eða MinGW-w64
  \item Síðan þarf að passa að GCC sé í PATH
 \end{itemize}
 \item Windows (cygwin)
 \begin{itemize}
  \item Setja upp Cygwin með \texttt{gcc-core} og \texttt{gcc-g++} pökkunum
 \end{itemize}
 \item Makkar: 
 \begin{itemize}
  \item Setja upp homebrew og ``xcode command line tools''
 \end{itemize}
 \item Linux:
 \begin{itemize}
  \item Það er í pakkakerfinu ef ekki foruppsett
  \item \texttt{sudo apt-get install build-essential} á Ubuntu
 \end{itemize}
 \item Hægt er að komast fram hjá uppsetningu með því að nota (gamla) útgáfu af GCC á \texttt{hekla.rhi.hi.is} í gegnum ssh
\end{itemize}
\end{frame}

\begin{frame}{Ritlar og þróunarumhverfi}
Val á ritli og/eða þróunarumhverfi fyrir C++ skiptir ekki höfuðmáli í þessu námskeiði. Vitað er að \href{https://code.visualstudio.com/}{Microsoft VS Code} (ritill) og \href{https://www.jetbrains.com/student/}{CLion} (þróunarumhverfi) ``virka''.

\begin{center}
\includegraphics[width=\linewidth]{editors}
\end{center}
\end{frame}

\section{Málfræði C++}

\begin{frame}[fragile]{Grundvallarmálfræði}
Mikið af grundvallarmálfræði (en ekki allri virkni!) C++ ætti að vera kunnuglegt þeim sem þekkja til Java.
\begin{columns}
\column{0.6\textwidth}
\cppfile[firstline=5,lastline=10,gobble=4,fontsize=\small, linenos=false]{Code/w1/javacomparison.cpp}
\column{0.4\textwidth}
\cppfile[firstline=13,lastline=15,fontsize=\small, linenos=false]{Code/w1/javacomparison.cpp}
\end{columns}
Leggjum héðan af áherslu á þá hluta sem eru öðru vísi en í Java.
\end{frame}

\begin{frame}{\#include}
\begin{itemize}
 \item Í C++ hefjast skipanir til forþýðandans á \#
 \item Það sem við gerum langoftast er:
 \begin{itemize}
  \item ``Taktu skipanir úr þessari skrá og settu þær hingað''
  \item Höfum þegar séð skipunina \texttt{\#include <iostream>}
 \end{itemize}
 \item \texttt{\#include} vinnur á hausskrám (e. \emph{header files}), sem innihalda eingöngu forritslýsingar
 \item Útfærslan á þeim forritum sem er síðan ``annars staðar'' og er tengdar seinna
\end{itemize}
\end{frame}

\begin{frame}{Nafnarými}
\texttt{cout} skipunin (og mun fleiri) eru í \texttt{std} nafnarýminu (e. \emph{namespace}). Við getum sagt þýðandanum að leita í ákveðnum nafnarýmum:
\cppfile[label=namespace.cpp]{Code/w1/namespace.cpp}
\end{frame}

\begin{frame}{Nákvæmari innflutningur}
\cppfile[label=namespacespecific.cpp]{Code/w1/namespacespecific.cpp}
\end{frame}

\section{Straumar, inntak og úttak}

\begin{frame}{Inntak og úttak}
Í C++ eru straumarnir mjög sýnilegir.
\cppfile[label=input.cpp, firstline=7, lastline=17, gobble=4, fontsize=\small]{Code/w1/input.cpp}
\end{frame}

\begin{frame}[fragile]{Skipanalínan}
Skipanalína og straumar virka vel saman.
\begin{columns}
\column{0.3\textwidth}
\inputminted[frame=lines,label=numbers.txt]{bash}{Code/w1/numbers.txt}
\column{0.6\textwidth}
\cppfile[firstline=7, lastline=12, gobble=4, label=loopyinput.cpp]{Code/w1/loopyinput.cpp}
\end{columns}
\begin{minted}[frame=lines]{bash}
$ cat numbers.txt | ./loopyinput 
29
\end{minted}
\end{frame}

\begin{frame}[fragile]{Skilatáknið}
Skilatákn (e. \emph{exit code}) forrita sem keyra á skipanalínunni eru aðgengilegt á skipanalínunni. Í bash (Linux skel) er síðasta skilatákn geymt í breytunni \texttt{\$?}.

\begin{minted}[frame=lines]{bash}
$ g++ hello.cpp -o hello
$ echo $?
0
\end{minted}
\end{frame}

\section{Inngangur að fylkjum og strengjum}

\begin{frame}[fragile]{Fylki}
Líkt og í Java eru venjuleg fylki af fastri stærð. Þýðandinn tekur frá minnisblokk af ákveðinni stærð fyrir fylki og notar fjarlægðina frá upphafi blokkarinnar til að vísa í stök fylkisins.

Dæmi um fylki:
\begin{minted}[frame=lines]{cpp}
int numbers1[5] = {}; // initializes all integers to 0
int numbers2[5] = {34, 56, -21, 5002, 365};
int numbers3[] = {2016, 2052, -525}; // 3 elements
\end{minted}
Vísar í fylki í C++ byrja í 0.

\end{frame}

\begin{frame}{Varúð!}
C++ forrit þýðast og keyrast þó vísað sé út fyrir fylki.
\cppfile[firstline=2, label=arraydanger.cpp]{Code/w1/arraydanger.cpp}
\end{frame}

\begin{frame}[fragile]{Oftast betri kostur}
\texttt{vector} gagnagerðin í staðalsafninu er oftast þægilegri - býður meðal annars upp á að stækka vigurinn.
\cppfile[firstline=7, lastline=17, fontsize=\small, label=vectorexample.cpp]{Code/w1/vectorexample.cpp}
\end{frame}

\begin{frame}{Strengir}
C++ getur unnið með strengi eins og þeir eru í forritunarmálinu C - fylki af táknum. Strengirnir sem við höfum séð hingað til hafa flestir verið af slíkri gerð.
\cppfile[firstline=6, lastline=12, fontsize=\small, label=nullterm.cpp]{Code/w1/nullterm.cpp}
Strengjunum er ``lokað'' með sértákninu \texttt{'\textbackslash0'}. Strengjavinnsla hættir á þeim stað.
\end{frame}

\begin{frame}{Strengir}
Strengir úr C++ staðalsafninu eru skilvirkir og hafa ýmsa þægilega eiginleika.
\cppfile[firstline=15, lastline=17, gobble=4, fontsize=\small, label=stdstring.cpp]{Code/w1/stdstring.cpp}

\end{frame}


\section{Inngangur að bendum}

\begin{frame}{Bendar}
\begin{itemize}
 \item Bendir er breyta sem inniheldur staðsetningu minnissvæðis
 \begin{itemize}
  \item Hún ``bendir á'' minnissvæðið
 \end{itemize}
 \item Hægt er að nota bendinn til að fá aðgang að gögnunum sem geymd eru í minnissvæðinu
\end{itemize}
\begin{center}
\includegraphics[width=\textwidth]{pointer-visualization}
\end{center}
\end{frame}

\begin{frame}[fragile]{Bendar}
Við búum til bendi sem vísar á breytu af gerðinni \texttt{gerd} með \texttt{gerd*}. Við náum í staðsetningu minnissvæðis með \texttt{\&} virkjanum.
\cppfile[firstline=6, lastline=10, gobble=4, fontsize=\small, label=pointerintro.cpp]{Code/w1/pointerintro.cpp}
\end{frame}

\begin{frame}[fragile]{Bendar}
Við getum sótt gögn sem bendir vísar á með \texttt{*} virkjanum.
\cppfile[firstline=6, lastline=15, gobble=4, fontsize=\small, label=dereferencing.cpp]{Code/w1/dereferencing.cpp}
\end{frame}


\section{Lokaorð}

\begin{frame}{Þessi glærupakki}
Tengill á fyrirlestraræfingu: \url{https://goo.gl/forms/qaB2Hoxwd6PFEN9I2}
\vspace{1cm}

Öll nafngreind forrit í þessum glærupakka, ásamt glærupakkanum sjálfum, má finna á  \href{https://github.com/Ernir/kennsluefni/tree/master/T2/Code/w1}{Github}.

\end{frame}


\begin{frame}{Næst}
Meira um minnismódelið í C++, hlutbundin forritun í C++.
\end{frame}


\end{document}
