\documentclass{article}

\input{../Vor2017skil.tex}

\title{Tölvunarfræði 2, \semester \\ Skilaverkefni 10}
\author{}

\begin{document}
\maketitle
\hypersetup{pdftitle={Tölvunarfræði 2 - Skilaverkefni 10}}

\paragraph{Skilavefur} Skila skal þessum verkefnum á \href{https://gradescope.com/courses/5640}{Gradescope}.

\paragraph{Vinnubrögð og frágangur} Skrifa þarf forrit í Java %og C++ 
sem útfæra ýmsar aðferðir. Í öllum tilvikum skal skila \texttt{main} fall/aðferð sem býr til prufugögn og sýnir virkni hverrar aðferðar um sig. Skilið öllum forritskóða og sýnið dæmi um keyrslu. Leitist við að nota beinagrindur óbreyttar þegar þær eru gefnar. Vönduð framsetning og læsilegur kóði er hluti af verkefninu, sjá \href{https://piazza.com/class/ixkicfen49l111?cid=52}{glósu um frágang og framsetningu}.

\paragraph{Samþykki til dreifingar} Dæmatímakennarar velja framúrskarandi lausnir til birtingar undir nafni í lausnasafni. Sé þess óskað að einhverjar þinna úrlausna séu ekki birtar nema nafnlaust eða alls ekki birtar yfir höfuð skal taka slíkt fram í hverju dæmi sem takmörkunin á við.

\section{Spurning 1}
Skráin \href{https://github.com/Ernir/kennsluefni/tree/master/T2/Code/w10/tinyDG.txt}{tinyDG.txt} inniheldur gögn sem nota mætti til að skilgreina stefnt net á hátt sem \texttt{Digraph} klasinn skilur. 

Teiknið netið. Merkið hnútana með númerum og notið örvar til að tákna stefnu leggjanna.

\paragraph{Ábending} Til að sjá grennslalistaframsetningu á netinu, notið skjalið til að búa til tilvik af \texttt{Digraph} klasanum og kallið á \texttt{toString()} aðferðina.

\section{Spurning 2}
Búið til útgáfu af \texttt{Graph} sem leyfir hvorki lykkjur (leggi sem byrja og enda í sama hnút) og né endurtekna leggi. Meðhöndlið endurtekningar á þann hátt að lykkjur valdi villu (t.d. \texttt{IllegalArgumentException}) en endurteknir leggir hafi engin áhrif. 

Nefnið útgáfuna \texttt{SimpleGraph}. Sýnið dæmi um keyrslu þar sem nýja virknin er sýnd.

\section{Spurning 3}
Ein leið til að herma útbreiðslu smitsjúkdóma á milli staða er að nota netaframsetningu. Táknum hverja staðsetningu með hnúti og tengingu þeirra á milli með legg. Þá gætum við t.d. athugað hvað gerist ef sýking kemur upp á flugvelli og breiðist út til tengdra flugvalla.

Klárið forritið \href{https://github.com/Ernir/kennsluefni/tree/master/T2/Code/w10/Epidemic.java}{Epidemic.java}, sem hermir eftir endurteknum faröldrum sem breiðast út frá flugvöllunum í \href{https://github.com/Ernir/kennsluefni/tree/master/T2/Code/w10/routes.txt}{routes.txt}.

Dæmi um mögulega keyrslu:
\begin{verbatim}
Outbreak in JFK!
Infected: ATL, JFK, MCO, ORD
Outbreak in MCO!
Infected: ATL, HOU, JFK, MCO, ORD
Outbreak in HOU!
Infected: ATL, DFW, HOU, JFK, MCO, ORD
Outbreak in HOU!
Infected: ATL, DFW, HOU, JFK, MCO, ORD
Outbreak in ORD!
Infected: ATL, DEN, DFW, HOU, JFK, MCO, ORD, PHX
Outbreak in DFW!
Infected: ATL, DEN, DFW, HOU, JFK, MCO, ORD, PHX
Outbreak in HOU!
Infected: ATL, DEN, DFW, HOU, JFK, MCO, ORD, PHX
Outbreak in PHX!
Infected: ATL, DEN, DFW, HOU, JFK, LAS, LAX, MCO, ORD, PHX
All is lost.
\end{verbatim}
Í hverri ítrun er sýktur flugvöllur valinn af handahófi til að breiða út sjúkdóminn. Hver keyrsla getur því verið mismunandi, en henni lýkur ekki á annan hátt en að allir flugvellirnir séu sýktir.

\section{Spurning 4}
Það að ákvarða Bacon-tölu ýmissa leikara er skemmtilegur\footnote{frá sjónarhóli netafræðinnar} samkvæmisleikur. Þegar ákvarða skal Bacon tölu fyrir leikara fær leikarinn Kevin Bacon Bacon-töluna 0, leikari sem leikið hefur í mynd með Kevin Bacon fær Bacon-töluna 1, leikari sem leikið hefur í mynd með leikara sem hefur leikið í mynd með Kevin Bacon fær Bacon-töluna 2 (nema viðkomandi leikari eigi rétt á lægri Bacon-tölu) og svo koll af kolli. Sé ekki mögulegt að finna leið á milli leikarans og Kevins Bacons í gegnum sameiginleg leikverk er Bacon-tala viðkomandi leikara óskilgreind.

Skrifið forrit sem leitar í gegnum \href{https://github.com/Ernir/kennsluefni/tree/master/T2/Code/w10/movies.txt}{movies.txt}, finnur Bacon-tölu allra leikaranna sem þar eru taldir upp og skrifar út fjölda leikara sem eru með hverja tölu, á eftirfarandi sniði:

\begin{verbatim}
0: 1
1: 1324
2: 70717
3: 40862
4: 1591
5: 125
621 actors have no defined Bacon number
\end{verbatim}

Sem hér (eðlilega) segir okkur að nákvæmlega einn leikari hafi Bacon-töluna 0, 1324 leikarar Bacon-töluna 1, og svo framvegis.

Nota má hugmyndir í \href{https://github.com/Ernir/kennsluefni/tree/master/T2/Code/w10/SixDegrees.java}{SixDegrees.java}.

% \vfill
% \includegraphics[width=0.5\linewidth]{hi-von-logo}
\end{document}