\documentclass{article}

\usepackage{Vor2018skil}

\title{Tölvunarfræði 2, \semester \\ Skilaverkefni 9}
\author{}

\begin{document}
\maketitle
\hypersetup{pdftitle={Tölvunarfræði 2 - Skilaverkefni 9}}

Skila skal þessum verkefnum á \href{https://gradescope.com/courses/14122}{Gradescope}.

Þegar forriti er skilað inn til yfirferðar er mikilvægt að láta \textbf{niðurstöðurnar fylgja}. Öllum forritskóða skal skila framsettum með jafnbilaletri. Hann þarf að vera afritanlegur úr .pdf skjalinu. Vönduð framsetning og læsilegur kóði er hluti af verkefninu.

\section{Spurning 1}
Teiknið 2-3 tréð sem fæst eftir að lyklarnir \texttt{E A S Y Q U T I O N} eru settir inn í það í röð, væri tréð tómt í upphafi. Látið gildin vera númer lykilsins (í innsetningarröð).

\section{Spurning 2}
Nafnatöflur eru útfærðar í C++ með \href{http://www.cplusplus.com/reference/unordered_map/unordered_map/}{\texttt{std::unordered\_map}} og \href{http://www.cplusplus.com/reference/map/map/}{\texttt{std::map}}. Tilgreint er að uppfletting, innsetning og eyðing í \texttt{std::unordered\_map} skuli vera framkvæmanleg á föstum tíma í meðaltilfellinu, en á logratíma í \texttt{std::map}.

Framkvæmið tímamælingar á eftirfarandi aðgerðum fyrir \emph{hvora gagnagrind um sig}:

\begin{itemize}
 \item Innsetningu þar sem orðin í \href{http://introcs.cs.princeton.edu/java/data/words.txt}{\texttt{words.txt}} eru notuð sem lyklar og sett inn í slembinni röð. Gildin skipta ekki máli, en þægilegt er að nota stígandi heiltölur.
 \item Leit að hverju einasta orði í \texttt{words.txt}
\end{itemize}
Samtals er því um 4 mælingar að ræða. Hægt er að mæla keyrslutíma forrits í C++ með \href{http://www.cplusplus.com/reference/ctime/clock/}{clock}. Beinagrind er ekki gefin, en sjá \href{https://raw.githubusercontent.com/Ernir/kennsluefni/master/T2/Code/w10/fibotime.cpp}{fibotime.cpp} sem dæmi um tímamælingu í C++.

\textbf{Ábending:} Séu fyrstu niðurstöður ekki afgerandi, prófið að keyra hverja mælingu oft og taka meðaltímann.

\section{Spurning 3}

Búið til útgáfu af \texttt{SeparateChainingHashST} sem notar \texttt{RedBlackBST} í stað \texttt{SequentialSearchST} til þess að útkljá árekstra. Nefnið hana \texttt{RedBlackHashST}.

Skrifið líka forrit sem framkvæmir tímamælingu þar sem keyrslutími \texttt{RedBlackHashST} er marktækt betri en keyrslutími \texttt{SeparateChainingHashST}.

\textbf{Ábending:} Þessi hegðun kemur hratt fram séu lyklarnir mismunandi m.t.t. \texttt{compareTo} en eins m.t.t. \texttt{hashCode}.

%\vfill
%\includegraphics[width=0.5\linewidth]{hi-von-logo}
\end{document}