\documentclass{article}

\usepackage{Vor2018skil}

\title{Tölvunarfræði 2, \semester \\ Skilaverkefni 4}
\author{}

\begin{document}
\maketitle
\hypersetup{pdftitle={Tölvunarfræði 2 - Skilaverkefni 4}}

Skila skal þessum verkefnum á \href{https://gradescope.com/courses/14122}{Gradescope}.

Þegar forriti er skilað inn til yfirferðar er mikilvægt að láta \textbf{niðurstöðurnar fylgja}. Öllum forritskóða skal skila framsettum með jafnbilaletri. Hann þarf að vera afritanlegur úr .pdf skjalinu. Vönduð framsetning og læsilegur kóði er hluti af verkefninu.

Athugið að fyrstu tvö verkefni þessarar viku eru verkefni í Java. Verkefnin krefjast þess að forritssafn kennslubókarinnar (algs4.jar) sé sett upp. Sjá \href{http://algs4.cs.princeton.edu/code/}{leiðbeiningar neðarlega á síðu bókarinnar}.

\question

Gefinn er Java-forritið \href{https://raw.githubusercontent.com/Ernir/kennsluefni/master/T2/Code/w5/ArrayList.java}{ArrayList.java} sem líkir að takmörkuðu leyti eftir \href{https://docs.oracle.com/javase/8/docs/api/java/util/ArrayList.html}{java.util.ArrayList}.  Betrumbætið klasann svo hann útfæri \texttt{Iterable}.

Ekki breyta \texttt{main} aðferðinni nema þar sem tekið er fram.

Keyrsludæmi:

\begin{verbatim}
$ java ArrayList
[ B A N A N I ]
\end{verbatim}

\question
Turninn í Hanoi (e. \emph{Tower of Hanoi}) er stærðfræðiþraut sem gengur út á að færa skífur á milli þriggja hlaða. Sjá \href{https://www.tutorialspoint.com/data_structures_algorithms/images/tower_of_hanoi.gif}{hreyfimynd}.

Í byrjun eru $n$ skífur á einum hlaðanum í stærðarröð, stærsta skífan neðst. Þrautin er unnin þegar allar skífurnar hafa verið færðar frá fyrsta hlaðanum yfir á annan hlaða, en fylgja þarf eftirfarandi reglum þegar skífurnar eru færðar:
\begin{enumerate}
	\item Einungis má færa eina skífu í einu
	\item Þegar skífa er færð er hún tekinn efst af einum af hlaðanum og færð efst á annan hlaða
	\item Aldrei má setja skífu ofan á minni skífu
\end{enumerate}

Skrifið forrit í Java sem skrifar út öll skref sem þarf til að leysa Turninn í Hanoi.

Tákna skal hlaðana þrjá með tilvikum af klasanum \texttt{Stack} úr \texttt{algs4.jar}. Notið beinagrindina í \href{https://raw.githubusercontent.com/Ernir/kennsluefni/master/T2/Code/w5/Hanoi.java}{Hanoi.java}. Notið tilviksbreyturnar sem eru gefnar og ekki breyta \texttt{main} fallinu eða smiðnum.

Forritið ætti (fræðilega) að geta leyst þrautina fyrir hvaða fjölda skrefa sem er. Fjöldi skrefa sem þarf til að leysa þrautina vex þó gríðarlega hratt, mælt er gegn því að bíða eftir keyrslu fyrir stærri tilvik.

Keyrsludæmi:

\begin{verbatim}
$ java Hanoi
Vinstri : 1 2 3 
Miðja   : 
Hægri   : 
#####################
Vinstri : 2 3 
Miðja   : 
Hægri   : 1 
#####################
Vinstri : 3 
Miðja   : 2 
Hægri   : 1 
#####################
Vinstri : 3 
Miðja   : 1 2 
Hægri   : 
#####################
Vinstri : 
Miðja   : 1 2 
Hægri   : 3 
#####################
Vinstri : 1 
Miðja   : 2 
Hægri   : 3 
#####################
Vinstri : 1 
Miðja   : 
Hægri   : 2 3 
#####################
Vinstri : 
Miðja   : 
Hægri   : 1 2 3 
#####################    
\end{verbatim}

\newpage

\question
Skrifið klasa í C++ sem uppfyllir eftirfarandi skil fyrir hlaða heiltalna:

\begin{center}
	\begin{tabularx}{\textwidth}{rlX}
		\toprule
		\multicolumn{3}{c}{\texttt{public class Stack}}                                                                                        \\
		\midrule
		-             & \texttt{Stack()}     & Smiður, upphafsstillir tóman hlaða                                                              \\
		\texttt{void} & \texttt{push(int n)} & Bætir heiltölunni n efst á hlaðann                                                              \\
		\texttt{int}  & \texttt{pop()}       & Fjarlægir þá tölu sem styst hefur verið á hlaðanum og skilar henni                              \\
		\texttt{int}  & \texttt{peek()}      & Skilar gildi þeirrar tölu sem styst hefur verið á hlaðanum en skilur við hlaðann í sama ástandi \\
		\bottomrule
	\end{tabularx}
\end{center}

Notið eintengdan lista til að geyma gögn hlaðans. Útfærið eyði við hæfi. Ekki breyta \texttt{main} aðferðinni eða hausum aðferðanna sem eru gefnar í beinagrindinni \href{https://raw.githubusercontent.com/Ernir/kennsluefni/master/T2/Code/w5/Stack.cpp}{Stack.cpp}.

Dæmi um keyrslu:

\begin{verbatim}
$ ./stack
Setjum tölurnar 1, 2 og 4 á hlaðann s.
Stelum efsta stakinu, 4, af s.
Setjum tölurnar 8, 16 og 32 á s.
Kíkjum á efsta stak s, það er 32.
Fjarlægjum nú tölurnar af s:
32 16 8 2 1
Ekki tókst að kíkja á s, enda tómur
Ekki tókst að fjarlægja stak af s, enda tómur
\end{verbatim}

\vfill
\includegraphics[width=0.5\linewidth]{hi-von-logo}
\end{document}