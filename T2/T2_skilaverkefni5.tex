\documentclass{article}

\usepackage{Vor2018skil}

\title{Tölvunarfræði 2, \semester \\ Skilaverkefni 5}
\author{}

\hyphenation{StdRandom}

\begin{document}
\maketitle
\hypersetup{pdftitle={Tölvunarfræði 2 - Skilaverkefni 5}}

Skila skal þessum verkefnum á \href{https://gradescope.com/courses/14122}{Gradescope}.

Þegar forriti er skilað inn til yfirferðar er mikilvægt að láta \textbf{niðurstöðurnar fylgja}. Öllum forritskóða skal skila framsettum með jafnbilaletri. Hann þarf að vera afritanlegur úr .pdf skjalinu. Vönduð framsetning og læsilegur kóði er hluti af verkefninu.

Athugið að verkefni þessarar viku eru í Java. Verkefnin krefjast þess að forritssafn kennslubókarinnar (algs4.jar) sé sett upp. Sjá \href{http://algs4.cs.princeton.edu/code/}{leiðbeiningar neðarlega á síðu bókarinnar}.

Verkefnin eru \emph{ekki óháð} í þetta skiptið. Mælt er með að leysa þau í röð. Byrjið snemma og leitið hjálpar ef þið lendið í vandræðum. Þegar forrit í seinni verkefnunum notar forrit sem var útfært fyrir fyrra verkefni þarf ekki að láta það fylgja með þegar seinna verkefninu er skilað til yfirferðar.

\question

Til að prófa forrit þurfum við oft að nota slembitölur. Í algs4 safninu er klasinn \href{https://algs4.cs.princeton.edu/code/edu/princeton/cs/algs4/StdRandom.java.html}{StdRandom} sem inniheldur gagnlegar hjálparaðferðir til að búa til slembitölur.

Eitt vantar okkur þó - aðferð til að búa til slembið fylki af \texttt{Double} tölum.

Skrifið slíka aðferð í samræmi við lýsingu í \href{https://raw.githubusercontent.com/Ernir/kennsluefni/master/T2/Code/w6/MoreRandom.java}{beinagrind}. Ekki breyta \texttt{main} aðferðinni sem er gefin eða haus aðferðarinnar.

Útskrift forritsins ætti ekki að breytast á milli keyrslna og vera á eftirfarandi formi:

\begin{verbatim}
$ java MoreRandom
0.736073918590018 0.5765058389895272 0.13046441762918415
\end{verbatim}

\question

Skrifið forrit sem framkvæmir endurteknar mælingar á keyrslutíma innsetningarröðunar (úr algs4) til að sannfærast um að aðferðin sé í alvörunni $O(n^2)$ reiknirit. Örlítil \href{https://raw.githubusercontent.com/Ernir/kennsluefni/master/T2/Code/w6/PlotRunningTimes.java}{beinagrind} er gefin.

Mælið keyrslutíma reikniritsins á fylkjum sem innihalda frá 0 upp í 5000 slembnar fleytitölur. Prófið stærðirnar $0, 10, 20, 30, \ldots , 5000$. Til að fá gott mat á keyrslutímana skuluð þið framkvæma röðunina 20 sinnum fyrir hverja stærð á fylki og reikna meðalkeyrslutímann.

Teiknið að lokum niðurstöðurnar upp sem safn af punktum, með stærð fylkjanna á lárétta ásnum og meðalkeyrslutímann fyrir fylki af þeirri stærð á lóðrétta ásnum. Niðurstaðan ætti að vera svipuð og á mynd \ref{mynd:1}.

\begin{figure}[h]
	\caption{Keyrslutími innsetningarröðunar eftir inntaksstærð. Ef vel er að gáð má sjá fleygboga.}
	\label{mynd:1}
	\begin{center}
		\includegraphics[width=0.5\textwidth]{insertion-sort-times}
	\end{center}
\end{figure}

\paragraph{Ábendingar}

\begin{enumerate}
	\item Gert er ráð fyrir að grunnnotkun á \texttt{StdDraw} klasanum sé þekkt. Sé svo ekki, athugið umfjöllun um teikniaðferðir sem hefst á bls. 42, sérstaklega fyrsta sýnidæmið á bls. 45.
    \item Fjallað er um útfærslur röðunarreikniritanna í \texttt{algs4} og hvernig má nota þau í köflum 2.1 og 2.2 í Algorithms.
    \item Látið slembitölugjafann vinna út frá föstu ``seed''. Sjá verkefni 1.
	\item Hægt er að nota \texttt{System.nanoTime()} eða \texttt{StopWatchCPU} úr \texttt{algs4} til að mæla tíma.
	\item Forritið tók rúma mínútu í keyrslu á tölvu kennarans. Gagnlegt er að nota aðrar tölur (t.d. lægra \texttt{maxArraySize} og hærra \texttt{stride}) meðan á forritun stendur.
\end{enumerate}

\question

Því er oft haldið fram að innsetningarröðun sé skilvirk fyrir lítil fylki en dragist svo afturúr. Staðfestið þetta með tilraunum.

Tímamælið annars vegar innsetningarröðun og hins vegar sameiningarröðun (úr algs4) á 10 staka slembnu fleytitölufylki 50000 sinnum og reiknið meðalkeyrslutímana. Innsetningarröðun ætti þá að vera hraðvirkari.

Hækkið svo stærð fylkisins um 1 og athugið aftur hvort innsetningarröðun sé hraðvirkari. Haldið áfram að hækka þar til sameiningarröðun tekur fram úr eða þar til komið er upp í fylki af stærðinni 1000.

Gefin er örlítil \href{https://raw.githubusercontent.com/Ernir/kennsluefni/master/T2/Code/w6/SortingCompetition.java}{beinagrind}. Skilið niðurstöðu tilraunanna.

\paragraph{Ábending:} Látið slembitölugjafann vinna út frá föstu ``seed''. Sjá verkefni 1.
%\vfill
%\includegraphics[width=0.5\linewidth]{hi-von-logo}
\end{document}