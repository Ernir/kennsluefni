\documentclass{beamer}

\input{../Vor2017glærur}

\title{Tölvunarfræði 2}
\subtitle{Vika 10}

\begin{document}

\begin{frame}
\titlepage
\end{frame}

\section{Inngangur}

\begin{frame}{Í þessum tíma}
\begin{itemize}
 \item Skoðum net!
 \item Notum glærur Sedgewick \& Wayne stíft:
 \begin{itemize}
  \item \url{http://algs4.cs.princeton.edu/lectures/41UndirectedGraphs.pdf}
  \item \url{http://algs4.cs.princeton.edu/lectures/42DirectedGraphs.pdf}
 \end{itemize}
\end{itemize}
\end{frame}

\section{Óstefnd net}

\section{Stefnd net}

\section{Leit í netum}

\section{Lok}

\begin{frame}{Þessi glærupakki}
Tengill á fyrirlestraræfingu: \url{https://goo.gl/forms/XZpf2C6DcNE3H4Cb2}
\vspace{1cm}

Kóða fyrir algs4 reiknirit má finna á \url{http://algs4.cs.princeton.edu/code/}.

Gagnaskrár má finna á \href{https://github.com/Ernir/kennsluefni/tree/master/T2/Code/w10}{Github} og algs4-data.zip.
\end{frame}

\begin{frame}{Næst}
Spanntré, vegin net
\end{frame}

\end{document}
