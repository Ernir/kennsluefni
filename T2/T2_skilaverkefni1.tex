\documentclass{article}

\usepackage{Vor2018skil}

\title{Tölvunarfræði 2, \semester \\ Skilaverkefni 1}
\author{}

\begin{document}
\maketitle
\hypersetup{pdftitle={Tölvunarfræði 2 - Skilaverkefni 1}}

Skila skal þessum verkefnum á \href{https://gradescope.com/courses/14122}{Gradescope}.

Þegar forriti er skilað inn til yfirferðar er mikilvægt að láta \textbf{niðurstöðurnar fylgja}. Öllum forritskóða skal skila framsettum með jafnbilaletri. Hann þarf að vera afritanlegur úr .pdf skjalinu. Vönduð framsetning og læsilegur kóði er hluti af verkefninu.

\question

Skrifið C++ fallið \texttt{swap} sem tekur inn tvo benda á heiltölur og víxlar á gildunum sem bendarnir benda á.

Beinagrind að verkefninu má finna í \href{https://raw.githubusercontent.com/Ernir/kennsluefni/master/T2/Code/w2/pointerswap.cpp}{pointerswap.cpp}. Ekki breyta \texttt{main} fallinu eða haus \texttt{swap} fallsins. Þegar forritið er keyrt ætti útskriftin að vera:

\begin{verbatim}
*pointsToA = 2
*pointsToB = 3
Skiptum á gildum
*pointsToA = 3
*pointsToB = 2    
\end{verbatim}

\question 

Gefinn er \href{https://raw.githubusercontent.com/Ernir/kennsluefni/master/T2/Code/w2/fraction.cpp}{vísir að klasanum \texttt{Fraction}}, sem táknar almenn brot. Teljari brots er táknaður með eiginleikanum \texttt{num} og nefnari þess með eiginleikanum \texttt{den}.

Endurbætið klasann á eftirfarandi vegu:

\begin{enumerate}[a)]
    \item Látið smið klasans skrifa villu á \texttt{cerr} sé reynt að skilgreina nefnarann sem 0.
    \item Klárið \texttt{display} aðferð klasans svo að hún skrifi brot út á sniðinu \texttt{teljari/nefnari}. Hún þarf ekki að ráða við fleiri jaðartilvik en birtast í keyrsludæminu að neðan.
    \item Klárið \texttt{plus} aðferð klasans svo að hún geti lagt saman tvö almenn brot.
    \item Klárið \texttt{reduce} aðferð klasans svo að hún noti útreikninga á stærsta samdeili (e. \emph{greatest common divisor}) til að búa til stytta útgáfu af broti. Hægt er að reikna samdeili tveggja talna með \href{https://en.wikipedia.org/wiki/Euclidean_algorithm}{reikniriti Evklíðs}. (Dæmi: $\frac{8}{12} = \frac{8/gcd(12,8)}{12/gcd(12,8)} = \frac{8/4}{12/4} = \frac{2}{3}$)
\end{enumerate}

Ekki breyta \texttt{main} fallinu eða hausum fallanna í klasanum. Þegar forritið er keyrt ætti útskriftin að vera:

\newpage

\begin{verbatim}
Leggjum saman brotin 1/2 og 1/3 og birtum niðurstöðuna:
5/6

Búum til og birtum brotið 4/6:
4/6
Styttum brotið 4/6 og birtum það aftur:
2/3

Búum til brot með 0 í nefnara:
Warning, zero denominator!
\end{verbatim}


\paragraph{Athugasemd:} Í C++ er reyndar hægt að gera betur en að skilgreina bara aðferðir sem heita \texttt{plus} og \texttt{display} fyrir klasa og nota þær eins og hverja aðra aðferð. Sjá fjölbindingu virkja í næstu viku.

\question 

Skrifið tvö föll sem vinna með hnúta í eintengdum lista \eng{singly linked list} af heiltölum.

\begin{itemize}
    \item \texttt{length}: Tekur inn bendi á hnút. Skilar fjölda hnúta í listanum sem hefst í hnútnum.
    \item \texttt{search}: Tekur inn bendi á hnút ásamt heiltölu. Sé heiltalan í listanum sem hefst í hnútnum skal fallið skila bendi á fyrsta hnútinn sem inniheldur þá tölu, annars skal það skila núllbendinum.
\end{itemize}

Beinagrind að verkefninu má finna í \href{https://raw.githubusercontent.com/Ernir/kennsluefni/master/T2/Code/w2/singlylinked.cpp}{singlylinked.cpp}. Ekki breyta \texttt{main} fallinu eða haus fallsstúfanna. Þegar forritið er keyrt á útskriftin að vera:

\begin{verbatim}
Lengd listans er:                    6
Lengdin sem lengdarfallið skilar er: 6

Lengd tóma listans er:               0
Lengdin sem lengdarfallið skilar er: 0

Minnissvæðið sem inniheldur hnút með gögnin "3" er:        0x55672a8aac20
Minnissvæðið sem leitaraðferðin finnur með leit að "3" er: 0x55672a8aac20

Ekkert minnissvæði inniheldur hnút með gögnin "6".
Minnissvæðið sem leitaraðferðin finnur með leit að "6" er: 0
\end{verbatim}

fyrir utan númer minnissvæðisins.

\paragraph{Ábending:} Við tökum hér örlítið forskot á sæluna. Hægt er að lesa meira um eintengda lista í fyrsta kafla \emph{Algorithms} kennslubókarinnar.

\vfill
\includegraphics[width=0.5\linewidth]{hi-von-logo}
\end{document}