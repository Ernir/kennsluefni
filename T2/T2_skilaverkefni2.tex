\documentclass{article}

\usepackage{Vor2017skil}

\title{Tölvunarfræði 2, \semester \\ Skilaverkefni 2}
\author{}

\begin{document}
\maketitle
\hypersetup{pdftitle={Tölvunarfræði 2 - Skilaverkefni 2}}

Skila skal þessum verkefnum á \href{https://gradescope.com/courses/5640}{Gradescope}. Öllum forritskóða skal skila framsettum með jafnbilaletri. Vönduð framsetning og læsilegur kóði er hluti af verkefninu.

\section{Spurning 1}
Skrifið C++ fall \texttt{readToArray} sem tekur tölu $n$ sem inntaksbreytu, les inn $n$ tölur af \texttt{std::cin} og geymir þær í ``gamaldags'' heiltölufylki. Fallið skal skila bendi á fyrsta stak fylkisins.

Skrifið síðan C++ fall \texttt{readToVector} sem gerir það sama, en les inn í og skilar \texttt{std::vector} breytu.

Skrifið að lokum \texttt{main} fall sem kallar á bæði föllin með $n = 3$ og skrifar út summu allra talnanna (6 alls). Gætið þess að öllu minni sé skilað. Sýnið dæmi um keyrslu.

\section{Spurning 2}
Eintengdur listi (e. \emph{singly linked list}) er einföld gagnagerð. Hún samanstendur af hnútum sem innihalda gögn og tengingum þeirra á milli. Tengingin er í eina átt - hver hnútur vísar á nákvæmlega einn annan hnút og tengingin myndar aldrei rás.

Við getum búið til eintengdan lista í C++ með því að skilgreina klasa sem táknar einn hnút. Eiginleikar klasans eru þá annars vegar gögn (hér heiltölur) og hins vegar bendir á næsta hnút:

\begin{minted}[frame=lines]{cpp}
class Node {
  public:
    
    Node* next;
    int data;
    
    Node(int data, Node* next) {
      this->data = data;
      this->next = next;
    }    
};
\end{minted}

Hér er síðasti hnúturinn í listanum látinn benda á tómt gildi. Við köllum þann hnút sem enginn hnútur bendir á ``haus'' listans.

En til að gagnlegt sé að geyma eitthvað í listanum er nauðsynlegt að útfæra fyrir hann aðgerðir. Klárið að útfæra eftirfarandi:

\begin{verbatim}
int length(Node* head) {

}

Node* search(Node* head, int n) {

}
\end{verbatim}

Hér skal \texttt{length} taka inn bendi á haus og skila fjölda hnúta í listanum. \texttt{search} skal taka inn bendi á haus og heiltölu $n$, og skila bendi á hnút sem inniheldur þá tölu.

Sýnið dæmi um keyrslu á föllunum.

\section{Spurning 3}

Skrifið fjóra C++ klasa þar sem erfðir koma við sögu. Klasarnir eru: 

\begin{itemize}
 \item \texttt{Shape}. Shape hefur \texttt{virtual} aðferðina \texttt{area} til að reikna út flatarmál.
 \item \texttt{Circle}, sem erfir frá \texttt{Shape} og hefur að auki eiginleikann \texttt{radius}. Smiður \texttt{Circle} skal taka inn eina breytu, radíusinn.
 \item \texttt{Rectangle}, sem erfir frá \texttt{Shape} og hefur að auki eiginleikana \texttt{height} og \texttt{width}. Smiður \texttt{Rectangle} skal taka inn tvær breytur, hæð og breidd.
 \item \texttt{Square}, sem erfir frá \texttt{Rectangle}. Smiður \texttt{Square} skal taka inn eina breytu, hliðarlengd.
\end{itemize}
Sýnið dæmi um keyrslu þar sem tilvik af \texttt{Circle}, \texttt{Rectangle} og \texttt{Square} eru búin til og flatarmál þeirra skrifuð út, með útskýringum.

\newpage

\section{Spurning 4}
Tvinntölur (e. \emph{complex numbers}) eru gagnleg útvíkkun á rauntalnahugtakinu. Tvinntölur fást með því að skilgreina stærðina $i$ svo að $i^2 = -1$. Þá má setja fram tvinntölu $z$ sem $z = a+bi$ þar sem $a$ og $b$ eru rauntölur.

Ýmislegt skemmtilegt gildir um tvinntölur, t.d. það að allar margliður með tvinntölustuðla eigi sér tvinntölurót, en við höfum aðallega áhuga á að geta reiknað með þeim í C++.

Skrifið klasann \texttt{Complex} sem bætir tvinntölustuðningi við C++. Tvinntöluklasinn skal styðja eftirfarandi aðgerðir:

\begin{itemize}
 \item Samlagningu tveggja tvinntalna. Samlagning tvinntalna $z = a + bi$ og $w = c + di$ skilgreinist af $z + w = (a+c) + (b+d)i$. Samlagning skal vera studd með fjölbindingu á \texttt{+} virkjanum.
 \item Frádrátt einnar tvinntölu frá annarri. Frádráttur tvinntalna $z = a + bi$ og $w = c + di$ skilgreinist af $z - w = (a-c) + (b-d)i$. Frádráttur skal vera studdur með fjölbindingu á \texttt{-} virkjanum.
 \item Margföldun tveggja tvinntalna. Margföldun tvinntalna $z = a + bi$ og $w = c + di$ skilgreinist af $zw = (ac -bd) + (bc+ad)i$.
 Margföldun skal vera studd með fjölbindingu á \texttt{*} virkjanum.
 \item Útreikning samokatölu. Samokatala $z = a+bi$ er $\overline{z} = a-bi$. Útreikning á samokatölu skal útfæra í aðferðinni \texttt{conjugate}.
 \item Útskrift á skipanalínu á forminu \texttt{a + bi}.
\end{itemize}
Sýnið dæmi um keyrslu þar sem tölurnar $z = -3.5 + 2i$ og $w = -1+3i$ eru lagðar saman, $w$ er dregin frá $z$, þær margfaldaðar saman og samokatala $z$ er reiknuð. Skrifið út niðurstöður útreikninganna með útskriftaraðferðinni.

\vfill
\includegraphics[width=0.5\linewidth]{hi-von-logo}
\end{document}