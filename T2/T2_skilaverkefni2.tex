\documentclass{article}

\usepackage{Vor2018skil}

\title{Tölvunarfræði 2, \semester \\ Skilaverkefni 2}
\author{}

\begin{document}
\maketitle
\hypersetup{pdftitle={Tölvunarfræði 2 - Skilaverkefni 2}}

Skila skal þessum verkefnum á \href{https://gradescope.com/courses/14122}{Gradescope}.

Þegar forriti er skilað inn til yfirferðar er mikilvægt að láta \textbf{niðurstöðurnar fylgja}. Öllum forritskóða skal skila framsettum með jafnbilaletri. Hann þarf að vera afritanlegur úr .pdf skjalinu. Vönduð framsetning og læsilegur kóði er hluti af verkefninu.

\question

Skrifið C++ forrit sem les inn tölur af staðalinntaki. Safnið tölunum saman í vigur (\texttt{std::vector} breytu) af \texttt{double} tölum. Reiknið og skrifið út miðgildi \eng{median} talnanna og dreifni \eng{variance} þeirra.

Dæmi:
\begin{verbatim}
$ ./statistics
5
6
11
7
Miðgildi talnanna er 6.5
Dreifni talnanna er  5.1875
\end{verbatim}

\paragraph{Áminning:} Til að finna miðgildi talnasafns með oddatölufjölda talna má byrja á að raða safninu og velja töluna sem er í miðjunni. Til að finna miðgildi talnasafns með sléttum fjölda talna þarf að taka meðaltal stakanna tveggja sem er í miðjunni. Dreifni talnasafns $x$ með $N$ tölum má reikna með formúlunni $\frac{1}{N} \sum_{i=1}^N(x_i - \mu)^2$ þar sem $\mu$ er meðaltal talnanna.

\question

Gefinn er vísir að klasanum \href{https://raw.githubusercontent.com/Ernir/kennsluefni/master/T2/Code/w3/singlylinkedlist.cpp}{hluti af klasanum \texttt{SinglyLinkedList}} sem notar eintengdan lista til að geyma tákn. Fullbúna útfærslu væri hægt að nota á svipaðan hátt og \texttt{std::vector}. En við látum okkur duga að bæta eftirfarandi atriðum við klasann:

\begin{enumerate}[a)]
	\item Bætið við aðferðinni \texttt{prepend} sem bætir einum hnút fremst í listann. Nýji hnúturinn skal vera geymdur í kös. Fallið þarf að uppfæra eiginleikana \texttt{head} og \texttt{length}.
	\item Notið fjölbindingu til að láta \texttt{[]} virkjann sækja gögnin í hnút eftir númeri (eins og hægt er að gera við fylki og \texttt{std::vector}). Vísun með of háu númeri er villa.
	\item Bætið viðeigandi eyði \eng{destructor} við klasann. Hann þarf að sjá til þess að öllu minni sem úthlutað var við keyrslu \texttt{prepend} aðferðarinnar sé skilað.
\end{enumerate}

Ekki breyta \texttt{main} fallinu sem er gefið. Þegar forritið er keyrt ætti útskriftin að vera:

\begin{verbatim}
$ ./singlylinkedlist
T -> O -> L -> 2 -> 0 -> 3 -> G -> Ø
Stak 0 í listanum er T
Stak 3 í listanum er 2
Kallað hefur verið á eyðinn fyrir listann
\end{verbatim}

\question

Við unnum með almenn brot í Skilaverkefni 1. Nú skal nota fjölbindingu virkja og erfðir til að gera betur.

Gefinn er \href{https://raw.githubusercontent.com/Ernir/kennsluefni/master/T2/Code/w3/mixedfraction.cpp}{hluti af klasanum \texttt{Fraction}}, sem táknar almenn brot. Forritið inniheldur líka vísi að klasanum \texttt{MixedFraction}, sem virkar eins og \texttt{Fraction} en hefur útskrift á þann hátt að heiltöluhluti er brotinn frá brothlutanum þegar teljari er stærri en nefnari og brothlutanum er sleppt þegar brotið er í raun heiltala.

Bætið eftirfarandi atriðum við forritið:

\begin{enumerate}[a)]
	\item Fjölbindingu á \texttt{*} virkjanum í \texttt{Fraction} svo hægt sé að margfalda brot saman.
    \item Smið fyrir \texttt{MixedFraction} sem getur tekið við þremur inntökum, þannig að fyrsta inntakið tákni heiltöluhluta.
    \item Fjölbindingu á \texttt{<<} virkjanum svo að \texttt{MixedFraction} skrifist út á strauminn í samræmi við lýsingu.
\end{enumerate}

Ekki breyta \texttt{main} fallinu sem er gefið nema þar sem það er sérstaklega tekið fram. Þegar forritið er keyrt ætti útskriftin að vera:

\begin{verbatim}
$ ./mixedfraction
Búum til og birtum brotið 4/3 sem Fraction: 4/3
Margföldum saman Fraction-in 1/2 og 2/3: 1/3
Búum til og birtum brotið 1/4 sem MixedFraction: 1/4
Búum til og birtum brotið 4/1 sem MixedFraction: 4
Búum til og birtum brotið 0/2 sem MixedFraction: 0
Búum til og birtum brotið 7/6 sem MixedFraction: 1 1/6
Búum til og birtum brotið 1 1/3 sem MixedFraction: 1 1/3
\end{verbatim}

\vfill
\includegraphics[width=0.5\linewidth]{hi-von-logo}
\end{document}