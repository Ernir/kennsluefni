\documentclass{article}

\input{../Vor2017skil.tex}

\title{Tölvunarfræði 2, \semester \\ Skilaverkefni 3}
\author{}

\begin{document}
\maketitle
\hypersetup{pdftitle={Tölvunarfræði 2 - Skilaverkefni 3}}

Skila skal þessum verkefnum á \href{https://gradescope.com/courses/5640}{Gradescope}. Vönduð framsetning og læsilegur kóði er hluti af verkefninu, sjá \href{https://piazza.com/class/ixkicfen49l111?cid=52}{glósu um frágang og framsetningu}.


\section{Spurning 1}
Spegilstrengur (e. \emph{palindrome}) er strengur sem er eins hvort sem hann er lesinn afturábak eða áfram. Þannig er t.d. strengurinn \texttt{ABBA} spegilstrengur, en \texttt{ABAB} ekki.

Skrifið C++ fallið \texttt{isPalindrome} sem tekur inn \texttt{string} breytu og skilar rökbreytu sem táknar hvort strengurinn sé spegilstrengur eða ekki. Sýnið dæmi um keyrslu fallsins á streng sem er spegilstrengur og keyrslu á streng sem er það ekki. Skilið öllum forritskóða.

\section{Spurning 2}
Flest könnumst við við íslenskar kennitölur. Vel þekkt er að fyrstu sex tölustafirnir í kennitölu einstaklings er mynduð með fæðingardagsetningu viðkomandi og að tíundi stafurinn segir til um öldina.
Minna þekkt er að stafir sjö og átta eru merkingarlausar raðtölur og að níundi tölustafurinn er ``vartala'' sem notuð er til að reikna hvort að kennitalan sé lögleg.

Lesa má um regluna sem notuð er til að ákvarða hvort kennitala sé lögleg á vef þjóðskrár, \url{http://www.skra.is/thjodskra/um-thjodskra-/um-kennitolur/}.

Skrifið C++ fallið \texttt{isLegal}, sem tekur inn \texttt{string} sem inniheldur tíu tölustafi og skilar rökbreytu sem táknar hvort að kennitalan sé lögleg eða ekki.

Einungis þarf að athuga vartöluna, ekki hvort að kennitalan sé að öðru leyti gild.

Sýnið dæmi um keyrslu fallsins á kennitölunum \texttt{2110873909} og \texttt{2110873919}. Skilið öllum forritskóða.

\paragraph{Ábending:} C++11 fallið \href{http://www.cplusplus.com/reference/string/stoi/}{stoi} er líklegt til að koma að notum.
\newpage
\section{Spurning 3}
Í hverjum löngum streng getum við fundið algengustu hlutstrengi. Til dæmis eru algengustu hlutstrengir af lengd 3 í textanum \texttt{PANAMASKANDALABRANDARAR} strengirnir \texttt{NDA} og \texttt{AND}.

Skrifið C++ fallið \texttt{mostFrequentWords} sem tekur inn \texttt{string} breytu \texttt{text} og heiltölu \texttt{k} og skilar \texttt{unordered\_set} sem inniheldur alla algengustu hlutstrengi af lengd \texttt{k} sem finna má í \texttt{text}.

Sýnið dæmi um keyrslur fallsins á heiltölunni 5 og strengjunum tveimur í \href{https://raw.githubusercontent.com/Ernir/kennsluefni/master/T2/Code/w3/dna.txt}{dna.txt} þar sem innihald mengjanna er skrifað út. Í fyrra tilvikinu ætti að vera um að ræða strenginn \texttt{AACAA} sem er skrifaður út, í hinu strengina \texttt{AAAAT}, \texttt{GGGGT} og \texttt{TTTTA}. Skilið öllum forritskóða.

\section{Spurning 4}
Sækið sögusafnið ``The Memoirs of Sherlock Holmes'' af \href{https://www.gutenberg.org/ebooks/834}{Project Gutenberg}, á ``Plain Text UTF-8'' sniði.

Skrifið C++ fallið \texttt{findInBook} sem tekur inn tvær breytur, breytuna \texttt{text} sem inniheldur bókatextann og aðra sem inniheldur strenginn \texttt{pattern} sem leita skal að í bókinni. Fallið skal skila línunúmerum allra lína þar sem \texttt{pattern} kemur fyrir.

Notið fallið til að svara eftirfarandi spurningum:

\begin{enumerate}[a)]
 \item Á hvaða línum kemur orðið ``\texttt{Elementary}'' fyrir?
 \item Á hvaða línum kemur orðasambandið ``\texttt{, my dear Watson}'' fyrir?
\end{enumerate}

Sýnið hvernig forritið skrifar út svör við spurningunum. Skilið öllum forritskóða.

\paragraph{Ábending:} Fallið \texttt{getline} getur lesið inn heila línu af texta í einu.

\vfill
\includegraphics[width=0.5\linewidth]{hi-von-logo}
\end{document}