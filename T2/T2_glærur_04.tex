\documentclass{beamer}

\usepackage{Vor2018glærur}

\title{Tölvunarfræði 2}
\subtitle{Vika 4}

\begin{document}

\begin{frame}
\titlepage
\end{frame}

\section{Inngangur}

\begin{frame}{Af hverju þetta námsefni?}
\begin{itemize}
 \item Það sem eftir er af námskeiðinu snýst um þekkt reiknirit \eng{algorithms} og gagnagrindur \eng{data structures}
 \begin{itemize}
  \item Fræðilegt!
 \end{itemize}
 \item Fátt sem kemur jafn víða við, þó e.t.v. munið þið ekki koma til með að útfæra grundvallarreiknirit sjálf
 \begin{itemize}
  \item Gagnlegt að vita hvernig innbyggð reiknirit virka
  \item Innsýn mun nýtast í öðrum samhengjum
 \end{itemize}
 \item Almennara en aðrar leiðir til að læra forritun
\end{itemize}
\end{frame}

\section{Java}

\begin{frame}{Java}
\begin{columns}
\column{0.6\textwidth}
\begin{itemize}
 \item Munum héðan af líka forrita í Java, auk C++
 \item Þekkjum það vonandi flest úr Tölvunarfræði 1!
 \begin{itemize}
  \item Ef ekki - setja upp Java-þýðanda (\texttt{javac}) sem allra fyrst, ásamt forritssafni bókahöfunda (\texttt{algs4.jar})
  \item Leiðbeiningar á \href{http://algs4.cs.princeton.edu/code/}{síðu bókarinnar}
 \end{itemize}
 \item Java líkist C++ að ýmsu leyti
\end{itemize}
\column{0.4\textwidth}
\begin{center}
\includegraphics[width=\textwidth]{java}
\end{center}
\end{columns}
\end{frame}

\subsection{Hlutbundin forritun í Java}

\begin{frame}{Lykilorðið \texttt{static}}
\begin{itemize}
 \item Í Java tilheyra allar aðferðir klösum, líka \texttt{main} aðferðir
 \item Þurfum að ákvarða hvort að aðferðir sem við búum til tilheyri \emph{klasa} eða \emph{tilviki af klasa}
 \begin{itemize}
  \item Tilviksaðferðir \texttt{instance methods} tilheyra tilviki, er sjálfgefið
  \item Klasaaðferðir \eng{class methods} tilheyra klasa
 \end{itemize}
 \item Tilsvarandi á við um breytur
 \item Grundvallaratriði sem mikilvægt er að átta sig á!
\end{itemize}
\end{frame}

\begin{frame}{Dæmi}
\javafile[firstline=3, lastline=21, fontsize=\scriptsize, label=StaticExample.java]{Code/w4/StaticExample.java}
\end{frame}

\begin{frame}[fragile]{Breytur og tilvik}
\begin{columns}
\column{0.5\textwidth}
\begin{itemize}
 \item Í Java höfum við tvær gerðir breyta
 \begin{itemize}
  \item Einfaldar gerðir \eng{primitive types}, örfáar innbyggðar
  \item Tilvísunargerðir \emph{reference types}, allt sem skilgreint er af klösum
 \end{itemize}
\end{itemize}
\column{0.5\textwidth}
\begin{itemize}
 \item Þegar Java forrit notar \texttt{new}:
 \begin{itemize}
  \item Er minnissvæði í kös frátekið
  \item Kallað er á smið
  \item Tilvísun (e. \emph{reference}) á hlutinn skilað
 \end{itemize}
\end{itemize}
\end{columns}
\begin{center}
Tilvik af klasanum \texttt{A} búið til:

\vspace{-0.5cm}
\end{center}
\begin{columns}
\column{0.5\textwidth}
\begin{minted}[frame=lines, label=Java]{java}
A a = new A();
\end{minted}
\column{0.5\textwidth}
\begin{minted}[frame=lines, label=C++]{cpp}
A *a = new A();
\end{minted}
\end{columns}
\end{frame}

\begin{frame}{Dæmi}
Lítum á \texttt{ObjectInstantiation.java}.
\end{frame}

\begin{frame}{Jafngildi hluta}
\begin{itemize}
 \item Sjáum að jafngildi er skilgreint með \texttt{equals} aðferð
 \begin{itemize}
  \item \texttt{equals} aðferð á að vera jafngildisvensl \eng{equivalence relation}
  \begin{itemize}
   \item Sjálfhverft \eng{reflexive}: \texttt{x.equals(x)} skal vera satt
   \item Samhverft \eng{symmetric}: \texttt{x.equals(y)} þá og því aðeins að \texttt{y.equals(x)}
   \item Gegnvirk \eng{transitive}: Ef \texttt{x.equals(y)} og \texttt{y.equals(z)} þá skal \texttt{x.equals(z)}
  \end{itemize}
  \item Einnig skal \texttt{x.equals(y)} alltaf skila sama gildi og \texttt{x.equals(null)} alltaf skila ósönnu
 \end{itemize}
 \item Okkar klasar gætu þurft að útfæra \texttt{equals} aðferð
\end{itemize}
\end{frame}


\begin{frame}{Gildisveiting breyta}
Gildra: Gildisveiting afritar gildi á einföldum breytum, en afritar vísun á breytu af tilvísunargerð!

\begin{columns}
\column{0.6\textwidth}
\javafile[lastline=13, fontsize=\scriptsize, label=MyInt.java]{Code/w4/MyInt.java}
\column{0.4\textwidth}
\javafile[firstline=5, lastline=10, gobble=8, fontsize=\scriptsize, label=Assignment.java]{Code/w4/Assignment.java}
\end{columns}
\end{frame}

\begin{frame}{Breytur og aðferðir}
Gildra: Þegar aðferðir taka við breytum, þá eru breyturnar afritaðar. Þegar breytan er einföld er gildi hennar afritað, en sé hún af tilvísunargerð er vísunin í hana afrituð.

\javafile[firstline=8, lastline=17, gobble=4, fontsize=\scriptsize, label=PassingVariables.java]{Code/w4/PassingVariables.java}
\end{frame}

\subsection{Stofnrænir klasar}

\begin{frame}{Stofnrænir klasar}
\begin{columns}
\column{0.5\textwidth}
\begin{itemize}
 \item Hingað til hefur verið ákveðin takmörkun á reikniritum og aðferðum - þær virka bara á eina gagnagerð
 \item Getum reddað ýmsu með fjölbindingu, en ómögulegt er að sjá alla notkunarmöguleika fyrir 
\end{itemize}
\column{0.5\textwidth}
\begin{center}
\javafile[fontsize=\scriptsize, label=IntegerBox.java]{Code/w4/IntegerBox.java}

Kassi fyrir heiltölur. Það væri óþægilegt að búa til kassa fyrir allar gagnagerðir.
\end{center}
\end{columns}
\end{frame}

\begin{frame}{Dæmi}
\begin{columns}
\column{0.5\textwidth}
\begin{itemize}
 \item Við getum útvíkkað klasana okkar og aðferðir með því að skilgreina þau sem stofnræn (e. \emph{generic})
 \item Stofnrænn klasi eða aðferð inniheldur sínar eigin tögunarskilgreiningar
\end{itemize}
\column{0.5\textwidth}
\begin{center}
\javafile[fontsize=\scriptsize, firstline=3, lastline=12,label=Genericbox.java]{Code/w4/GenericBox.java}

Kassi fyrir alls kyns hluti.
\end{center}
\end{columns}
\end{frame}

\section{Almennar lýsingar}

\subsection{Skil}

\begin{frame}{Skil}
\begin{itemize}
 \item Við skrifum ekki öll okkar forrit frá grunni, við notum forrit annarra
 \item Til að auðvelda samskipti forrita skilgreinum við oft fyrir þau skil (e. \emph{Application Programming Interfaces}, APIs)
 \item Í skilum felast notkunar- og samskiptalýsingar á aðferðum og sviðum
 \item Skil eru sérstaklega mikilvæg fyrir forritasöfn
 \item Hugmyndin - ef við vitum hvernig hitt forritið hegðar sér þarf forritið okkar ekki að hafa aðgang að útfærslu þess
 \item Felur í sér n.k. samning á milli forrita
\end{itemize}
\end{frame}

\begin{frame}{Vandamál!}
\begin{itemize}
 \item Skil geta verið\ldots
 \begin{itemize}
  \item \ldots of erfið í útfærslu
  \item \ldots of erfið í notkun
  \item \ldots of þröng (of fáar aðferðir)
  \item \ldots of víð (of margar aðferðir)
  \item \ldots of almenn
  \item \ldots of sérhæfð
  \item \ldots of háð útfærslunni
 \end{itemize}
 \item En samt nauðsynleg!
\end{itemize}
\end{frame}

\subsection{Hugrænar gagnagerðir}

\begin{frame}{Hugrænar gagnagerðir}
\begin{itemize}
 \item Hugræn gagnagerð (e. \emph{Abstract Data Type, ADT}) er gagnagerð þar sem útfærslan er falin fyrir þeim sem nota gagnagerðina
 \begin{itemize}
  \item Þegar við notum hugræna gagnagerð beitum við aðferðunum sem skil hennar skilgreina
  \item Þegar við búum til útfærslu á hugrænni gagnagerð einbeitum við okkur að uppröðun gagnanna og gætum þess að skilin séu uppfyllt
 \end{itemize}
 \item Munum sjá mörg dæmi um hugrænar gagnagerðir!
\end{itemize}
\end{frame}

\section{Lokaorð}

\begin{frame}{Þessi glærupakki}
Tengill á fyrirlestraræfingu: \url{https://goo.gl/forms/iuOvOdwwWohyPu6g2}
\vspace{1cm}

Öll nafngreind forrit í þessum glærupakka, ásamt glærupakkanum sjálfum, má finna á  \href{https://github.com/Ernir/kennsluefni/tree/master/T2/Code/w4}{Github}. 

Fleiri forrit: \url{http://algs4.cs.princeton.edu/code/}
\end{frame}


\begin{frame}{Næst}
Skjóður, biðraðir, hlaðar, greining reiknirita
\end{frame}


\end{document}
