\documentclass{beamer}

\usepackage[utf8]{inputenc} % Language and font encoding
\usepackage[icelandic]{babel}
\usepackage[T1]{fontenc}


\usepackage{tikz}
\usepackage[listings,theorems]{tcolorbox}
\usepackage{booktabs}
\usepackage{minted} %Minted and configuration
\usemintedstyle{default}

\renewcommand{\theFancyVerbLine}{\sffamily \arabic{FancyVerbLine}}

%%%%%%%%%%%%%%%%%%%%%%
% Beamer configuration
%%%%%%%%%%%%%%%%%%%%%%
\setbeamertemplate{navigation symbols}{}
\usecolortheme{dove}
\setbeamercolor{frametitle}{fg=white}

\usebackgroundtemplate%
{%
\vbox to \paperheight{
\includegraphics[width=\paperwidth]{Pics/hi-slide-head-2016}

\vfill
\hspace{0.5cm}\includegraphics[width=0.3\paperwidth]{Pics/hi-von-logo}
\vspace{0.4cm}
    }%
}

\AtBeginSection[]
{
  \begin{frame}<beamer>
    \frametitle{Yfirlit}
    \tableofcontents[currentsection]
  \end{frame}
}

\setbeamerfont{frametitle}{size=\normalsize}
\addtobeamertemplate{frametitle}{}{\vspace*{0.5cm}}

%%%%%%%%%%%%%%%%%%%%%%%%%
% tcolorbox configuration
%%%%%%%%%%%%%%%%%%%%%%%%%

% Setup from: http://tex.stackexchange.com/a/43329/21638
\tcbset{%
    noparskip,
    colback=gray!10, %background color of the box
    colframe=gray!40, %color of frame and title background
    coltext=black, %color of body text
    coltitle=black, %color of title text 
    fonttitle=\bfseries,
    alerted/.style={coltitle=red, colframe=gray!40},
    example/.style={coltitle=black, colframe=green!20, colback=green!5},
}


% Shortcuts
\newcommand{\Mod}[1]{\ \text{mod}\ #1}
\newcommand{\eng}[1]{(e.\ \emph{#1})}
\newmintedfile[cppfile]{cpp}{frame=lines, linenos=true}
\newmintedfile[javafile]{java}{frame=lines, linenos=true}


%%%%%%%%%%%%%%%%%%%%%%%
% Further configuration
%%%%%%%%%%%%%%%%%%%%%%%
\hypersetup{colorlinks=true,pdfauthor={Eirikur Ernir Thorsteinsson},linkcolor=blue,urlcolor=blue}
\graphicspath{{./Pics/}}


\author{Eiríkur Ernir Þorsteinsson}
\institute{Háskóli Íslands}
\date{Vor 2017}

\title{Tölvunarfræði 2}
\subtitle{Vika 7}

\begin{document}

\begin{frame}
\titlepage
\end{frame}

\section{Inngangur}

\begin{frame}{Upprifjun: Tré}
\begin{columns}
\column{0.6\textwidth}
\begin{itemize}
 \item Algengt er í tölvunarfræði að raða gögnum upp í tré
 \item Hugmyndin eins og við munum nota hana:
 \begin{itemize}
  \item Höfum hnúta, hver hnútur inniheldur vísun í tvö eða fleiri \emph{börn}
  \begin{itemize}
   \item Vísunin getur verið tóm
  \end{itemize}
  \item Getum kallað hnút sem er ekki barn neins annars \emph{rót}, hnút sem á engin ekki-tóm börn \emph{lauf}
 \end{itemize}
 \item Sértilvik: tvíundartré \eng{binary tree}, þar sem barnafjöldinn er 2
\end{itemize}
\column{0.4\textwidth}
\includegraphics[width=\linewidth]{tree-example}
\end{columns}
\end{frame}


\section{Hrúgur}

\begin{frame}{Hrúgur}
Hrúga \eng{heap} er tvíundartré sem uppfyllir hrúguskilyrði \eng{heap property}. Við notum fylki til að geyma hrúgur.

Skoðum glærur 16-17 í \href{http://algs4.cs.princeton.edu/lectures/24PriorityQueues.pdf}{PriorityQueues}.
\end{frame}

\begin{frame}{Aðgerðir}
\begin{itemize}
 \item Notkun hrúgu til að útfæra forgangsbiðröð krefst tveggja aðgerða
 \item Innsetning
 \begin{itemize}
  \item Setjum stakið aftast og látum það synda upp
 \end{itemize}
 \item Eyðing stærsta staks
 \begin{itemize}
  \item Skiptum á rót og síðasta stakinu í fylkinu, sökkvum því sem við færðum upp
 \end{itemize}
 \item Skoðum glærur 20-23 í \href{http://algs4.cs.princeton.edu/lectures/24PriorityQueues.pdf}{PriorityQueues}.
\end{itemize}
\end{frame}

\begin{frame}{Tími}
\begin{itemize}
 \item Athugum - um fullskipuð tré er að ræða
 \item Innsetning og eyðing felur í sér færslu á milli hæða í trénu
 \item Fjöldi aðgerða takmarkast af hæð trésins
 \begin{itemize}
  \item Hæð fullskipaðs trés með $N$ hnútum er $\lfloor \log_2 n \rfloor$
 \end{itemize}
 \item Praktísk vandræði við útfærsluna eins og henni hefur verið lýst er að langt er á milli staka, hentar illa fyrir minnisuppbyggingu raun
\end{itemize}
\end{frame}

\section{Minnisuppbygging tölva}

\begin{frame}{Minnisuppbygging}
\begin{columns}
\column{0.6\textwidth}
\begin{itemize}
 \item Sumir geymslustaðir fyrir gögn eru hraðvirkari en aðrir
 \item Gróf uppbygging í nútímatölvu:
 \begin{enumerate}
  \item Minni örgjörvans \eng{internal memory} - gisti \eng{registers} og skyndiminni \eng{cache}
  \item Aðalminni \eng{main memory}  - RAM
  \item Aukageymsla \eng{secondary storage} - SSD diskar/harðir diskar
  \item Útvær geymsla \eng{off-line storage} - geymsla utan stjórnar örgjörvans
 \end{enumerate}
\end{itemize}
\column{0.4\textwidth}
\includegraphics[width=\linewidth]{computer-memory-hierarchy}
\begin{center}
\href{https://en.wikipedia.org/wiki/Memory\_hierarchy}{Mynd af Wikipedia}
\end{center}
\end{columns}
\end{frame}

\section{Nafnatöflur}

\begin{frame}{Nafnatöflur}
\begin{columns}
\column{0.5\textwidth}
\begin{itemize}
 \item Nafnatafla \eng{symbol table} er mjög almenn hugræn gagnagrind
 \item Snýst um að tengja saman lykla (e. \emph{keys}) og gildi (e. \emph{values})
 \begin{itemize}
  \item ``Hvert er gildið fyrir þennan lykil?''
 \end{itemize}
 \item Höfum þegar séð útfærslu á nafnatöflu - \texttt{std::map} í C++
\end{itemize}
\column{0.5\textwidth}
\begin{center}
DNS tafla
\begin{tabular}{cc}
\toprule
Lykill&Gildi\\
\midrule
mbl.is&92.43.192.110\\
visir.is&82.221.81.10\\
hi.is&130.208.165.207\\
ru.is&52.48.55.82\\
\bottomrule
\end{tabular}
\end{center}
\end{columns}
\end{frame}

\begin{frame}{API}
Möguleg skil fyrir nafnatöflu:
\begin{center}
\begin{tabularx}{\textwidth}{rlX}
\toprule
\multicolumn{3}{c}{\texttt{public class ST<Key, Value>}}\\
\midrule
-&\texttt{ST()}& Smiður, býr til tóma nafnatöflu\\
\texttt{void}&\texttt{void put(Key k, Value v)}&Setja lykil-gildis par í töfluna\\
\texttt{Value}&\texttt{get(Key key)}&Skilar gildinu sem svarar til \texttt{key}\\
\bottomrule
\end{tabularx}
\end{center}
Auk þess mætti skilgreina \texttt{delete(Key key)}, \texttt{contains(Key\ key)}, \texttt{size()} og \texttt{isEmpty()} aðferðir
\end{frame}

\begin{frame}{Útfærsla á nafnatöflu}
\begin{itemize}
 \item Við gætum sett gögnin okkar í fylki eða eintengdan lista
 \begin{itemize}
  \item Gerir leit í stórum töflum erfiða
 \end{itemize}
 \item Getum notað öflugri aðferðir ef við getum gert ráð fyrir meiru um lyklana
 \begin{itemize}
  \item Viljum hafa lykla sem eru samanburðarhæfir (sjá \texttt{Comparable})
  \item Seinna: Lykla sem hægt er að taka af \texttt{hashCode}
 \end{itemize}
 \item Almennt ráðlagt: Nota óbreytanlega \eng{immutable} lykla
 \begin{itemize}
  \item Gott: \texttt{Integer}, \texttt{String}, \texttt{Double},\ldots
  \item Slæmt: T.d. fylki
 \end{itemize}
\end{itemize}
\end{frame}

\section{Helmingunarleit}

\begin{frame}{Helmingunarleit}
%TODO fjalla um helmingunarleit
\end{frame}


\begin{frame}{Þessi glærupakki}
Tengill á fyrirlestraræfingu: \url{}
\vspace{1cm}

\texttt{} má finna á \href{https://github.com/Ernir/kennsluefni/tree/master/T2/Code/w8}{Github}. 

Kóða fyrir algs4 reiknirit má finna á \url{http://algs4.cs.princeton.edu/code/}
\end{frame}


\end{document}
