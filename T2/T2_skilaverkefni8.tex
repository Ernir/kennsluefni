\documentclass{article}

\usepackage{Vor2018skil}

\title{Tölvunarfræði 2, \semester \\ Skilaverkefni 8}
\author{}

\begin{document}
\maketitle
\hypersetup{pdftitle={Tölvunarfræði 2 - Skilaverkefni 8}}

Skila skal þessum verkefnum á \href{https://gradescope.com/courses/14122}{Gradescope}.

Þegar forriti er skilað inn til yfirferðar er mikilvægt að láta \textbf{niðurstöðurnar fylgja}. Öllum forritskóða skal skila framsettum með jafnbilaletri. Hann þarf að vera afritanlegur úr .pdf skjalinu. Vönduð framsetning og læsilegur kóði er hluti af verkefninu.

\question

Eintengdir listar hafa skemmt okkur vel hingað til. En við hingað til ekki útfært aðferð sem eyðir staki úr eintengdum lista eftir sætisnúmeri. Gefin er \href{https://raw.githubusercontent.com/Ernir/kennsluefni/master/T2/Code/w9/SinglyLinkedList.java}{beinagrind í Java}. Bætið \texttt{delete} aðferðinni við beinagrindina skv. lýsingu.

Skilið eingöngu \texttt{delete} aðferðinni sem þið skrifið, gert verður ráð fyrir að aðrir hlutar forritsins séu óbreyttir við yfirferð.

Hluti útskriftar (sjá einnig verkefni 2):

\begin{verbatim}
Listinn er í upphafi: 1 -> 2 -> 3 -> 4 -> Ø

Hendum nú út öllum hnútunum í slembinni röð
Hendum staki númer 2. Eftir eyðingu er listinn: 1 -> 2 -> 4 -> Ø
Hendum staki númer 0. Eftir eyðingu er listinn: 2 -> 4 -> Ø
Hendum staki númer 0. Eftir eyðingu er listinn: 4 -> Ø
Hendum staki númer 0. Eftir eyðingu er listinn: Ø
\end{verbatim}

\question

Skrifið aðferðina \texttt{swap}, sem víxlar á staðsetningum tveggja hnúta í eintengdum lista. Beinagrindin úr verkefninu á undan er notuð aftur. Framkvæmið víxlunina með því að uppfæra vísanirnar í hnútana, ekki bara gögn þeirra.

Skilið eingöngu \texttt{swap} aðferðinni sem þið skrifið, gert verður ráð fyrir að aðrir hlutar forritsins séu óbreyttir við yfirferð.

Hluti útskriftar:

\begin{verbatim}
Listinn er í upphafi:          1 -> 2 -> 3 -> 4 -> Ø

Skiptum á hnútum 1 og 3, fáum: 1 -> 4 -> 3 -> 2 -> Ø
Lagfærum listann aftur, fáum:  1 -> 2 -> 3 -> 4 -> Ø
Prófum að skipta á endunum:    4 -> 2 -> 3 -> 1 -> Ø
Lagfærum aftur:                1 -> 2 -> 3 -> 4 -> Ø
\end{verbatim}

\question

Hægt er að nota það sem við höfum lært um skilvirkar útfærslur á forgangsbiðröðum til að útfæra röðunarreiknirit sem heitir heapsort.

Heapsort á fylki fer fram með tveimur lykkjum.

\begin{itemize}
	\item Í fyrri lykkjunni er fylkinu endurraðað svo það uppfylli hrúguskilyrði.

	\item Í seinni lykkjunni notum við lítum við á vinstri hluta fylkisins sem óraðaðan og hægri hluta fylkisins sem raðaðan. Finnum endurtekið stærsta stak í vinstri hlutanum, færum það yfir til raðaða hlutans, lögum óraðaða hlutann svo hann uppfylli aftur hrúguskilyrði og færum mörk hlutanna til vinstri um eitt sæti.
\end{itemize}
Reikniritinu svipar þannig til valröðunar, nema hvað leitin í óraðaða hluta fylkisins er mun skilvirkari.

Fjallað er um heapsort á bls. 323 til 327 í Algorithms. Skrifið útfærslu á heapsort fyrir í C++ sem svipar til Java-útfærslunnar sem finna má á bls. 324 og \href{https://algs4.cs.princeton.edu/code/edu/princeton/cs/algs4/Heap.java.html}{síðu bókarinnar}.

Gefin er \href{https://raw.githubusercontent.com/Ernir/kennsluefni/master/T2/Code/w9/heapsort.cpp}{beinagrind}. Skilið \texttt{heapsort} fallinu ásamt öðrum kóða sem þið bætið við, gert verður ráð fyrir að \texttt{main} og \texttt{issorted} föllin séu óbreytt við yfirferð.

\vfill
\includegraphics[width=0.5\linewidth]{hi-von-logo}
\end{document}