\documentclass{article}

\input{../Vor2017skil.tex}

\title{Tölvunarfræði 2, \semester \\ Skilaverkefni 8}
\author{}

\begin{document}
\maketitle
\hypersetup{pdftitle={Tölvunarfræði 2 - Skilaverkefni 8}}

\paragraph{Skilavefur} Skila skal þessum verkefnum á \href{https://gradescope.com/courses/5640}{Gradescope}.

\paragraph{Vinnubrögð og frágangur} Skrifa þarf forrit í Java og C++ sem útfæra ýmsar aðferðir. Í öllum tilvikum skal skila \texttt{main} fall/aðferð sem býr til prufugögn og sýnir virkni hverrar aðferðar um sig. Skilið öllum forritskóða og sýnið dæmi um keyrslu. Leitist við að nota beinagrindur óbreyttar þegar þær eru gefnar. Vönduð framsetning og læsilegur kóði er hluti af verkefninu, sjá \href{https://piazza.com/class/ixkicfen49l111?cid=52}{glósu um frágang og framsetningu}.

\paragraph{Samþykki til dreifingar} Dæmatímakennarar velja framúrskarandi lausnir til birtingar undir nafni í lausnasafni. Sé þess óskað að einhverjar þinna úrlausna séu ekki birtar nema nafnlaust eða alls ekki birtar yfir höfuð skal taka slíkt fram í hverju dæmi sem takmörkunin á við.

\section{Spurning 1}
Teiknaðu tvíleitartréð sem myndi liggja að baki nafnatöflu sem útfærð var með \texttt{BST.java} eftir að stök með lyklana
\begin{center}
\texttt{E A S Y Q U E S T I O N} 
\end{center}
væru sett inn í hana í röð, ef hún væri tóm í upphafi. Láttu gildi hvers staks vera númer lykilsins.

\section{Spurning 2}
Skrifið forrit í Java sem les inn orð af staðalinntaki og athugar hversu mörg þeirra séu leyfileg orð í Scrabble skv. \href{http://introcs.cs.princeton.edu/java/data/ospd.txt}{ospd.txt} (Official Scrabble Player's Dictionary). Forritið skal framkvæma leitina tvisvar, annarsvegar með línulegri leit og hins vegar með helmingunarleit. Það skal skrifa út hversu mörg orðin eru og keyrslutímann sem hvor aðferð um sig þarf.

Prófið forritið með orðasafninu í \href{http://introcs.cs.princeton.edu/java/data/words.txt}{words.txt}.

Ekki þarf að útfæra eigin helmingunarleit, nota má forrit úr \texttt{algs4.jar} sem framkvæmir helmingunarleit af einhverju tagi.

Notið beinagrindina í \href{https://github.com/Ernir/kennsluefni/tree/master/T2/Code/w8/ScrabbleSearch.java}{ScrabbleSearch.java}.

\section{Spurning 3}
Skrifið helmingunarleit í C++ sem vinnur á \texttt{std::vector} af aðskildum heiltölum í \emph{lækkandi} röð.

Notið beinagrindina í \href{https://github.com/Ernir/kennsluefni/tree/master/T2/Code/w8/DescendingBinarySearch.cpp}{DescendingBinarySearch.cpp}.

\section{Spurning 4}
Við höfum skoðað fylkjaútfærslu á hrúgu. Skrifið nú hrúgu í Java sem útfærð er með því að skrifa þrítengt tré (þar sem hver hnútur veit af foreldri sínu og báðum börnum) og notið hana til að útfæra forgangsbiðröð.

Notið beinagrindina í \href{https://github.com/Ernir/kennsluefni/tree/master/T2/Code/w8/TreeHeapPriQueue.java}{TreeHeapPriQueue.java}.
\vfill
\includegraphics[width=0.5\linewidth]{hi-von-logo}
\end{document}