\documentclass{article}

\usepackage{Vor2018skil}

\title{Tölvunarfræði 2, \semester \\ Upprifjunarverkefni}
\author{}

\begin{document}
\maketitle
\hypersetup{pdftitle={Tölvunarfræði 2 - Upprifjunarverkefni}}

Þessi verkefni eru ætluð til upprifjunar og æfingar fyrir próf, full einkunn er gefin fyrir allar heiðarlegar tilraunir. Hafið í huga að best er að meðhöndla þetta sem alvöru próftexta, prófið að leysa þau án tölvu.

Skila skal þessum verkefnum á \href{https://gradescope.com/courses/14122}{Gradescope}.

Öllum forritskóða skal skila framsettum með jafnbilaletri. Hann þarf að vera afritanlegur úr .pdf skjalinu. Vönduð framsetning og læsilegur kóði er hluti af verkefninu. 

\question
Skrifið Java-aðferðina \texttt{reverseStack} sem tekur inn hlaða og snýr við röð stakanna sem í honum eru.

\question
Skrifið Java-aðferðina \texttt{farthestM} sem tekur inn skjóðu af hlutum sem tákna punkta í tvívíðu hnitakerfi ásamt heiltölu \texttt{m}. Aðferðin skal reikna út fjarlægð hvers punkts frá upphafspunkti hnitakerfisins og skila skjóðu af þeim \texttt{m} fjarlægðum sem lengstar eru.

\begin{minted}[frame=lines]{java}
static Bag<Double> farthestM(Bag<Point2D> points, int m) {

}
\end{minted}

\question

Útfærið innsetningarröðun í C++ fyrir \texttt{std::vector} heiltalna. Hjálparaðferðin \texttt{exch} er gefin.

\begin{minted}[frame=lines]{cpp}
void exch(vector<int> &v, int i, int j) {
    int swap = v[i];
    v[i] = v[j];
    v[j] = swap;
}

void insertionSort(vector<int> &v) {

}
\end{minted}

\end{document}