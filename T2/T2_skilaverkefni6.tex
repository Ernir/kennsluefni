\documentclass{article}

\usepackage{Vor2018skil}

\title{Tölvunarfræði 2, \semester \\ Skilaverkefni 6}
\author{}

\usepackage[normalem]{ulem}

\begin{document}
\maketitle
\hypersetup{pdftitle={Tölvunarfræði 2 - Skilaverkefni 6}}

Skila skal þessum verkefnum á \href{https://gradescope.com/courses/14122}{Gradescope}.

Þegar forriti er skilað inn til yfirferðar er mikilvægt að láta \textbf{niðurstöðurnar fylgja}. Öllum forritskóða skal skila framsettum með jafnbilaletri. Hann þarf að vera afritanlegur úr .pdf skjalinu. Vönduð framsetning og læsilegur kóði er hluti af verkefninu.

\question

Til að sjá fyrir sér virkni forrita getur verið gagnlegt að skrifa út milliniðurstöður þeirra.

Gefið er forritið \href{https://github.com/Ernir/kennsluefni/tree/master/T2/Code/w7/NotQuick.java}{NotQuick.java}, sem er útgáfa af \texttt{Quick.java} sem sleppir því að stokka fylkið í upphafi.\footnote{Aðrir hlutar forritsins sem eru verkefninu óviðkomandi hafa líka verið fjarlægðir.}

Dæmi um útskrift á ástöndunum undir lok \texttt{partition} aðferðarinnar þegar bókstöfunum í \texttt{KRATELEPUIMQCXOS} er raðað er eftirfarandi\footnote{Þessi stafaruna er hluti af sýnidæmi á bls. 289 í Algorithms.}:

\begin{verbatim}
Fyrir röðun: K R A T E L E P U I M Q C X O S 
Vendistak K, raðað frá  0 til 15: E C A I E K L P U T M Q R X O S 
Vendistak E, raðað frá  0 til  4: E C A E I K L P U T M Q R X O S 
Vendistak E, raðað frá  0 til  2: A C E E I K L P U T M Q R X O S 
Vendistak A, raðað frá  0 til  1: A C E E I K L P U T M Q R X O S 
Vendistak L, raðað frá  6 til 15: A C E E I K L P U T M Q R X O S 
Vendistak P, raðað frá  7 til 15: A C E E I K L M O P T Q R X U S 
Vendistak M, raðað frá  7 til  8: A C E E I K L M O P T Q R X U S 
Vendistak T, raðað frá 10 til 15: A C E E I K L M O P S Q R T U X 
Vendistak S, raðað frá 10 til 12: A C E E I K L M O P R Q S T U X 
Vendistak R, raðað frá 10 til 11: A C E E I K L M O P Q R S T U X 
Vendistak U, raðað frá 14 til 15: A C E E I K L M O P Q R S T U X 
Eftir röðun: A C E E I K L M O P Q R S T U X 
\end{verbatim}

Breytið NotQuick svo það skrifi út ástandið undir lok hvers kalls á \texttt{partition} aðferðina. Vendistakið, mörk undirfylkisins sem verið var að raða og fylkið í heild sinni þurfa að koma fram í hverri útskrift. Ekki breyta öðrum aðferðum en \texttt{partition} aðferðinni.

\newpage

\question

Skrifið aðferð í Java sem tekur inn tvær forgangsbiðraðir og sameinar þær. 
Hún skal búa til nýja forgangsbiðröð sem inniheldur öll stök sem finna mátti í forgangsbiðröðunum sem hún tók inn. 

Notið beinagrindina í \href{https://github.com/Ernir/kennsluefni/tree/master/T2/Code/w7/PriQueueMerge.java}{PriQueueMerge.java}. Ekki breyta \texttt{main} aðferðinni eða haus \texttt{merge} aðferðarinnar.

Í réttri útfærslu ætti prufuforritið að skrifa út tölurnar \texttt{9 8 7 6 5 4 3 2 1 0}.

\question

Útfærið quicksort reikniritið í C++ fyrir \texttt{std::vector} heiltalna. Notið \texttt{Quick.java} til hliðsjónar.

Notið beinagrindina í \href{https://github.com/Ernir/kennsluefni/tree/master/T2/Code/w7/quicksort.cpp}{quicksort.cpp}. Ekki breyta \texttt{main} fallinu, \texttt{issorted} fallinu eða haus \texttt{quicksort} fallsins.

\paragraph{Ábending 1:} Hægt er að stokka vigur með því að nota \href{http://www.cplusplus.com/reference/algorithm/random_shuffle/}{\texttt{std::random\_shuffle}} (sjá notkunardæmi í \texttt{main} fallinu).
\paragraph{Ábending 2:} \&-merkið í fallsinntakinu lætur fallið meðhöndla inntakið sem vísun, á svipaðan hátt og við erum farin að venjast í Java. 

\vfill
\includegraphics[width=0.5\linewidth]{hi-von-logo}
\end{document}