\documentclass{beamer}

\usepackage[utf8]{inputenc}
\usepackage[icelandic]{babel}
\usepackage[T1]{fontenc}

\usepackage{booktabs}
\usepackage{minted} %Minted and configuration
\usepackage{framed}
\usepackage{tikz}
\usemintedstyle{default}
\renewcommand{\theFancyVerbLine}{\sffamily \arabic{FancyVerbLine}}
\newcommand{\Mod}[1]{\ \text{mod}\ #1}

% \makeatletter
% \minted@define@extra{label}
% \makeatother

\usebackgroundtemplate%
{%
\vbox to \paperheight{
\includegraphics[width=\paperwidth]{Pics/hi-slide-head}

\vfill
\hspace{0.5cm}\includegraphics[width=0.3\paperwidth]{Pics/hi-von-logo}
\vspace{0.5cm}
    }%
}

% \makeatletter
% \newcommand{\minted@write@detok}[1]{%
%   \immediate\write\FV@OutFile{\detokenize{#1}}}%
%   
%   \newcommand{\minted@FVB@VerbatimOut}[1]{%
%   \@bsphack
%   \begingroup
%     \FV@UseKeyValues
%     \FV@DefineWhiteSpace
%     \def\FV@Space{\space}%
%     \FV@DefineTabOut
%     %\def\FV@ProcessLine{\immediate\write\FV@OutFile}% %Old, non-Unicode version
%     \let\FV@ProcessLine\minted@write@detok %Patch for Unicode
%     \immediate\openout\FV@OutFile #1\relax
%     \let\FV@FontScanPrep\relax
% %% DG/SR modification begin - May. 18, 1998 (to avoid problems with ligatures)
%     \let\@noligs\relax
% %% DG/SR modification end
%     \FV@Scan}
%     \let\FVB@VerbatimOut\minted@FVB@VerbatimOut
%     
%     \renewcommand\minted@savecode[1]{
%   \immediate\openout\minted@code\jobname.pyg
%   \immediate\write\minted@code{\expandafter\detokenize\expandafter{#1}}%
%   \immediate\closeout\minted@code}
%   
% \makeatother

\setbeamertemplate{navigation symbols}{}
\usecolortheme{dove}
\setbeamercolor{frametitle}{fg=white}
\hypersetup{colorlinks=true,pdfauthor={Eirikur Ernir Thorsteinsson},linkcolor=blue,urlcolor=blue}

\AtBeginSection[]
{
  \begin{frame}<beamer>
    \frametitle{Yfirlit}
    \tableofcontents[currentsection]
  \end{frame}
}

\author{Eiríkur Ernir Þorsteinsson}
\institute{Háskóli Íslands}
\date{Haust 2015}

\title{Tölvunarfræði 1a}
\subtitle{Vika 14, seinni fyrirlestur}

\begin{document}

\begin{frame}
\titlepage
\end{frame}

\section{Inngangur}

\begin{frame}{Í síðasta þætti\ldots}
\begin{itemize}
 \item Ýmis grunnatriði í forritun, borið saman við Java og Python
\end{itemize}
Kaflar: 1-6
\end{frame}

\section{Námsmat}

\begin{frame}{Námsmat}
\begin{itemize}
 \item Vikuleg heimaverkefnaskil
 \begin{itemize}
  \item Gilda 20\% samtals
  \item \textbf{9} bestu verkefni gilda til einkunnar
 \end{itemize}
 \item Fyrirlestraæfingar
 \begin{itemize}
  \item Gilda 10\% samtals
  \item 20 skil gefa fulla einkunn
 \end{itemize}
 \item Miðmisserispróf
 \begin{itemize}
  \item Gildir 20\% ef það leiðir til hækkunar, annars 0\%
 \end{itemize}
 \item Lokapróf
 \begin{itemize}
  \item Gildir 50\% ef miðmisserispróf leiðir til hækkunar, annars 70\%
  \item Lágmarkseinkunn á prófi og námskeiðinu í heild er 5 (\href{https://ugla.hi.is/kennsluskra/?tab=skoli&chapter=content&id=34690}{prófreglur})
  \item Fyrirhugað að gefa í heilum og hálfum
 \end{itemize}
\end{itemize}
\end{frame}

\section{Efni til prófs}

\begin{frame}{Námsefni til lokaprófs}
\begin{itemize}
 \item Allar glærur
 \item Öll verkefni (skila, dæmatíma- og fyrirlestra)
 \item Kaflar 1-11 í kennslubókinni, með eftirfarandi undantekningum:
 \begin{itemize}
  \item Kafli 3.8 (föll sem skipanir)
  \item Kafli 4.5 (\texttt{menu} fallið)
  \item Kafli 6.3 (\texttt{menu} + einingaforritun)
  \item Kafli 6.4.1 (\texttt{persistent} breytur)
  \item Kafli 6.5 (aflúsun - fórum yfir, en er ekki til prófs)
  \item Kafli 8.2.5 (færslur af færslum)
  \item Kafli 11.7 (vistun mynda)
 \end{itemize}
 \item Kaflar 12.1, 12.2 og 14.1 í kennslubókinni  
\end{itemize}
\end{frame}

\section{Dæmi}

\begin{frame}{Lítum á dæmi}
\begin{itemize}
 \item Haustpróf 2008, dæmi 6
 \item Haustpróf 2011, dæmi 5, liður a)
 \item \texttt{parentfun.m}
 \item Dæmi 7, kafla 10 í bók
 \item Haustpróf 2010, dæmi 4
\end{itemize}

\end{frame}


\end{document}
