\documentclass{beamer}

\usepackage[utf8]{inputenc}
\usepackage[icelandic]{babel}
\usepackage[T1]{fontenc}

\usepackage{booktabs}
\usepackage{minted} %Minted and configuration
\usepackage{framed}
\usepackage{tikz}
\usemintedstyle{default}
\renewcommand{\theFancyVerbLine}{\sffamily \arabic{FancyVerbLine}}
\newcommand{\Mod}[1]{\ \text{mod}\ #1}

% \makeatletter
% \minted@define@extra{label}
% \makeatother

\usebackgroundtemplate%
{%
\vbox to \paperheight{
\includegraphics[width=\paperwidth]{Pics/hi-slide-head}

\vfill
\hspace{0.5cm}\includegraphics[width=0.3\paperwidth]{Pics/hi-von-logo}
\vspace{0.5cm}
    }%
}

% \makeatletter
% \newcommand{\minted@write@detok}[1]{%
%   \immediate\write\FV@OutFile{\detokenize{#1}}}%
%   
%   \newcommand{\minted@FVB@VerbatimOut}[1]{%
%   \@bsphack
%   \begingroup
%     \FV@UseKeyValues
%     \FV@DefineWhiteSpace
%     \def\FV@Space{\space}%
%     \FV@DefineTabOut
%     %\def\FV@ProcessLine{\immediate\write\FV@OutFile}% %Old, non-Unicode version
%     \let\FV@ProcessLine\minted@write@detok %Patch for Unicode
%     \immediate\openout\FV@OutFile #1\relax
%     \let\FV@FontScanPrep\relax
% %% DG/SR modification begin - May. 18, 1998 (to avoid problems with ligatures)
%     \let\@noligs\relax
% %% DG/SR modification end
%     \FV@Scan}
%     \let\FVB@VerbatimOut\minted@FVB@VerbatimOut
%     
%     \renewcommand\minted@savecode[1]{
%   \immediate\openout\minted@code\jobname.pyg
%   \immediate\write\minted@code{\expandafter\detokenize\expandafter{#1}}%
%   \immediate\closeout\minted@code}
%   
% \makeatother

\setbeamertemplate{navigation symbols}{}
\usecolortheme{dove}
\setbeamercolor{frametitle}{fg=white}
\hypersetup{colorlinks=true,pdfauthor={Eirikur Ernir Thorsteinsson},linkcolor=blue,urlcolor=blue}

\AtBeginSection[]
{
  \begin{frame}<beamer>
    \frametitle{Yfirlit}
    \tableofcontents[currentsection]
  \end{frame}
}

\author{Eiríkur Ernir Þorsteinsson}
\institute{Háskóli Íslands}
\date{Haust 2015}

\title{Tölvunarfræði 1a}
\subtitle{Vika 10, fyrri fyrirlestur}

\begin{document}

\begin{frame}
\titlepage
\end{frame}

\section{Inngangur}

\begin{frame}{Í síðasta þætti\ldots}
\begin{itemize}
 \item Almenn skráarvinnsla
 \item Excel-skrár
 \item \texttt{.mat} skrár
\end{itemize}
Kaflar: 9.1 - 9.3 
\end{frame}

\begin{frame}{Um heimaverkefni}
\begin{itemize}
 \item Í þessari viku er verið að forrita \emph{í tveggja manna hópum}
 \item Ráð og trikk fyrir paraforritun:
 \begin{itemize}
  \item Pör þar sem annar aðilinn er sterkari forritari eru mjög góð
  \begin{itemize}
   \item Veikari lærir af þeim sterkari
   \item Sterkari lærir af kennslunni
  \end{itemize}
  \item Forritið á \emph{einni} tölvu
  \begin{itemize}
   \item Hlutverk ykkar: ``Driver'' (situr við tölvu) og ``navigator'' (hugsar fram í tímann, grípur villur)
   \item Látið veikari forritarann vera driver til að byrja með
   \item Skiptið um hlutverk þegar þið eruð farin að detta í Facebook
  \end{itemize}
 \end{itemize}
\end{itemize}
\end{frame}


\section{Nafnlaus föll (10.1)}

\begin{frame}[fragile]{Nafnlaus föll}
\begin{itemize}
 \item Nafnlaus föll (e. \emph{anonymous functions}) eru einföld einnar línu föll
 \begin{itemize}
  \item Þurfa ekki að vera í sérstakri \texttt{.m}-skrá
  \item Geta einfaldað forritun töluvert
 \end{itemize}
 \item Almennt snið:
\begin{verbatim}
handfang = @(viðföng) fallsegð
\end{verbatim}
 \begin{itemize}
  \item Þar sem \texttt{handfang} er fallshandfang (e. \emph{function handle})
  \item \texttt{viðföng} er 0 eða fleiri breytur sem unnið er með
  \item \texttt{fallsegð} er ein Matlab-segð (e. \emph{expression})
 \end{itemize}
\end{itemize}
\end{frame}

\begin{frame}[fragile]{Dæmi um nafnlaus föll}
\begin{minted}{matlab}
>> circleArea = @(r) pi*r.^2 % Fallið skilgreint
circleArea =
@(r) pi * r .^ 2
>> circleArea(10) % Fallið notað
ans =  314.16
>> circleArea(1:3) % Notuðum .^, svo vigrar virka
ans =
    3.1416   12.5664   28.2743
\end{minted}
\end{frame}

\begin{frame}[fragile]{Dæmi um nafnlaus föll}
\begin{minted}{matlab}
>> f = @(x,y) y*sin(x) + x*cos(y); % Tvö inntök
>> f(1,2)
ans =
    1.2668
\end{minted}
\end{frame}

\begin{frame}[fragile]{Dæmi um nafnlaus föll}
Nafnlaust fall án inntaksbreytu:
\begin{minted}{matlab}
>> randomPrint = @() fprintf('%.2f\n', rand());
>> randomPrint() % Kallað á fallið
0.81
>> randomPrint % Engir svigar - fallið bara sýnt
randomPrint = 
    @()fprintf('%.2f\n',rand())
\end{minted}
\end{frame}

\section{Fallshandföng (10.2)}

\begin{frame}[fragile]{Fallshandföng}
\vspace{\baselineskip}
\begin{itemize}
 \item Fallshandfang er leið til að vísa til falls
 \begin{itemize}
  \item Koma líka fyrir í öðrum samhengjum en þegar nafnlaus föll eru búin til
 \end{itemize}
 \item Getum fengið handfang fyrir ýmis föll (þar á meðal innbyggð)
 \begin{itemize}
  \item Notum til þess virkjann (e. \emph{operator}) @
 \end{itemize}
\end{itemize}
\begin{minted}{matlab}
>> sinHandle = @sin
sinHandle = 
    @sin
>> sinHandle(pi/4)
ans =
    0.7071
\end{minted}
\end{frame}

\begin{frame}{Af hverju fallshandföng?}
\begin{itemize}
 \item Hægt er að vinna með fallshandföng eins og hver önnur gildi, t.d. sett þau sem viðföng í önnur föll
 \item Þetta kallast að föll séu ``fyrsta stigs hlutir'' (e. \emph{first class objects})
 \begin{itemize}
  \item Þýðir að þeir hegða sér eins og grunnhlutir í málinu
  \item Hægt í sumum forritunarmálum: Scheme, Javascript, Python\ldots
  \item Ekki hægt í öðrum: Java, C
 \end{itemize}
\end{itemize}
\end{frame}

\begin{frame}{Dæmi um fallshandföng}
Lítum á \texttt{fnfnexamp.m} a blaðsíðu 324!

Fallið tekur við fallshandfangi og teiknar það á ákveðnu bili.
\end{frame}

\begin{frame}[fragile]{Hagnýting handfanga: \texttt{fplot} fallið}
Fallið \texttt{fplot} teiknar fall á gefnu bili. Almennt snið:
\begin{verbatim}
>> fplot(fallshandfang, [lággildi hágildi])
\end{verbatim}
Fallið sem handfangið vísar til verður að taka við einu gildi og skila einu gildi. Það verður líka að ráða við að inntakið sé vigur.
\end{frame}

\begin{frame}{Fyrirlestraræfing}
\begin{enumerate}
 \item Skilgreinið nafnlaust fall til að reikna rúmmál kúlu
 \[
  \text{formúla: } V = \frac{4}{3} \pi r^3
 \]
 \item Notið \texttt{fplot} til að teikna $f(x) = x\cos(5x)$ á bilinu 0 til 10
\end{enumerate}
\end{frame}

\section{Breytilegur fjöldi viðfanga (10.3)}

\begin{frame}{Breytilegur fjöldi viðfanga}
\begin{itemize}
 \item Hingað til hafa öll föll sem við skrifum tekið fastan fjölda viðfanga
 \item Matlab leyfir breytilegan fjölda inntaksbreyta og breytilegan fjölda skilabreyta
 \begin{itemize}
  \item Notum hólfavigrana \texttt{varargin} og \texttt{varargout} til að geyma inntaks- og skilabreyturnar
  \item Fjöldinn fæst með \texttt{nargin} og \texttt{nargout}
 \end{itemize}
\end{itemize}
\end{frame}

\begin{frame}[fragile]{Dæmi um \texttt{nargin} og \texttt{varargin}}
\begin{minted}[frame=lines, fontsize=\small]{matlab}
function varDisp(firstArg, varargin)
fprintf('Fjöldi viðfanga er %d\n', nargin)
fprintf('Fasta viðfangið er %d\n', firstArg)

fprintf('Fjöldi aukaviðfanga er %d, þau eru: \n', nargin-1)
for i = 1:(nargin-1)
    fprintf('%10d\n', varargin{i})
end
end
\end{minted}
\end{frame}

\begin{frame}{Dæmi um \texttt{nargin} og \texttt{varargin}}
Lítum á \texttt{areafori.m} á blaðsíðu 327!

Fallið reiknar út flatarmál hrings með radíus gefinn í fetum. Það tekur við einu eða tveimur inntaksbreytum. Ef seinni breytan er gefin og með gildið \texttt{'i'} skilar fallið niðurstöðunni í fertommum, annars ferfetum.
\end{frame}

\begin{frame}[fragile]{Breytilegur fjöldi skilabreyta}
Getum skilað vigri af breytilegri lengd. Almennt snið:
\begin{verbatim}
function [skilA, skilB, ..., varargout] = fall(inntök)
  ...
end
\end{verbatim}
\texttt{varargout} er hólfavigur sem fallið fyllir í eftir þörfum.
\end{frame}

\begin{frame}{Dæmi um \texttt{varargout}}
Lítum á \texttt{typesize.m} blaðsíðu 330!

Fallið skilar 's' ef inntakið er skalarbreyta, 'v' ef það er vigur, og 'm' ef það er fylki. Það skilar einnig víddunum ef það er vigur eða fylki.
\end{frame}

\section{Hreiðruð föll (10.4)}

\begin{frame}{Hreiðruð föll}
\begin{itemize}
 \item Matlab leyfir hreiðruð föll (e. \emph{nested functions})
 \begin{itemize}
  \item Þá er skilgreining eins falls algjörlega innan í öðru falli
  \item Notað til að skipuleggja forrit
 \end{itemize}
\end{itemize}
\end{frame}

\begin{frame}[fragile]{Dæmi um hreiðrað fall}
Skoðum \texttt{nestedVolume.m} á blaðsíðu 333. Það reiknar út rúmmál tenings.
\end{frame}


\begin{frame}{Gildissvið breyta}
\vspace{\baselineskip}
\begin{itemize}
 \item Gildissvið breytu (e. \emph{scope of a variable}) er gildissvið ysta falls sem notar hana
 \begin{itemize}
  \item Innra fall hefur aðgang að breytum ytra falls
  \item Ytra fall hefur aðgang að breytum innra falls
  \item Ef ytra fall notar breytu innra falls ekki, þá er hún staðvær (e. \emph{local}) í innra falli
 \end{itemize}
 \item Breytur í ytri föllum virka því sem víðværar (e. \emph{global}) breytur í innri föllum - ekki þarf að senda þær inn í föllin sem inntaksbreytur
 \item Bókin eyðir ekki miklu púðri í þessar pælingar - \href{http://se.mathworks.com/help/matlab/matlab_prog/nested-functions.html?refresh=true}{skjölunin á Matlab-síðunni} er með meira fyrir fróðleiksfúsa
\end{itemize}
\end{frame}

\begin{frame}[fragile]{Fyrirlestraræfing}
\begin{columns}
\column{0.55\textwidth}
\begin{enumerate}
 \setcounter{enumi}{2}
 \item Sýnið haus á falli sem getur fengið núll eða fleiri inntaksbreytur
 \item Hvernig má skrifa fall sem getur tekið 0, 1 eða 2 inntaksbreytur, en skilar villu ef þær eru fleiri?
 \item Hvert er gildi skilabreytanna \texttt{a} og \texttt{b} í fallinu hér til hliðar?
\end{enumerate}
\column{0.45\textwidth}
\begin{minted}[frame=lines, fontsize=\small]{matlab}
function [a, b] = nestTest()
a = nested1();
b = nested2();

    function x = nested1()
        x = 1;
    end

    function x = nested2()
        x = 2;
    end
end
\end{minted}

\end{columns}
\end{frame}


\end{document}
