\documentclass{beamer}

\usepackage{Haust2016glærur}

\title{Tölvunarfræði 1a}
\subtitle{Vika 11, seinni fyrirlestur}

\begin{document}

\begin{frame}
\titlepage
\end{frame}

\section{Inngangur}

\begin{frame}{Í síðasta þætti\ldots}
\begin{itemize}
 \item Hlutbundin forritun
 \item Teiknihlutir
 \item Klasar
 \begin{itemize}
  \item Yfirskrifun aðferða og virkja
 \end{itemize}
\end{itemize}
Kaflar: 11.1 til 11.3
\end{frame}

\begin{frame}[fragile]{Upprifjun}
\begin{minted}[frame=lines, fontsize=\small]{matlab}
classdef Fraction

    properties
        n % Teljari / numerator
        d % Nefnari / denomenator
    end
    
    methods
        function frac = Fraction(numerator, denomenator)
            frac.n = numerator;
            frac.d = denomenator;
        end
    end
    
end
\end{minted}
\end{frame}

\begin{frame}[fragile]{Upprifjun}
Tveimur aðferðum bætt við (undir ``methods'')
\begin{minted}[frame=lines]{matlab}
function disp(fr)
    fprintf('%d/%d\n', fr.n, fr.d)
end

function newFrac = plus(frac1, frac2)
    newN = frac1.n * frac2.d + frac2.n * frac1.d;
    newD = frac1.d * frac2.d;
    newFrac = Fraction(newN, newD);
end
\end{minted}
\end{frame}

\section{Erfðir}

\begin{frame}{Annað dæmi}
\begin{itemize}
 \item Skoðum annað dæmi um hlut sem tákna má með hlutbundinni forritun - rétthyrning
 \begin{itemize}
  \item Látum rétthyrning hafa eiginleika:
  \begin{itemize}
   \item Lengd
   \item Breidd
  \end{itemize}
  \item Látum rétthyrning hafa aðferðir:
  \begin{itemize}
   \item Reikna flatarmál
   \item \texttt{disp}-aðferð
   \item Samanburðaraðferð sem segir hvor tveggja rétthyrninga sé minni að flatarmáli
  \end{itemize}
 \end{itemize}
 \item Sýnidæmi - bls. 383 í 4. útgáfu bókarinnar
\end{itemize}
\end{frame}

\begin{frame}{Erfðir}
\begin{itemize}
 \item Við getum notað einn klasa til að mynda annan
 \item Látum þá einn klasa erfa (e. \emph{inherit}) aðferðir og eiginleika frá öðrum
 \begin{itemize}
  \item Klasinn sem erfir er þá undirklasi (e. \emph{subclass})
  \item Klasinn sem erft er frá er þá yfirklasi (e. \emph{superclass})
 \end{itemize}
\end{itemize}
\end{frame}

\begin{frame}[fragile]{Að erfa}
\begin{itemize}
 \item Til að skilgreina undirklasa:
\begin{minted}{matlab}
classdef UndirKlasi < Yfirklasi
\end{minted}
 \item Til að vísa í smið yfirklasa úr undirklasa:
\begin{minted}{matlab}
obj = obj@MySuperClass(SuperClassArguments);
\end{minted}
\item Til að vísa í aðra aðferð í yfirklasa:
\begin{minted}{matlab}
disp@MySuperClass(obj)
\end{minted}
\end{itemize}
\end{frame}


\begin{frame}{Sýnidæmi}
Búum til nýjan klasa sem notar rétthyrningsklasann til að tákna kassa.

\vspace{1cm}
Skoðum bls. 385 í 4. útgáfu bókarinnar
\end{frame}

\section{Handfangsklasar og gildisklasar}

\begin{frame}{Tvær gerðir klasa}
\begin{itemize}
 \item Í Matlab höfum við tvær gerðir klasa
 \item Gildisklasar (e. \emph{value classes})
 \begin{itemize}
  \item Ef við tökum ekkert annað fram, þá er klasi sem við búum til gildisklasi
  \item Þegar við notum gildisklasa til að búa til nýja hluti, þá er hver hlutur sem búinn er til óháður öðrum
 \end{itemize}
 \item Handfangsklasar (e. \emph{handle classes})
 \begin{itemize}
  \item Handfangsklasi er búinn til með því að erfa frá innbyggða klasanum \texttt{handle}
  \item Allir hlutir sem búnir eru til með handfangsklasa vísa í sama fyrirbrigðið
 \end{itemize}
\end{itemize}
\end{frame}

\section{``Attributes''}

\begin{frame}{Aðgangsstýring}
\begin{itemize}
 \item Við getum stýrt því hvaðan eiginleikar og aðferðir hvers klasa eru aðgengileg
 \item Erum þá með margar \texttt{properties} og/eða \texttt{methods} blokkir
 \item Þetta getur haldið fólki frá slæmum forritunarvenjum og/eða villum
\end{itemize}
\end{frame}

\begin{frame}{Aðgangsstýring eiginleika}
\begin{itemize}
 \item Aðgangsmöguleikar eiginleika:
 \begin{itemize}
  \item \texttt{public} - eiginleikinn er aðgengilegur alls staðar frá (sjálfgefið)
  \item \texttt{private} - eiginleikinn er eingöngu aðgengilegur innan sama klasa
  \item \texttt{protected} - eiginleikinn er eingöngu aðgengilegur innan sama klasa eða í undirklösum hans 
 \end{itemize}
 \item Gerðir aðgangs:
 \begin{itemize}
  \item \texttt{GetAccess}: Lesaðgangur að eiginleika
  \item \texttt{SetAccess}: Skrifaðgangur að eiginleika
  \item \texttt{Access}: Bæði les- og skrifaðgangur
 \end{itemize}
 \item Dæmi: Blokkin \texttt{properties (SetAccess = protected)} inniheldur eiginleika sem aðeins eru breytanlegir innan sama klasa, en hægt er að skoða alls staðar frá
\end{itemize}
\end{frame}

\begin{frame}{Aðgangsstýring aðferða}
\begin{itemize}
 \item Hægt er að stilla aðgang aðferða með lykilorðunum \texttt{public}, \texttt{private} og \texttt{protected}
 \item Eina gerðin er \texttt{Access}, sem stýrir því hvaðan er hægt að kalla á aðferðina
 \item Dæmi: Blokkin \texttt{methods (Access = protected)} inniheldur aðferðir sem aðeins er hægt að kalla á úr sama klasa
\end{itemize}
\end{frame}

\begin{frame}{Fleiri ``attributes''}
\begin{itemize}
 \item Fleiri lykilorð eru til sem stýra hegðun aðferða
 \item Þau taka öll \texttt{logical} gildi og \texttt{false} er sjálfgefið
 \begin{itemize}
  \item \texttt{Abstract} - ef satt, þá er aðferðin ekki útfærð (heldur þarf undirklasa)
  \item \texttt{Hidden} - ef satt, þá er aðferðin ekki sýnd þegar beðið er um lista af aðferðum
  \item \texttt{Sealed} - ef satt er ekki hægt að endurskilgreina aðferðina í undirklösum
  \item \texttt{Static} - ef satt tilheyrir aðferðin klasanum sjálfum frekar en hlutum sem hann býr til
 \end{itemize}

\end{itemize}

\end{frame}


\begin{frame}[fragile]{Fyrirlestraræfing}
Skráið ykkur inn á \url{http://socrative.com/} og klárið æfinguna.

Herbergisnúmer = \texttt{TOL105G2016}

Notendanafn = HÍ-tölvupóstfang
\end{frame}


\end{document}
