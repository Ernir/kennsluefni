\documentclass{beamer}

\input{../Haust2015glærur}

\title{Tölvunarfræði 1a}
\subtitle{Vika 14, seinni fyrirlestur}

\begin{document}

\begin{frame}
\titlepage
\end{frame}

\section{Inngangur}

\begin{frame}{Í síðasta þætti\ldots}
\begin{itemize}
 \item Ýmis grunnatriði í forritun, borið saman við Java og Python
\end{itemize}
Kaflar: 1-6
\end{frame}

\section{Námsmat}

\begin{frame}{Námsmat}
\begin{itemize}
 \item Vikuleg heimaverkefnaskil
 \begin{itemize}
  \item Gilda 20\% samtals
  \item \textbf{9} bestu verkefni gilda til einkunnar
 \end{itemize}
 \item Fyrirlestraæfingar
 \begin{itemize}
  \item Gilda 10\% samtals
  \item 20 skil gefa fulla einkunn
 \end{itemize}
 \item Miðmisserispróf
 \begin{itemize}
  \item Gildir 20\% ef það leiðir til hækkunar, annars 0\%
 \end{itemize}
 \item Lokapróf
 \begin{itemize}
  \item Gildir 50\% ef miðmisserispróf leiðir til hækkunar, annars 70\%
  \item Lágmarkseinkunn á prófi og námskeiðinu í heild er 5 (\href{https://ugla.hi.is/kennsluskra/?tab=skoli&chapter=content&id=34690}{prófreglur})
  \item Fyrirhugað að gefa í heilum og hálfum
 \end{itemize}
\end{itemize}
\end{frame}

\section{Efni til prófs}

\begin{frame}{Námsefni til lokaprófs}
\begin{itemize}
 \item Allar glærur
 \item Öll verkefni (skila, dæmatíma- og fyrirlestra)
 \item Kaflar 1-11 í kennslubókinni, með eftirfarandi undantekningum:
 \begin{itemize}
  \item Kafli 3.8 (föll sem skipanir)
  \item Kafli 4.5 (\texttt{menu} fallið)
  \item Kafli 6.3 (\texttt{menu} + einingaforritun)
  \item Kafli 6.4.1 (\texttt{persistent} breytur)
  \item Kafli 6.5 (aflúsun - fórum yfir, en er ekki til prófs)
  \item Kafli 8.2.5 (færslur af færslum)
  \item Kafli 11.7 (vistun mynda)
 \end{itemize}
 \item Kaflar 12.1, 12.2 og 14.1 í kennslubókinni  
\end{itemize}
\end{frame}

\section{Dæmi}

\begin{frame}{Lítum á dæmi}
\begin{itemize}
 \item Haustpróf 2008, dæmi 6
 \item Haustpróf 2011, dæmi 5, liður a)
 \item \texttt{parentfun.m}
 \item Dæmi 7, kafla 10 í bók
 \item Haustpróf 2010, dæmi 4
\end{itemize}

\end{frame}


\end{document}
