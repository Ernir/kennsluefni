\documentclass{beamer}

\usepackage{Haust2016glærur}

\title{Tölvunarfræði 1a}
\subtitle{Vika 9, fyrri fyrirlestur}

\begin{document}

\begin{frame}
\titlepage
\end{frame}

\section{Inngangur}

\begin{frame}{Í síðasta þætti\ldots}
\begin{itemize}
 \item Færslur
\end{itemize}
Kafli: 8.2
\end{frame}

\section{Skráarvinnsla (9.3)}

\subsection{Upprifjun}

\begin{frame}{Upprifjun Einfalt I/O}
\begin{itemize}
 \item Þegar unnið er með mikið gagnamagn er gott að geyma gögn í skrám
 \item Skipunin \texttt{save skráarnafn fylki -ascii} vistar gögnin í $fylki$ í skrá
 \begin{itemize}
  \item Ef skráin er þegar til er hún yfirskrifuð
  \item Skrifa má aftast í skrá í stað yfirskriftar með því að gefa valmöguleikann \texttt{-append}
 \end{itemize}
 \item Skipunin \texttt{load skráarnafn} les innihald skrár inní fylki
  \begin{itemize}
   \item Fylkið heitir sama nafni og skráin (án endingar)
  \end{itemize}
 \end{itemize}
\end{frame}

\begin{frame}[fragile]{Dæmi um skráarinntak og skráarúttak}
\begin{minted}[frame=lines]{matlab}
>> a = randi(5,2,2) % 2x2 slembifylki búið til
a =
   3   2
   3   4
>> save gogn.dat a -ascii % a skrifað í skrána gogn.dat
>> load gogn.dat % innihald gogn.dat sett í breytuna gogn
>> gogn
gogn =
   3   2
   3   4
\end{minted}
\end{frame}

\subsection{Lestur úr skrám}

\begin{frame}{Skráarvinnsla}
\begin{itemize}
 \item \texttt{load} og \texttt{save} duga oft, en stundum þarf öflugri aðferðir
 \item Í almennri skráarvinnslu felst:
 \begin{enumerate}
  \item Opna skrá
  \item Skrifa í skrá/lesa úr skrá
  \item Loka skrá
 \end{enumerate}
\end{itemize}
\end{frame}

\begin{frame}[fragile]{Opna skrá}
\begin{itemize}
 \item Almennt form:
\begin{minted}[frame=lines]{matlab}
>> fid = fopen('skráarnafn'); % eða
>> fid = fopen('skráarnafn', 'heimildir');
\end{minted}
 \item Skilagildið er -1 ef ekki tókst að opna skrána:
\begin{minted}[frame=lines]{matlab}
fid = fopen('daemi.dat');
if fid == -1
  error('Tókst ekki að opna skrá');
else
  % Skrifa kóða sem vinnur með skrána
end
\end{minted}
\end{itemize}
\end{frame}

\begin{frame}[fragile]{Loka skrá}
\begin{itemize}
 \item Þegar við höfum lokið við að nota skrá, þá eigum við að loka henni.
\begin{minted}[frame=lines]{matlab}
>> res = fclose(fid); % Lokar skrá með auðkennið fid
\end{minted}
\begin{minted}[frame=lines]{matlab}
>> res = fclose('all'); % Lokar öllum opnum skrám
\end{minted}
 \item Ef ekki tókst að loka þá er skilagildið -1
 \begin{itemize}
  \item Gerist reyndar ekki oft að slíkt mistakist
 \end{itemize}
\end{itemize}
\end{frame}

\begin{frame}{Að lesa úr skrá}
Getum lesið skrár á tvo vegu:
\begin{enumerate}
 \item Línu fyrir línu (með notkun \texttt{fgetl}) (\textbf{f}ile \textbf{get} \textbf{l}ine)
 \begin{itemize}
  \item Fáum þá hverja línu sem streng
  \item Þurfum lykkju til að lesa alla skrána
 \end{itemize}
 \item Alla skrána inn í eina gagnagrind (með \texttt{textscan} eða \texttt{fscanf})
 \begin{itemize}
  \item Gefum snið (e. \emph{format}) skráarinnar í skipuninni
 \end{itemize}
\end{enumerate}
\end{frame}


\begin{frame}[fragile]{Notkun á fgetl}
\vspace{\baselineskip}
Notum lykkju til að lesa. Fallið \texttt{feof} (\textbf{f}ile \textbf{e}nd \textbf{o}f \textbf{f}ile) segir til um hvort við erum komin út á enda skráarinnar.
\begin{minted}[frame=lines]{matlab}
fid = fopen('skráarheiti');
if fid ~= -1
    while ~feof(fid)
        line = fgetl(fid);
        % Gerum eitthvað við línuna
    end
else
    error('Ekki tókst að opna skrá');
end
\end{minted}
\end{frame}

\begin{frame}[fragile]{Upprifjun: Að skipta upp streng}
\begin{itemize}
 \item Oft þegar verið er að lesa inn gögn þurfum við að skipta streng upp í búta
 \item Höfum tvær aðalleiðir: 
 \begin{enumerate}
  \item \texttt{strtok} skiptir streng upp í tvo búta, ``haus'' og ``hala''
  \begin{itemize}
   \item Leitar að ``skiptitákni'' - allt sem er fyrir framan táknið er hausinn, afgangur strengsins er halinn
   \item Sjálfgefið skiptitákn er bilstafur
  \end{itemize}
  \item \texttt{strsplit} skiptir streng upp í hólfavigur af strengjum
  \begin{itemize}
   \item Inntaksstrengurinn klipptur í sundur á öllum stöðum þar sem skiptitáknið kemur fyrir
  \end{itemize}
 \end{enumerate}
\end{itemize}
\end{frame}

\begin{frame}[fragile]{Skrá sem við munum vinna með}
Áður en hafist er handa við að lesa inn skrá þurfum við að ákveða skrá til að vinna með.

Skráin \texttt{subjexp.dat}, í bók (bls. 299 í 3. útgáfu, 323 í 4. útgáfu), er eftirfarandi:
\begin{minted}[frame=lines]{matlab}
5.3	a
2.2	b
3.3	a
4.4	a
1.1	b
\end{minted}

\end{frame}


\begin{frame}[fragile]{Dæmi}
Forritið byggt á forriti í bók (bls. 300 í 3. útgáfu, 323 í 4. útgáfu)
\begin{minted}[frame=lines]{matlab}
fid = fopen('subjexp.dat');
if fid ~= -1
    while ~feof(fid)
        line = fgetl(fid);
        [number, character] = strtok(line);
        fprintf('%.2f%s\n', str2double(number), character)
    end
else
    error('Ekki tókst að opna skrá');
end
fclose(fid);
\end{minted}
\end{frame}

\begin{frame}{Fyrirlestraræfing}
Skráið ykkur inn á \url{http://socrative.com/} og fyrstu tvær spurningarnar.

Herbergisnúmer = \texttt{TOL105G2016}

Notendanafn = HÍ-tölvupóstfang
\end{frame}

\begin{frame}[fragile]{\texttt{textscan}}
\begin{itemize}
 \item Fallið \texttt{textscan} les alla skrána inn í hólfavigur
\begin{minted}[frame=lines]{matlab}
>> cellArray = textscan(fid, 'snið')
\end{minted}
 \item Sniðið getur innihaldið sniðslýsingar á svipaðan hátt og \texttt{fprintf}
 \item Úttakið er hólfavigur, með eitt hólf fyrir hvern dálk í skránni
\end{itemize}
\end{frame}

\begin{frame}[fragile]{Dæmi um notkun \texttt{textscan}}
\begin{minted}[frame=lines]{matlab}
>> fid = fopen('subjexp.dat');
>> cellArray = textscan(fid, '%f %c')
cellArray = 
    [5x1 double]    [5x1 char]
>> cellArray{1}'
ans =
    5.3000    2.2000    3.3000    4.4000    1.1000
\end{minted}
Í sniðslýsingunni: \texttt{f} fyrir double, \texttt{c} fyrir char
\end{frame}

\begin{frame}[fragile]{\texttt{textscan} fyrir strengi}
Hægt er að nota \texttt{textscan} fallið á streng til að skipta honum niður í hólfavigur:
\begin{minted}[frame=lines]{matlab}
>> myString = '0.45 abc 123';
>> secondCellArray = textscan(myString, '%f %s %d')
secondCellArray = 
    [0.4500]    {1x1 cell}    [123]
\end{minted}
\end{frame}

\begin{frame}[fragile]{\texttt{fscanf}}
Fallið \texttt{fscanf} er svipað \texttt{textscan}, en skilar fylki.

\begin{minted}{matlab}
>> fid = fopen('subjexp.dat');
>> mat = fscanf(fid, '%f %c', [2, inf])
mat =
    5.3000    2.2000    3.3000    4.4000    1.1000
   97.0000   98.0000   97.0000   97.0000   98.0000
\end{minted}


\end{frame}


\subsection{Að skrifa í skrár}

\begin{frame}[fragile]{\texttt{fprintf} í skrá}
\begin{itemize}
 \item Aðalskipunin til að skrifa í skrár er \texttt{fprintf}
 \begin{itemize}
  \item Þetta er sama \texttt{fprintf} fall og við könnumst við
  \item Til að skrifa í skrána þarf að láta fallið fá skráarauðkenni sem fyrsta viðfang
 \end{itemize}
 \item Notkunin verður þá á borð við \verb|fprintf(fid, 'snið', breyta1, breyta2...)|
\end{itemize}
\end{frame}

\begin{frame}[fragile]{\texttt{fprintf} í skrá}
Dæmi:
\begin{minted}[frame=lines]{matlab}
fid = fopen('tilraun.txt', 'w');
for i = 1:5
    fprintf(fid, '%d: %d\n', i, randi(50));
end
fclose(fid);
\end{minted}
Ath. að hér er skráin \emph{yfirskrifuð} í hvert skipti. Til að skrifa aftast í skrá þarf að gefa \texttt{fopen} aðrar heimildir.
\end{frame}

\begin{frame}{Skráarheimildir}
\begin{center}
\begin{tabular}{ll}
\toprule
Regla&Lýsing\\
\midrule
\texttt{'r'}&Opnar skrá til lesturs\\
\texttt{'w'}&Opnar eða býr til skrá til skriftar, yfirskrifar\\
\texttt{'a'}&Opnar eða býr til skrá til skriftar, skrifar aftast\\
\texttt{'r+'}&Opnar skrá til skriftar eða lesturs\\
\texttt{'w+'}&Opnar eða býr til skrá til skriftar eða lesturs, yfirskrifar\\
\texttt{'a+'}&Opnar eða býr til skrá til skriftar eða lesturs, skrifar aftast\\
\bottomrule
\end{tabular}

\end{center}

\end{frame}

\begin{frame}[fragile]{Fyrirlestraræfing}
\begin{columns}
\column{0.7\textwidth}
Skráið ykkur inn á \url{http://socrative.com/} og klárið æfinguna.

Herbergisnúmer = \texttt{TOL105G2016}

Notendanafn = HÍ-tölvupóstfang
\column{0.3\textwidth}
\begin{center}
\begin{tabular}{ll}
\toprule
Klst&$C^\circ$\\
\midrule
0&12.5\\
3&12.4\\
6&12.3\\
9&12.8\\
12&13.4\\
15&14.0\\
18&13.1\\
21&12.8\\
\bottomrule
\end{tabular}
\end{center}
\end{columns}
\end{frame}



\end{document}
