\documentclass[handout]{beamer}

\usepackage{Haust2017glærur}

\title{Tölvunarfræði 1a}
\subtitle{Vika 10, fyrri fyrirlestur}

\begin{document}

\begin{frame}
\titlepage
\end{frame}

\section{Inngangur}

\begin{frame}{Í síðasta þætti\ldots}
    \begin{itemize}
        \item Að skrifa í skrár
        \item Excel-skrár
        \item \texttt{.mat} skrár
    \end{itemize}
\end{frame}

\section{Nafnlaus föll (10.1)}

\begin{frame}[fragile]{Nafnlaus föll}
\begin{itemize}
 \item Nafnlaus föll (e. \emph{anonymous functions}) eru einföld einnar línu föll
 \begin{itemize}
  \item Þurfa ekki að vera í sérstakri \texttt{.m}-skrá
  \item Geta einfaldað forritun töluvert
 \end{itemize}
 \item Almennt snið:
\begin{verbatim}
handfang = @(viðföng) fallsegð
\end{verbatim}
 \begin{itemize}
  \item Þar sem \texttt{handfang} er fallshandfang (e. \emph{function handle})
  \item \texttt{viðföng} er 0 eða fleiri breytur sem unnið er með
  \item \texttt{fallsegð} er ein Matlab-segð (e. \emph{expression})
 \end{itemize}
\end{itemize}
\end{frame}

\begin{frame}[fragile]{Dæmi um nafnlaus föll}
\begin{minted}{matlab}
>> circleArea = @(r) pi*r.^2 % Fallið skilgreint
circleArea =
@(r) pi * r .^ 2
>> circleArea(10) % Fallið notað
ans =  314.16
>> circleArea(1:3) % Notuðum .^, svo vigrar virka
ans =
    3.1416   12.5664   28.2743
\end{minted}
\end{frame}

\begin{frame}[fragile]{Dæmi um nafnlaus föll}
\begin{minted}{matlab}
>> f = @(x,y) y*sin(x) + x*cos(y); % Tvö inntök
>> f(1,2)
ans =
    1.2668
\end{minted}
\end{frame}

\begin{frame}[fragile]{Dæmi um nafnlaus föll}
Nafnlaust fall án inntaksbreytu:
\begin{minted}{matlab}
>> randomPrint = @() fprintf('%.2f\n', rand());
>> randomPrint() % Kallað á fallið
0.81
>> randomPrint % Engir svigar - fallið bara sýnt
randomPrint = 
    @()fprintf('%.2f\n',rand())
\end{minted}
\end{frame}

\section{Fallshandföng (10.2)}

\begin{frame}[fragile]{Fallshandföng}
\vspace{\baselineskip}
\begin{itemize}
 \item Fallshandfang er leið til að vísa til falls
 \begin{itemize}
  \item Koma líka fyrir í öðrum samhengjum en þegar nafnlaus föll eru búin til
 \end{itemize}
 \item Getum fengið handfang fyrir ýmis föll (þar á meðal innbyggð)
 \begin{itemize}
  \item Notum til þess virkjann (e. \emph{operator}) @
 \end{itemize}
\end{itemize}
\begin{minted}{matlab}
>> sinHandle = @sin
sinHandle = 
    @sin
>> sinHandle(pi/4)
ans =
    0.7071
\end{minted}
\end{frame}

\begin{frame}{Af hverju fallshandföng?}
\begin{itemize}
 \item Hægt er að vinna með fallshandföng eins og hver önnur gildi, t.d. sett þau sem viðföng í önnur föll
 \item Þetta kallast að föll séu ``fyrsta stigs hlutir'' (e. \emph{first class objects}) í Matlab
 \begin{itemize}
  \item Þýðir að þeir hegða sér eins og grunnhlutir í málinu
  \item Hægt í sumum forritunarmálum: Scheme, Javascript, Python\ldots
  \item Ekki hægt í öðrum: Java, C
 \end{itemize}
\end{itemize}
\end{frame}

\begin{frame}[fragile]{Dæmi um fallshandföng}
\begin{minted}[frame=lines, label='fnfnexamp.m']{matlab}
function fnfnexamp(funh)
% fnfnexamp receives the handle of a function
% and plots that function of x (which is 1:.25:6)
% Format: fnfnexamp(function handle)
x = 1:.25:6;
y = funh(x);
plot(x,y,'ko')
xlabel('x')
ylabel('fn(x)')
title(func2str(funh))
end
\end{minted}

\end{frame}

\begin{frame}[fragile]{Hagnýting handfanga: \texttt{fplot} fallið}
Fallið \texttt{fplot} teiknar fall á gefnu bili. Almennt snið:
\begin{verbatim}
>> fplot(fallshandfang, [minnstaX stærstaX])
\end{verbatim}
Fallið sem handfangið vísar til verður að taka við einu gildi og skila einu gildi. Það verður líka að ráða við að inntakið sé vigur.
\end{frame}

\begin{frame}{Fyrirlestraræfing}
    \begin{enumerate}
     \item Skilgreinið nafnlaust fall til að reikna rúmmál kúlu
     \[
      \text{formúla: } V = \frac{4}{3} \pi r^3
     \]
     \item Notið \texttt{fplot} til að teikna $f(x) = x\cos(5x)$ á bilinu 0 til 10
    \end{enumerate}
\end{frame}

\section{Breytilegur fjöldi viðfanga (10.3)}

\begin{frame}{Breytilegur fjöldi viðfanga}
\begin{itemize}
 \item Hingað til hafa öll föll sem við skrifum tekið fastan fjölda viðfanga
 \item Matlab leyfir breytilegan fjölda inntaksbreyta og breytilegan fjölda skilabreyta
 \begin{itemize}
  \item Notum hólfavigrana \texttt{varargin} og \texttt{varargout} til að geyma inntaks- og skilabreyturnar
  \item Fjöldinn fæst með \texttt{nargin} og \texttt{nargout}
 \end{itemize}
\end{itemize}
\end{frame}

\begin{frame}[fragile]{Dæmi um \texttt{nargin} og \texttt{varargin}}
\begin{minted}[frame=lines, fontsize=\small]{matlab}
function vardisp(firstArg, varargin)
fprintf('Fjöldi viðfanga er %d\n', nargin)
fprintf('Fasta viðfangið er %d\n', firstArg)

fprintf('Fjöldi aukaviðfanga er %d, þau eru: \n', nargin-1)
for i = 1:(nargin-1)
    fprintf('%10d\n', varargin{i})
end
end
\end{minted}
\end{frame}

\begin{frame}[fragile]{Dæmi um \texttt{nargin} og \texttt{varargin}}
\begin{minted}[frame=lines, label=areafori.m]{matlab}
function area = areafori(varargin)
n = nargin; % number of input arguments
radius = varargin{1}; % Given in feet by default
if n == 2
  unit = varargin{2};
  % if inches is specified, convert the radius
  if unit == 'i'
    radius = radius * 12;
  end
end
area = pi * radius .^ 2;
end
\end{minted}

\end{frame}

\begin{frame}[fragile]{Breytilegur fjöldi skilabreyta}
Getum skilað vigri af breytilegri lengd. Almennt snið:
\begin{verbatim}
function [skilA, skilB, ..., varargout] = fall(inntök)
  ...
end
\end{verbatim}
\texttt{varargout} er hólfavigur sem fallið fyllir í eftir þörfum.
\end{frame}

\begin{frame}[fragile]{Dæmi um \texttt{varargout}}
\begin{minted}[frame=lines, label=mysize.m]{matlab}
function [row, col, varargout] = mysize(mat)
% mysize returns dimensions of input argument
% and possibly also total # of elements
% Format: mysize(inputArgument)
[row, col] = size(mat);
if nargout == 3
  varargout{1} = row*col;
end
end
\end{minted}

\end{frame}

\section{Hreiðruð föll (10.4)}

\begin{frame}{Hreiðruð föll}
\begin{itemize}
 \item Matlab leyfir hreiðruð föll (e. \emph{nested functions})
 \begin{itemize}
  \item Þá er skilgreining eins falls algjörlega innan í öðru falli
  \item Notað til að skipuleggja forrit
 \end{itemize}
\end{itemize}
\end{frame}

\begin{frame}[fragile]{Dæmi um hreiðrað fall}
\begin{minted}[frame=lines, label=nestedvolume.m]{matlab}
function outvol = nestedvolume(len, wid, ht)
% nestedvolume receives the lenght, width, and
% height of a cube and returns the volume; it calls
% a nested function that returns the area of the base
% Format: nestedvolume(length,width,height)
outvol = base * ht;
  function outbase = base
  % returns the area of the base
  outbase = len * wid;
  end % base function
end % nestedvolume function
\end{minted}

\end{frame}


\begin{frame}{Gildissvið breyta}
\vspace{\baselineskip}
\begin{itemize}
 \item Gildissvið breytu (e. \emph{scope of a variable}) er gildissvið ysta falls sem notar hana
 \begin{itemize}
  \item Innra fall hefur aðgang að breytum ytra falls
  \item Ytra fall hefur aðgang að breytum innra falls
  \item Ef ytra fall notar breytu innra falls ekki, þá er hún staðvær (e. \emph{local}) í innra falli
 \end{itemize}
 \item Breytur í ytri föllum virka því sem víðværar (e. \emph{global}) breytur í innri föllum - ekki þarf að senda þær inn í föllin sem inntaksbreytur
 \item Bókin eyðir ekki miklu púðri í þessar pælingar - \href{http://se.mathworks.com/help/matlab/matlab_prog/nested-functions.html?refresh=true}{skjölunin á Matlab-síðunni} er með meira fyrir fróðleiksfúsa
\end{itemize}
\end{frame}

\begin{frame}[fragile]{Fyrirlestraræfing}
    \begin{columns}
    \column{0.55\textwidth}
    \begin{enumerate}
     \setcounter{enumi}{2}
     \item Sýnið haus á falli sem getur fengið núll eða fleiri inntaksbreytur
     \item Hvernig má skrifa fall sem getur tekið 0, 1 eða 2 inntaksbreytur, en skilar villu ef þær eru fleiri?
     \item Hvert er gildi skilabreytanna \texttt{a} og \texttt{b} í fallinu hér til hliðar?
    \end{enumerate}
    \column{0.45\textwidth}
    \begin{minted}[frame=lines, fontsize=\small]{matlab}
function [a, b] = nestTest()
a = nested1();
b = nested2();

    function x = nested1()
        x = 1;
    end

    function x = nested2()
        x = 2;
    end
end
    \end{minted}
    
    \end{columns}
\end{frame}


\end{document}
