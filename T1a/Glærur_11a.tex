\documentclass{beamer}

\usepackage[utf8]{inputenc}
\usepackage[icelandic]{babel}
\usepackage[T1]{fontenc}

\usepackage{booktabs}
\usepackage{minted} %Minted and configuration
\usepackage{framed}
\usepackage{tikz}
\usemintedstyle{default}
\renewcommand{\theFancyVerbLine}{\sffamily \arabic{FancyVerbLine}}
\newcommand{\Mod}[1]{\ \text{mod}\ #1}

% \makeatletter
% \minted@define@extra{label}
% \makeatother

\usebackgroundtemplate%
{%
\vbox to \paperheight{
\includegraphics[width=\paperwidth]{Pics/hi-slide-head}

\vfill
\hspace{0.5cm}\includegraphics[width=0.3\paperwidth]{Pics/hi-von-logo}
\vspace{0.5cm}
    }%
}

% \makeatletter
% \newcommand{\minted@write@detok}[1]{%
%   \immediate\write\FV@OutFile{\detokenize{#1}}}%
%   
%   \newcommand{\minted@FVB@VerbatimOut}[1]{%
%   \@bsphack
%   \begingroup
%     \FV@UseKeyValues
%     \FV@DefineWhiteSpace
%     \def\FV@Space{\space}%
%     \FV@DefineTabOut
%     %\def\FV@ProcessLine{\immediate\write\FV@OutFile}% %Old, non-Unicode version
%     \let\FV@ProcessLine\minted@write@detok %Patch for Unicode
%     \immediate\openout\FV@OutFile #1\relax
%     \let\FV@FontScanPrep\relax
% %% DG/SR modification begin - May. 18, 1998 (to avoid problems with ligatures)
%     \let\@noligs\relax
% %% DG/SR modification end
%     \FV@Scan}
%     \let\FVB@VerbatimOut\minted@FVB@VerbatimOut
%     
%     \renewcommand\minted@savecode[1]{
%   \immediate\openout\minted@code\jobname.pyg
%   \immediate\write\minted@code{\expandafter\detokenize\expandafter{#1}}%
%   \immediate\closeout\minted@code}
%   
% \makeatother

\setbeamertemplate{navigation symbols}{}
\usecolortheme{dove}
\setbeamercolor{frametitle}{fg=white}
\hypersetup{colorlinks=true,pdfauthor={Eirikur Ernir Thorsteinsson},linkcolor=blue,urlcolor=blue}

\AtBeginSection[]
{
  \begin{frame}<beamer>
    \frametitle{Yfirlit}
    \tableofcontents[currentsection]
  \end{frame}
}

\author{Eiríkur Ernir Þorsteinsson}
\institute{Háskóli Íslands}
\date{Haust 2015}

\title{Tölvunarfræði 1a}
\subtitle{Vika 11, fyrri fyrirlestur}

\begin{document}

\begin{frame}
\titlepage
\end{frame}

\section{Inngangur}

\begin{frame}{Í síðasta þætti\ldots}
\begin{itemize}
 \item Hreiðruð föll
 \item Endurkvæmni
\end{itemize}
Kaflar: 10.4 - 10.5
\end{frame}

\begin{frame}{Næstu skilaverkefni}
\begin{itemize}
 \item Það fer að styttast í önninni
 \begin{itemize}
  \item Heimadæmum 9 skal skila 6. nóvember
  \item Heimadæmum 10 skal skila 13. nóvember
  \item Heimadæmum 11 skal skila 20. nóvember
  \item Ein aukadæmi verða eftir það
  \begin{itemize}
   \item Þau verða sérstök - gefa einkunnina 10 (skil) eða 0 (ekki-skil), ætluð til prófundirbúnings
  \end{itemize}
 \end{itemize}
 \item Bestu 9 skammtarnir gilda til einkunnar
\end{itemize}
\end{frame}

\section{Færslur (8.2)}

\begin{frame}{Skoðum kafla 8 aftur!}
\begin{itemize}
 \item Færslur (e. \emph{structures})
 \begin{itemize}
  \item Gagnagrind sem hópar gildi sem eiga saman
  \item Dæmi: Upplýsingar um vöru: vörunúmer, lýsing, innkaupsverð, söluverð
 \end{itemize}
 \item Færslur hafa svið (e. \emph{fields}) og gildi (e. \emph{values})
\end{itemize}
\end{frame}

\begin{frame}{Að smíða færslur}
\begin{itemize}
 \item Hægt er að búa til færslu á tvo mismunandi vegu
 \begin{itemize}
  \item Með því að nota \texttt{struct} fallið
  \item Með því að nota punktvirkja
 \end{itemize}
 \item Náð er í gögn með punktvirkjanum
\end{itemize}
\end{frame}

\begin{frame}[fragile]{\texttt{struct} fallið}
Í struct-fallinu koma til skiptis nafn sviðs og gildi sviðsins.
\begin{minted}{matlab}
>> student = struct('id', '12345', 'grades', [6 7 8])
student = 
        id: '12345'
    grades: [6 7 8]
\end{minted}
\end{frame}

\begin{frame}[fragile]{Að ná í gögn úr færslu}
Punktvirkjann (e. \emph{dot operator}) má nota til að ná í gögn úr færslu.
\begin{minted}{matlab}
>> student.id
ans =
12345
\end{minted}
\end{frame}

\begin{frame}[fragile]{Punktvirkinn}
\begin{itemize}
 \item Punktvirkjann má líka nota til að búa til færslur
 \item Þá skrifum við nafn breytu, punkt, og svo nafn sviðsins (breytan þarf ekki að vera skilgreind fyrirfram)
\begin{minted}{matlab}
>> student2.id = '23456';
>> student2.grades = [7 8 9];
\end{minted}
 \item Ef við viljum að \texttt{student2} sé af sömu gerð og \texttt{student} þá verðum við að nota nákvæmlega sömu sviðsnöfn og gagnagerðir
\end{itemize}
\end{frame}

\begin{frame}{Nánar um punktvirkjann}
\begin{itemize}
 \item Þegar punktvirkinn er notaður til að búa til færslubreytu þá verður breytan til smátt og smátt
 \begin{itemize}
  \item Einu sviði bætt við í einu
 \end{itemize}
 \item Þetta hefur svipuð vandamál í för með sér og þegar vigur er stækkaður inni í lykkju - nýjar minnisúthlutanir þurfa að eiga sér stað
 \begin{itemize}
  \item Það er því hraðvirkara að nota \texttt{struct}-fallið
 \end{itemize}
\end{itemize}
\end{frame}

\begin{frame}{Að prenta út færslur}
\begin{itemize}
 \item Fallið \texttt{disp} birtir öll svið færslu
 \item \texttt{fprintf} getur einungis birt einstök svið
 \begin{itemize}
  \item En það er e.t.v. ekki undarlegt - \texttt{fprintf} er nákvæmnistól
 \end{itemize}
\end{itemize}
\end{frame}

\begin{frame}{Færsluföll}
\begin{center}
\begin{tabular}{lp{7cm}}
\toprule
Nafn falls&Hlutverk\\
\midrule
\texttt{isstruct}&Skilar \texttt{true} ef inntak er færsla\\
\texttt{isfield}&Skilar \texttt{true} ef gefin færsla hefur svið með gefnu nafni\\
\texttt{fieldnames}&Skilar hólfavigri með nöfnum allra sviða í gefinni færslu\\
\bottomrule
\end{tabular}
\end{center}
\end{frame}

\begin{frame}{Færsluvigrar}
\begin{itemize}
 \item Oft viljum við geyma margar færslur af sömu gerð en með mismunandi gildum
 \begin{itemize}
  \item Dæmi: Nemendaskrá, vörulisti
 \end{itemize}
 \item Búum þá til vigur af færslum
 \begin{itemize}
  \item Munum: Allt í Matlab er fylki
 \end{itemize}
\end{itemize}
\end{frame}

\begin{frame}[fragile]{Að nota færsluvigur}
\begin{itemize}
 \item Getum notað færsluvigur eins og aðra vigra
 \begin{itemize}
  \item Vísum í sæti vigurs með númeri
  \item Vísum í svið færslu með punktvirkjanum
 \end{itemize}
\end{itemize}
\begin{minted}[fontsize=\small]{matlab}
>> products = struct('name', 'Torfstrekkjari', 'price', 12000);
>> products(2) = struct('name', 'Blómalím', 'price', 1500);
>> products.price
ans =
       12000
ans =
        1500
\end{minted}

\end{frame}

\begin{frame}{Fyrirlestraræfing}
\begin{enumerate}
 \item Búið til $1 \times 1$ færslubreytu sem inniheldur upplýsingar um nafn ykkar og kennitölu
 \item Búið til færsluvigur með a.m.k. 2 færslum sem inniheldur upplýsingar um nafn bjórs, nafn brugghúss, og seldan lítrafjölda
 \item Finnið heildarfjölda seldra lítra skv. færsluvigrinum í liðnum á undan með lykkju eða vigurkóðun
\end{enumerate}
\end{frame}

\section{Hlutbundin forritun}

\begin{frame}{Hlutbundin forritun}
\begin{itemize}
 \item Hingað til höfum við verið að raða gögnunum okkar niður í vigra, hólfafylki, færslur\ldots
 \begin{itemize}
  \item Föll taka gögnin síðan inn, vinna með þau og skila öðrum gögnum
 \end{itemize}
 \item Hægt er að nálgast forritun á annan hátt - hlutbundinn (e. \emph{object oriented})
 \begin{itemize}
  \item Hvaða fyrirbrigði/hlutir koma við sögu í vandamálinu?
  \item Hvernig hegða þessir hlutir sér?
 \end{itemize}
\end{itemize}
\end{frame}

\begin{frame}{Hvað er hlutur?}
\pause
\begin{itemize}
 \item Hvaða fyrirbrigði sem skilgreinast af gögnum og aðgerðum á þau geta verið hlutir
 \begin{itemize}
  \item Mjög opið hugtak! \pause
 \end{itemize}
 \item Dæmi um fyrirbrigði sem gætu verið hlutir í forriti \pause
 \begin{itemize}
  \item Allt sem sést í tölvuleikjum \pause
  \item Hringur
  \begin{itemize}
   \item Gögn/eiginleikar: Staðsetning miðpunkts, radíus
   \item Aðgerðir: Teikna, reikna flatarmál
  \end{itemize} \pause
  \item Almennt brot
  \begin{itemize}
   \item Gögn/eiginleikar: Nefnari, teljari
   \item Aðgerðir: Leggja saman, stytta\ldots
  \end{itemize}
 \end{itemize}
\end{itemize}
\end{frame}

\begin{frame}[fragile]{Hvernig búum við til hluti í Matlab?}
\begin{columns}
\column{0.5\textwidth}
\begin{itemize}
 \item Til að búa til hluti í Matlab notum við svokallaða klasa (e. \emph{classes})
 \item Matlab-klasi á heima í sér \texttt{.m} skrá. Hann inniheldur
 \begin{itemize}
  \item Eiginleika (e. \emph{properties}) sem lýsa gögnum klasans
  \begin{itemize}
   \item Þetta eru breytur
  \end{itemize}
  \item Aðferðir (e. \emph{methods}) sem lýsa því hvernig gögnin hegða sér
  \begin{itemize}
   \item Þetta eru föll
  \end{itemize}
 \end{itemize}
\end{itemize}
\column{0.5\textwidth}
\begin{minted}[frame=lines]{matlab}
classdef ClassName % Nafn

    properties
        % Eiginleikar hingað
    end
    
    methods
        % Aðferðir hingað
    end
    
end
\end{minted}

\end{columns}
\end{frame}

\begin{frame}[fragile]{Dæmi: Almennt brot}
\begin{minted}[frame=lines, fontsize=\small]{matlab}
classdef Fraction

    properties
        n % Teljari / numerator
        d % Nefnari / denomenator
    end
    
    methods
        function frac = Fraction(numerator, denomenator)
            frac.n = numerator;
            frac.d = denomenator;
        end
    end
    
end
\end{minted}

\end{frame}

\begin{frame}{Í Matlab}
\begin{center}
\includegraphics[width=\textheight]{Pics/fraction-workspace}
\end{center}
\end{frame}


\begin{frame}{Hvað vorum við að horfa á?}
\begin{itemize}
 \item Við vorum að skoða klasa með\ldots
 \item Tveimur eiginleikum
 \begin{itemize}
  \item \texttt{n} og \texttt{d}, sem tákna teljara og nefnara
  \item Einni aðferð, sem ``smíðar'' nýja hluti
  \begin{itemize}
   \item Slík aðferð er kölluð smiður (e. \emph{constructor})
   \item Notuð til að upphafsstilla gildi, athuga hvort að breyturnar séu af leyfilegum gerðum o.fl.
   \item Flestir klasar hafa smið
  \end{itemize}
 \end{itemize}
\end{itemize}
\end{frame}

\begin{frame}[fragile]{Fleiri aðferðir}
\begin{columns}
\column{0.45\textwidth}
\begin{itemize}
 \item Til að klasinn okkar virki vel þarf hann fleiri aðferðir
 \item Oftast viljum við að klasinn okkar prentist á skynsamlegan hátt út þegar við köllum á \texttt{disp} eða þegar við gefum breytum gildi af hans tagi
 \item Hvernig gerum við það?
 \begin{itemize}
  \item Bætum við \texttt{disp} aðferð á klasann!
 \end{itemize}
\end{itemize}
\column{0.55\textwidth}

\begin{minted}[frame=lines, fontsize=\small]{matlab}
function disp(fr)
    fprintf('%d/%d\n', fr.n, fr.d)
end
\end{minted}

Þetta fer inn í ``\texttt{methods}'' blokk klasans
\end{columns}
\end{frame}

\begin{frame}[fragile]{Látum klasann vinna vinnu}
\vspace{\baselineskip}
Við kunnum að leggja saman brot:
\[
\frac{c}{d} + \frac{a}{b} = \frac{cb+ad}{db} 
\]
Þetta má gera í klasanum okkar með því að bæta við aðferðinni
\begin{minted}[frame=lines]{matlab}
function newFrac = add(frac1, frac2)
    newN = frac1.n * frac2.d + frac2.n * frac1.d;
    newD = frac1.d * frac2.d;
    newFrac = Fraction(newN, newD);
end
\end{minted}
\end{frame}

\begin{frame}[fragile]{Af hverju hlutbundin forritun?}
\begin{itemize}
 \item Mörgum ``kerfum'' er auðvelt að lýsa með hlutbundnum aðferðum
 \begin{itemize}
  \item Sérstaklega kerfum sem eiga að herma eftir bókstaflegum hlutum
 \end{itemize}
 \item Ýmis atriði í hlutbundinni forritun hjálpa til við að gera kóða endurnýtanlegan og skilanlegri
 \begin{itemize}
  \item Erfðir, hjúpun, \ldots
 \end{itemize}
\end{itemize}
\end{frame}

\begin{frame}{Fyrirlestraræfing}
\begin{enumerate}
 \setcounter{enumi}{3}
 \item Breytið \texttt{disp} aðferðinni í \texttt{Fraction} klasanum okkar svo að tölur með 1 í nefnara séu sýndar án nefnara (t.d. sýna töluna \texttt{3} en ekki \texttt{3/1})
 \item Bætið margföldunaraðferð við \texttt{Fraction} klasann
\end{enumerate}

\end{frame}


\end{document}
