\documentclass[handout]{beamer}

\usepackage{Haust2017glærur}

\title{Tölvunarfræði 1a}
\subtitle{Vika 13, fyrri fyrirlestur}

\begin{document}

\begin{frame}
\titlepage
\end{frame}

\section{Inngangur}

\begin{frame}{Í síðasta þætti\ldots}
    \begin{itemize}
        \item Margliður
        \item Nálgun og brúun
    \end{itemize}
\end{frame}

\section{Tölfræði}

\begin{frame}{Tölfræðiföll í Matlab}
\begin{itemize}
 \item Matlab hefur mörg föll sem reikna tölfræði fyrir gagnarunur
 \item Gerum ráð fyrir að gögn séu í línuvigrinum $x$
\[
 x = [x_1,x_2, \ldots, x_n]
\]
 \item Tölfræðiföllin gefa okkur nánari upplýsingar um gerð og eiginleika gagna
 \begin{itemize}
  \item Verkfræðingar og raunvísindafólk þurfa oft að spyrja sig spurninga um gögnin
  \item Hvernig eru gögnin dreifð? Hver er þróunin? Eru þau öll jafn gild?
 \end{itemize}
\end{itemize}

\end{frame}

\begin{frame}[fragile]{Kunnuglegar slóðir}
Við höfum séð föllin \texttt{min} og \texttt{max}
\begin{minted}[frame=lines]{matlab}
>> x = [9 10 5 9 8 7 2 8 10 10];
>> min(x)
ans =  2
\end{minted}
Þau geta líka skilað staðsetningu stakanna
\begin{minted}[frame=lines]{matlab}
>> [maxVal, maxIndex] = max(x)
maxVal =  10
maxIndex =  2
\end{minted}
Hér fær \texttt{maxIndex} gildið ``staðsetning fyrsta hágildis''
\end{frame}

\begin{frame}[fragile]{Mörg hágildi}
Til að fá staðsetningar allra há- eða lággilda í vigri þurfum við að beita smá brellum. \pause
Möguleg leið:
\begin{minted}[frame=lines]{matlab}
>> x = [9 10 5 9 8 7 2 8 10 10];
>> find(x == max(x))
ans =
    2    9   10
\end{minted}
\end{frame}

\begin{frame}[fragile]{Samanburður vigra}
\texttt{min} og \texttt{max} geta tekið inn tvo vigra og skilað vigri sem hefur lægsta/hæsta gildið í hverju sæti
\begin{minted}[frame=lines]{matlab}
>> x = [3 5 8 2 11];
>> y = [2 6 4 5 10];
>> min(x,y)
ans =
    2    5    4    2   10
>> max(x,y)
ans =
    3    6    8    5   11
\end{minted}
\end{frame}

\begin{frame}[fragile]{Meðaltal}
\begin{itemize}
 \item Meðaltal (e. \emph{arithmetic mean}) vigurs er summa vigursins deilt með fjölda staka í honum.
 \item Fallið \texttt{mean} reiknar meðaltal
\end{itemize}
\begin{minted}[frame=lines]{matlab}
>> x = [9 10 5 9 8 7 2 8 10 10];
>> sum(x)/length(x)
ans =  7.8000
>> mean(x)
ans =  7.8000
\end{minted}
\end{frame}

\begin{frame}[fragile]{Villandi meðaltal}
Einfarar (e. \emph{outliers}) í gögnum geta þýtt að meðaltal er ekki lengur dæmigert fyrir flest gildin
\begin{minted}[frame=lines]{matlab}
>> x2 = [9 10 5 9 8 100 7 2 8 10 10];
>> mean(x2)
ans =  16.182
\end{minted}
Getum hent út hæstu og lægstu gildum:
\begin{minted}[frame=lines]{matlab}
>> x3 = x2(x2~=min(x2) & x2~=max(x2))
x3 =
    9   10    5    9    8    7    8   10   10
\end{minted}
\end{frame}

\begin{frame}{Staðalfrávik}
Staðalfrávik (e. \emph{standard deviation} og fervik (e. \emph{variance}) mæla hversu dreifð stök í vigri eru
\[
 var(x) = \frac{\sum_{i=1}^n (x_i - mean)^2}{n-1}
\]
\[
  std(x) = \sqrt{var(x)}
\]
\end{frame}

\begin{frame}[fragile]{Staðalfrávik}
Föllin \texttt{var} og \texttt{std} reikna út fervik og staðalfrávik í Matlab.
\begin{minted}[frame=lines]{matlab}
>> x = [9 10 5 9 8 7 2 8 10 10];
>> var(x)
ans =  6.6222
>> std(x)
ans =  2.5734
\end{minted}
\end{frame}

\begin{frame}[fragile]{Algengasta gildi}
Algengasta gildið getur gefið okkur ákveðnar upplýsingar um gögnin. Í Matlab má finna það með \texttt{mode} fallinu.
\begin{minted}[frame=lines]{matlab}
>> x = [9 10 5 9 8 7 2 8 10 10];
>> mode(x)
ans =  10
>> mode([1 1 2 2 5])
ans =  1
\end{minted}
Séu mörg gildi jafn algeng þá skilar \texttt{mode} því lægsta.
\end{frame}

\begin{frame}[fragile]{Miðgildi}
\begin{itemize}
 \item Gildið sem er í miðjunni í stærðarröð rununnar er oft ``dæmigert'' gildi
 \begin{itemize}
  \item Ef fjöldinn er slétt tala þá er miðgildið meðaltal talnanna tveggja í miðjunni
 \end{itemize}
 \item Til að finna miðgildi þurfa gögnin að vera í röð
 \begin{itemize}
  \item Fallið \texttt{median} sér um þetta fyrir okkur
 \end{itemize}
\begin{minted}[frame=lines]{matlab}
>> x = [9 10 5 9 8 7 2 8 10 10];
>> median(x)
ans =  8.5000
\end{minted}
\end{itemize}
\end{frame}

\begin{frame}{Fyrirlestraræfing}
    \begin{enumerate}
        \item Skrifið nafnlaust Matlab-fall sem reiknar spannarmiðju (e. \emph{mid-range}), sem er meðaltal hágildis og lággildis
        \item Einfarar eru oft skilgreindir sem þau gildi sem eru meira en tvö staðalfrávik frá meðaltalinu. Skrifið vigurkóða til að finna alla einfara í vigri
        \item Skrifið Matlab-fall sem reiknar út staðalfrávik vigurs ef kallað er á það með einni úttaksbreytu, en staðalfrávik og fervik ef kallað er á það með tveimur úttaksbreytum
    \end{enumerate}
\end{frame}

\begin{frame}{Meiri forritun}
    \begin{itemize}
        \item Lítum á tvö sýnidæmi um flóknari hlutbundin forrit
        \begin{itemize}
            \item Hermun á bolta
            \item Útfærsla á nafnatöflu (e. \emph{symbol table}) með tvíleitartré (e. \emph{binary search tree})
        \end{itemize}
    \end{itemize}
\end{frame}

\begin{frame}{Hugmyndin um nafnatöflu}
    \begin{itemize}
        \item Höfum stök þar sem hvert samanstendur af lykli (e. \emph{key}) og gildi (e. \emph{value})
        \item Notum lykilinn til að fletta upp gildum
        \begin{itemize}
            \item Dæmi: Flettum upp nafni með því að nota kennitölu
        \end{itemize}
        \item Ef við notum fylki til að geyma stökin vex keyrslutíminn línulega með fjölda staka í nafnatöflunni
        \begin{itemize}
            \item Útfærsla með tvíleitartré er mun hraðvirkari þegar töflurnar eru stórar
        \end{itemize}
    \end{itemize}
\end{frame}

\end{document}
