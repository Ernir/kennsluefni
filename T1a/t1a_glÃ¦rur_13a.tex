\documentclass[handout]{beamer}

\usepackage{Haust2016glærur}

\title{Tölvunarfræði 1a}
\subtitle{Vika 13, fyrri fyrirlestur}

\begin{document}

\begin{frame}
\titlepage
\end{frame}

\section{Inngangur}

\begin{frame}{Í síðasta þætti\ldots}
\begin{itemize}
 \item Image fallið og litavarpanir
 \item Teiknihandföng
 \item Grunnhlutir 
\end{itemize}
\end{frame}

\section{Tölfræði}

\begin{frame}{Tölfræðiföll í Matlab}
\begin{itemize}
 \item Matlab hefur mörg föll sem reikna tölfræði fyrir gagnarunur
 \item Gerum ráð fyrir að gögn séu í línuvigrinum $x$
\[
 x = [x_1,x_2, \ldots, x_n]
\]
 \item Tölfræðiföllin gefa okkur nánari upplýsingar um gerð og eiginleika gagna
 \begin{itemize}
  \item Verkfræðingar og raunvísindafólk þurfa oft að spyrja sig spurninga um gögnin
  \item Hvernig eru gögnin dreifð? Hver er þróunin? Eru þau öll jafn gild?
 \end{itemize}
\end{itemize}

\end{frame}

\begin{frame}[fragile]{Kunnuglegar slóðir}
Við höfum séð föllin \texttt{min} og \texttt{max}
\begin{minted}[frame=lines]{matlab}
>> x = [9 10 5 9 8 7 2 8 10 10];
>> min(x)
ans =  2
\end{minted}
Þau geta líka skilað staðsetningu stakanna
\begin{minted}[frame=lines]{matlab}
>> [maxVal, maxIndex] = max(x)
maxVal =  10
maxIndex =  2
\end{minted}
Hér fær \texttt{maxIndex} gildið ``staðsetning fyrsta hágildis''
\end{frame}

\begin{frame}[fragile]{Mörg hágildi}
Til að fá staðsetningar allra há- eða lággilda í vigri þurfum við að beita smá brellum. \pause
Möguleg leið:
\begin{minted}[frame=lines]{matlab}
>> x = [9 10 5 9 8 7 2 8 10 10];
>> find(x == max(x))
ans =
    2    9   10
\end{minted}
\end{frame}

\begin{frame}[fragile]{Samanburður vigra}
\texttt{min} og \texttt{max} geta tekið inn tvo vigra og skilað vigri sem hefur lægsta/hæsta gildið í hverju sæti
\begin{minted}[frame=lines]{matlab}
>> x = [3 5 8 2 11];
>> y = [2 6 4 5 10];
>> min(x,y)
ans =
    2    5    4    2   10
>> max(x,y)
ans =
    3    6    8    5   11
\end{minted}
\end{frame}

\begin{frame}[fragile]{Meðaltal}
\begin{itemize}
 \item Meðaltal (e. \emph{arithmetic mean}) vigurs er summa vigursins deilt með fjölda staka í honum.
 \item Fallið \texttt{mean} reiknar meðaltal
\end{itemize}
\begin{minted}[frame=lines]{matlab}
>> x = [9 10 5 9 8 7 2 8 10 10];
>> sum(x)/length(x)
ans =  7.8000
>> mean(x)
ans =  7.8000
\end{minted}
\end{frame}

\begin{frame}[fragile]{Villandi meðaltal}
Einfarar (e. \emph{outliers}) í gögnum geta þýtt að meðaltal er ekki lengur dæmigert fyrir flest gildin
\begin{minted}[frame=lines]{matlab}
>> x2 = [9 10 5 9 8 100 7 2 8 10 10];
>> mean(x2)
ans =  16.182
\end{minted}
Getum hent út hæstu og lægstu gildum:
\begin{minted}[frame=lines]{matlab}
>> x3 = x2(x2~=min(x2) & x2~=max(x2))
x3 =
    9   10    5    9    8    7    8   10   10
\end{minted}
\end{frame}

\begin{frame}{Staðalfrávik}
Staðalfrávik (e. \emph{standard deviation} og fervik (e. \emph{variance}) mæla hversu dreifð stök í vigri eru
\[
 var(x) = \frac{\sum_{i=1}^n (x_i - mean)^2}{n-1}
\]
\[
  std(x) = \sqrt{var(x)}
\]
\end{frame}

\begin{frame}[fragile]{Staðalfrávik}
Föllin \texttt{var} og \texttt{std} reikna út fervik og staðalfrávik í Matlab.
\begin{minted}[frame=lines]{matlab}
>> x = [9 10 5 9 8 7 2 8 10 10];
>> var(x)
ans =  6.6222
>> std(x)
ans =  2.5734
\end{minted}
\end{frame}

\begin{frame}[fragile]{Algengasta gildi}
Algengasta gildið getur gefið okkur ákveðnar upplýsingar um gögnin. Í Matlab má finna það með \texttt{mode} fallinu.
\begin{minted}[frame=lines]{matlab}
>> x = [9 10 5 9 8 7 2 8 10 10];
>> mode(x)
ans =  10
>> mode([1 1 2 2 5])
ans =  1
\end{minted}
Séu mörg gildi jafn algeng þá skilar \texttt{mode} því lægsta.
\end{frame}

\begin{frame}[fragile]{Miðgildi}
\begin{itemize}
 \item Gildið sem er í miðjunni í stærðarröð rununnar er oft ``dæmigert'' gildi
 \begin{itemize}
  \item Ef fjöldinn er slétt tala þá er miðgildið meðaltal talnanna tveggja í miðjunni
 \end{itemize}
 \item Til að finna miðgildi þurfa gögnin að vera í röð
 \begin{itemize}
  \item Fallið \texttt{median} sér um þetta fyrir okkur
 \end{itemize}
\begin{minted}[frame=lines]{matlab}
>> x = [9 10 5 9 8 7 2 8 10 10];
>> median(x)
ans =  8.5000
\end{minted}
\end{itemize}
\end{frame}

\begin{frame}[fragile]{Fyrirlestraræfing}
Skráið ykkur inn á \url{http://socrative.com/} og gerið fyrstu tvær spurningarnar.

Herbergisnúmer = \texttt{TOL105G2016}

Notendanafn = HÍ-tölvupóstfang
\end{frame}

\section{Mengi}

\begin{frame}{Mengi í Matlab}
Hægt er að nota vigra í Matlab sem mengi. Skoðum nokkrar nýjar aðgerðir:
\begin{center}
\begin{tabular}{ll}
\toprule
Fall&Aðgerð\\
\midrule
\texttt{union}&sammengi $A \cup B$\\
\texttt{intersect}&sniðmengi $A \cap B$\\
\texttt{setdiff}&mengjamismunur $ A \setminus B$\\
\texttt{setxor}&samhverfur mengjamunur $A \Delta B$\\
\bottomrule
\end{tabular}
\end{center}
\end{frame}

\begin{frame}{Mengi?}
\begin{itemize}
 \item Mengi koma furðu oft fyrir í forritun
 \begin{itemize}
  \item Almenn hugtök eiga víða við
 \end{itemize}
 \item Mengjahugsunarháttur er allsráðandi í gagnagrunnsvinnslu
 \item Oft sem gögn þurfa að hafa mengjaeiginleika
\end{itemize}
\end{frame}


\begin{frame}[fragile]{Sammengi}
Sammengi tveggja vigra skilar vigri með öllum stökunum úr þeim báðum, án endurtekninga.
\begin{minted}[frame=lines]{matlab}
a = [6 4 2 0 -1];
b = [-1 0 1 1 3];
>> c = union(a, b)
c =
  -1   0   1   2   3   4   6
\end{minted}
Athugum: Úttakið er raðað!
\end{frame}

\begin{frame}[fragile]{Sniðmengi}
Sniðmengi tveggja vigra skilar vigri með öllum sameiginlegum stökunum, án endurtekninga
\begin{minted}[frame=lines]{matlab}
>> piano = {'Ari', 'Erna', 'Hallur'};
>> fotbolti = {'Ásta', 'Bjarni', 'Erna', 'Óli'};
>> d = intersect(piano, fotbolti)
d = 
    'Erna'
\end{minted}
\end{frame}

\begin{frame}[fragile]{Mengjamismunur}
Mengjamunur tveggja vigra skilar vigri með þeim stökum fyrri vigursins sem ekki koma fyrir í þeim seinni
\begin{minted}[frame=lines]{matlab}
>> piano = {'Ari', 'Erna', 'Hallur'};
>> fotbolti = {'Ásta', 'Bjarni', 'Erna', 'Óli'};
>> setdiff(piano, fotbolti)
ans = 
    'Ari'    'Hallur'
>> setdiff(fotbolti, piano)
ans = 
    'Bjarni'    'Ásta'    'Óli'
\end{minted}

\end{frame}

\begin{frame}[fragile]{Samhverfur mengjamismunur}
Samhverfur mengjamunur tveggja vigra skilar vigri með þeim stökum vigranna sem ekki eru í sniðmengi þeirra
\begin{minted}[frame=lines]{matlab}
>> piano = {'Ari', 'Erna', 'Hallur'};
>> fotbolti = {'Ásta', 'Bjarni', 'Erna', 'Óli'};
>> setxor(piano, fotbolti)
ans = 
   'Ari'   'Bjarni'   'Hallur'   'Ásta'   'Óli'
\end{minted}
\[
A \Delta B = (A \setminus B) \cup (B \setminus A) = (A \cup B) \setminus (A \cap B)
\]
\end{frame}

\begin{frame}[fragile]{Einkvæmni}
\begin{itemize}
 \item Í stærðfræði kemur hvert stak aðeins einu sinni fyrir í mengjum
 \begin{itemize}
  \item Munum að ``mengin'' í Matlab eru bara vigrar sem við kjósum að líta á sem mengi
  \item \ldots og vigrar leyfa endurtekningar
 \end{itemize}
 \item Fallið \texttt{unique} gerir stökin einkvæm
\end{itemize}
\begin{minted}[frame=lines]{matlab}
>> unique([3 2 2 4 2 3])
ans =
   2   3   4
>> unique([3 2 2 4 2 3], 'stable')
ans =
   3   2   4
\end{minted}
\end{frame}

\begin{frame}[fragile]{Vísar sem úttak}
Mengjaföllin skila líka vísum inn í upphaflegu vigrana.
\begin{minted}[frame=lines]{matlab}
>> a = [6 4 2 0 -1];     
>> b = [-1 0 1 1 3];
>> [c, aIndex, bIndex] = intersect(a, b)
c =
  -1   0
aIndex =
   5   4
bIndex =
   1   2
\end{minted}
\end{frame}

\begin{frame}[fragile]{\texttt{ismember} fallið}
Fallið \texttt{ismember} athugar hvort stök vigurs séu stök í öðrum vigri.
\begin{minted}[frame=lines]{matlab}
>> piano = {'Ari', 'Erna', 'Hallur'};
>> ismember({'Gunna', 'Ari', 'Bjarni'}, piano)
ans =
   0   1   0
\end{minted}
Fáum út rökvigur sem er jafn stór og fyrri vigurinn.
\end{frame}

\begin{frame}[fragile]{\texttt{ismember} fallið}
Röð inntaksbreytanna skiptir máli í \texttt{ismember} fallinu.
\begin{minted}[frame=lines]{matlab}
>> v1 = [3 5 2 7 8];
>> v2 = [4 1 5];
>> ismember(v1, v2) % hvaða stök í v1 eru í v2?
ans =
   0   1   0   0   0
>> ismember(v2, v1) % hvaða stök í v2 eru í v1?
ans =
   0   0   1
\end{minted}
\end{frame}

\begin{frame}[fragile]{Fyrirlestraræfing}
Skráið ykkur inn á \url{http://socrative.com/} og klárið æfinguna.

Herbergisnúmer = \texttt{TOL105G2016}

Notendanafn = HÍ-tölvupóstfang
\end{frame}

\end{document}
