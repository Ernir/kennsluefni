\documentclass{beamer}

\usepackage[utf8]{inputenc}
\usepackage[icelandic]{babel}
\usepackage[T1]{fontenc}

\usepackage{booktabs}
\usepackage{minted} %Minted and configuration
\usepackage{framed}
\usepackage{tikz}
\usemintedstyle{default}
\renewcommand{\theFancyVerbLine}{\sffamily \arabic{FancyVerbLine}}
\newcommand{\Mod}[1]{\ \text{mod}\ #1}

% \makeatletter
% \minted@define@extra{label}
% \makeatother

\usebackgroundtemplate%
{%
\vbox to \paperheight{
\includegraphics[width=\paperwidth]{Pics/hi-slide-head}

\vfill
\hspace{0.5cm}\includegraphics[width=0.3\paperwidth]{Pics/hi-von-logo}
\vspace{0.5cm}
    }%
}

% \makeatletter
% \newcommand{\minted@write@detok}[1]{%
%   \immediate\write\FV@OutFile{\detokenize{#1}}}%
%   
%   \newcommand{\minted@FVB@VerbatimOut}[1]{%
%   \@bsphack
%   \begingroup
%     \FV@UseKeyValues
%     \FV@DefineWhiteSpace
%     \def\FV@Space{\space}%
%     \FV@DefineTabOut
%     %\def\FV@ProcessLine{\immediate\write\FV@OutFile}% %Old, non-Unicode version
%     \let\FV@ProcessLine\minted@write@detok %Patch for Unicode
%     \immediate\openout\FV@OutFile #1\relax
%     \let\FV@FontScanPrep\relax
% %% DG/SR modification begin - May. 18, 1998 (to avoid problems with ligatures)
%     \let\@noligs\relax
% %% DG/SR modification end
%     \FV@Scan}
%     \let\FVB@VerbatimOut\minted@FVB@VerbatimOut
%     
%     \renewcommand\minted@savecode[1]{
%   \immediate\openout\minted@code\jobname.pyg
%   \immediate\write\minted@code{\expandafter\detokenize\expandafter{#1}}%
%   \immediate\closeout\minted@code}
%   
% \makeatother

\setbeamertemplate{navigation symbols}{}
\usecolortheme{dove}
\setbeamercolor{frametitle}{fg=white}
\hypersetup{colorlinks=true,pdfauthor={Eirikur Ernir Thorsteinsson},linkcolor=blue,urlcolor=blue}

\AtBeginSection[]
{
  \begin{frame}<beamer>
    \frametitle{Yfirlit}
    \tableofcontents[currentsection]
  \end{frame}
}

\author{Eiríkur Ernir Þorsteinsson}
\institute{Háskóli Íslands}
\date{Haust 2015}

\title{Tölvunarfræði 1a}
\subtitle{Vika 14, fyrri fyrirlestur}

\begin{document}

\begin{frame}
\titlepage
\end{frame}

\section{Inngangur}

\begin{frame}{Í síðasta þætti\ldots}
\begin{itemize}
 \item Margliður
 \item Nálgun
 \item Brúun
\end{itemize}
Kafli: 14.1
\end{frame}

\begin{frame}{Upprifjun}
\begin{itemize}
 \item Förum í dag yfir ýmis grundvallaratriði (aftur)
 \item Beru þau saman við hvernig þessi sömu atriði eru útfærð í Java annars vegar og Python hins vegar
 \item Einn mögulegur listi yfir helstu forritunarmál: \href{http://www.tiobe.com/index.php/content/paperinfo/tpci/index.html}{Tiobe}
\end{itemize}
\end{frame}


\section{Breytur og virkjar}

\begin{frame}[fragile]{Breytur}
\vspace{1cm}
Í Matlab er breytum gefið gildi á eftirfarandi máta:
\begin{minted}[frame=lines]{matlab}
myNumber = 1;
myFraction = 1.5;
myString = 'Jól';
\end{minted}
Í Java þarf líka að skrifa tag (gerð) breytunnar á undan 
\begin{minted}[frame=lines]{java}
Integer myNumber = 1;
Double myFraction = 1.5;
String myString = "Jól"
\end{minted}
\end{frame}

\begin{frame}[fragile]{Samanburðarvirkjar}
Flestir samanburðarvirkjar (\texttt{>} \texttt{<}, \texttt{>=}, \texttt{<=}, \texttt{==}) eru nokkuð staðlaðir í forritunarmálum. Helst er munur á framsetningu ekki-jafnt-og.

Ekki-jafnt-og í Matlab:
\begin{minted}[frame=lines]{matlab}
a ~= b
\end{minted}

Í Java, Python og mörgum öðrum málum:
\begin{minted}[frame=lines]{java}
a != b
\end{minted}
\end{frame}

\section{Vigrar og fylki}

\begin{frame}{Vigrar og fylki}
\begin{itemize}
 \item Í Matlab eru vigrar og fylki lang-mest áberandi gagnagrindurnar
 \begin{itemize}
  \item Þetta er ekki sérstaklega algengt
  \item Rökvísanir, stakvísar grunnaðgerðir og fylkjaalgebra fylgja sjaldnast með forritunarmálum
 \end{itemize}
 \item Mörg forritunarmál gera greinarmun á listum (e. \emph{lists}) og fylkjum (e. \emph{arrays})
\end{itemize}
\end{frame}

\begin{frame}[fragile]{Fylki í Matlab}
Í Matlab má búa til vigra og fylki á nokkurn veginn sama máta. Hægt er að stækka fylki eftir að þau eru búin til.
\begin{minted}[frame=lines]{matlab}
a = [3 4 5]; % 1x3 vigur
b = [3 4 5;1 7 8]; % 2x3 fylki
b(:,4) = [2 3]' % fylkið stækkað
b(1,4) % Vísað í fylkið með einu svigapari
\end{minted}
\end{frame}

\begin{frame}[fragile]{Fylki í Java}
\vspace{\baselineskip}
Í Java hafa fylki fyrirfram ákveðna stærð. 
\begin{minted}[frame=lines]{java}
// Kallað á smið Integer klasans til að búa til
// einvítt 10 staka fylki ("vigur")
Integer[] myArray = new Integer[10];
// 2x2 strengjafylki
String[][] myStringArray = new String [][] {
    { "a", "b"}, { "c", "d"}, 
};
String myString = myStrings[0][1]; // Náð í "b"
\end{minted}
Listar eru aðskilið fyrirbrigði í Java.
\end{frame}

\begin{frame}[fragile]{Listar í Python}
Í Python gegna listar svipuðu hlutverki og fylki gegna í Matlab.
\begin{minted}[frame=lines]{python}
a = [3, 4, 5]  # listi skilgreindur
x = a[0]  # náð í fyrsta stakið í a
a.append(7)  # 7 bætt aftast í listann
y = a[:2]  # náð í fyrstu tvö stökin í a
\end{minted}
Til að fá margvíð fylki þarf að nota lista af listum eða viðbætur við grunnmálið (sér í lagi \texttt{numpy}).
\end{frame}

\section{Inntak og úttak}

\begin{frame}[fragile]{Inntak og úttak}
\vspace{\baselineskip}
Matlab og Python taka við gögnum og skrifa gögn á svipaðan hátt.
\begin{minted}[frame=lines]{matlab}
>> s = input('Sladu inn stafi: ', 's');
Sladu inn stafi: brabra
>> disp(s)
brabra
\end{minted}

\begin{minted}[frame=lines]{python}
>>> s = input('Sladu inn stafi: ')  #gert ráð fyrir streng
Sladu inn stafi: abracadabra
>>> print(s)
abracadabra
\end{minted}

\end{frame}

\begin{frame}[fragile]{Inntak og úttak í Java}
\vspace{\baselineskip}
Java gerir þetta á nokkuð frábrugðinn hátt. 

Kóði:
\begin{minted}[frame=lines]{java}
Scanner scan = new Scanner(System.in);
System.out.print("Sláðu inn stafi: ");
String s = scan.nextLine();
System.out.println("Stafirnir eru: " + s);
\end{minted}

Keyrsla:
\begin{verbatim}
Sláðu inn stafi: blah!
Stafirnir eru: blah!
\end{verbatim}
\end{frame}

\section{if og else}

\begin{frame}[fragile]{if og else}
\texttt{if} og \texttt{else} eru með mjög svipuðu sniði í öllum málunum. Munurinn felst fyrst og fremst í smáatriðunum.

\begin{verbatim}
ef skilyrði
  keyra blokk skipana
annars 
  keyra aðra blokk
\end{verbatim}

\end{frame}

\begin{frame}[fragile]{Skilyrði í Matlab}
Í Matlab þarf að loka skilyrðinu með \texttt{end}
\begin{minted}[frame=lines]{matlab}
if a < b
  c = 0;
else
  c = 1;
end
\end{minted}
\end{frame}

\begin{frame}[fragile]{Skilyrði í Java}
Í Java þarf að afmarka skilyrði með svigum og blokkir með slaufusvigum
\begin{minted}[frame=lines]{matlab}
if (a < b) {
  c = 0;
} else {
  c = 1;
}
\end{minted}
(Reyndar má sleppa slaufusvigunum ef ``blokkin'' er bara ein lína, eins og hér)
\end{frame}

\begin{frame}[fragile]{Skilyrði í Python}
Í Python byrjar blokk í \texttt{if-else} á tvípunkti og afmarkast af inndrætti
\begin{minted}[frame=lines]{matlab}
if a < b:
    c = 1
else:
    c = 0
\end{minted}
\end{frame}

\section{Lykkjur}

\begin{frame}[fragile]{Lykkjur}
For-lykkjur í Matlab og Python eru nokkuð líkar.
\begin{columns}
\column{0.5\textwidth}
\begin{minted}[frame=lines]{matlab}
numbers = [3 5 7];
total = 0;
for n = numbers
  total = total + n;
end
\end{minted}
\column{0.5\textwidth}
\begin{minted}[frame=lines]{python}
numbers = [3, 5, 7]
total = 0
for n in numbers:
  total += n
\end{minted}
\end{columns}
\vspace{\baselineskip}
Í báðum tilfellum fær breytan \texttt{total} gildið 15.
\end{frame}

\begin{frame}[fragile]{for-lykkjur í Java}
Tvær gerðir \texttt{for}-lykkja eru til í Java. Sú fyrri ítrar yfir fylki eða svipaða gagnagrind, líkt og lykkjur í Matlab og Python.
\begin{minted}[frame=lines]{java}
Integer[] numbers = {3, 5, 7};
Integer total = 0;
for (Integer n : numbers) {
  total += n;
}
\end{minted}
\end{frame}

\begin{frame}[fragile]{for-lykkjur í Java}
Hin gerð \texttt{for}-lykkju í Java heldur sérstaklega utan um teljara.
\begin{minted}[frame=lines]{java}
Integer[] numbers = {3, 5, 7};
Integer total = 0;
for (Integer n = 0; n < numbers.length; n++){
  total += numbers[n];
}
\end{minted}
Þessi gerð er líklega þó nokkuð meira notuð.
\end{frame}

\section{Föll}

\begin{frame}[fragile]{Föll í Matlab}
Í Matlab eru föll skilgreind með lykilorðinu \texttt{function}. Inntaks- og úttaksbreytur eru skilgreindar í fallshaus.
\begin{minted}[frame=lines]{matlab}
function z = twoSum(x, y)
  z = x + y;
end
\end{minted}
\end{frame}

\begin{frame}[fragile]{Föll í Python}
Í Python eru föll skilgreind með lykilorðinu \texttt{def}. Inntaksbreytur eru skilgreindar í fallshaus, úttaksbreyta er tilgreind með \texttt{return} lykilorðinu.
\begin{minted}[frame=lines]{python}
def two_sum(x, y):
    z = x + y
    return z
\end{minted}
Ekkert \texttt{end} þarf, lok fallsins skilgreinast af inndrætti.
\end{frame}

\begin{frame}[fragile]{Föll í Python}
Í Java þarf ekki sérstakt lykilorð til að skilgreina aðferð (fall). Þess í stað eru tög (gerðir) inntaks- og úttaksbreyta skilgreind. Hér eru inntökin og úttakið heiltölur, svo orðið \texttt{Integer} kemur víða við.
\begin{minted}[frame=lines]{java}
Integer twoSum(Integer x, Integer y) {
  Integer z = x + y;
  return z;
}
\end{minted}
Aðferðir eru alltaf inni í klösum í Java.
\end{frame}

\begin{frame}[fragile]{Föll í Scheme}
Til gamans: summufallið okkar í forritunarmálinu Scheme, ásamt kalli á það:
\begin{minted}[frame=lines]{scheme}
(define (twoSum x y)
  (+ x y)
)
(display (twoSum 1 2))
\end{minted}

\end{frame}


\begin{frame}[fragile]{Fyrirlestraræfing}
Farið inn á \url{http://www.tutorialspoint.com/codingground.htm}, finnið Java og Python 3 á listanum og keyrið nokkur af sýnidæmunum. Getið þið breytt þeim til að leysa flóknari verkefni?

\vspace{1cm}
Skipanir í Java þurfa að fara inn í aðferðina sem heitir
\begin{minted}{java}
public static void main(String []args)
\end{minted}
\end{frame}

\end{document}
