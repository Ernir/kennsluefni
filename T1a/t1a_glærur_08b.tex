\documentclass[handout]{beamer}

\usepackage{Haust2017glærur}

\title{Tölvunarfræði 1a}
\subtitle{Vika 8, seinni fyrirlestur}

\begin{document}

\begin{frame}
\titlepage
\end{frame}

\section{Inngangur}

\begin{frame}{Í síðasta þætti\ldots}
\begin{itemize}
 \item Miðmisserispróf
 \item Miðmisseriskönnun
 \item Hólfavigrar
\end{itemize}
\end{frame}

\section{Færslur (8.2)}

\begin{frame}{Færslur}
\begin{itemize}
 \item Lítum nú á færslur (e. \emph{structures})
 \begin{itemize}
  \item Gagnagrind sem hópar gildi sem eiga saman
  \item Dæmi: Upplýsingar um vöru: vörunúmer, lýsing, innkaupsverð, söluverð
 \end{itemize}
 \item Færslur hafa svið (e. \emph{fields}) og gildi (e. \emph{values})
\end{itemize}
\end{frame}

\begin{frame}{Að smíða færslur}
\begin{itemize}
 \item Hægt er að búa til færslu á tvo mismunandi vegu
 \begin{itemize}
  \item Með því að nota \texttt{struct} fallið
  \item Með því að nota punktvirkja
 \end{itemize}
 \item Náð er í gögn með punktvirkjanum
\end{itemize}
\end{frame}

\begin{frame}[fragile]{\texttt{struct} fallið}
Í struct-fallinu koma til skiptis nafn sviðs og gildi sviðsins.
\begin{minted}{matlab}
>> student = struct('id', '12345', 'grades', [6 7 8])
student = 
        id: '12345'
    grades: [6 7 8]
\end{minted}
\end{frame}

\begin{frame}[fragile]{Að ná í gögn úr færslu}
Punktvirkjann (e. \emph{dot operator}) má nota til að ná í gögn úr færslu.
\begin{minted}{matlab}
>> student.id
ans =
12345
\end{minted}
\end{frame}

\begin{frame}[fragile]{Punktvirkinn}
\begin{itemize}
 \item Punktvirkjann má líka nota til að búa til færslur
 \item Þá skrifum við nafn breytu, punkt, og svo nafn sviðsins (breytan þarf ekki að vera skilgreind fyrirfram)
\begin{minted}{matlab}
>> student2.id = '23456';
>> student2.grades = [7 8 9];
\end{minted}
 \item Ef við viljum að \texttt{student2} sé af sömu gerð og \texttt{student} þá verðum við að nota nákvæmlega sömu sviðsnöfn og gagnagerðir
\end{itemize}
\end{frame}

\begin{frame}{Nánar um punktvirkjann}
\begin{itemize}
 \item Þegar punktvirkinn er notaður til að búa til færslubreytu þá verður breytan til smátt og smátt
 \begin{itemize}
  \item Einu sviði bætt við í einu
 \end{itemize}
 \item Þetta hefur svipuð vandamál í för með sér og þegar vigur er stækkaður inni í lykkju - nýjar minnisúthlutanir þurfa að eiga sér stað
 \begin{itemize}
  \item Það er því hraðvirkara að nota \texttt{struct}-fallið
 \end{itemize}
\end{itemize}
\end{frame}

\begin{frame}{Að prenta út færslur}
    \begin{itemize}
        \item Fallið \texttt{disp} birtir yfirirlitsupplýsingar um færslur
        \item \texttt{fprintf} getur einungis birt einstök svið
        \begin{itemize}
        \item En það er e.t.v. ekki undarlegt - \texttt{fprintf} er nákvæmnistól
        \end{itemize}
        \item Til að birta færsluvigur fallega þarf venjulega að sækja gildin
    \end{itemize}
\end{frame}

\begin{frame}{Færsluföll}
\begin{center}
\begin{tabular}{lp{7cm}}
\toprule
Nafn falls&Hlutverk\\
\midrule
\texttt{isstruct}&Skilar \texttt{true} ef inntak er færsla\\
\texttt{isfield}&Skilar \texttt{true} ef gefin færsla hefur svið með gefnu nafni\\
\texttt{fieldnames}&Skilar hólfavigri með nöfnum allra sviða í gefinni færslu\\
\bottomrule
\end{tabular}
\end{center}
\end{frame}

\begin{frame}{Færsluvigrar}
\begin{itemize}
 \item Oft viljum við geyma margar færslur af sömu gerð en með mismunandi gildum
 \begin{itemize}
  \item Dæmi: Nemendaskrá, vörulisti
 \end{itemize}
 \item Búum þá til vigur af færslum
 \begin{itemize}
  \item Munum: Allt í Matlab er fylki
 \end{itemize}
\end{itemize}
\end{frame}

\begin{frame}[fragile]{Að nota færsluvigur}
\begin{itemize}
 \item Getum notað færsluvigur eins og aðra vigra
 \begin{itemize}
  \item Vísum í sæti vigurs með númeri
  \item Vísum í svið færslu með punktvirkjanum
 \end{itemize}
\end{itemize}
\begin{minted}[fontsize=\small]{matlab}
>> products = struct('name', 'Torfstrekkjari', 'price', 12000);
>> products(2) = struct('name', 'Blómalím', 'price', 1500);
>> products.price
ans =
       12000
ans =
        1500
\end{minted}

\end{frame}

\begin{frame}[fragile]{Upphafsstilling með hólfavigrum}
Hægt er að nota hólfavigur til að búa til færsluvigur í einu skrefi:

\begin{minted}{matlab}
>> products = struct('name', {'Torfstrekkjari', ...
   'Blómalím'})
\end{minted}
\end{frame}

\begin{frame}[fragile]{Samskeyting}
\texttt{nafnfærsluvigurs.nafnsviðs} skilar mörgum breytum. Hægt er að skeyta þeim saman með því að umlykja skipunina með hornklofum:
\begin{minted}{matlab}
>> [p1, p2] = products.price
p1 =
       12000
p2 =
        1500
>> prices = [products.price]
prices =
       12000        1500
\end{minted}
\end{frame}

\begin{frame}[fragile]{Hreiðraðar færslur}
Hægt er að setja færslu inn í færslu.

\begin{minted}{matlab}
course = struct('code', 'TÖL105G', 'teacher', ...
   struct('name', 'Ernir'))
course = 
       code: 'TÖL105G'
    teacher: [1x1 struct]
>> course.teacher.name
ans =
Ernir
\end{minted}

\end{frame}

\begin{frame}{Fyrirlestraræfing}
    \begin{enumerate}
        \item Búið til $1 \times 1$ færslubreytu sem inniheldur upplýsingar um nafn ykkar og kennitölu
        \item Búið til færsluvigur með a.m.k. 2 færslum sem inniheldur upplýsingar um nafn bjórs, nafn brugghúss, og seldan lítrafjölda
        \item Finnið heildarfjölda seldra lítra skv. færsluvigrinum í liðnum á undan með lykkju eða vigurkóðun
    \end{enumerate}
\end{frame}

\begin{frame}{Viðbótardæmi}
    \begin{itemize}
        \item Dæmi 8.5
        \item Dæmi 8.6
        \item Dæmi 8.17
    \end{itemize}
\end{frame}

\end{document}
