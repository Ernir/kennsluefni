\documentclass[handout]{beamer}

\usepackage[utf8]{inputenc}
\usepackage[icelandic]{babel}
\usepackage[T1]{fontenc}

\usepackage{booktabs}
\usepackage{minted} %Minted and configuration
\usepackage{framed}
\usepackage{tikz}
\usemintedstyle{default}
\renewcommand{\theFancyVerbLine}{\sffamily \arabic{FancyVerbLine}}
\newcommand{\Mod}[1]{\ \text{mod}\ #1}

% \makeatletter
% \minted@define@extra{label}
% \makeatother

\usebackgroundtemplate%
{%
\vbox to \paperheight{
\includegraphics[width=\paperwidth]{Pics/hi-slide-head}

\vfill
\hspace{0.5cm}\includegraphics[width=0.3\paperwidth]{Pics/hi-von-logo}
\vspace{0.5cm}
    }%
}

% \makeatletter
% \newcommand{\minted@write@detok}[1]{%
%   \immediate\write\FV@OutFile{\detokenize{#1}}}%
%   
%   \newcommand{\minted@FVB@VerbatimOut}[1]{%
%   \@bsphack
%   \begingroup
%     \FV@UseKeyValues
%     \FV@DefineWhiteSpace
%     \def\FV@Space{\space}%
%     \FV@DefineTabOut
%     %\def\FV@ProcessLine{\immediate\write\FV@OutFile}% %Old, non-Unicode version
%     \let\FV@ProcessLine\minted@write@detok %Patch for Unicode
%     \immediate\openout\FV@OutFile #1\relax
%     \let\FV@FontScanPrep\relax
% %% DG/SR modification begin - May. 18, 1998 (to avoid problems with ligatures)
%     \let\@noligs\relax
% %% DG/SR modification end
%     \FV@Scan}
%     \let\FVB@VerbatimOut\minted@FVB@VerbatimOut
%     
%     \renewcommand\minted@savecode[1]{
%   \immediate\openout\minted@code\jobname.pyg
%   \immediate\write\minted@code{\expandafter\detokenize\expandafter{#1}}%
%   \immediate\closeout\minted@code}
%   
% \makeatother

\setbeamertemplate{navigation symbols}{}
\usecolortheme{dove}
\setbeamercolor{frametitle}{fg=white}
\hypersetup{colorlinks=true,pdfauthor={Eirikur Ernir Thorsteinsson},linkcolor=blue,urlcolor=blue}

\AtBeginSection[]
{
  \begin{frame}<beamer>
    \frametitle{Yfirlit}
    \tableofcontents[currentsection]
  \end{frame}
}

\author{Eiríkur Ernir Þorsteinsson}
\institute{Háskóli Íslands}
\date{Haust 2015}

\title{Tölvunarfræði 1a}
\subtitle{Vika 12, seinni fyrirlestur}

\begin{document}

\begin{frame}
\titlepage
\end{frame}

\section{Inngangur}

\begin{frame}{Í síðasta þætti\ldots}
\begin{itemize}
 \item Teikning skoðuð betur (11.1)
 \item Hreyfimyndir (11.2)
 \item Óvissubil (utan bókar)
 \item Þrívíddarteikningar (11.3)
\end{itemize}
\end{frame}

\section{Óskalisti}

\begin{frame}[fragile]{Að teikna bestu línu}
Hægt er að teikna bestu línu í Matlab-teikningaglugganum.

Byrjum á að teikna mynd, líkt og í síðasta tíma
\begin{minted}{matlab}
>> x = 1:6;
>> y = [1 1.4 1.9 2.6 3 3.6];
>> e = [0.09 0.1 0.1 0.1 0.09 0.1];
>> errorbar(x,y,e)
\end{minted}
\end{frame}

\begin{frame}{Að teikna bestu línu}

\small Förum svo í \emph{Tools} - \emph{Basic Fitting}

\begin{center}
\includegraphics[width=\textwidth]{Pics/errorbar-line}
\end{center}
\end{frame}

\begin{frame}[fragile]{Að teikna hæðarlínur}
Hægt er að teikna hæðarlínur á mjög svipaðan hátt og yfirborð eru teiknuð með \texttt{mesh} eða \texttt{surf}. Notum einfaldlega \texttt{contour} eða \texttt{contour3} í staðinn.

\begin{minted}{matlab}
>> [x,y] = meshgrid(linspace(-2,2),linspace(-2,2));
>> z = sin(x).*exp(y);
>> contour(x,y,z)
\end{minted}

\end{frame}


\section{Teiknihandföng}

\begin{frame}{Teiknihandföng}
\begin{itemize}
 \item Allar teikningar í Matlab samanstanda af teiknihlutum (e. \emph{graphics objects})
 \begin{itemize}
  \item Vísa má í teiknihluti með handföngum (e. \emph{handles})
  \item Handfang er tala sem vísar einkvæmt í hlutinn
 \end{itemize}
 \item Teiknihlutir eru af ýmsum gerðum
 \begin{itemize}
  \item Ásar, línur, texti, \ldots
  \item Þessir hlutir hafa mismunandi eiginleika
 \end{itemize}
\end{itemize}
\end{frame}

\begin{frame}[fragile]{Plot objects}
\begin{itemize}
 \item Teikniföllin (\texttt{plot}, \texttt{bar}, \texttt{pie}) skila handfangi fyrir teiknihlutinn
 \begin{itemize}
  \item Höfum ekki verið að nota handfangið hingað til
 \end{itemize}
\end{itemize}
\begin{minted}{matlab}
>> x = -2*pi : 0.1 : 2*pi;
>> y = sin(x);
>> handle = plot(x,y)
handle = 
  Line with properties:
  ...
\end{minted}
\end{frame}

\begin{frame}[fragile]{Eiginleikar teiknihluta}
\begin{itemize}
 \item Getum náð í gildi einstakra eiginleika með \texttt{get}
 \begin{minted}{matlab}
 >> get(handle)
  AlignVertexCenters: 'off'
  ...
 \end{minted}
 \item Getum breytt gildum eiginleika með \texttt{set}
 \begin{minted}{matlab}
 >> set(handle, 'LineWidth', 2.5)
 \end{minted}
 \item Einnig er hægt að setja þá í upphaflegu teikniskipunina
 \begin{minted}{matlab}
 >> handle= plot(x,y,'LineWidth', 2.5);
 \end{minted}
\end{itemize}
\end{frame}

\begin{frame}[fragile]{Core objects}
\begin{itemize}
 \item Matlab er með ákveðna grunnhluti (e. \emph{core objects}) til teiknunar
 \begin{itemize}
  \item Lína (\texttt{line}), texti (\texttt{text}), rétthyrningur (\texttt{rectangle})
 \end{itemize}
 \item Hægt er að teikna þá með innbyggðum föllum
\begin{minted}[frame=lines]{matlab}
>> h1 = line([3 1], [2 4], 'Color', 'r');
\end{minted}
\end{itemize}
\end{frame}

\begin{frame}{Línur}
\begin{itemize}
 \item Línur eru kunnuglegur teiknihlutur
 \begin{itemize}
  \item \texttt{plot} fallið býr til línur
 \end{itemize}
 \item Línur hafa ýmsa eiginleika
 \begin{itemize}
  \item \texttt{LineStyle} - heil lína, punktalína, engin lína
  \item \texttt{LineWidth} - þykkt línunnar
  \item \texttt{Marker} - hvaða tákn á að sýna í gagnapunktum
 \end{itemize}
\end{itemize}
\end{frame}

\begin{frame}[fragile]{Texti}
\begin{itemize}
 \item Fallið \texttt{text} setur texta inn á teikninguna
 \begin{itemize}
  \item Almennt snið: \texttt{text(x,y,'strengur')}
  \item \texttt{x} og \texttt{y} eru staðsetning neðra vinstra horns textakassans
 \end{itemize}
 \item Hægt er að nota sumar \LaTeX-skipanir í strengnum
 \begin{itemize}
  \item Líka í föllum eins og \texttt{title} og \texttt{xlabel}
 \end{itemize}
\end{itemize}

\begin{minted}{matlab}
>> x = 0:0.01:5;
>> y = exp(x);
>> line(x,y,'LineWidth',3);
>> textHandle = text(3,50,'e^x \rightarrow');
>> set(textHandle, 'Fontsize', 20);
\end{minted}

\end{frame}

\begin{frame}{Texti}
\begin{center}
\includegraphics[width=0.8\textwidth]{Pics/exp-label}
\end{center}
\end{frame}

\begin{frame}[fragile]{Rétthyrningar}
\begin{itemize}
 \item Fallið \texttt{rectangle} teiknar rétthyrning
 \begin{itemize}
  \item Staðsetning (\texttt{Position}) hans er gefin með vigri \texttt{[x, y, w, h]}
  \begin{itemize}
   \item Staðsetning vinstra neðra horns, breidd og hæð
  \end{itemize}
 \end{itemize}
 \item Sveigja (\texttt{Curvature}) hans er tveggja staka vigur með gildi frá \texttt{[0,0]} til \texttt{[1,1]}
\end{itemize}
\begin{minted}{matlab}
>> rectangleHandle = rectangle('Position', [2 1 4 2]);
>> axis([0 8 0 4])
>> set(rectangleHandle, 'Curvature', [0.2 0.2])
\end{minted}

\end{frame}

\begin{frame}{Rétthyrningar}
\begin{center}
\includegraphics[width=0.8\textwidth]{Pics/rectangle-example}
\end{center}
\end{frame}

\begin{frame}[fragile]{Hringir}
\pause
\begin{itemize}
 \item Hægt er að teikna hringi og sporöskjur með því að búa til ``rétthyrninga'' með sveigjuna \texttt{[1 1]}
 \begin{itemize}
  \item Fyrra gildið í \texttt{Curvature} vigri er hversu hátt hlutfall af láréttu línunum er sveigt, hið seinna á við lóðréttu línurnar
 \end{itemize}
\end{itemize}
\begin{minted}{matlab}
>> circleHandle = rectangle('Curvature',[1 1]);
>> axis('square')
>> axis([-1 2 -1 2])
\end{minted}
\end{frame}

\begin{frame}{Hringir}
\begin{center}
\includegraphics[width=0.6\textwidth]{Pics/circle-example}
\end{center}
\end{frame}

\begin{frame}[fragile]{Marghyrningar}
\begin{columns}
\column{0.5\textwidth}
\begin{itemize}
 \item Til að teikna marghyrninga má nota fallið \texttt{patch}
 \begin{itemize}
  \item Fær inn $x$ og $y$ hnit
 \end{itemize}
\end{itemize}
\begin{minted}{matlab}
>> xCoords = [1 4 3 2];
>> yCoords = [1 1 2 2];
>> patch(xCoords,yCoords,'r')
>> axis([0 5 0 3])
\end{minted}
\column{0.5\textwidth}
\includegraphics[width=\textwidth]{Pics/patch-example}
\end{columns}
\end{frame}


\begin{frame}[fragile]{Fyrirlestraræfing}
\begin{columns}
\column{0.5\textwidth}
\begin{enumerate}
 \setcounter{enumi}{0}
 \item Rissið upp myndina sem kóðinn hér til hliðar teiknar
 \item Teiknið umferðarmerkið hér til hliðar með tveimur \texttt{patch} skipunum. 
 
 Vísbending: Hornpunktar innri þríhyrningsins eru $(0.85,0.9), (0.15,0.9), (0.5,0.15)$
\end{enumerate}
\column{0.5\textwidth}
\begin{minted}{matlab}
>> rectangleH = rectangle();
>> set(rectangleH, 'Position', [1 1 2 1])
>> axis([0 4 0 3])
\end{minted}
\begin{center}
\includegraphics[width=0.8\linewidth]{Pics/bidskylda}
\end{center}
\end{columns}
\end{frame}



\end{document}
