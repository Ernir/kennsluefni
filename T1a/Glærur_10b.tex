\documentclass{beamer}

\usepackage[utf8]{inputenc}
\usepackage[icelandic]{babel}
\usepackage[T1]{fontenc}

\usepackage{booktabs}
\usepackage{minted} %Minted and configuration
\usepackage{framed}
\usepackage{tikz}
\usemintedstyle{default}
\renewcommand{\theFancyVerbLine}{\sffamily \arabic{FancyVerbLine}}
\newcommand{\Mod}[1]{\ \text{mod}\ #1}

% \makeatletter
% \minted@define@extra{label}
% \makeatother

\usebackgroundtemplate%
{%
\vbox to \paperheight{
\includegraphics[width=\paperwidth]{Pics/hi-slide-head}

\vfill
\hspace{0.5cm}\includegraphics[width=0.3\paperwidth]{Pics/hi-von-logo}
\vspace{0.5cm}
    }%
}

% \makeatletter
% \newcommand{\minted@write@detok}[1]{%
%   \immediate\write\FV@OutFile{\detokenize{#1}}}%
%   
%   \newcommand{\minted@FVB@VerbatimOut}[1]{%
%   \@bsphack
%   \begingroup
%     \FV@UseKeyValues
%     \FV@DefineWhiteSpace
%     \def\FV@Space{\space}%
%     \FV@DefineTabOut
%     %\def\FV@ProcessLine{\immediate\write\FV@OutFile}% %Old, non-Unicode version
%     \let\FV@ProcessLine\minted@write@detok %Patch for Unicode
%     \immediate\openout\FV@OutFile #1\relax
%     \let\FV@FontScanPrep\relax
% %% DG/SR modification begin - May. 18, 1998 (to avoid problems with ligatures)
%     \let\@noligs\relax
% %% DG/SR modification end
%     \FV@Scan}
%     \let\FVB@VerbatimOut\minted@FVB@VerbatimOut
%     
%     \renewcommand\minted@savecode[1]{
%   \immediate\openout\minted@code\jobname.pyg
%   \immediate\write\minted@code{\expandafter\detokenize\expandafter{#1}}%
%   \immediate\closeout\minted@code}
%   
% \makeatother

\setbeamertemplate{navigation symbols}{}
\usecolortheme{dove}
\setbeamercolor{frametitle}{fg=white}
\hypersetup{colorlinks=true,pdfauthor={Eirikur Ernir Thorsteinsson},linkcolor=blue,urlcolor=blue}

\AtBeginSection[]
{
  \begin{frame}<beamer>
    \frametitle{Yfirlit}
    \tableofcontents[currentsection]
  \end{frame}
}

\author{Eiríkur Ernir Þorsteinsson}
\institute{Háskóli Íslands}
\date{Haust 2015}

\title{Tölvunarfræði 1a}
\subtitle{Vika 10, seinni fyrirlestur}

\begin{document}

\begin{frame}
\titlepage
\end{frame}

\section{Inngangur}

\begin{frame}{Í síðasta þætti\ldots}
\begin{itemize}
 \item Nafnlaus föll
 \item Fallshandföng
 \item Breytilegur fjöldi viðfanga
\end{itemize}
Kaflar: 10.1 - 10.3 
\end{frame}

\section{Hreiðruð föll (10.4)}

\begin{frame}{Hreiðruð föll}
\begin{itemize}
 \item Matlab leyfir hreiðruð föll (e. \emph{nested functions})
 \begin{itemize}
  \item Þá er skilgreining eins falls algjörlega innan í öðru falli
  \item Notað til að skipuleggja forrit
 \end{itemize}
\end{itemize}
\end{frame}

\begin{frame}[fragile]{Dæmi um hreiðrað fall}
Skoðum \texttt{nestedVolume.m} á blaðsíðu 333. Það reiknar út rúmmál tenings.
\end{frame}


\begin{frame}{Gildissvið breyta}
\vspace{\baselineskip}
\begin{itemize}
 \item Gildissvið breytu (e. \emph{scope of a variable}) er gildissvið ysta falls sem notar hana
 \begin{itemize}
  \item Innra fall hefur aðgang að breytum ytra falls
  \item Ytra fall hefur aðgang að breytum innra falls
  \item Ef ytra fall notar breytu innra falls ekki, þá er hún staðvær (e. \emph{local}) í innra falli
 \end{itemize}
 \item Breytur í ytri föllum virka því sem víðværar (e. \emph{global}) breytur í innri föllum - ekki þarf að senda þær inn í föllin sem inntaksbreytur
 \item Bókin eyðir ekki miklu púðri í þessar pælingar - \href{http://se.mathworks.com/help/matlab/matlab_prog/nested-functions.html?refresh=true}{skjölunin á Matlab-síðunni} er með meira fyrir fróðleiksfúsa
\end{itemize}
\end{frame}

\begin{frame}[fragile]{Fyrirlestraræfing}
\begin{columns}
\column{0.55\textwidth}
\begin{enumerate}
 \item Hvert er gildi skilabreytanna \texttt{a} og \texttt{b} í fallinu hér til hliðar?
\end{enumerate}

\column{0.45\textwidth}
\begin{minted}[frame=lines, fontsize=\small]{matlab}
function [a, b] = nestTest()
a = nested1();
b = nested2();

    function x = nested1()
        x = 1;
    end

    function x = nested2()
        x = 2;
    end
end
\end{minted}

\end{columns}
\end{frame}

\section{Endurkvæmni (10.5)}

\begin{frame}[fragile]{Endurkvæmni}
\begin{columns}
\column{0.6\textwidth}
\begin{itemize}
 \item Hægt er að skilgreina hugtök með því að nota hugtakið sjálft
 \begin{itemize}
  \item Slík skilgreining er kölluð endurkvæm
  \item Samanstendur af tveimur hlutum: Grunntilfelli (e. \emph{base case}) og almennu tilfelli (e. \emph{general case})
 \end{itemize}
 \item Grunntilfellið er notað til að ``loka'' endurkvæmninni
 \begin{itemize}
  \item Ef grunntilfellið vantar verður endurkvæmnin óendanleg
 \end{itemize}
\end{itemize}
\column{0.4\textwidth}

\vspace{0.1cm}
Endurkvæm skilgreining á hrópmerkingu:
\begin{align*}
1! &= 1\\
n! &= n\cdot(n-1)!\\
\end{align*}

% \[
% n! = \left\{
% \begin{array}{lll}
% 1&,& n = 1\\
% n\cdot(n-1)!&,& n \geq 1\\
% \end{array}
% \right.
% \]

\end{columns}
\end{frame}

\begin{frame}{(Óendanleg) endurkvæmni á ýmsum stöðum}
\begin{columns}
\column{0.6\textwidth}
\begin{itemize}
 \item Endurkvæmar skammstafanir:
 \begin{itemize}
  \item VISA stendur fyrir ``\textbf{V}ISA \textbf{I}nternational \textbf{S}ervice \textbf{A}ssociation''
  \item Bing stendur fyrir ``\textbf{B}ing \textbf{I}s \textbf{N}ot \textbf{G}oogle''
 \end{itemize}
 \item ``Vinir vina minna eru vinir mínir''
\end{itemize}
\column{0.4\textwidth}
\includegraphics[width=\linewidth]{Pics/google-recursion}
\end{columns}
\end{frame}

\begin{frame}{Endurkvæmt mynstur}
Sierpinski-þríhyrningurinn:
\begin{center}
\includegraphics[width=0.6\linewidth]{Pics/sierpinski}
\end{center}
\end{frame}

\begin{frame}[fragile]{Endurkvæmni í Matlab}
Föll í Matlab geta verið endurkvæm (bls. 335):
\begin{minted}[frame=lines]{matlab}
function facn = fact(n)
if n == 1
  facn = 1; % Grunntilfelli
else
  facn = n * fact(n-1); % Almennt tilfelli
end
end
\end{minted}
Í almenna tilfellinu er kallað aftur á fallið sjálft!
\end{frame}

\begin{frame}[fragile]{Dæmi: Endurkvæm útprentun}
Fall sem prentar út orð í setningu í öfugri röð (bls. 337):
\begin{minted}[frame=lines]{matlab}
function prtwords(sent)

  [word, rest] = strtok(sent);
  if ~isempty(rest) % Grunntilfelli: rest er tómur
    prtwords(rest);
  end
  disp(word)
end
\end{minted}
\end{frame}

\begin{frame}[fragile]{Dæmi: Endurkvæmt hjálparfall}
\vspace{\baselineskip}
\begin{minted}[frame=lines, fontsize=\scriptsize]{matlab}
function maximum = recursiveMax(inputVector)
  
  maximum = maxHelper(inputVector, -inf);
  
  function partialMaximum = maxHelper(partialV, currentMax)
    if isempty(partialV)
      partialMaximum = currentMax;
    else
       if partialV(1) > currentMax
         partialMaximum = maxHelper(partialV(2:end), partialV(1));
       else
         partialMaximum = maxHelper(partialV(2:end), currentMax);
       end
    end
  end
  
end
\end{minted}

\end{frame}



\begin{frame}{Tengd hugtök}
\begin{itemize}
 \item Endurkvæmni (í forritun) byggir á svipuðum hugmyndum og þrepun í stærðfræði
 \item Endurkvæmni er afar skyld lykkjum
 \begin{itemize}
  \item Lykkjur eru leið til að keyra sama kóðann oft, með smá breytingum í hverri ítrun
  \item Endurkvæmni er leið til að keyra sama kóðann oft, með smá breytingum í hverju fallskalli
 \end{itemize}
\end{itemize}

\end{frame}

\begin{frame}{Hvenær á að nota endurkvæmni?}
\begin{itemize}
 \item Endurkvæm útgáfa af forriti er oftast dýrari í keyrslu en útgáfa sem notar ítrun
 \begin{itemize}
  \item Það kostar mun meira að framkvæma kall á fall en að framkvæma eina ítrun
 \end{itemize}
 \item En stundum er endurkvæma útgáfan skiljanlegri
 \begin{itemize}
  \item Tími forritarans er oft meira virði en tími tölvunnar
  \item Til eru forritunarmál sem hafa engar lykkjur, bara endurkvæmni
 \end{itemize}
\end{itemize}
\end{frame}




\begin{frame}[fragile]{Fyrirlestraræfing}
\begin{columns}
\column{0.4\textwidth}
\begin{enumerate}
 \setcounter{enumi}{1}
 \item Breytið fallinu \texttt{prtwords} (á fyrri glæru) svo það prenti setningu út í réttri röð, ekki öfugri (ráðlegging: Færið \texttt{disp}ið)
 \item Hverju skilar fallskallið \texttt{f(v, 5)}, þar sem \texttt{v} er vigur af tölum?
\end{enumerate}

\column{0.6\textwidth}
\begin{minted}[frame=lines, fontsize=\small]{matlab}
function r = f(v, x)
  if isempty(v)
    r = false;
  else
    r = (v(1)==x) || f(v(2:end),x);
 end
end
\end{minted}

\end{columns}
\end{frame}


\end{document}
