\documentclass[handout]{beamer}

\usepackage{Haust2017glærur}

\title{Tölvunarfræði 1a}
\subtitle{Vika 10, seinni fyrirlestur}

\begin{document}

\begin{frame}
\titlepage
\end{frame}

\section{Inngangur}

\begin{frame}{Í síðasta þætti\ldots}
\begin{itemize}
 \item Nafnlaus föll
 \item Fallshandföng
 \item Breytilegur fjöldi viðfanga
\end{itemize}
Kaflar: 10.1 - 10.4 
\end{frame}

\section{Endurkvæmni (10.5)}

\begin{frame}[fragile]{Endurkvæmni}
\begin{columns}
\column{0.6\textwidth}
\begin{itemize}
 \item Hægt er að skilgreina hugtök með því að nota hugtakið sjálft
 \begin{itemize}
  \item Slík skilgreining er kölluð endurkvæm
  \item Samanstendur af tveimur hlutum: Grunntilfelli (e. \emph{base case}) og almennu tilfelli (e. \emph{general case})
 \end{itemize}
 \item Grunntilfellið er notað til að ``loka'' endurkvæmninni
 \begin{itemize}
  \item Ef grunntilfellið vantar verður endurkvæmnin óendanleg
 \end{itemize}
\end{itemize}
\column{0.4\textwidth}

\vspace{0.1cm}
Endurkvæm skilgreining á hrópmerkingu:
\begin{align*}
1! &= 1\\
n! &= n\cdot(n-1)!\\
\end{align*}

\end{columns}
\end{frame}

\begin{frame}{Endurkvæmt mynstur}
Sierpinski-þríhyrningurinn:
\begin{center}
\includegraphics[width=0.6\linewidth]{Pics/sierpinski}
\end{center}
\end{frame}

\begin{frame}[fragile]{Endurkvæmni í Matlab}
Föll í Matlab geta verið endurkvæm (bls. 335):
\begin{minted}[frame=lines, label=fact.m]{matlab}
function facn = fact(n)
if n == 0
  facn = 1; % Grunntilfelli
else
  facn = n * fact(n-1); % Almennt tilfelli
end
end
\end{minted}
Í almenna tilfellinu er kallað aftur á fallið sjálft!
\end{frame}

\begin{frame}[fragile]{Dæmi: Endurkvæm útprentun}
Fall sem prentar út orð í setningu í öfugri röð (úr bók):
\begin{minted}[frame=lines, label=prtwords.m]{matlab}
function prtwords(sent)

  [word, rest] = strtok(sent);
  if ~isempty(rest) % Grunntilfelli: rest er tómur
    prtwords(rest);
  end
  disp(word)
end
\end{minted}
\end{frame}

\begin{frame}[fragile]{Dæmi: Endurkvæmt hjálparfall}
\vspace{\baselineskip}
\begin{minted}[frame=lines, fontsize=\scriptsize, label=recursivemax.m]{matlab}
function maximum = recursivemax(inputVector)
maximum = maxhelper(inputVector, -inf);

function partialmaximum = maxhelper(partialV, currentMax)
	if isempty(partialV)
	partialmaximum = currentMax;
	else
		if partialV(1) > currentMax
		partialmaximum = maxhelper(partialV(2:end), partialV(1));
		else
		partialmaximum = maxhelper(partialV(2:end), currentMax);
		end
	end
end
end
\end{minted}
\end{frame}

\begin{frame}{Tengd hugtök}
\begin{itemize}
 \item Endurkvæmni (í forritun) byggir á svipuðum hugmyndum og þrepun í stærðfræði
 \begin{itemize}
   \item Helsti munurinn - í endurkvæmni brjótum við vandamálið niður, í þrepun byggjum við okkur upp
 \end{itemize}
 \item Reikningar sem framkvæmdir eru af endurkvæmni eru mjög svipaðir þeim sem lykkjur framkvæma
 \begin{itemize}
  \item Lykkjur eru leið til að keyra sama kóðann oft, með smá breytingum í hverri ítrun
  \item Endurkvæmni er leið til að keyra sama kóðann oft, með smá breytingum í hverju fallskalli
 \end{itemize}
\end{itemize}

\end{frame}

\begin{frame}{Hvenær á að nota endurkvæmni?}
\begin{itemize}
 \item Endurkvæm útgáfa af forriti er oftast dýrari í keyrslu en útgáfa sem notar ítrun
 \begin{itemize}
  \item Það kostar mun meira að framkvæma kall á fall en að framkvæma eina ítrun
 \end{itemize}
 \item En stundum er endurkvæma útgáfan skiljanlegri
 \begin{itemize}
  \item Tími forritarans er oft meira virði en tími tölvunnar
  \item Til eru forritunarmál sem hafa engar lykkjur, bara endurkvæmni
 \end{itemize}
\end{itemize}
\end{frame}

\begin{frame}[fragile]{Fyrirlestraræfing}
\begin{columns}
\column{0.4\textwidth}
\begin{enumerate}
    \item Breytið fallinu \texttt{prtwords} (á fyrri glæru) svo það prenti setningu út í réttri röð, ekki öfugri (ráðlegging: Færið \texttt{disp}ið)
    \item Hverju skilar fallskallið \texttt{f(v, 5)}, þar sem \texttt{v} er vigur af tölum?
\end{enumerate}

\column{0.6\textwidth}
\begin{minted}[frame=lines, fontsize=\small]{matlab}
function r = f(v, x)
  if isempty(v)
    r = false;
  else
    r = (v(1)==x) || f(v(2:end),x);
  end
end
\end{minted}

\end{columns}
\end{frame}

\begin{frame}{Viðbótardæmi}
    \begin{itemize}
        \item Dæmi 10.14 (Hreiðruð föll)
		\item Dæmi 10.15 (Nafnlaust fall)
		\item Dæmi 10.20 (Fallshandfang)
		\item Dæmi 10.25 (Endurkvæmni)
    \end{itemize}
\end{frame}

\end{document}
