\documentclass{beamer}

\usepackage{Haust2016glærur}

\title{Tölvunarfræði 1a}
\subtitle{Vika 10, seinni fyrirlestur}

\begin{document}

\begin{frame}
\titlepage
\end{frame}

\section{Inngangur}

\begin{frame}{Í síðasta þætti\ldots}
\begin{itemize}
 \item Nafnlaus föll
 \item Fallshandföng
 \item Breytilegur fjöldi viðfanga
\end{itemize}
Kaflar: 10.1 - 10.3 
\end{frame}

\section{Hreiðruð föll (10.4)}

\begin{frame}{Hreiðruð föll}
\begin{itemize}
 \item Matlab leyfir hreiðruð föll (e. \emph{nested functions})
 \begin{itemize}
  \item Þá er skilgreining eins falls algjörlega innan í öðru falli
  \item Notað til að skipuleggja forrit
 \end{itemize}
\end{itemize}
\end{frame}

\begin{frame}[fragile]{Dæmi um hreiðrað fall}
\begin{minted}[frame=lines, label=nestedvolume.m]{matlab}
function outvol = nestedvolume(len, wid, ht)
% nestedvolume receives the length, width, and
% height of a cube and returns the volume; it calls
% a nested function that returns the area of the base
% Format: nestedvolume(length,width,height)
outvol = base * ht;
  function outbase = base
  % returns the area of the base
  outbase = len * wid;
  end % base function
end % nestedvolume function
\end{minted}

\end{frame}


\begin{frame}{Gildissvið breyta}
\vspace{\baselineskip}
\begin{itemize}
 \item Gildissvið breytu (e. \emph{scope of a variable}) er gildissvið ysta falls sem notar hana
 \begin{itemize}
  \item Innra fall hefur aðgang að breytum ytra falls
  \item Ytra fall hefur aðgang að breytum innra falls
  \item Ef ytra fall notar breytu innra falls ekki, þá er hún staðvær (e. \emph{local}) í innra falli
 \end{itemize}
 \item Breytur í ytri föllum virka því sem víðværar (e. \emph{global}) breytur í innri föllum - ekki þarf að senda þær inn í föllin sem inntaksbreytur
 \item Bókin eyðir ekki miklu púðri í þessar pælingar - \href{http://se.mathworks.com/help/matlab/matlab_prog/nested-functions.html?refresh=true}{skjölunin á Matlab-síðunni} er með meira fyrir fróðleiksfúsa
\end{itemize}
\end{frame}

\begin{frame}[fragile]{Fyrirlestraræfing}
\begin{columns}
\column{0.55\textwidth}
Skráið ykkur inn á \url{http://socrative.com/} og gerið fyrstu spurninguna.

Herbergisnúmer = \texttt{TOL105G2016}

Notendanafn = HÍ-tölvupóstfang
\column{0.45\textwidth}
\begin{minted}[frame=lines, fontsize=\small]{matlab}
function [a, b] = nestTest()
a = nested1();
b = nested2();

    function x = nested1()
        x = 1;
    end

    function x = nested2()
        x = 2;
    end
end
\end{minted}

\end{columns}
\end{frame}

\section{Endurkvæmni (10.5)}

\begin{frame}[fragile]{Endurkvæmni}
\begin{columns}
\column{0.6\textwidth}
\begin{itemize}
 \item Hægt er að skilgreina hugtök með því að nota hugtakið sjálft
 \begin{itemize}
  \item Slík skilgreining er kölluð endurkvæm
  \item Samanstendur af tveimur hlutum: Grunntilfelli (e. \emph{base case}) og almennu tilfelli (e. \emph{general case})
 \end{itemize}
 \item Grunntilfellið er notað til að ``loka'' endurkvæmninni
 \begin{itemize}
  \item Ef grunntilfellið vantar verður endurkvæmnin óendanleg
 \end{itemize}
\end{itemize}
\column{0.4\textwidth}

\vspace{0.1cm}
Endurkvæm skilgreining á hrópmerkingu:
\begin{align*}
1! &= 1\\
n! &= n\cdot(n-1)!\\
\end{align*}

\end{columns}
\end{frame}

\begin{frame}{Endurkvæmt mynstur}
Sierpinski-þríhyrningurinn:
\begin{center}
\includegraphics[width=0.6\linewidth]{Pics/sierpinski}
\end{center}
\end{frame}

\begin{frame}[fragile]{Endurkvæmni í Matlab}
Föll í Matlab geta verið endurkvæm (bls. 335):
\begin{minted}[frame=lines]{matlab}
function facn = fact(n)
if n == 1
  facn = 1; % Grunntilfelli
else
  facn = n * fact(n-1); % Almennt tilfelli
end
end
\end{minted}
Í almenna tilfellinu er kallað aftur á fallið sjálft!
\end{frame}

\begin{frame}[fragile]{Dæmi: Endurkvæm útprentun}
Fall sem prentar út orð í setningu í öfugri röð (úr bók):
\begin{minted}[frame=lines]{matlab}
function prtwords(sent)

  [word, rest] = strtok(sent);
  if ~isempty(rest) % Grunntilfelli: rest er tómur
    prtwords(rest);
  end
  disp(word)
end
\end{minted}
\end{frame}

\begin{frame}{Tengd hugtök}
\begin{itemize}
 \item Endurkvæmni (í forritun) byggir á svipuðum hugmyndum og þrepun í stærðfræði
 \item Endurkvæmni er afar skyld lykkjum
 \begin{itemize}
  \item Lykkjur eru leið til að keyra sama kóðann oft, með smá breytingum í hverri ítrun
  \item Endurkvæmni er leið til að keyra sama kóðann oft, með smá breytingum í hverju fallskalli
 \end{itemize}
\end{itemize}

\end{frame}

\begin{frame}{Hvenær á að nota endurkvæmni?}
\begin{itemize}
 \item Endurkvæm útgáfa af forriti er oftast dýrari í keyrslu en útgáfa sem notar ítrun
 \begin{itemize}
  \item Það kostar mun meira að framkvæma kall á fall en að framkvæma eina ítrun
 \end{itemize}
 \item En stundum er endurkvæma útgáfan skiljanlegri
 \begin{itemize}
  \item Tími forritarans er oft meira virði en tími tölvunnar
  \item Til eru forritunarmál sem hafa engar lykkjur, bara endurkvæmni
 \end{itemize}
\end{itemize}
\end{frame}

\begin{frame}[fragile]{Fyrirlestraræfing}
\begin{columns}
\column{0.4\textwidth}

Skráið ykkur inn á \url{http://socrative.com/} og klárið æfinguna.

Herbergisnúmer = \texttt{TOL105G2016}

Notendanafn = HÍ-tölvupóstfang

\column{0.6\textwidth}
\begin{minted}[frame=lines, fontsize=\small]{matlab}
function r = f(v, x)
  if isempty(v)
    r = false;
  else
    r = (v(1)==x) || f(v(2:end),x);
 end
end
\end{minted}

\end{columns}
\end{frame}


\end{document}
