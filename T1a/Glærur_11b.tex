\documentclass{beamer}

\usepackage[utf8]{inputenc}
\usepackage[icelandic]{babel}
\usepackage[T1]{fontenc}

\usepackage{booktabs}
\usepackage{minted} %Minted and configuration
\usepackage{framed}
\usepackage{tikz}
\usemintedstyle{default}
\renewcommand{\theFancyVerbLine}{\sffamily \arabic{FancyVerbLine}}
\newcommand{\Mod}[1]{\ \text{mod}\ #1}

% \makeatletter
% \minted@define@extra{label}
% \makeatother

\usebackgroundtemplate%
{%
\vbox to \paperheight{
\includegraphics[width=\paperwidth]{Pics/hi-slide-head}

\vfill
\hspace{0.5cm}\includegraphics[width=0.3\paperwidth]{Pics/hi-von-logo}
\vspace{0.5cm}
    }%
}

% \makeatletter
% \newcommand{\minted@write@detok}[1]{%
%   \immediate\write\FV@OutFile{\detokenize{#1}}}%
%   
%   \newcommand{\minted@FVB@VerbatimOut}[1]{%
%   \@bsphack
%   \begingroup
%     \FV@UseKeyValues
%     \FV@DefineWhiteSpace
%     \def\FV@Space{\space}%
%     \FV@DefineTabOut
%     %\def\FV@ProcessLine{\immediate\write\FV@OutFile}% %Old, non-Unicode version
%     \let\FV@ProcessLine\minted@write@detok %Patch for Unicode
%     \immediate\openout\FV@OutFile #1\relax
%     \let\FV@FontScanPrep\relax
% %% DG/SR modification begin - May. 18, 1998 (to avoid problems with ligatures)
%     \let\@noligs\relax
% %% DG/SR modification end
%     \FV@Scan}
%     \let\FVB@VerbatimOut\minted@FVB@VerbatimOut
%     
%     \renewcommand\minted@savecode[1]{
%   \immediate\openout\minted@code\jobname.pyg
%   \immediate\write\minted@code{\expandafter\detokenize\expandafter{#1}}%
%   \immediate\closeout\minted@code}
%   
% \makeatother

\setbeamertemplate{navigation symbols}{}
\usecolortheme{dove}
\setbeamercolor{frametitle}{fg=white}
\hypersetup{colorlinks=true,pdfauthor={Eirikur Ernir Thorsteinsson},linkcolor=blue,urlcolor=blue}

\AtBeginSection[]
{
  \begin{frame}<beamer>
    \frametitle{Yfirlit}
    \tableofcontents[currentsection]
  \end{frame}
}

\author{Eiríkur Ernir Þorsteinsson}
\institute{Háskóli Íslands}
\date{Haust 2015}

\title{Tölvunarfræði 1a}
\subtitle{Vika 11, seinni fyrirlestur}

\begin{document}

\begin{frame}
\titlepage
\end{frame}

\section{Inngangur}

\begin{frame}{Í síðasta þætti\ldots}
\begin{itemize}
 \item Færslur
 \item Hlutbundin forritun
\end{itemize}
Kafli: 8.2

\href{http://www.ce.berkeley.edu/~sanjay/e7/oop.pdf}{Lesefni um hlutbundna forritun}
\end{frame}

\subsection{Lausn á gamalli fyrirlestraræfingu}

\begin{frame}[fragile]{Lausn á fyrirlestraræfingu}
\vspace{\baselineskip}
Breytið \texttt{disp} aðferðinni í \texttt{Fraction} klasanum okkar svo að tölur með 1 í nefnara séu sýndar án nefnara (t.d. sýna töluna \texttt{3} en ekki \texttt{3/1})
\pause

Bætum við \texttt{if}-setningu:
\begin{minted}[frame=lines]{matlab}
function disp(frac)
    if frac.denom ~= 1
        fprintf('%d/%d\n',frac.numer, frac.denom)
    else
        fprint('%d\n', frac.numer)
    end
end
\end{minted}

\end{frame}

\begin{frame}[fragile]{Lausn á fyrirlestraræfingu}
\vspace{\baselineskip}
Bætið margföldunaraðferð við \texttt{Fraction} klasann
\pause

Mjög svipað og \texttt{add} aðferðin:
\begin{minted}[frame=lines]{matlab}
function newFrac = mult(frac1, frac2)
   newN = frac1.numer * frac2.numer;
   newD = frac1.denom * frac2.denom;
   newFrac = Fraction(newN, newD);
end
\end{minted}

\end{frame}

\section{Að skilgreina vandamál}

\begin{frame}{Hvað erum við eiginlega að gera hérna?}
\pause
\begin{itemize}
 \item Venjulega erum við ekki að skrifa kóða kóðans vegna
 \item Oftast forritum við til að \emph{leysa vandamál}
 \item Vandamál eru af ýmsum toga
 \begin{itemize}
  \item Stundum vel skilgreind vandamál
  \begin{itemize}
   \item ``Vinstra framdekkið á bílnum mínum er sprungið.''
  \end{itemize}
  \item Oftar óljós vandamál
  \begin{itemize}
   \item ``Bíllinn minn er bilaður.''
  \end{itemize}
 \end{itemize}
\end{itemize}
\end{frame}

\begin{frame}{Óljós vandamál}
\begin{itemize}
 \item Í fullkomnum heimi vitum við alltaf til hvers er ætlast af okkur/hvað við viljum gera
 \item Í alvöru heimi þurfum við oft að komast að því sjálf
 \begin{itemize}
  \item Þurfum að ákvarða lýsingu á vandamálinu
  \item Þurfum líka að ákvarða hvernig lausnin ætti að líta út
  \item Best að gera þetta \emph{áður} en byrjað er að forrita!
 \end{itemize}
\end{itemize}
\end{frame}

\begin{frame}{Að lýsa vandamáli betur}
\begin{itemize}
 \item Tökum okkur tíma í að skoða nákvæmlega til hvers er ætlast af okkur
 \begin{itemize}
  \item Ferlið getur verið mjög formlegt - gerð er þarfagreining og/eða kröfulýsing
  \begin{itemize}
   \item Sérstaklega algengt þegar um útselda forritunarvinnu er að ræða
  \end{itemize}
  \item Getur líka verið mjög óformlegt
  \begin{itemize}
   \item Sest niður og vandamálið rætt
   \item Erfiðast - að hafa agann til að sleppa því ekki!
  \end{itemize}
 \end{itemize}
\end{itemize}
\end{frame}

\begin{frame}{Abstract dæmi}
\begin{itemize}
 \item Fáum í hendurnar safn af mæligögnum (gagnapunktum) sem okkur grunar að innihaldi margar lélegar niðurstöður. Viljum henda lélegu niðurstöðunum en geyma þær góðu.
 \begin{itemize}
  \item Opin spurning: Hvað þurfum við að gera?
 \end{itemize}
\end{itemize}
\end{frame}

\section{Erfið vandamál}

\begin{frame}{Að leysa erfið vandamál}
\begin{columns}
\column{0.5\textwidth}
\begin{itemize}
 \item Fæst vandamál eru árennileg við fyrstu sýn
 \begin{itemize}
  \item Erfitt að vita hvar á að byrja
 \end{itemize}
 \item Afar stór hluti af því að verða góður forritari er að læra að brjóta vandamál niður
 \begin{itemize}
  \item Stórt vandamál er venjulega samansafn af litlum vandamálum
 \end{itemize}
\end{itemize}

\column{0.5\textwidth}
\begin{center}
\includegraphics[width=0.8\linewidth]{Pics/catcube}
\end{center}
\end{columns}
\end{frame}

\begin{frame}{Abstract vandamál}
\begin{itemize}
 \item Vandamál: Vinstra framdekkið á bílnum mínum er sprungið. Ákveðum að við þurfum að skipta um dekk. Hvernig leysum við það vandamál? \pause
 \begin{itemize}
  \item Tökum sprungna dekkið af
  \item Sækjum varadekkið
  \item Setjum varadekkið á
 \end{itemize}
\end{itemize}
\end{frame}

\begin{frame}{Abstract vandamál}
\begin{itemize}
 \item Vandamál: Hvernig tökum við sprungna dekkið af? \pause
 \begin{itemize}
  \item Losum um rærnar
  \item Tjökkum bílinn upp
  \item Losum rærnar alveg af
  \item Tökum sprungna dekkið af
 \end{itemize}
 \item Hugsunin: höldum áfram að brjóta vandamálið niður þar til það er orðið einfalt
\end{itemize}
\end{frame}

\begin{frame}{Niðurbrot á vandamálum}
\begin{itemize}
 \item Til að brjóta niður forritunarvandamál getum við spurt okkur ýmissa spurninga:
 \begin{itemize}
  \item Get ég leyst einhvern afmarkaðan hluta af vandamálinu?
  \item Gæti ég leyst aðeins einfaldari útgáfu af þessu verkefni?
  \begin{itemize}
   \item Get ég leyst ákveðin tilvik?
  \end{itemize}
  \item Er þetta vandamál á einhvern hátt svipað vandamáli sem ég kann að leysa?
  \item Hvaða gögn er ég með?
  \begin{itemize}
   \item Hvernig er best að setja þau fram?
  \end{itemize}
 \end{itemize}
\end{itemize}
\end{frame}

\begin{frame}{Þegar vandamálið hefur verið brotið niður}
\begin{itemize}
 \item Þegar vandamálið er orðið nógu smátt er auðvelt\footnote{Eða a.m.k. auðveldara} að leysa það
 \item Virðist vandamálið ennþá vera of stórt - brjótið það meira niður!
 \begin{itemize}
  \item Hugmyndin er að enda með eina ``einingu'' sem auðvelt er að útfæra
 \end{itemize}
 \item Forritunarmál hjálpa til við þetta ferli
 \begin{itemize}
  \item Eitt fall táknar venjulega eina gerð útreikninga
  \item Einn klasi táknar venjulega eitt fyrirbrigði
 \end{itemize}
\end{itemize}
\end{frame}


\begin{frame}{Erfiðleikar við niðurbrot}
\begin{itemize}
 \item Það er nokkur list að einangra leysanlega hluta vandamála
 \item Almennt gagnlegt
 \begin{itemize}
  \item Eyða meiri tíma í að nota tól (t.d. forritunarmál) sem þið þekkið
  \begin{itemize}
   \item Séu tólin kunnugleg er auðveldara að koma auga á lítil vandamál sem hægt er að leysa með þeim
  \end{itemize}
  \item Læra á fleiri tól
  \begin{itemize}
   \item Nýir hugsunarhættir fylgja nýjum tólum
  \end{itemize}
 \end{itemize}
\end{itemize}
\end{frame}

\begin{frame}{Lexía}
Þegar við erum með forritunarvandamál, gerum eftirfarandi:
\begin{enumerate}
 \item Greinum vandamálið vandlega
 \item Brjótum vandamálið niður í viðráðanlegar einingar
 \item Forritum hverja einingu fyrir sig
\end{enumerate}
\end{frame}


\begin{frame}{Fyrirlestraræfing}
\begin{enumerate}
 \item Við viljum skrifa leikinn Mylla (Tic Tac Toe) í Matlab. Hvernig gæti lausn á því vandamáli litið út?
 \item Við þurfum að framkvæma útreikninga með tvinntölum í hlutbundnu forritunarmáli, svo við þurfum að útfæra tvinntölur. Hvernig getum við brotið það vandamál niður?
\end{enumerate}

\end{frame}


\end{document}
