\documentclass[handout]{beamer}

\usepackage{Haust2017glærur}

\title{Tölvunarfræði 1a}
\subtitle{Vika 1, seinni fyrirlestur}

\begin{document}

\begin{frame}
\titlepage
\end{frame}

\section{Inngangur}

\begin{frame}{Í síðasta þætti\ldots}
\begin{itemize}
 \item Kynning á námskeiðinu
 \item Forritunarhugtakið
 \item Hvað Matlab er
 \begin{itemize}
  \item Hvernig gekk að setja það upp?
 \end{itemize}
 \item Bitar og tvíundartölur
\end{itemize}
\end{frame}

\section{Breytur}

\begin{frame}{Breytur}
\begin{itemize}
 \item Gefum okkur að við höfum aðgang að minni í tölvu (samanstendur af bætum)
 \item Breyta (e. \emph{variable}) er afmarkað minnishólf sem inniheldur gildi
 \item Til að geta vísað í minnishólfið gefum við því nafn (breytuheiti, e. \emph{variable name})
 \item Stærð breyta er mismunandi eftir því hvaða af hvaða tagi (e. \emph{type} eða \emph{class}) gögnin eru
 \begin{itemize}
  \item Dæmi um tög eru heiltölur, fleytitölur, bókstafir, \ldots
 \end{itemize}
\end{itemize}
\end{frame}

\begin{frame}[fragile]{Að búa til breytur}
Dæmi um hvernig búa má til breytu:
\begin{minted}[frame=lines]{matlab}
>> myNumber = 6
myNumber =  
     6
\end{minted}
Merking skipunarinnar í efstu línunni er ``settu gildið $6$ í breytuna \texttt{myNumber}''. Matlab svarar með gildi nýju breytunnar.

Sé semíkomma sett á eftir skipuninni svarar Matlab ekki:
\begin{minted}[frame=lines]{matlab}
>> myNumber = 6;
\end{minted}
\end{frame}

\begin{frame}{Workspace-glugginn}
\begin{columns}
\column{0.5\textwidth}
\begin{itemize}
 \item Workspace-glugginn sýnir ýmsar upplýsingar um skilgreindar breytur
 \item Hægri-smella má á dálkana til að sýna fleiri
\end{itemize}
\column{0.5\textwidth}
\includegraphics[width=\linewidth]{Pics/workspace-window}
\end{columns}
\end{frame}

\begin{frame}[fragile]{Breytan \texttt{ans}}
Breytan \texttt{ans} er sjálfkrafa gefið gildi þegar reiknisetning er framkvæmd 
\begin{minted}[frame=lines]{matlab}
>> 2+3
ans = 
     5
\end{minted}
\end{frame}

\begin{frame}[fragile]{Að nota breytur}
Breytur geymast, svo hægt er að nota þær áfram (annars væri lítið gagn í þeim)
\begin{minted}[frame=lines]{matlab}
>> x = 1;
>> y = 2;
>> x + y
ans =
     3
\end{minted}
Breyta helst skilgreind þar til henni er eytt eða þar til við förum út úr sviði (e. \emph{scope}) breytunnar (meira síðar)
\end{frame}

\begin{frame}{Breytuheiti}
\begin{itemize}
 \item Allar breytur hafa nöfn
 \begin{itemize}
  \item Nöfnin verða að byrja á bókstaf (úr stafrófinu)
  \begin{itemize}
   \item \texttt{number123} er í lagi, en  \texttt{123number}  ekki í lagi
   \item Breytuheiti mega ekki innihalda bil
  \end{itemize}
  \item Engir séríslenskir stafir leyfðir
  \begin{itemize}
   \item \texttt{upphaed}  er í lagi, en  \texttt{upphæð}  ekki í lagi
  \end{itemize}
  \item Greinarmunur gerður á hástaf og lágstaf
  \begin{itemize}
   \item \texttt{mynumber}, \texttt{Mynumber} og \texttt{MYNUMBER} eru allt aðskildar breytur
   \item Tillaga: Láta breytuheiti byrja á litlum staf, nota hástafi til að skilja á milli orða: \texttt{myNumber}
  \end{itemize}
  \item Nokkur lykilorð (key words) eru frátekin af Matlab og má ekki nota sem nöfn
  \begin{itemize}
   \item \texttt{>> if = 4;}    gefur villu
  \end{itemize}
 \end{itemize}
\end{itemize}
\end{frame}

\begin{frame}[fragile]{Meira um breytuheiti}
\begin{itemize}
 \item Reynum að hafa breytuheiti lýsandi
 \begin{itemize}
  \item \texttt{force = mass * acceleration} er skýrara en \texttt{z = x * y} 
 \end{itemize}
 \item Það að breytuheiti sé löglegt þýðir ekki endilega að það sé \emph{gott}
 \begin{itemize}
  \item T.d. er hægt að yfirskrifa innbyggðar breytur:
\begin{minted}[frame=lines]{matlab}
>> pi = 4;
>> radius = 2;
>> area = radius^2 * pi
area = 
      16
\end{minted}
 \end{itemize}
\end{itemize}
\end{frame}

\begin{frame}{Skipanir tengdar breytum}
\begin{center}
\begin{tabular}{ll}
\toprule
Skipun&Merking\\
\midrule
\texttt{who}&sýnir nöfn skilgreindra breyta\\
\texttt{whos}&sýnir upplýsingar um breytur\\
\texttt{clear}&eyðir öllum skilgreindum breytum\\
\texttt{clear <breyta>}&eyðir tiltekinni breytu\\
\bottomrule
\end{tabular}
\end{center}
\end{frame}

\begin{frame}{Tög breyta}
\begin{itemize}
 \item Allar breytur hafa tag
 \item Matlab ákvarðar tag breytu sjálfkrafa út frá gildinu sem breytunni er gefið
 \item Helstu tög eru:
 \begin{itemize}
  \item Kommutölur (e. \emph{floating point numbers}):  \texttt{single}, \texttt{double}
  \begin{itemize}
   \item \texttt{double} er sjálfgefna tagið fyrir tölur
  \end{itemize}
  \item Heiltölur (e. \emph{integers}): \texttt{int8}, \texttt{int16}, \texttt{int32}, \texttt{int64}
  \item Bókstafir (e. \emph{characters}): \texttt{char}
  \begin{itemize}
   \item T.d. \texttt{'a'} og \texttt{'halló heimur'}
  \end{itemize}
  \item Rökgildi (e. \emph{logical} eða \emph{boolean}): \texttt{logical}
  \begin{itemize}
   \item Bara tvö möguleg gildi: \texttt{true} og \texttt{false} (eða 1 og 0)
  \end{itemize}
 \end{itemize}
\end{itemize}
\end{frame}

\section{Útreikningar}

\begin{frame}[fragile]{Reiknisegðir}
\begin{columns}
\column{0.5\textwidth}
\begin{itemize}
 \item Við getum reiknað með Matlab á mjög svipaðan hátt og í (mjög öflugri) reiknivél
 \begin{itemize}
  \item \texttt{+}, \texttt{-}: samlagning og frádráttur
  \item \texttt{*}, \texttt{/}, \texttt{\textbackslash}: margföldun, deiling, deilt uppí
  \item \texttt{\^}: veldishafning
 \end{itemize}
 \item Munum að við getum notað breytur!
\end{itemize}
\column{0.5\textwidth}
\begin{minted}[frame=lines]{matlab}
>> height = 4;
>> width = 2.5;
>> area = height * width
area =  10
\end{minted}
\end{columns}
\end{frame}

\begin{frame}{Forgangur}
Líkt og í hefðbundinni stærðfræði, þá eru sumir virkjar (e. \emph{operators}) framkvæmdir á undan öðrum.
\begin{center}
\begin{tabular}{ll}
\toprule
Virki&Hópur\\
\midrule
\texttt{()}&Svigar\\
\texttt{\^}&Veldishafning\\
\texttt{-}&Viðsnúningur formerkis\\
\texttt{*, /, \textbackslash}&Margföldun og deiling\\
\texttt{+, -}&Samlagning og frádráttur\\
\bottomrule
\end{tabular}
\end{center}
Virkjar efst í töflunni eru framkvæmdir fyrst.
\end{frame}

\begin{frame}[fragile]{Forgangur aðgerða}
\begin{itemize}
 \item Matlab les frá vinstri til hægri ef um ``jafntefli'' er að ræða
\begin{minted}[frame=lines]{matlab}
>> 1/2*2
ans =
     1
\end{minted}
 \item Notið sviga ef þið eruð í vafa
\end{itemize}
\end{frame}

\section{Föll}

\begin{frame}[fragile]{Föll í Matlab}
\begin{itemize}
 \item Föll eru lykilfyrirbrigði í forritun.
 \item Matlab er með innbyggð föll sem hægt er að nota:
\end{itemize}
\begin{minted}[frame=lines]{matlab}
>> sin(pi/2)
ans =  1
>> cos(0)
ans =  1
\end{minted}
\end{frame}

\begin{frame}{Innbyggð föll}
\begin{itemize}
 \item Mjög mikið er til af innbyggðum stærðfræðiföllum
 \item \texttt{sin, cos, tan, asin, acos, atan, exp, log, log10, sqrt, abs, round, mod, rem,}\ldots
 \item Nokkur föll breyta kommutölum í heiltölur:
 \begin{itemize}
  \item \texttt{floor(x)}: rúnnar niður að $-\infty$
  \item \texttt{ceil(x)}: rúnnar upp að $+\infty$
  \item \texttt{fix(x)}: rúnnar að núlli
  \item \texttt{round(x)}: rúnnar að næstu heiltölu
 \end{itemize}
 \item \texttt{>> help elfun} gefur lista
\end{itemize}
\end{frame}

\begin{frame}[fragile]{Frekari upplýsingar}
Til að fá snöggt yfirlit yfirlit yfir innbyggt fall má skrifa \texttt{help} og svo nafn fallsins:
\begin{minted}[frame=lines]{matlab}
>> help sin
\end{minted}
eða \texttt{doc} og svo nafn fallsins til að skoða það í hjálpinni:
\begin{minted}[frame=lines]{matlab}
>> doc sin
\end{minted}
Hjálpin í Matlab er almennt mjög gagnleg.
\end{frame}

\begin{frame}[fragile, shrink]{Dæmi um útreikninga}
\vspace{1cm}
Lausn á annars stigs jöfnunni $x^2 + 3x - 4 = 0$
\begin{minted}[frame=lines]{matlab}
>> a = 1;
>> b = 3;
>> c = -4;
>> x1 = (-b + sqrt(b^2 - 4*a*c))/(2*a)
x1 =
      1
>> x2 = (-b - sqrt(b^2 - 4*a*c))/(2*a)
x2 =
     -4
\end{minted}
\end{frame}

\begin{frame}[fragile]{Deilingarafgangar}
\begin{itemize}
 \item Hugmyndin um ``deilingarafgang'' kemur furðu oft fyrir í tölvunarfræði
 \item Dæmi um deilingarafgang: Þegar 10 er deilt með 3 gengur einn af. Þetta er táknað með lykilorðinu \emph{mod}:
 \begin{itemize}
  \item $10 \Mod{3} = 1$
 \end{itemize}
 \item Í Matlab má nota fallið \texttt{mod}:
\end{itemize}
\begin{minted}[frame=lines]{matlab}
>> mod(10,3)
ans =
     1
\end{minted}
\begin{itemize}
 \item Einnig er til fallið \texttt{rem}, sem er svipað en höndlar neikvæðar tölur á annan hátt
\end{itemize}
\end{frame}

\begin{frame}{Fyrirlestraræfing}
    \begin{enumerate}
        \item Eru eftirfarandi breytuheiti lögleg í Matlab?
        \begin{itemize}
            \item 007JamesBond
            \item batman
            \item Leðurblökumaðurinn
        \end{itemize}
        \item Skrifið þrjár Matlab-skipanir. Þær skulu skilgreina breyturnar $G=9.8$, $L=10$ og reikna upp úr formúlunni $\sqrt{\frac{G}{L}}$.
    \end{enumerate}
\end{frame}

\section{Slembitölur}

\begin{frame}{Hvað eru slembitölur?}
\begin{itemize}
 \item Slembitölur (e. \emph{random numbers}) eru tölur búnar til ``af handahófi''
 \item Ein stök tala er ekki slembin
 \item Getum skoðað runu af tölum og leitt líkur að því að hún sé slembin
\end{itemize}
\end{frame}

\begin{frame}[fragile]{Slembitölur í Matlab}
\begin{itemize}
 \item Fallið \texttt{rand} býr til jafndreifðar slembitölur
 \item Býr til kommutölur frá 0 til 1 (hvorug meðtalin!)
\end{itemize}
\begin{minted}[frame=lines]{matlab}
>> r1 = rand()
r1 =
    0.8147
>> r2 = rand()
r2 =
    0.9058
\end{minted}
\end{frame}

\begin{frame}{Slembitölugjafar}
\begin{itemize}
 \item Erfitt er að búa til ``raunverulegan'' slembitölugjafa (e. \emph{random number generator})
 \item Við notum reiknirit (e. \emph{algorithms}) til að búa til slembitölur
 \begin{itemize}
  \item Oft kallaðar gervislembitölur (e. \emph{pseudo-random numbers})
  \item Reikniritin geta engu að síður orðið ansi góð \pause
  \item \ldots og ansi léleg
 \end{itemize}
\end{itemize}
\end{frame}

\begin{frame}{Slembitölugjafar í tölvum}
\begin{columns}
\column{0.6\textwidth}
Notum stærðfræðiformúlu:
\[
r_{n+1} = (ar_n + c) \Mod{m}
\]
veljum okkur $a=3$, $c=4$, $m=10$, setjum $r_0 = 0$ og reiknum:
\begin{align*}
r1 = (3 \cdot 0 + 4) \Mod{10}  &=  4\\
r2 = (3 \cdot 4 + 4) \Mod{10}  &=  6\\
r3 = (3 \cdot 6 + 4) \Mod{10}  &=  2\\
r4 = (3 \cdot 2 + 4) \Mod{10}  &=  0\\
\end{align*}
\column{0.4\textwidth}
\begin{itemize}
\item Nú erum við komin í hring, fáum alltaf sömu niðurstöður
 \begin{itemize}
  \item Góðir slembitölugjafar eru með mun lengri hring
 \end{itemize}
 \item Þessi aðferð kallast línuleg samleifaraðferð (e. \emph{linear congruential method}). Formúla af þessari gerð kallast rakningarvensl (e. \emph{recurrence relation})
\end{itemize}
\end{columns}
\end{frame}

\begin{frame}{Upphafsgildi slembitölugjafa}
\begin{itemize}
 \item Matlab alltaf sama upphafsgildið í upphafi, svo 0.8147 er alltaf fyrsta slembitalan eftir ræsingu
 \begin{itemize}
  \item Hægt að breyta því með fallinu \texttt{rng}
 \end{itemize}
 \item Hægt að setja upphafsgildið á marga vegu:
 \begin{itemize}
  \item \texttt{rng('shuffle')}:upphafsstillt með kerfisklukkunni
  \item \texttt{rng('default')}: upphaflegt upphafsgildi
  \item \texttt{rng(tala)}: upphafsstillt með gildinu \texttt{tala}
 \end{itemize}
\end{itemize}
\end{frame}

\begin{frame}[shrink]{Upphafsgildið}
\vspace{1cm}
\begin{itemize}
 \item Slembitölugjafi samanstendur af mjög löngum hring af slembitölum
 \begin{itemize}
  \item Upphafsgildið ræður því hvar í hringnum við byrjum
  \item \texttt{rand} skilar svo næstu tölu á hringnum
 \end{itemize}
\end{itemize}
\begin{center}
\includegraphics[width=0.7\textwidth]{Pics/random-circle}
\end{center}
\end{frame}

\begin{frame}[fragile]{Dæmi um slembitölur}
\vspace{-0.5cm}
\begin{columns}
\column{0.6\textwidth}
\begin{minted}[frame=lines]{matlab}
>> r = rand()*10
r =
    9.0579
\end{minted}
\column{0.4\textwidth}
Kommutala á bilinu $]0;10[$
\end{columns}

\begin{columns}
\column{0.6\textwidth}
\begin{minted}[frame=lines]{matlab}
>> low = 5;
>> high = 12;
>> r = rand()*(high-low)+low
r =  7.7613
\end{minted}
\column{0.4\textwidth}
Kommutala á bilinu $]5;12[$
\end{columns}
\end{frame}

\begin{frame}[fragile]{Slembnar heiltölur}
\texttt{rand} fallið býr til slembnar kommutölur - til að gera slembnar heiltölur þarf önnur föll (eða smá fimleika)
\begin{columns}
\column{0.6\textwidth}
\begin{minted}[frame=lines]{matlab}
>> randomInteger = randi(20)
randomInteger =  8
\end{minted}
\column{0.4\textwidth}
Heiltölur á bilinu $[1;20]$
\end{columns}

\begin{columns}
\column{0.6\textwidth}
\begin{minted}[frame=lines]{matlab}
>> randomInteger = randi([4, 8])
randomInteger =  6
\end{minted}
\column{0.4\textwidth}
Heiltala á bilinu $[4; 8]$. (Hornklofarnir í Matlab-kóðanum skýrast seinna)
\end{columns}
\end{frame}

\section{Bókstafir}

\begin{frame}[fragile]{Bókstafir}
\begin{itemize}
 \item Hægt er að vinna með bókstafi og (önnur tákn) í Matlab.
 \item Notum einfaldar gæsalappir til að tákna þá
 \begin{itemize}
  \item T.d. \texttt{'a'}, \texttt{'E'}, \texttt{'\%'} og \texttt{'4'}
 \end{itemize}
 \item Athugum að það er munur á stafnum \texttt{'4'} (kóðaður \texttt{0110100} í ASCII) og tölunni $4$ (kóðuð \texttt{00000100}).
\end{itemize}
\begin{minted}[frame=lines]{matlab}
>> myChar = 'E'
myChar =
E
\end{minted}
\end{frame}

\begin{frame}{Að geyma bókstafi í tölvu}
\begin{columns}
\column{0.6\textwidth}
\begin{itemize}
 \item Möguleg leið til að kóða bókstafi og önnur tákn er ASCII
 \item Einföld kóðunaraðferð, mjög gömul
 \begin{itemize}
  \item Tákn gefin með 7 bitum
  \begin{itemize}
   \item ``Efsti'' bitinn notaður sem varbiti (e. \emph{parity bit})
  \end{itemize}
 \end{itemize}
 \item Líklega mest notaða kóðunaraðferðin í dag: UTF-8
 \begin{itemize}
  \item ASCII er innifalið í UTF-8
 \end{itemize}
\end{itemize}
\column{0.4\textwidth}
\includegraphics[width=\linewidth]{Pics/ascii-table}
\\Hluti ASCII-töflunnar af \href{https://en.wikipedia.org/wiki/ASCII}{Wikipedia}
\end{columns}
\end{frame}

\begin{frame}[fragile]{Föll fyrir kóðun}
\begin{itemize}
 \item Við getum fundið kóðann fyrir einstaka bókstafi
 \item Dæmi: Stafurinn \texttt{'Þ'} sem 32 bita heiltala: 
\begin{minted}[frame=lines]{matlab}
>> int32('Þ')
ans =
         222
\end{minted}
 \item Einnig er hægt að fara í hina áttina:
\begin{minted}[frame=lines]{matlab}
>> char(222)
ans =
Þ
\end{minted}
\end{itemize}
\end{frame}

\section{Rökyrðingar}

\begin{frame}{Rökyrðingar}
\begin{itemize}
 \item Rökyrðingar (e. \emph{logical expressions}) vinna með \emph{sanngildi} í stað talna.
 \begin{itemize}
  \item Sanngildi eru tvö: \texttt{true} og ósatt \texttt{false}
  \item Í Matlab er oft notað $1$ og $0$ í stað \texttt{true} og \texttt{false}
 \end{itemize}
 \item Notum nýja virkja (e. \emph{operators}). Tvær gerðir:
 \begin{itemize}
  \item Samanburðarvirkjar (\emph{relational})
  \item Rökvirkjar (\emph{logical})
 \end{itemize}
\end{itemize}
\end{frame}

\subsection{Samanburðarvirkjar}
\begin{frame}{Samanburðarvirkjar}
\begin{columns}
\column{0.5\textwidth}
\begin{itemize}
 \item Samanburðarvirkjar bera saman gildi og skila sanngildi (e. \emph{truth value}).
 \item Sanngildi hafa sitt eigið tag, \texttt{logical}\footnotemark
\end{itemize}
\column{0.5\textwidth}

\vspace{0.5cm}
Samanburðarvirkjar í Matlab: 

\vspace{0.2cm}
\begin{tabular}{ll}
\toprule
Virki&Merking\\
\midrule
\texttt{>}&stærri en\\
\texttt{<}&minni en\\
\texttt{>=}&stærri en eða jafn ($\geq$)\\
\texttt{<=}&minni en eða jafn ($\leq$)\\
\texttt{==}&jafnt og\\
\texttt{\~}\texttt{=}&ekki jafnt og\\
\bottomrule
\end{tabular}
\end{columns}
\footnotetext{Við höfum núna líka séð \texttt{double}, \texttt{char} og \texttt{int32}.}
\end{frame}

\begin{frame}[fragile]{Sanngildi}
Í Matlab er sanngildið ``satt'' oftast táknað með \texttt{logical} gildinu \texttt{1} og ``ósatt'' með \texttt{logical} gildinu \texttt{0}.
\begin{columns}
\column{0.5\textwidth}
\begin{minted}[frame=lines]{matlab}
>> 2 < 3
ans =
     1
\end{minted}
\begin{minted}[frame=lines]{matlab}
>> 3 >= 2
ans =
     1
\end{minted}
\column{0.5\textwidth}
\begin{minted}[frame=lines]{matlab}
>> 'b' < 'a'
ans =
     0
\end{minted}

\vspace{0.09cm}
Öll svörin eru af taginu \texttt{logical}, þó þau líti út eins og venjulegar tölur (sem væru af tagi eins og \texttt{double} eða \texttt{int32}).
\end{columns}
\end{frame}

\subsection{Rökvirkjar}
\begin{frame}{Rökvirkjar}
Rökvirkjar vinna með sanngildi:
\begin{center}
\begin{tabular}{ll}
\toprule
Virki&Merking\\
\midrule
||& eða\footnotemark, skilar satt ef annað viðfangið er satt\\
\&\& & og\footnotemark[1], skilar satt ef bæði viðföngin eru sönn\\
\~{} &ekki, skilar öfugu við viðfangið\\
\bottomrule
\end{tabular}
\end{center}
Einnig er til rökfallið \texttt{xor}. Það tekur tvö viðföng og skilar ``annaðhvort eða''. Satt ef annað viðfangið er satt, en ekki ef bæði eru sönn.
\footnotetext{Þessir virkjar eiga við stök sanngildi. Virkjar fyrir vigra koma seinna.}
\end{frame}

\begin{frame}[fragile]{Dæmi um rökvirkja}
\begin{columns}
\column{0.5\textwidth}
\begin{minted}[frame=lines]{matlab}
>> (2 < 4) || (2 > 4)
ans =
     1
\end{minted}
\begin{minted}[frame=lines]{matlab}
>> (2 < 4) && (2 > 4)
ans =
     0
\end{minted}
\column{0.5\textwidth}
\begin{minted}[frame=lines]{matlab}
>> ~(2 < 4)
ans =
     0
\end{minted}
\begin{minted}[frame=lines]{matlab}
>> xor(2 < 4, 2 > 4)
ans =
     1
\end{minted}
\end{columns}
\end{frame}

\begin{frame}{Sanntöflur rökvirkja}
\begin{columns}
\column{0.33\textwidth}
\begin{center}
Eða\\
\begin{tabular}{ccc}
\toprule
\texttt{x}&\texttt{y}&\texttt{x||y}\\
\midrule
1&1&1\\
1&0&1\\
0&1&1\\
0&0&0\\
\bottomrule
\end{tabular}
\end{center}
\column{0.33\textwidth}
\begin{center}
Og\\
\begin{tabular}{ccc}
\toprule
\texttt{x}&\texttt{y}&\texttt{x\&\&y}\\
\midrule
1&1&1\\
1&0&0\\
0&1&0\\
0&0&0\\
\bottomrule
\end{tabular}
\end{center}
\column{0.33\textwidth}
\begin{center}
XOR\\
\begin{tabular}{ccc}
\toprule
\texttt{x}&\texttt{y}&\texttt{xor(x,y)}\\
\midrule
1&1&0\\
1&0&1\\
0&1&1\\
0&0&0\\
\bottomrule
\end{tabular}
\end{center}
\end{columns}
\end{frame}

\begin{frame}{Forgangur virkja}
\vspace{-0.5cm}
Stækkum forgangstöfluna:
\begin{center}
\small
\begin{tabular}{ll}
\toprule
Virki&Hópur\\
\midrule
\texttt{()}&Svigar\\
\texttt{'} og \texttt{\^}&Bylt og veldi\\
\texttt{-} og \texttt{\~{}}&Viðsnúningur formerkis og neitun\\
\texttt{*, /, \textbackslash}&Margföldun og deiling\\
\texttt{+, -}&Samlagning og frádráttur\\
\texttt{:}&Tvípunktur (fyrir vigra)\\
\texttt{<}, \texttt{<=}, \texttt{>}, \texttt{>=}, \texttt{==}, \texttt{\~{}=}&Samanburður\\
\texttt{\&\&}&Og\\
\texttt{||}&Eða\\
\texttt{=}&Gildisveiting\\
\bottomrule
\end{tabular}
\end{center}
Virkjar efst í töflunni eru framkvæmdir fyrst.
\end{frame}

\begin{frame}{Fyrirlestraræfing}
\begin{enumerate}
    \item Skrifið Matlab-skipanir sem búa til eftirfarandi (jafndreifðar) slembitölur:
    \begin{itemize}
        \item Kommutölur á bilinu $]0;\pi[$
        \item Heiltölur á bilinu $[1;4]$
        \item Heiltölurnar $0$ eða $1$.
    \end{itemize}
    \item Hvaða sanngildi hefur yrðingin \texttt{3 > 2 > 1}? (\emph{Hint}: Einn virki er athugaður í einu.)
    \item Skrifið yrðingu sem er sönn þegar gildi breytunnar \texttt{a} er hvorki 0 né 10.
\end{enumerate}
\end{frame}



\end{document}
